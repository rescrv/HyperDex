\chapter{Java API}
\label{chap:api:java}

\section{Client Library}
\label{sec:api:java:client}

HyperDex provides Java bindings in the package \code{org.hyperdex.client}.  This
package wraps the HyperDex C Client library and enables the use of native Java
data types.

This library was brought up-to-date following the 1.0.5 release.

\subsection{Building the HyperDex Java Binding}
\label{sec:api:java:building}

The HyperDex Java Binding must be requested at configure time as it is not
automatically built.  You can ensure that the Java bindings are always built by
providing the \code{--enable-java-bindings} option to \code{./configure} like
so:

\begin{consolecode}
% ./configure --enable-client --enable-java-bindings
\end{consolecode}

\subsection{Using Java Within Your Application}
\label{sec:api:java:using}

All client operation are defined in the \code{org.hyperdex.client} package.  You
can access this in your program with:

\begin{javacode}
import org.hyperdex.client.*;
\end{javacode}

\subsection{Hello World}
\label{sec:api:java:hello-world}

The following is a minimal application that stores the value "Hello World" and
then immediately retrieves the value:

\inputminted{java}{\topdir/java/client/HelloWorld.java}

You can run this example with:

\begin{consolecode}
% javac HelloWorld.java
% java -Djava.library.path=/usr/local/lib HelloWorld
put: true
got: {v=Hello World!}
\end{consolecode}

Right away, there are several points worth noting in this example:

\begin{itemize}
\item Every operation is synchronous.  The PUT and GET operations run to
completion by default.

\item Java types are automatically converted to HyperDex types.  There's no need
to specify information such as the length of each string, as one would do with
the C API.

\item We specify -Djava.library.path=/usr/local/lib.  This is necessary for
builds from source, but should not be necessary for Java bindings installed
using binary packages.
\end{itemize}

\subsection{Asynchronous Operations}
\label{sec:api:java:async-ops}

For convenience, the Java bindings treat every operation as synchronous.  This
enables you to write short programs without concern for asynchronous operations.
Most operations come with an asynchronous form, denoted by the \code{async\_}
prefix.  For example, the above Hello World example could be rewritten in
asynchronous fashion as such:

\inputminted{java}{\topdir/java/client/HelloWorldAsyncWait.java}

This enables applications to issue multiple requests simultaneously and wait for
their completion in an application-specific order.  It's also possible to use
the \code{loop} method on the client object to wait for the next request to
complete:

\inputminted{java}{\topdir/java/client/HelloWorldAsyncLoop.java}

\subsection{Data Structures}
\label{sec:api:java:data-structures}

The Java bindings automatically manage conversion of data types from Java to
HyperDex types, enabling applications to be written in idiomatic Java.

\subsubsection{Examples}
\label{sec:api:java:examples}

This section shows examples of Java data structures that are recognized by
HyperDex.  The examples here are for illustration purposes and are not
exhaustive.

\paragraph{Strings}

The HyperDex client recognizes Java's strings and automatically converts them to
HyperDex strings.  For example, the following call stores a string:
equivalent and have the same effect:

\begin{javacode}
Map<String, Object> attrs = new HashMap<String, Object>();
attrs.put("v", "someattrs");
c.put("kv", "somekey", attrs);
\end{javacode}

\paragraph{Integers}

The HyperDex client recognizes Java's integers and automatically converts them
to HyperDex integers.  For example:

\begin{javacode}
Map<String, Object> attrs = new HashMap<String, Object>();
attrs.put("v", 42);
c.put("kv", "somekey", attrs);
\end{javacode}

\paragraph{Floats}

The HyperDex client recognizes Java's floating point numbers and automatically
converts them to HyperDex floats.  For example:

\begin{javacode}
Map<String, Object> attrs = new HashMap<String, Object>();
attrs.put("v", 3.1415);
c.put("kv", "somekey", attrs);
\end{javacode}

\paragraph{Lists}

The HyperDex client recognizes Java lists and automatically converts them to
HyperDex lists.  For example:

\begin{javacode}
List<Object> list = new ArrayList<Object>();
list.add("a");
list.add("b");
list.add("c");
Map<String, Object> attrs = new HashMap<String, Object>();
attrs.put("v", list);
c.put("kv", "somekey", attrs);
\end{javacode}

\paragraph{Sets}

The HyperDex client recognizes Java sets and automatically converts them to
HyperDex sets.  For example:

\begin{javacode}
Set<Object> set = new HashSet<Object>();
set.add("a");
set.add("b");
set.add("c");
Map<String, Object> attrs = new HashMap<String, Object>();
attrs.put("v", set);
c.put("kv", "somekey", attrs);
\end{javacode}

\paragraph{Maps}

The HyperDex client recognizes Java maps and automatically converts them to
HyperDex maps.  For example:

\begin{javacode}
Map<Object, Object> map = new HashMap<Object, Object>();
map.put("k", "v");
Map<String, Object> attrs = new HashMap<String, Object>();
attrs.put("v", map);
c.put("kv", "somekey", attrs);
\end{javacode}

\subsection{Attributes}
\label{sec:api:java:attributes}

Attributes in Java are specified in the form of a map from attribute names to
their values.  As you can see in the examples above, attributes are specified in
the form:

\begin{javacode}
Map<String, Object> attrs = new HashMap<String, Object>();
\end{javacode}

\subsection{Map Attributes}
\label{sec:api:java:map-attributes}

Map attributes in Java are specified in the form of a nested map.  The outer
map key specifies the name, while the inner map key-value pair's specify the
key-value pair of the map.  For example:

\begin{javacode}
Map<String, Map<Object, Object>> mapattrs = new HashMap<String, Map<Object, Object>>();
\end{javacode}

\subsection{Predicates}
\label{sec:api:java:predicates}

Predicates in Java are specified in the form of a hash from attribute names to
their predicates.  In the simple case, the predicate is just a value to be
compared against:

\begin{javacode}
Map<String, Object> checks = new HashMap<String, Object>();
checks.put("v", "value");
\end{javacode}

This is the same as saying:

\begin{javacode}
Map<String, Object> checks = new HashMap<String, Object>();
checks.put("v", new Equals("value"));
\end{javacode}

The Java bindings support the full range of predicates supported by HyperDex
itself.  For example:

\begin{javacode}
checks.put("v", new LessEqual(5));
checks.put("v", new GreaterEqual(5));
checks.put("v", new RangeEqual(5, 10));
checks.put("v", new Regex("^s.*"));
checks.put("v", new LengthEquals(5));
checks.put("v", new LengthLessEqual(5));
checks.put("v", new LengthGreaterEqual(5));
checks.put("v", new Contains('value'));
\end{javacode}

\subsection{Error Handling}
\label{sec:api:java:error-handling}

All error handling within the Java bindings is done via the
\code{try}/\code{catch} mechanism of Java.  Errors will be thrown by the package
and should be handled by your application.  For example, if we were trying to
store an integer (5) as attribute \code{"v"}, where \code{"v"} is actually a
string, we'd generate an error.

\begin{javacode}
try
{
    attrs.put("v", 5);
    System.out.println("put: " + c.put("kv", "k", attrs));
}
catch (HyperDexClientException e)
{
    System.out.println(e.status());
    System.out.println(e.symbol());
    System.out.println(e.message());
}
\end{javacode}

Errors of type \code{HyperDexClientException} will contain both a message
indicating what went wrong, as well as the underlying \code{enum
hyperdex\_client\_returncode}.  The member \code{status} indicates the numeric
value of this enum, while \code{symbol} returns the enum as a string.  The above
code will fail with the following output:

\begin{verbatim}
8525
HYPERDEX_CLIENT_WRONGTYPE
invalid attribute "v": attribute has the wrong type
\end{verbatim}

\subsection{Operations}
\label{sec:api:java:ops}

% Copyright (c) 2014, Cornell University
% All rights reserved.
%
% Redistribution and use in source and binary forms, with or without
% modification, are permitted provided that the following conditions are met:
%
%     * Redistributions of source code must retain the above copyright notice,
%       this list of conditions and the following disclaimer.
%     * Redistributions in binary form must reproduce the above copyright
%       notice, this list of conditions and the following disclaimer in the
%       documentation and/or other materials provided with the distribution.
%     * Neither the name of HyperDex nor the names of its contributors may be
%       used to endorse or promote products derived from this software without
%       specific prior written permission.
%
% THIS SOFTWARE IS PROVIDED BY THE COPYRIGHT HOLDERS AND CONTRIBUTORS "AS IS"
% AND ANY EXPRESS OR IMPLIED WARRANTIES, INCLUDING, BUT NOT LIMITED TO, THE
% IMPLIED WARRANTIES OF MERCHANTABILITY AND FITNESS FOR A PARTICULAR PURPOSE ARE
% DISCLAIMED. IN NO EVENT SHALL THE COPYRIGHT OWNER OR CONTRIBUTORS BE LIABLE
% FOR ANY DIRECT, INDIRECT, INCIDENTAL, SPECIAL, EXEMPLARY, OR CONSEQUENTIAL
% DAMAGES (INCLUDING, BUT NOT LIMITED TO, PROCUREMENT OF SUBSTITUTE GOODS OR
% SERVICES; LOSS OF USE, DATA, OR PROFITS; OR BUSINESS INTERRUPTION) HOWEVER
% CAUSED AND ON ANY THEORY OF LIABILITY, WHETHER IN CONTRACT, STRICT LIABILITY,
% OR TORT (INCLUDING NEGLIGENCE OR OTHERWISE) ARISING IN ANY WAY OUT OF THE USE
% OF THIS SOFTWARE, EVEN IF ADVISED OF THE POSSIBILITY OF SUCH DAMAGE.

% This LaTeX file is generated by bindings/python.py

%%%%%%%%%%%%%%%%%%%% get %%%%%%%%%%%%%%%%%%%%
\pagebreak
\subsubsection{\code{get}}
\label{api:python:get}
\index{get!Python API}
Retreive the object with key "key" from space "space".

\noindent\textbf{Cost:}  Approximately one network round trip.


\input{api/shards/linearizable}


\paragraph{Definition:}
\begin{pythoncode}
def get(self, spacename, key)
\end{pythoncode}

\paragraph{Parameters:}
\begin{itemize}[noitemsep]
\item \code{spacename}\\
The name of an existing space.

\item \code{key}\\
The key for the operation where \code{key} is a bytestring and \code{key\_sz}
specifies the number of bytes in \code{key}.

\end{itemize}

\paragraph{Returns:}
This function returns via the provided callback.  In the normal case, the first
argument will indicate success or failure of the operation with one of the
following values:

\begin{itemize}[noitemsep]
\item A Javascript representation of the stored object.
\item \code{null} if the operation is a retrieval operation and no object was
    found.
\end{itemize}

If the operation encounters any error, the error argument will be provided and
will specify the error, in which case the first argument is undefined.


\pagebreak
\subsubsection{\code{async\_get}}
\label{api:python:async_get}
\index{async\_get!Python API}
Retreive the object with key "key" from space "space".

\noindent\textbf{Cost:}  Approximately one network round trip.


\input{api/shards/linearizable}


\paragraph{Definition:}
\begin{pythoncode}
def async_get(self, spacename, key)
\end{pythoncode}

\paragraph{Parameters:}
\begin{itemize}[noitemsep]
\item \code{spacename}\\
The name of an existing space.

\item \code{key}\\
The key for the operation where \code{key} is a bytestring and \code{key\_sz}
specifies the number of bytes in \code{key}.

\end{itemize}

\paragraph{Returns:}
A \code{Deferred} object with a \code{wait} method that returns the object if
found, \code{None} if not found.  Raises exception on error.


\paragraph{See also:}  This is the asynchronous form of \code{get}.

%%%%%%%%%%%%%%%%%%%% get_partial %%%%%%%%%%%%%%%%%%%%
\pagebreak
\subsubsection{\code{get\_partial}}
\label{api:python:get_partial}
\index{get\_partial!Python API}
Get part of an object by key.  This will return only the listed attribute names.

\paragraph{Behavior:}
\begin{itemize}[noitemsep]
\item This operation return to the user the requested object(s).

\end{itemize}


\paragraph{Definition:}
\begin{pythoncode}
def get_partial(self, spacename, key, attributenames)
\end{pythoncode}

\paragraph{Parameters:}
\begin{itemize}[noitemsep]
\item \code{spacename}\\
The name of an existing space.

\item \code{key}\\
The key for the operation where \code{key} is a bytestring and \code{key\_sz}
specifies the number of bytes in \code{key}.

\item \code{attributenames}\\
A list of attributes to return.  \code{attrnames} points to an array of length
\code{attrnames\_sz}.

\end{itemize}

\paragraph{Returns:}
This function returns via the provided callback.  In the normal case, the first
argument will indicate success or failure of the operation with one of the
following values:

\begin{itemize}[noitemsep]
\item A Javascript representation of the stored object.
\item \code{null} if the operation is a retrieval operation and no object was
    found.
\end{itemize}

If the operation encounters any error, the error argument will be provided and
will specify the error, in which case the first argument is undefined.


\pagebreak
\subsubsection{\code{async\_get\_partial}}
\label{api:python:async_get_partial}
\index{async\_get\_partial!Python API}
Get part of an object by key.  This will return only the listed attribute names.

\paragraph{Behavior:}
\begin{itemize}[noitemsep]
\item This operation return to the user the requested object(s).

\end{itemize}


\paragraph{Definition:}
\begin{pythoncode}
def async_get_partial(self, spacename, key, attributenames)
\end{pythoncode}

\paragraph{Parameters:}
\begin{itemize}[noitemsep]
\item \code{spacename}\\
The name of an existing space.

\item \code{key}\\
The key for the operation where \code{key} is a bytestring and \code{key\_sz}
specifies the number of bytes in \code{key}.

\item \code{attributenames}\\
A list of attributes to return.  \code{attrnames} points to an array of length
\code{attrnames\_sz}.

\end{itemize}

\paragraph{Returns:}
A \code{Deferred} object with a \code{wait} method that returns the object if
found, \code{None} if not found.  Raises exception on error.


\paragraph{See also:}  This is the asynchronous form of \code{get\_partial}.

%%%%%%%%%%%%%%%%%%%% put %%%%%%%%%%%%%%%%%%%%
\pagebreak
\subsubsection{\code{put}}
\label{api:python:put}
\index{put!Python API}
Store or update an object by key.  The object's attributes will be set to the
values specified by \code{attrs}.
If the object exists, it will be updated and all existing values not altered by
\code{attrs} will be preserved.  If the object does not exist, a new object will
be created, with its attributes initialized to their default values.



\paragraph{Definition:}
\begin{pythoncode}
def put(self, spacename, key, attributes)
\end{pythoncode}

\paragraph{Parameters:}
\begin{itemize}[noitemsep]
\item \code{spacename}\\
The name of an existing space.

\item \code{key}\\
The key for the operation where \code{key} is a bytestring and \code{key\_sz}
specifies the number of bytes in \code{key}.

\item \code{attributes}\\
The attributes to modify and their respective values.

\end{itemize}

\paragraph{Returns:}
True if the operation succeeded, False if any of the provided predicates failed.
Raises an exception on error.


\pagebreak
\subsubsection{\code{async\_put}}
\label{api:python:async_put}
\index{async\_put!Python API}
Store or update an object by key.  The object's attributes will be set to the
values specified by \code{attrs}.
If the object exists, it will be updated and all existing values not altered by
\code{attrs} will be preserved.  If the object does not exist, a new object will
be created, with its attributes initialized to their default values.



\paragraph{Definition:}
\begin{pythoncode}
def async_put(self, spacename, key, attributes)
\end{pythoncode}

\paragraph{Parameters:}
\begin{itemize}[noitemsep]
\item \code{spacename}\\
The name of an existing space.

\item \code{key}\\
The key for the operation where \code{key} is a bytestring and \code{key\_sz}
specifies the number of bytes in \code{key}.

\item \code{attributes}\\
The attributes to modify and their respective values.

\end{itemize}

\paragraph{Returns:}
A Deferred object with a \code{wait} method that returns True if the operation
succeeded, or False if any of the provided predicates failed.  Raises an
exception on error.


\paragraph{See also:}  This is the asynchronous form of \code{put}.

%%%%%%%%%%%%%%%%%%%% cond_put %%%%%%%%%%%%%%%%%%%%
\pagebreak
\subsubsection{\code{cond\_put}}
\label{api:python:cond_put}
\index{cond\_put!Python API}
Conditionally update an the object stored under \code{key} in \code{space}.
Existing values will be overwritten with the values specified by \code{attrs}.
Values not specified by \code{attrs} will remain unchanged.
This operation requires a pre-existing object in order to complete successfully.
If no object exists, the operation will fail with \code{NOTFOUND}.


This operation will succeed if and only if the predicates specified by
\code{checks} hold on the pre-existing object.  If any of the predicates are not
true for the existing object, then the operation will have no effect and fail
with \code{CMPFAIL}.

All checks are atomic with the write.  HyperDex guarantees that no other
operation will come between validating the checks, and writing the new version
of the object.



\paragraph{Definition:}
\begin{pythoncode}
def cond_put(self, spacename, key, predicates, attributes)
\end{pythoncode}

\paragraph{Parameters:}
\begin{itemize}[noitemsep]
\item \code{spacename}\\
The name of an existing space.

\item \code{key}\\
The key for the operation where \code{key} is a bytestring and \code{key\_sz}
specifies the number of bytes in \code{key}.

\item \code{predicates}\\
The predicates to check against.  \code{checks} is an object with the individual
predicates as properties.

\item \code{attributes}\\
The attributes to modify and their respective values.

\end{itemize}

\paragraph{Returns:}
True if the operation succeeded, False if any of the provided predicates failed.
Raises an exception on error.


\pagebreak
\subsubsection{\code{async\_cond\_put}}
\label{api:python:async_cond_put}
\index{async\_cond\_put!Python API}
Conditionally update an the object stored under \code{key} in \code{space}.
Existing values will be overwritten with the values specified by \code{attrs}.
Values not specified by \code{attrs} will remain unchanged.
This operation requires a pre-existing object in order to complete successfully.
If no object exists, the operation will fail with \code{NOTFOUND}.


This operation will succeed if and only if the predicates specified by
\code{checks} hold on the pre-existing object.  If any of the predicates are not
true for the existing object, then the operation will have no effect and fail
with \code{CMPFAIL}.

All checks are atomic with the write.  HyperDex guarantees that no other
operation will come between validating the checks, and writing the new version
of the object.



\paragraph{Definition:}
\begin{pythoncode}
def async_cond_put(self, spacename, key, predicates, attributes)
\end{pythoncode}

\paragraph{Parameters:}
\begin{itemize}[noitemsep]
\item \code{spacename}\\
The name of an existing space.

\item \code{key}\\
The key for the operation where \code{key} is a bytestring and \code{key\_sz}
specifies the number of bytes in \code{key}.

\item \code{predicates}\\
The predicates to check against.  \code{checks} is an object with the individual
predicates as properties.

\item \code{attributes}\\
The attributes to modify and their respective values.

\end{itemize}

\paragraph{Returns:}
A Deferred object with a \code{wait} method that returns True if the operation
succeeded, or False if any of the provided predicates failed.  Raises an
exception on error.


\paragraph{See also:}  This is the asynchronous form of \code{cond\_put}.

%%%%%%%%%%%%%%%%%%%% cond_put_or_create %%%%%%%%%%%%%%%%%%%%
\pagebreak
\subsubsection{\code{cond\_put\_or\_create}}
\label{api:python:cond_put_or_create}
\index{cond\_put\_or\_create!Python API}
Conditionally update an the object stored under \code{key} in \code{space}.
Existing values will be overwritten with the values specified by \code{attrs}.
Values not specified by \code{attrs} will remain unchanged.  If the object
exists, this is equivalent to a \code{cond\_put}.  If the object does not exist,
this is equivalent to a \code{put}

This operation will succeed if and only if the predicates specified by
\code{checks} hold on the pre-existing object.  If any of the predicates are not
true for the existing object, then the operation will have no effect and fail
with \code{CMPFAIL}.

All checks are atomic with the write.  HyperDex guarantees that no other
operation will come between validating the checks, and writing the new version
of the object.



\paragraph{Definition:}
\begin{pythoncode}
def cond_put_or_create(self, spacename, key, predicates, attributes)
\end{pythoncode}

\paragraph{Parameters:}
\begin{itemize}[noitemsep]
\item \code{spacename}\\
The name of an existing space.

\item \code{key}\\
The key for the operation where \code{key} is a bytestring and \code{key\_sz}
specifies the number of bytes in \code{key}.

\item \code{predicates}\\
The predicates to check against.  \code{checks} is an object with the individual
predicates as properties.

\item \code{attributes}\\
The attributes to modify and their respective values.

\end{itemize}

\paragraph{Returns:}
True if the operation succeeded, False if any of the provided predicates failed.
Raises an exception on error.


\pagebreak
\subsubsection{\code{async\_cond\_put\_or\_create}}
\label{api:python:async_cond_put_or_create}
\index{async\_cond\_put\_or\_create!Python API}
Conditionally update an the object stored under \code{key} in \code{space}.
Existing values will be overwritten with the values specified by \code{attrs}.
Values not specified by \code{attrs} will remain unchanged.  If the object
exists, this is equivalent to a \code{cond\_put}.  If the object does not exist,
this is equivalent to a \code{put}

This operation will succeed if and only if the predicates specified by
\code{checks} hold on the pre-existing object.  If any of the predicates are not
true for the existing object, then the operation will have no effect and fail
with \code{CMPFAIL}.

All checks are atomic with the write.  HyperDex guarantees that no other
operation will come between validating the checks, and writing the new version
of the object.



\paragraph{Definition:}
\begin{pythoncode}
def async_cond_put_or_create(self, spacename, key, predicates, attributes)
\end{pythoncode}

\paragraph{Parameters:}
\begin{itemize}[noitemsep]
\item \code{spacename}\\
The name of an existing space.

\item \code{key}\\
The key for the operation where \code{key} is a bytestring and \code{key\_sz}
specifies the number of bytes in \code{key}.

\item \code{predicates}\\
The predicates to check against.  \code{checks} is an object with the individual
predicates as properties.

\item \code{attributes}\\
The attributes to modify and their respective values.

\end{itemize}

\paragraph{Returns:}
A Deferred object with a \code{wait} method that returns True if the operation
succeeded, or False if any of the provided predicates failed.  Raises an
exception on error.


\paragraph{See also:}  This is the asynchronous form of \code{cond\_put\_or\_create}.

%%%%%%%%%%%%%%%%%%%% group_put %%%%%%%%%%%%%%%%%%%%
\pagebreak
\subsubsection{\code{group\_put}}
\label{api:python:group_put}
\index{group\_put!Python API}
Update all objects stored in \code{space} that match \code{checks}.  Existing
values will be overwritten with the values specified by \code{attrs}.  Values not
specified by \code{attrs} will remain unchanged.

This operation will only affect objects that match the provided \code{checks}.
Objects that do not match \code{checks} will be unaffected by the group call.
Each object that matches \code{checks} will be atomically updated with the check
on the object.  HyperDex guarantees that no object will be altered if the
\code{checks} do not pass at the time of the write.  Objects that are updated
concurrently with the group call may or may not be updated; however, regardless
of any other concurrent operations, the preceding guarantee will always hold.



\paragraph{Definition:}
\begin{pythoncode}
def group_put(self, spacename, predicates, attributes)
\end{pythoncode}

\paragraph{Parameters:}
\begin{itemize}[noitemsep]
\item \code{spacename}\\
The name of an existing space.

\item \code{predicates}\\
The predicates to check against.  \code{checks} is an object with the individual
predicates as properties.

\item \code{attributes}\\
The attributes to modify and their respective values.

\end{itemize}

\paragraph{Returns:}
A count of the number of objects, and a \code{client.Error} object indicating
the status of the operation.


\pagebreak
\subsubsection{\code{async\_group\_put}}
\label{api:python:async_group_put}
\index{async\_group\_put!Python API}
Update all objects stored in \code{space} that match \code{checks}.  Existing
values will be overwritten with the values specified by \code{attrs}.  Values not
specified by \code{attrs} will remain unchanged.

This operation will only affect objects that match the provided \code{checks}.
Objects that do not match \code{checks} will be unaffected by the group call.
Each object that matches \code{checks} will be atomically updated with the check
on the object.  HyperDex guarantees that no object will be altered if the
\code{checks} do not pass at the time of the write.  Objects that are updated
concurrently with the group call may or may not be updated; however, regardless
of any other concurrent operations, the preceding guarantee will always hold.



\paragraph{Definition:}
\begin{pythoncode}
def async_group_put(self, spacename, predicates, attributes)
\end{pythoncode}

\paragraph{Parameters:}
\begin{itemize}[noitemsep]
\item \code{spacename}\\
The name of an existing space.

\item \code{predicates}\\
The predicates to check against.  \code{checks} is an object with the individual
predicates as properties.

\item \code{attributes}\\
The attributes to modify and their respective values.

\end{itemize}

\paragraph{Returns:}
This asynchronous operation returns a \code{Deferred} object with a
\code{waitForIt} method which blocks and returns a number indicating the number
of objects counted.

On error, this function will raise a \code{HyperDexClientException} describing
the error.


\paragraph{See also:}  This is the asynchronous form of \code{group\_put}.

%%%%%%%%%%%%%%%%%%%% put_if_not_exist %%%%%%%%%%%%%%%%%%%%
\pagebreak
\subsubsection{\code{put\_if\_not\_exist}}
\label{api:python:put_if_not_exist}
\index{put\_if\_not\_exist!Python API}
Store an object in space "space" under key "key" if and only if it does not
already exist.

The object will be create, and the attributes specified by \texttt{attrs} will
be set to their respective values.  Any attributes not specified by
\texttt{attrs} will be initialized to their default values.  The check is atomic
with the write, and is guaranteed to never overwrite an existing object.

\noindent\textbf{Cost:}  Approximately one traversal of the value-dependent
chain.


\input{api/shards/linearizable}


\paragraph{Definition:}
\begin{pythoncode}
def put_if_not_exist(self, spacename, key, attributes)
\end{pythoncode}

\paragraph{Parameters:}
\begin{itemize}[noitemsep]
\item \code{spacename}\\
The name of an existing space.

\item \code{key}\\
The key for the operation where \code{key} is a bytestring and \code{key\_sz}
specifies the number of bytes in \code{key}.

\item \code{attributes}\\
The attributes to modify and their respective values.

\end{itemize}

\paragraph{Returns:}
True if the operation succeeded, False if any of the provided predicates failed.
Raises an exception on error.


\pagebreak
\subsubsection{\code{async\_put\_if\_not\_exist}}
\label{api:python:async_put_if_not_exist}
\index{async\_put\_if\_not\_exist!Python API}
Store an object in space "space" under key "key" if and only if it does not
already exist.

The object will be create, and the attributes specified by \texttt{attrs} will
be set to their respective values.  Any attributes not specified by
\texttt{attrs} will be initialized to their default values.  The check is atomic
with the write, and is guaranteed to never overwrite an existing object.

\noindent\textbf{Cost:}  Approximately one traversal of the value-dependent
chain.


\input{api/shards/linearizable}


\paragraph{Definition:}
\begin{pythoncode}
def async_put_if_not_exist(self, spacename, key, attributes)
\end{pythoncode}

\paragraph{Parameters:}
\begin{itemize}[noitemsep]
\item \code{spacename}\\
The name of an existing space.

\item \code{key}\\
The key for the operation where \code{key} is a bytestring and \code{key\_sz}
specifies the number of bytes in \code{key}.

\item \code{attributes}\\
The attributes to modify and their respective values.

\end{itemize}

\paragraph{Returns:}
A Deferred object with a \code{wait} method that returns True if the operation
succeeded, or False if any of the provided predicates failed.  Raises an
exception on error.


\paragraph{See also:}  This is the asynchronous form of \code{put\_if\_not\_exist}.

%%%%%%%%%%%%%%%%%%%% del %%%%%%%%%%%%%%%%%%%%
\pagebreak
\subsubsection{\code{del}}
\label{api:python:del}
\index{del!Python API}
Delete an object by key.

%%% Generated below here
\paragraph{Behavior:}
\begin{itemize}[noitemsep]
If no object exists, the operation will fail with \code{NOTFOUND}.

\end{itemize}


\paragraph{Definition:}
\begin{pythoncode}
def delete(self, spacename, key)
\end{pythoncode}

\paragraph{Parameters:}
\begin{itemize}[noitemsep]
\item \code{spacename}\\
The name of an existing space.

\item \code{key}\\
The key for the operation where \code{key} is a bytestring and \code{key\_sz}
specifies the number of bytes in \code{key}.

\end{itemize}

\paragraph{Returns:}
True if the operation succeeded, False if any of the provided predicates failed.
Raises an exception on error.


\pagebreak
\subsubsection{\code{async\_del}}
\label{api:python:async_del}
\index{async\_del!Python API}
Delete an object by key.

%%% Generated below here
\paragraph{Behavior:}
\begin{itemize}[noitemsep]
If no object exists, the operation will fail with \code{NOTFOUND}.

\end{itemize}


\paragraph{Definition:}
\begin{pythoncode}
def async_delete(self, spacename, key)
\end{pythoncode}

\paragraph{Parameters:}
\begin{itemize}[noitemsep]
\item \code{spacename}\\
The name of an existing space.

\item \code{key}\\
The key for the operation where \code{key} is a bytestring and \code{key\_sz}
specifies the number of bytes in \code{key}.

\end{itemize}

\paragraph{Returns:}
A Deferred object with a \code{wait} method that returns True if the operation
succeeded, or False if any of the provided predicates failed.  Raises an
exception on error.


\paragraph{See also:}  This is the asynchronous form of \code{del}.

%%%%%%%%%%%%%%%%%%%% cond_del %%%%%%%%%%%%%%%%%%%%
\pagebreak
\subsubsection{\code{cond\_del}}
\label{api:python:cond_del}
\index{cond\_del!Python API}
Conditionally delete an object by key.

%%% Generated below here
\paragraph{Behavior:}
\begin{itemize}[noitemsep]
If no object exists, the operation will fail with \code{NOTFOUND}.

This operation will succeed if and only if the predicates specified by
\code{checks} hold on the pre-existing object.  If any of the predicates are not
true for the existing object, then the operation will have no effect and fail
with \code{CMPFAIL}.

All checks are atomic with the write.  HyperDex guarantees that no other
operation will come between validating the checks, and writing the new version
of the object.

\end{itemize}


\paragraph{Definition:}
\begin{pythoncode}
def cond_del(self, spacename, key, predicates)
\end{pythoncode}

\paragraph{Parameters:}
\begin{itemize}[noitemsep]
\item \code{spacename}\\
The name of an existing space.

\item \code{key}\\
The key for the operation where \code{key} is a bytestring and \code{key\_sz}
specifies the number of bytes in \code{key}.

\item \code{predicates}\\
The predicates to check against.  \code{checks} is an object with the individual
predicates as properties.

\end{itemize}

\paragraph{Returns:}
True if the operation succeeded, False if any of the provided predicates failed.
Raises an exception on error.


\pagebreak
\subsubsection{\code{async\_cond\_del}}
\label{api:python:async_cond_del}
\index{async\_cond\_del!Python API}
Conditionally delete an object by key.

%%% Generated below here
\paragraph{Behavior:}
\begin{itemize}[noitemsep]
If no object exists, the operation will fail with \code{NOTFOUND}.

This operation will succeed if and only if the predicates specified by
\code{checks} hold on the pre-existing object.  If any of the predicates are not
true for the existing object, then the operation will have no effect and fail
with \code{CMPFAIL}.

All checks are atomic with the write.  HyperDex guarantees that no other
operation will come between validating the checks, and writing the new version
of the object.

\end{itemize}


\paragraph{Definition:}
\begin{pythoncode}
def async_cond_del(self, spacename, key, predicates)
\end{pythoncode}

\paragraph{Parameters:}
\begin{itemize}[noitemsep]
\item \code{spacename}\\
The name of an existing space.

\item \code{key}\\
The key for the operation where \code{key} is a bytestring and \code{key\_sz}
specifies the number of bytes in \code{key}.

\item \code{predicates}\\
The predicates to check against.  \code{checks} is an object with the individual
predicates as properties.

\end{itemize}

\paragraph{Returns:}
A Deferred object with a \code{wait} method that returns True if the operation
succeeded, or False if any of the provided predicates failed.  Raises an
exception on error.


\paragraph{See also:}  This is the asynchronous form of \code{cond\_del}.

%%%%%%%%%%%%%%%%%%%% group_del %%%%%%%%%%%%%%%%%%%%
\pagebreak
\subsubsection{\code{group\_del}}
\label{api:python:group_del}
\index{group\_del!Python API}
Asynchronously delete all objects that match the specified \code{checks}.

\paragraph{Behavior:}
\begin{itemize}[noitemsep]
\item This operation is roughly equivalent to a client manually deleting every
    object returned from a search, but saves HyperDex from sending to the client
    objects that are soon to be deleted.
\end{itemize}


\paragraph{Definition:}
\begin{pythoncode}
def group_del(self, spacename, predicates)
\end{pythoncode}

\paragraph{Parameters:}
\begin{itemize}[noitemsep]
\item \code{spacename}\\
The name of an existing space.

\item \code{predicates}\\
The predicates to check against.  \code{checks} is an object with the individual
predicates as properties.

\end{itemize}

\paragraph{Returns:}
A count of the number of objects, and a \code{client.Error} object indicating
the status of the operation.


\pagebreak
\subsubsection{\code{async\_group\_del}}
\label{api:python:async_group_del}
\index{async\_group\_del!Python API}
Asynchronously delete all objects that match the specified \code{checks}.

\paragraph{Behavior:}
\begin{itemize}[noitemsep]
\item This operation is roughly equivalent to a client manually deleting every
    object returned from a search, but saves HyperDex from sending to the client
    objects that are soon to be deleted.
\end{itemize}


\paragraph{Definition:}
\begin{pythoncode}
def async_group_del(self, spacename, predicates)
\end{pythoncode}

\paragraph{Parameters:}
\begin{itemize}[noitemsep]
\item \code{spacename}\\
The name of an existing space.

\item \code{predicates}\\
The predicates to check against.  \code{checks} is an object with the individual
predicates as properties.

\end{itemize}

\paragraph{Returns:}
This asynchronous operation returns a \code{Deferred} object with a
\code{waitForIt} method which blocks and returns a number indicating the number
of objects counted.

On error, this function will raise a \code{HyperDexClientException} describing
the error.


\paragraph{See also:}  This is the asynchronous form of \code{group\_del}.

%%%%%%%%%%%%%%%%%%%% atomic_add %%%%%%%%%%%%%%%%%%%%
\pagebreak
\subsubsection{\code{atomic\_add}}
\label{api:python:atomic_add}
\index{atomic\_add!Python API}
Add the specified number to the existing value for each attribute.
This operation requires a pre-existing object in order to complete successfully.
If no object exists, the operation will fail with \code{NOTFOUND}.



\paragraph{Definition:}
\begin{pythoncode}
def atomic_add(self, spacename, key, attributes)
\end{pythoncode}

\paragraph{Parameters:}
\begin{itemize}[noitemsep]
\item \code{spacename}\\
The name of an existing space.

\item \code{key}\\
The key for the operation where \code{key} is a bytestring and \code{key\_sz}
specifies the number of bytes in \code{key}.

\item \code{attributes}\\
The attributes to modify and their respective values.

\end{itemize}

\paragraph{Returns:}
True if the operation succeeded, False if any of the provided predicates failed.
Raises an exception on error.


\pagebreak
\subsubsection{\code{async\_atomic\_add}}
\label{api:python:async_atomic_add}
\index{async\_atomic\_add!Python API}
Add the specified number to the existing value for each attribute.
This operation requires a pre-existing object in order to complete successfully.
If no object exists, the operation will fail with \code{NOTFOUND}.



\paragraph{Definition:}
\begin{pythoncode}
def async_atomic_add(self, spacename, key, attributes)
\end{pythoncode}

\paragraph{Parameters:}
\begin{itemize}[noitemsep]
\item \code{spacename}\\
The name of an existing space.

\item \code{key}\\
The key for the operation where \code{key} is a bytestring and \code{key\_sz}
specifies the number of bytes in \code{key}.

\item \code{attributes}\\
The attributes to modify and their respective values.

\end{itemize}

\paragraph{Returns:}
A Deferred object with a \code{wait} method that returns True if the operation
succeeded, or False if any of the provided predicates failed.  Raises an
exception on error.


\paragraph{See also:}  This is the asynchronous form of \code{atomic\_add}.

%%%%%%%%%%%%%%%%%%%% cond_atomic_add %%%%%%%%%%%%%%%%%%%%
\pagebreak
\subsubsection{\code{cond\_atomic\_add}}
\label{api:python:cond_atomic_add}
\index{cond\_atomic\_add!Python API}
Conditionally add the specified number to the existing value for each attribute.

%%% Generated below here
\paragraph{Behavior:}
\begin{itemize}[noitemsep]
This operation requires a pre-existing object in order to complete successfully.
If no object exists, the operation will fail with \code{NOTFOUND}.

This operation will succeed if and only if the predicates specified by
\code{checks} hold on the pre-existing object.  If any of the predicates are not
true for the existing object, then the operation will have no effect and fail
with \code{CMPFAIL}.

All checks are atomic with the write.  HyperDex guarantees that no other
operation will come between validating the checks, and writing the new version
of the object.

\end{itemize}


\paragraph{Definition:}
\begin{pythoncode}
def cond_atomic_add(self, spacename, key, predicates, attributes)
\end{pythoncode}

\paragraph{Parameters:}
\begin{itemize}[noitemsep]
\item \code{spacename}\\
The name of an existing space.

\item \code{key}\\
The key for the operation where \code{key} is a bytestring and \code{key\_sz}
specifies the number of bytes in \code{key}.

\item \code{predicates}\\
The predicates to check against.  \code{checks} is an object with the individual
predicates as properties.

\item \code{attributes}\\
The attributes to modify and their respective values.

\end{itemize}

\paragraph{Returns:}
True if the operation succeeded, False if any of the provided predicates failed.
Raises an exception on error.


\pagebreak
\subsubsection{\code{async\_cond\_atomic\_add}}
\label{api:python:async_cond_atomic_add}
\index{async\_cond\_atomic\_add!Python API}
Conditionally add the specified number to the existing value for each attribute.

%%% Generated below here
\paragraph{Behavior:}
\begin{itemize}[noitemsep]
This operation requires a pre-existing object in order to complete successfully.
If no object exists, the operation will fail with \code{NOTFOUND}.

This operation will succeed if and only if the predicates specified by
\code{checks} hold on the pre-existing object.  If any of the predicates are not
true for the existing object, then the operation will have no effect and fail
with \code{CMPFAIL}.

All checks are atomic with the write.  HyperDex guarantees that no other
operation will come between validating the checks, and writing the new version
of the object.

\end{itemize}


\paragraph{Definition:}
\begin{pythoncode}
def async_cond_atomic_add(self, spacename, key, predicates, attributes)
\end{pythoncode}

\paragraph{Parameters:}
\begin{itemize}[noitemsep]
\item \code{spacename}\\
The name of an existing space.

\item \code{key}\\
The key for the operation where \code{key} is a bytestring and \code{key\_sz}
specifies the number of bytes in \code{key}.

\item \code{predicates}\\
The predicates to check against.  \code{checks} is an object with the individual
predicates as properties.

\item \code{attributes}\\
The attributes to modify and their respective values.

\end{itemize}

\paragraph{Returns:}
A Deferred object with a \code{wait} method that returns True if the operation
succeeded, or False if any of the provided predicates failed.  Raises an
exception on error.


\paragraph{See also:}  This is the asynchronous form of \code{cond\_atomic\_add}.

%%%%%%%%%%%%%%%%%%%% group_atomic_add %%%%%%%%%%%%%%%%%%%%
\pagebreak
\subsubsection{\code{group\_atomic\_add}}
\label{api:python:group_atomic_add}
\index{group\_atomic\_add!Python API}
Add the specified number to the existing value for each object in \code{space}
that matches \code{checks}.

This operation will only affect objects that match the provided \code{checks}.
Objects that do not match \code{checks} will be unaffected by the group call.
Each object that matches \code{checks} will be atomically updated with the check
on the object.  HyperDex guarantees that no object will be altered if the
\code{checks} do not pass at the time of the write.  Objects that are updated
concurrently with the group call may or may not be updated; however, regardless
of any other concurrent operations, the preceding guarantee will always hold.



\paragraph{Definition:}
\begin{pythoncode}
def group_atomic_add(self, spacename, predicates, attributes)
\end{pythoncode}

\paragraph{Parameters:}
\begin{itemize}[noitemsep]
\item \code{spacename}\\
The name of an existing space.

\item \code{predicates}\\
The predicates to check against.  \code{checks} is an object with the individual
predicates as properties.

\item \code{attributes}\\
The attributes to modify and their respective values.

\end{itemize}

\paragraph{Returns:}
A count of the number of objects, and a \code{client.Error} object indicating
the status of the operation.


\pagebreak
\subsubsection{\code{async\_group\_atomic\_add}}
\label{api:python:async_group_atomic_add}
\index{async\_group\_atomic\_add!Python API}
Add the specified number to the existing value for each object in \code{space}
that matches \code{checks}.

This operation will only affect objects that match the provided \code{checks}.
Objects that do not match \code{checks} will be unaffected by the group call.
Each object that matches \code{checks} will be atomically updated with the check
on the object.  HyperDex guarantees that no object will be altered if the
\code{checks} do not pass at the time of the write.  Objects that are updated
concurrently with the group call may or may not be updated; however, regardless
of any other concurrent operations, the preceding guarantee will always hold.



\paragraph{Definition:}
\begin{pythoncode}
def async_group_atomic_add(self, spacename, predicates, attributes)
\end{pythoncode}

\paragraph{Parameters:}
\begin{itemize}[noitemsep]
\item \code{spacename}\\
The name of an existing space.

\item \code{predicates}\\
The predicates to check against.  \code{checks} is an object with the individual
predicates as properties.

\item \code{attributes}\\
The attributes to modify and their respective values.

\end{itemize}

\paragraph{Returns:}
This asynchronous operation returns a \code{Deferred} object with a
\code{waitForIt} method which blocks and returns a number indicating the number
of objects counted.

On error, this function will raise a \code{HyperDexClientException} describing
the error.


\paragraph{See also:}  This is the asynchronous form of \code{group\_atomic\_add}.

%%%%%%%%%%%%%%%%%%%% atomic_sub %%%%%%%%%%%%%%%%%%%%
\pagebreak
\subsubsection{\code{atomic\_sub}}
\label{api:python:atomic_sub}
\index{atomic\_sub!Python API}
Subtract the specified number from the existing value for each attribute.

%%% Generated below here
\paragraph{Behavior:}
\begin{itemize}[noitemsep]
This operation requires a pre-existing object in order to complete successfully.
If no object exists, the operation will fail with \code{NOTFOUND}.

\end{itemize}


\paragraph{Definition:}
\begin{pythoncode}
def atomic_sub(self, spacename, key, attributes)
\end{pythoncode}

\paragraph{Parameters:}
\begin{itemize}[noitemsep]
\item \code{spacename}\\
The name of an existing space.

\item \code{key}\\
The key for the operation where \code{key} is a bytestring and \code{key\_sz}
specifies the number of bytes in \code{key}.

\item \code{attributes}\\
The attributes to modify and their respective values.

\end{itemize}

\paragraph{Returns:}
True if the operation succeeded, False if any of the provided predicates failed.
Raises an exception on error.


\pagebreak
\subsubsection{\code{async\_atomic\_sub}}
\label{api:python:async_atomic_sub}
\index{async\_atomic\_sub!Python API}
Subtract the specified number from the existing value for each attribute.

%%% Generated below here
\paragraph{Behavior:}
\begin{itemize}[noitemsep]
This operation requires a pre-existing object in order to complete successfully.
If no object exists, the operation will fail with \code{NOTFOUND}.

\end{itemize}


\paragraph{Definition:}
\begin{pythoncode}
def async_atomic_sub(self, spacename, key, attributes)
\end{pythoncode}

\paragraph{Parameters:}
\begin{itemize}[noitemsep]
\item \code{spacename}\\
The name of an existing space.

\item \code{key}\\
The key for the operation where \code{key} is a bytestring and \code{key\_sz}
specifies the number of bytes in \code{key}.

\item \code{attributes}\\
The attributes to modify and their respective values.

\end{itemize}

\paragraph{Returns:}
A Deferred object with a \code{wait} method that returns True if the operation
succeeded, or False if any of the provided predicates failed.  Raises an
exception on error.


\paragraph{See also:}  This is the asynchronous form of \code{atomic\_sub}.

%%%%%%%%%%%%%%%%%%%% cond_atomic_sub %%%%%%%%%%%%%%%%%%%%
\pagebreak
\subsubsection{\code{cond\_atomic\_sub}}
\label{api:python:cond_atomic_sub}
\index{cond\_atomic\_sub!Python API}
Conditionally subtract the specified number from the existing value for each attribute.

%%% Generated below here
\paragraph{Behavior:}
\begin{itemize}[noitemsep]
This operation requires a pre-existing object in order to complete successfully.
If no object exists, the operation will fail with \code{NOTFOUND}.

This operation will succeed if and only if the predicates specified by
\code{checks} hold on the pre-existing object.  If any of the predicates are not
true for the existing object, then the operation will have no effect and fail
with \code{CMPFAIL}.

All checks are atomic with the write.  HyperDex guarantees that no other
operation will come between validating the checks, and writing the new version
of the object.

\end{itemize}


\paragraph{Definition:}
\begin{pythoncode}
def cond_atomic_sub(self, spacename, key, predicates, attributes)
\end{pythoncode}

\paragraph{Parameters:}
\begin{itemize}[noitemsep]
\item \code{spacename}\\
The name of an existing space.

\item \code{key}\\
The key for the operation where \code{key} is a bytestring and \code{key\_sz}
specifies the number of bytes in \code{key}.

\item \code{predicates}\\
The predicates to check against.  \code{checks} is an object with the individual
predicates as properties.

\item \code{attributes}\\
The attributes to modify and their respective values.

\end{itemize}

\paragraph{Returns:}
True if the operation succeeded, False if any of the provided predicates failed.
Raises an exception on error.


\pagebreak
\subsubsection{\code{async\_cond\_atomic\_sub}}
\label{api:python:async_cond_atomic_sub}
\index{async\_cond\_atomic\_sub!Python API}
Conditionally subtract the specified number from the existing value for each attribute.

%%% Generated below here
\paragraph{Behavior:}
\begin{itemize}[noitemsep]
This operation requires a pre-existing object in order to complete successfully.
If no object exists, the operation will fail with \code{NOTFOUND}.

This operation will succeed if and only if the predicates specified by
\code{checks} hold on the pre-existing object.  If any of the predicates are not
true for the existing object, then the operation will have no effect and fail
with \code{CMPFAIL}.

All checks are atomic with the write.  HyperDex guarantees that no other
operation will come between validating the checks, and writing the new version
of the object.

\end{itemize}


\paragraph{Definition:}
\begin{pythoncode}
def async_cond_atomic_sub(self, spacename, key, predicates, attributes)
\end{pythoncode}

\paragraph{Parameters:}
\begin{itemize}[noitemsep]
\item \code{spacename}\\
The name of an existing space.

\item \code{key}\\
The key for the operation where \code{key} is a bytestring and \code{key\_sz}
specifies the number of bytes in \code{key}.

\item \code{predicates}\\
The predicates to check against.  \code{checks} is an object with the individual
predicates as properties.

\item \code{attributes}\\
The attributes to modify and their respective values.

\end{itemize}

\paragraph{Returns:}
A Deferred object with a \code{wait} method that returns True if the operation
succeeded, or False if any of the provided predicates failed.  Raises an
exception on error.


\paragraph{See also:}  This is the asynchronous form of \code{cond\_atomic\_sub}.

%%%%%%%%%%%%%%%%%%%% group_atomic_sub %%%%%%%%%%%%%%%%%%%%
\pagebreak
\subsubsection{\code{group\_atomic\_sub}}
\label{api:python:group_atomic_sub}
\index{group\_atomic\_sub!Python API}
Subtract the specified number from the existing value for each object in
\code{space} that matches \code{checks}.

This operation will only affect objects that match the provided \code{checks}.
Objects that do not match \code{checks} will be unaffected by the group call.
Each object that matches \code{checks} will be atomically updated with the check
on the object.  HyperDex guarantees that no object will be altered if the
\code{checks} do not pass at the time of the write.  Objects that are updated
concurrently with the group call may or may not be updated; however, regardless
of any other concurrent operations, the preceding guarantee will always hold.



\paragraph{Definition:}
\begin{pythoncode}
def group_atomic_sub(self, spacename, predicates, attributes)
\end{pythoncode}

\paragraph{Parameters:}
\begin{itemize}[noitemsep]
\item \code{spacename}\\
The name of an existing space.

\item \code{predicates}\\
The predicates to check against.  \code{checks} is an object with the individual
predicates as properties.

\item \code{attributes}\\
The attributes to modify and their respective values.

\end{itemize}

\paragraph{Returns:}
A count of the number of objects, and a \code{client.Error} object indicating
the status of the operation.


\pagebreak
\subsubsection{\code{async\_group\_atomic\_sub}}
\label{api:python:async_group_atomic_sub}
\index{async\_group\_atomic\_sub!Python API}
Subtract the specified number from the existing value for each object in
\code{space} that matches \code{checks}.

This operation will only affect objects that match the provided \code{checks}.
Objects that do not match \code{checks} will be unaffected by the group call.
Each object that matches \code{checks} will be atomically updated with the check
on the object.  HyperDex guarantees that no object will be altered if the
\code{checks} do not pass at the time of the write.  Objects that are updated
concurrently with the group call may or may not be updated; however, regardless
of any other concurrent operations, the preceding guarantee will always hold.



\paragraph{Definition:}
\begin{pythoncode}
def async_group_atomic_sub(self, spacename, predicates, attributes)
\end{pythoncode}

\paragraph{Parameters:}
\begin{itemize}[noitemsep]
\item \code{spacename}\\
The name of an existing space.

\item \code{predicates}\\
The predicates to check against.  \code{checks} is an object with the individual
predicates as properties.

\item \code{attributes}\\
The attributes to modify and their respective values.

\end{itemize}

\paragraph{Returns:}
This asynchronous operation returns a \code{Deferred} object with a
\code{waitForIt} method which blocks and returns a number indicating the number
of objects counted.

On error, this function will raise a \code{HyperDexClientException} describing
the error.


\paragraph{See also:}  This is the asynchronous form of \code{group\_atomic\_sub}.

%%%%%%%%%%%%%%%%%%%% atomic_mul %%%%%%%%%%%%%%%%%%%%
\pagebreak
\subsubsection{\code{atomic\_mul}}
\label{api:python:atomic_mul}
\index{atomic\_mul!Python API}
\input{\topdir/client/fragments/atomic_mul}

\paragraph{Definition:}
\begin{pythoncode}
def atomic_mul(self, spacename, key, attributes)
\end{pythoncode}

\paragraph{Parameters:}
\begin{itemize}[noitemsep]
\item \code{spacename}\\
The name of an existing space.

\item \code{key}\\
The key for the operation where \code{key} is a bytestring and \code{key\_sz}
specifies the number of bytes in \code{key}.

\item \code{attributes}\\
The attributes to modify and their respective values.

\end{itemize}

\paragraph{Returns:}
True if the operation succeeded, False if any of the provided predicates failed.
Raises an exception on error.


\pagebreak
\subsubsection{\code{async\_atomic\_mul}}
\label{api:python:async_atomic_mul}
\index{async\_atomic\_mul!Python API}
\input{\topdir/client/fragments/atomic_mul}

\paragraph{Definition:}
\begin{pythoncode}
def async_atomic_mul(self, spacename, key, attributes)
\end{pythoncode}

\paragraph{Parameters:}
\begin{itemize}[noitemsep]
\item \code{spacename}\\
The name of an existing space.

\item \code{key}\\
The key for the operation where \code{key} is a bytestring and \code{key\_sz}
specifies the number of bytes in \code{key}.

\item \code{attributes}\\
The attributes to modify and their respective values.

\end{itemize}

\paragraph{Returns:}
A Deferred object with a \code{wait} method that returns True if the operation
succeeded, or False if any of the provided predicates failed.  Raises an
exception on error.


\paragraph{See also:}  This is the asynchronous form of \code{atomic\_mul}.

%%%%%%%%%%%%%%%%%%%% cond_atomic_mul %%%%%%%%%%%%%%%%%%%%
\pagebreak
\subsubsection{\code{cond\_atomic\_mul}}
\label{api:python:cond_atomic_mul}
\index{cond\_atomic\_mul!Python API}
Multiply the existing value by the specified number for each attribute if and
only if the \code{checks} hold on the object.
This operation requires a pre-existing object in order to complete successfully.
If no object exists, the operation will fail with \code{NOTFOUND}.


This operation will succeed if and only if the predicates specified by
\code{checks} hold on the pre-existing object.  If any of the predicates are not
true for the existing object, then the operation will have no effect and fail
with \code{CMPFAIL}.

All checks are atomic with the write.  HyperDex guarantees that no other
operation will come between validating the checks, and writing the new version
of the object.



\paragraph{Definition:}
\begin{pythoncode}
def cond_atomic_mul(self, spacename, key, predicates, attributes)
\end{pythoncode}

\paragraph{Parameters:}
\begin{itemize}[noitemsep]
\item \code{spacename}\\
The name of an existing space.

\item \code{key}\\
The key for the operation where \code{key} is a bytestring and \code{key\_sz}
specifies the number of bytes in \code{key}.

\item \code{predicates}\\
The predicates to check against.  \code{checks} is an object with the individual
predicates as properties.

\item \code{attributes}\\
The attributes to modify and their respective values.

\end{itemize}

\paragraph{Returns:}
True if the operation succeeded, False if any of the provided predicates failed.
Raises an exception on error.


\pagebreak
\subsubsection{\code{async\_cond\_atomic\_mul}}
\label{api:python:async_cond_atomic_mul}
\index{async\_cond\_atomic\_mul!Python API}
Multiply the existing value by the specified number for each attribute if and
only if the \code{checks} hold on the object.
This operation requires a pre-existing object in order to complete successfully.
If no object exists, the operation will fail with \code{NOTFOUND}.


This operation will succeed if and only if the predicates specified by
\code{checks} hold on the pre-existing object.  If any of the predicates are not
true for the existing object, then the operation will have no effect and fail
with \code{CMPFAIL}.

All checks are atomic with the write.  HyperDex guarantees that no other
operation will come between validating the checks, and writing the new version
of the object.



\paragraph{Definition:}
\begin{pythoncode}
def async_cond_atomic_mul(self, spacename, key, predicates, attributes)
\end{pythoncode}

\paragraph{Parameters:}
\begin{itemize}[noitemsep]
\item \code{spacename}\\
The name of an existing space.

\item \code{key}\\
The key for the operation where \code{key} is a bytestring and \code{key\_sz}
specifies the number of bytes in \code{key}.

\item \code{predicates}\\
The predicates to check against.  \code{checks} is an object with the individual
predicates as properties.

\item \code{attributes}\\
The attributes to modify and their respective values.

\end{itemize}

\paragraph{Returns:}
A Deferred object with a \code{wait} method that returns True if the operation
succeeded, or False if any of the provided predicates failed.  Raises an
exception on error.


\paragraph{See also:}  This is the asynchronous form of \code{cond\_atomic\_mul}.

%%%%%%%%%%%%%%%%%%%% group_atomic_mul %%%%%%%%%%%%%%%%%%%%
\pagebreak
\subsubsection{\code{group\_atomic\_mul}}
\label{api:python:group_atomic_mul}
\index{group\_atomic\_mul!Python API}
Multiply the existing value by the specified number for each object in
\code{space} that matches \code{checks}.

This operation will only affect objects that match the provided \code{checks}.
Objects that do not match \code{checks} will be unaffected by the group call.
Each object that matches \code{checks} will be atomically updated with the check
on the object.  HyperDex guarantees that no object will be altered if the
\code{checks} do not pass at the time of the write.  Objects that are updated
concurrently with the group call may or may not be updated; however, regardless
of any other concurrent operations, the preceding guarantee will always hold.



\paragraph{Definition:}
\begin{pythoncode}
def group_atomic_mul(self, spacename, predicates, attributes)
\end{pythoncode}

\paragraph{Parameters:}
\begin{itemize}[noitemsep]
\item \code{spacename}\\
The name of an existing space.

\item \code{predicates}\\
The predicates to check against.  \code{checks} is an object with the individual
predicates as properties.

\item \code{attributes}\\
The attributes to modify and their respective values.

\end{itemize}

\paragraph{Returns:}
A count of the number of objects, and a \code{client.Error} object indicating
the status of the operation.


\pagebreak
\subsubsection{\code{async\_group\_atomic\_mul}}
\label{api:python:async_group_atomic_mul}
\index{async\_group\_atomic\_mul!Python API}
Multiply the existing value by the specified number for each object in
\code{space} that matches \code{checks}.

This operation will only affect objects that match the provided \code{checks}.
Objects that do not match \code{checks} will be unaffected by the group call.
Each object that matches \code{checks} will be atomically updated with the check
on the object.  HyperDex guarantees that no object will be altered if the
\code{checks} do not pass at the time of the write.  Objects that are updated
concurrently with the group call may or may not be updated; however, regardless
of any other concurrent operations, the preceding guarantee will always hold.



\paragraph{Definition:}
\begin{pythoncode}
def async_group_atomic_mul(self, spacename, predicates, attributes)
\end{pythoncode}

\paragraph{Parameters:}
\begin{itemize}[noitemsep]
\item \code{spacename}\\
The name of an existing space.

\item \code{predicates}\\
The predicates to check against.  \code{checks} is an object with the individual
predicates as properties.

\item \code{attributes}\\
The attributes to modify and their respective values.

\end{itemize}

\paragraph{Returns:}
This asynchronous operation returns a \code{Deferred} object with a
\code{waitForIt} method which blocks and returns a number indicating the number
of objects counted.

On error, this function will raise a \code{HyperDexClientException} describing
the error.


\paragraph{See also:}  This is the asynchronous form of \code{group\_atomic\_mul}.

%%%%%%%%%%%%%%%%%%%% atomic_div %%%%%%%%%%%%%%%%%%%%
\pagebreak
\subsubsection{\code{atomic\_div}}
\label{api:python:atomic_div}
\index{atomic\_div!Python API}
Divide the existing value by the number specified for each attribute.

The division is atomic with the write.  If the object does not exist, the
operation will fail.

\noindent\textbf{Cost:}  Approximately one traversal of the value-dependent
chain.


\input{\topdir/api/shards/linearizable}


\paragraph{Definition:}
\begin{pythoncode}
def atomic_div(self, spacename, key, attributes)
\end{pythoncode}

\paragraph{Parameters:}
\begin{itemize}[noitemsep]
\item \code{spacename}\\
The name of an existing space.

\item \code{key}\\
The key for the operation where \code{key} is a bytestring and \code{key\_sz}
specifies the number of bytes in \code{key}.

\item \code{attributes}\\
The attributes to modify and their respective values.

\end{itemize}

\paragraph{Returns:}
True if the operation succeeded, False if any of the provided predicates failed.
Raises an exception on error.


\pagebreak
\subsubsection{\code{async\_atomic\_div}}
\label{api:python:async_atomic_div}
\index{async\_atomic\_div!Python API}
Divide the existing value by the number specified for each attribute.

The division is atomic with the write.  If the object does not exist, the
operation will fail.

\noindent\textbf{Cost:}  Approximately one traversal of the value-dependent
chain.


\input{\topdir/api/shards/linearizable}


\paragraph{Definition:}
\begin{pythoncode}
def async_atomic_div(self, spacename, key, attributes)
\end{pythoncode}

\paragraph{Parameters:}
\begin{itemize}[noitemsep]
\item \code{spacename}\\
The name of an existing space.

\item \code{key}\\
The key for the operation where \code{key} is a bytestring and \code{key\_sz}
specifies the number of bytes in \code{key}.

\item \code{attributes}\\
The attributes to modify and their respective values.

\end{itemize}

\paragraph{Returns:}
A Deferred object with a \code{wait} method that returns True if the operation
succeeded, or False if any of the provided predicates failed.  Raises an
exception on error.


\paragraph{See also:}  This is the asynchronous form of \code{atomic\_div}.

%%%%%%%%%%%%%%%%%%%% cond_atomic_div %%%%%%%%%%%%%%%%%%%%
\pagebreak
\subsubsection{\code{cond\_atomic\_div}}
\label{api:python:cond_atomic_div}
\index{cond\_atomic\_div!Python API}
\input{\topdir/client/fragments/cond_atomic_div}

\paragraph{Definition:}
\begin{pythoncode}
def cond_atomic_div(self, spacename, key, predicates, attributes)
\end{pythoncode}

\paragraph{Parameters:}
\begin{itemize}[noitemsep]
\item \code{spacename}\\
The name of an existing space.

\item \code{key}\\
The key for the operation where \code{key} is a bytestring and \code{key\_sz}
specifies the number of bytes in \code{key}.

\item \code{predicates}\\
The predicates to check against.  \code{checks} is an object with the individual
predicates as properties.

\item \code{attributes}\\
The attributes to modify and their respective values.

\end{itemize}

\paragraph{Returns:}
True if the operation succeeded, False if any of the provided predicates failed.
Raises an exception on error.


\pagebreak
\subsubsection{\code{async\_cond\_atomic\_div}}
\label{api:python:async_cond_atomic_div}
\index{async\_cond\_atomic\_div!Python API}
\input{\topdir/client/fragments/cond_atomic_div}

\paragraph{Definition:}
\begin{pythoncode}
def async_cond_atomic_div(self, spacename, key, predicates, attributes)
\end{pythoncode}

\paragraph{Parameters:}
\begin{itemize}[noitemsep]
\item \code{spacename}\\
The name of an existing space.

\item \code{key}\\
The key for the operation where \code{key} is a bytestring and \code{key\_sz}
specifies the number of bytes in \code{key}.

\item \code{predicates}\\
The predicates to check against.  \code{checks} is an object with the individual
predicates as properties.

\item \code{attributes}\\
The attributes to modify and their respective values.

\end{itemize}

\paragraph{Returns:}
A Deferred object with a \code{wait} method that returns True if the operation
succeeded, or False if any of the provided predicates failed.  Raises an
exception on error.


\paragraph{See also:}  This is the asynchronous form of \code{cond\_atomic\_div}.

%%%%%%%%%%%%%%%%%%%% group_atomic_div %%%%%%%%%%%%%%%%%%%%
\pagebreak
\subsubsection{\code{group\_atomic\_div}}
\label{api:python:group_atomic_div}
\index{group\_atomic\_div!Python API}
\input{\topdir/client/fragments/group_atomic_div}

\paragraph{Definition:}
\begin{pythoncode}
def group_atomic_div(self, spacename, predicates, attributes)
\end{pythoncode}

\paragraph{Parameters:}
\begin{itemize}[noitemsep]
\item \code{spacename}\\
The name of an existing space.

\item \code{predicates}\\
The predicates to check against.  \code{checks} is an object with the individual
predicates as properties.

\item \code{attributes}\\
The attributes to modify and their respective values.

\end{itemize}

\paragraph{Returns:}
A count of the number of objects, and a \code{client.Error} object indicating
the status of the operation.


\pagebreak
\subsubsection{\code{async\_group\_atomic\_div}}
\label{api:python:async_group_atomic_div}
\index{async\_group\_atomic\_div!Python API}
\input{\topdir/client/fragments/group_atomic_div}

\paragraph{Definition:}
\begin{pythoncode}
def async_group_atomic_div(self, spacename, predicates, attributes)
\end{pythoncode}

\paragraph{Parameters:}
\begin{itemize}[noitemsep]
\item \code{spacename}\\
The name of an existing space.

\item \code{predicates}\\
The predicates to check against.  \code{checks} is an object with the individual
predicates as properties.

\item \code{attributes}\\
The attributes to modify and their respective values.

\end{itemize}

\paragraph{Returns:}
This asynchronous operation returns a \code{Deferred} object with a
\code{waitForIt} method which blocks and returns a number indicating the number
of objects counted.

On error, this function will raise a \code{HyperDexClientException} describing
the error.


\paragraph{See also:}  This is the asynchronous form of \code{group\_atomic\_div}.

%%%%%%%%%%%%%%%%%%%% atomic_mod %%%%%%%%%%%%%%%%%%%%
\pagebreak
\subsubsection{\code{atomic\_mod}}
\label{api:python:atomic_mod}
\index{atomic\_mod!Python API}
Store the existing value modulo the specified number for each attribute.
This operation requires a pre-existing object in order to complete successfully.
If no object exists, the operation will fail with \code{NOTFOUND}.



\paragraph{Definition:}
\begin{pythoncode}
def atomic_mod(self, spacename, key, attributes)
\end{pythoncode}

\paragraph{Parameters:}
\begin{itemize}[noitemsep]
\item \code{spacename}\\
The name of an existing space.

\item \code{key}\\
The key for the operation where \code{key} is a bytestring and \code{key\_sz}
specifies the number of bytes in \code{key}.

\item \code{attributes}\\
The attributes to modify and their respective values.

\end{itemize}

\paragraph{Returns:}
True if the operation succeeded, False if any of the provided predicates failed.
Raises an exception on error.


\pagebreak
\subsubsection{\code{async\_atomic\_mod}}
\label{api:python:async_atomic_mod}
\index{async\_atomic\_mod!Python API}
Store the existing value modulo the specified number for each attribute.
This operation requires a pre-existing object in order to complete successfully.
If no object exists, the operation will fail with \code{NOTFOUND}.



\paragraph{Definition:}
\begin{pythoncode}
def async_atomic_mod(self, spacename, key, attributes)
\end{pythoncode}

\paragraph{Parameters:}
\begin{itemize}[noitemsep]
\item \code{spacename}\\
The name of an existing space.

\item \code{key}\\
The key for the operation where \code{key} is a bytestring and \code{key\_sz}
specifies the number of bytes in \code{key}.

\item \code{attributes}\\
The attributes to modify and their respective values.

\end{itemize}

\paragraph{Returns:}
A Deferred object with a \code{wait} method that returns True if the operation
succeeded, or False if any of the provided predicates failed.  Raises an
exception on error.


\paragraph{See also:}  This is the asynchronous form of \code{atomic\_mod}.

%%%%%%%%%%%%%%%%%%%% cond_atomic_mod %%%%%%%%%%%%%%%%%%%%
\pagebreak
\subsubsection{\code{cond\_atomic\_mod}}
\label{api:python:cond_atomic_mod}
\index{cond\_atomic\_mod!Python API}
Conditionally store the existing value modulo the specified number for each
attribute.

%%% Generated below here
\paragraph{Behavior:}
\begin{itemize}[noitemsep]
This operation requires a pre-existing object in order to complete successfully.
If no object exists, the operation will fail with \code{NOTFOUND}.

This operation will succeed if and only if the predicates specified by
\code{checks} hold on the pre-existing object.  If any of the predicates are not
true for the existing object, then the operation will have no effect and fail
with \code{CMPFAIL}.

All checks are atomic with the write.  HyperDex guarantees that no other
operation will come between validating the checks, and writing the new version
of the object.

\end{itemize}


\paragraph{Definition:}
\begin{pythoncode}
def cond_atomic_mod(self, spacename, key, predicates, attributes)
\end{pythoncode}

\paragraph{Parameters:}
\begin{itemize}[noitemsep]
\item \code{spacename}\\
The name of an existing space.

\item \code{key}\\
The key for the operation where \code{key} is a bytestring and \code{key\_sz}
specifies the number of bytes in \code{key}.

\item \code{predicates}\\
The predicates to check against.  \code{checks} is an object with the individual
predicates as properties.

\item \code{attributes}\\
The attributes to modify and their respective values.

\end{itemize}

\paragraph{Returns:}
True if the operation succeeded, False if any of the provided predicates failed.
Raises an exception on error.


\pagebreak
\subsubsection{\code{async\_cond\_atomic\_mod}}
\label{api:python:async_cond_atomic_mod}
\index{async\_cond\_atomic\_mod!Python API}
Conditionally store the existing value modulo the specified number for each
attribute.

%%% Generated below here
\paragraph{Behavior:}
\begin{itemize}[noitemsep]
This operation requires a pre-existing object in order to complete successfully.
If no object exists, the operation will fail with \code{NOTFOUND}.

This operation will succeed if and only if the predicates specified by
\code{checks} hold on the pre-existing object.  If any of the predicates are not
true for the existing object, then the operation will have no effect and fail
with \code{CMPFAIL}.

All checks are atomic with the write.  HyperDex guarantees that no other
operation will come between validating the checks, and writing the new version
of the object.

\end{itemize}


\paragraph{Definition:}
\begin{pythoncode}
def async_cond_atomic_mod(self, spacename, key, predicates, attributes)
\end{pythoncode}

\paragraph{Parameters:}
\begin{itemize}[noitemsep]
\item \code{spacename}\\
The name of an existing space.

\item \code{key}\\
The key for the operation where \code{key} is a bytestring and \code{key\_sz}
specifies the number of bytes in \code{key}.

\item \code{predicates}\\
The predicates to check against.  \code{checks} is an object with the individual
predicates as properties.

\item \code{attributes}\\
The attributes to modify and their respective values.

\end{itemize}

\paragraph{Returns:}
A Deferred object with a \code{wait} method that returns True if the operation
succeeded, or False if any of the provided predicates failed.  Raises an
exception on error.


\paragraph{See also:}  This is the asynchronous form of \code{cond\_atomic\_mod}.

%%%%%%%%%%%%%%%%%%%% group_atomic_mod %%%%%%%%%%%%%%%%%%%%
\pagebreak
\subsubsection{\code{group\_atomic\_mod}}
\label{api:python:group_atomic_mod}
\index{group\_atomic\_mod!Python API}
Store the existing value modulo the specified number for each object in
\code{space} that matches \code{checks}.

This operation will only affect objects that match the provided \code{checks}.
Objects that do not match \code{checks} will be unaffected by the group call.
Each object that matches \code{checks} will be atomically updated with the check
on the object.  HyperDex guarantees that no object will be altered if the
\code{checks} do not pass at the time of the write.  Objects that are updated
concurrently with the group call may or may not be updated; however, regardless
of any other concurrent operations, the preceding guarantee will always hold.



\paragraph{Definition:}
\begin{pythoncode}
def group_atomic_mod(self, spacename, predicates, attributes)
\end{pythoncode}

\paragraph{Parameters:}
\begin{itemize}[noitemsep]
\item \code{spacename}\\
The name of an existing space.

\item \code{predicates}\\
The predicates to check against.  \code{checks} is an object with the individual
predicates as properties.

\item \code{attributes}\\
The attributes to modify and their respective values.

\end{itemize}

\paragraph{Returns:}
A count of the number of objects, and a \code{client.Error} object indicating
the status of the operation.


\pagebreak
\subsubsection{\code{async\_group\_atomic\_mod}}
\label{api:python:async_group_atomic_mod}
\index{async\_group\_atomic\_mod!Python API}
Store the existing value modulo the specified number for each object in
\code{space} that matches \code{checks}.

This operation will only affect objects that match the provided \code{checks}.
Objects that do not match \code{checks} will be unaffected by the group call.
Each object that matches \code{checks} will be atomically updated with the check
on the object.  HyperDex guarantees that no object will be altered if the
\code{checks} do not pass at the time of the write.  Objects that are updated
concurrently with the group call may or may not be updated; however, regardless
of any other concurrent operations, the preceding guarantee will always hold.



\paragraph{Definition:}
\begin{pythoncode}
def async_group_atomic_mod(self, spacename, predicates, attributes)
\end{pythoncode}

\paragraph{Parameters:}
\begin{itemize}[noitemsep]
\item \code{spacename}\\
The name of an existing space.

\item \code{predicates}\\
The predicates to check against.  \code{checks} is an object with the individual
predicates as properties.

\item \code{attributes}\\
The attributes to modify and their respective values.

\end{itemize}

\paragraph{Returns:}
This asynchronous operation returns a \code{Deferred} object with a
\code{waitForIt} method which blocks and returns a number indicating the number
of objects counted.

On error, this function will raise a \code{HyperDexClientException} describing
the error.


\paragraph{See also:}  This is the asynchronous form of \code{group\_atomic\_mod}.

%%%%%%%%%%%%%%%%%%%% atomic_and %%%%%%%%%%%%%%%%%%%%
\pagebreak
\subsubsection{\code{atomic\_and}}
\label{api:python:atomic_and}
\index{atomic\_and!Python API}
Store the bitwise AND of the existing value and the specified number for
each attribute.
This operation requires a pre-existing object in order to complete successfully.
If no object exists, the operation will fail with \code{NOTFOUND}.



\paragraph{Definition:}
\begin{pythoncode}
def atomic_and(self, spacename, key, attributes)
\end{pythoncode}

\paragraph{Parameters:}
\begin{itemize}[noitemsep]
\item \code{spacename}\\
The name of an existing space.

\item \code{key}\\
The key for the operation where \code{key} is a bytestring and \code{key\_sz}
specifies the number of bytes in \code{key}.

\item \code{attributes}\\
The attributes to modify and their respective values.

\end{itemize}

\paragraph{Returns:}
True if the operation succeeded, False if any of the provided predicates failed.
Raises an exception on error.


\pagebreak
\subsubsection{\code{async\_atomic\_and}}
\label{api:python:async_atomic_and}
\index{async\_atomic\_and!Python API}
Store the bitwise AND of the existing value and the specified number for
each attribute.
This operation requires a pre-existing object in order to complete successfully.
If no object exists, the operation will fail with \code{NOTFOUND}.



\paragraph{Definition:}
\begin{pythoncode}
def async_atomic_and(self, spacename, key, attributes)
\end{pythoncode}

\paragraph{Parameters:}
\begin{itemize}[noitemsep]
\item \code{spacename}\\
The name of an existing space.

\item \code{key}\\
The key for the operation where \code{key} is a bytestring and \code{key\_sz}
specifies the number of bytes in \code{key}.

\item \code{attributes}\\
The attributes to modify and their respective values.

\end{itemize}

\paragraph{Returns:}
A Deferred object with a \code{wait} method that returns True if the operation
succeeded, or False if any of the provided predicates failed.  Raises an
exception on error.


\paragraph{See also:}  This is the asynchronous form of \code{atomic\_and}.

%%%%%%%%%%%%%%%%%%%% cond_atomic_and %%%%%%%%%%%%%%%%%%%%
\pagebreak
\subsubsection{\code{cond\_atomic\_and}}
\label{api:python:cond_atomic_and}
\index{cond\_atomic\_and!Python API}
Conditionally store the bitwise AND of the existing value and the specified
number for each attribute.

%%% Generated below here
\paragraph{Behavior:}
\begin{itemize}[noitemsep]
This operation requires a pre-existing object in order to complete successfully.
If no object exists, the operation will fail with \code{NOTFOUND}.

This operation will succeed if and only if the predicates specified by
\code{checks} hold on the pre-existing object.  If any of the predicates are not
true for the existing object, then the operation will have no effect and fail
with \code{CMPFAIL}.

All checks are atomic with the write.  HyperDex guarantees that no other
operation will come between validating the checks, and writing the new version
of the object.

\end{itemize}


\paragraph{Definition:}
\begin{pythoncode}
def cond_atomic_and(self, spacename, key, predicates, attributes)
\end{pythoncode}

\paragraph{Parameters:}
\begin{itemize}[noitemsep]
\item \code{spacename}\\
The name of an existing space.

\item \code{key}\\
The key for the operation where \code{key} is a bytestring and \code{key\_sz}
specifies the number of bytes in \code{key}.

\item \code{predicates}\\
The predicates to check against.  \code{checks} is an object with the individual
predicates as properties.

\item \code{attributes}\\
The attributes to modify and their respective values.

\end{itemize}

\paragraph{Returns:}
True if the operation succeeded, False if any of the provided predicates failed.
Raises an exception on error.


\pagebreak
\subsubsection{\code{async\_cond\_atomic\_and}}
\label{api:python:async_cond_atomic_and}
\index{async\_cond\_atomic\_and!Python API}
Conditionally store the bitwise AND of the existing value and the specified
number for each attribute.

%%% Generated below here
\paragraph{Behavior:}
\begin{itemize}[noitemsep]
This operation requires a pre-existing object in order to complete successfully.
If no object exists, the operation will fail with \code{NOTFOUND}.

This operation will succeed if and only if the predicates specified by
\code{checks} hold on the pre-existing object.  If any of the predicates are not
true for the existing object, then the operation will have no effect and fail
with \code{CMPFAIL}.

All checks are atomic with the write.  HyperDex guarantees that no other
operation will come between validating the checks, and writing the new version
of the object.

\end{itemize}


\paragraph{Definition:}
\begin{pythoncode}
def async_cond_atomic_and(self, spacename, key, predicates, attributes)
\end{pythoncode}

\paragraph{Parameters:}
\begin{itemize}[noitemsep]
\item \code{spacename}\\
The name of an existing space.

\item \code{key}\\
The key for the operation where \code{key} is a bytestring and \code{key\_sz}
specifies the number of bytes in \code{key}.

\item \code{predicates}\\
The predicates to check against.  \code{checks} is an object with the individual
predicates as properties.

\item \code{attributes}\\
The attributes to modify and their respective values.

\end{itemize}

\paragraph{Returns:}
A Deferred object with a \code{wait} method that returns True if the operation
succeeded, or False if any of the provided predicates failed.  Raises an
exception on error.


\paragraph{See also:}  This is the asynchronous form of \code{cond\_atomic\_and}.

%%%%%%%%%%%%%%%%%%%% group_atomic_and %%%%%%%%%%%%%%%%%%%%
\pagebreak
\subsubsection{\code{group\_atomic\_and}}
\label{api:python:group_atomic_and}
\index{group\_atomic\_and!Python API}
\input{\topdir/client/fragments/group_atomic_and}

\paragraph{Definition:}
\begin{pythoncode}
def group_atomic_and(self, spacename, predicates, attributes)
\end{pythoncode}

\paragraph{Parameters:}
\begin{itemize}[noitemsep]
\item \code{spacename}\\
The name of an existing space.

\item \code{predicates}\\
The predicates to check against.  \code{checks} is an object with the individual
predicates as properties.

\item \code{attributes}\\
The attributes to modify and their respective values.

\end{itemize}

\paragraph{Returns:}
A count of the number of objects, and a \code{client.Error} object indicating
the status of the operation.


\pagebreak
\subsubsection{\code{async\_group\_atomic\_and}}
\label{api:python:async_group_atomic_and}
\index{async\_group\_atomic\_and!Python API}
\input{\topdir/client/fragments/group_atomic_and}

\paragraph{Definition:}
\begin{pythoncode}
def async_group_atomic_and(self, spacename, predicates, attributes)
\end{pythoncode}

\paragraph{Parameters:}
\begin{itemize}[noitemsep]
\item \code{spacename}\\
The name of an existing space.

\item \code{predicates}\\
The predicates to check against.  \code{checks} is an object with the individual
predicates as properties.

\item \code{attributes}\\
The attributes to modify and their respective values.

\end{itemize}

\paragraph{Returns:}
This asynchronous operation returns a \code{Deferred} object with a
\code{waitForIt} method which blocks and returns a number indicating the number
of objects counted.

On error, this function will raise a \code{HyperDexClientException} describing
the error.


\paragraph{See also:}  This is the asynchronous form of \code{group\_atomic\_and}.

%%%%%%%%%%%%%%%%%%%% atomic_or %%%%%%%%%%%%%%%%%%%%
\pagebreak
\subsubsection{\code{atomic\_or}}
\label{api:python:atomic_or}
\index{atomic\_or!Python API}
Store the bitwise OR of the existing value and the specified number for each
attribute.

%%% Generated below here
\paragraph{Behavior:}
\begin{itemize}[noitemsep]
This operation requires a pre-existing object in order to complete successfully.
If no object exists, the operation will fail with \code{NOTFOUND}.

\end{itemize}


\paragraph{Definition:}
\begin{pythoncode}
def atomic_or(self, spacename, key, attributes)
\end{pythoncode}

\paragraph{Parameters:}
\begin{itemize}[noitemsep]
\item \code{spacename}\\
The name of an existing space.

\item \code{key}\\
The key for the operation where \code{key} is a bytestring and \code{key\_sz}
specifies the number of bytes in \code{key}.

\item \code{attributes}\\
The attributes to modify and their respective values.

\end{itemize}

\paragraph{Returns:}
True if the operation succeeded, False if any of the provided predicates failed.
Raises an exception on error.


\pagebreak
\subsubsection{\code{async\_atomic\_or}}
\label{api:python:async_atomic_or}
\index{async\_atomic\_or!Python API}
Store the bitwise OR of the existing value and the specified number for each
attribute.

%%% Generated below here
\paragraph{Behavior:}
\begin{itemize}[noitemsep]
This operation requires a pre-existing object in order to complete successfully.
If no object exists, the operation will fail with \code{NOTFOUND}.

\end{itemize}


\paragraph{Definition:}
\begin{pythoncode}
def async_atomic_or(self, spacename, key, attributes)
\end{pythoncode}

\paragraph{Parameters:}
\begin{itemize}[noitemsep]
\item \code{spacename}\\
The name of an existing space.

\item \code{key}\\
The key for the operation where \code{key} is a bytestring and \code{key\_sz}
specifies the number of bytes in \code{key}.

\item \code{attributes}\\
The attributes to modify and their respective values.

\end{itemize}

\paragraph{Returns:}
A Deferred object with a \code{wait} method that returns True if the operation
succeeded, or False if any of the provided predicates failed.  Raises an
exception on error.


\paragraph{See also:}  This is the asynchronous form of \code{atomic\_or}.

%%%%%%%%%%%%%%%%%%%% cond_atomic_or %%%%%%%%%%%%%%%%%%%%
\pagebreak
\subsubsection{\code{cond\_atomic\_or}}
\label{api:python:cond_atomic_or}
\index{cond\_atomic\_or!Python API}
\input{\topdir/client/fragments/cond_atomic_or}

\paragraph{Definition:}
\begin{pythoncode}
def cond_atomic_or(self, spacename, key, predicates, attributes)
\end{pythoncode}

\paragraph{Parameters:}
\begin{itemize}[noitemsep]
\item \code{spacename}\\
The name of an existing space.

\item \code{key}\\
The key for the operation where \code{key} is a bytestring and \code{key\_sz}
specifies the number of bytes in \code{key}.

\item \code{predicates}\\
The predicates to check against.  \code{checks} is an object with the individual
predicates as properties.

\item \code{attributes}\\
The attributes to modify and their respective values.

\end{itemize}

\paragraph{Returns:}
True if the operation succeeded, False if any of the provided predicates failed.
Raises an exception on error.


\pagebreak
\subsubsection{\code{async\_cond\_atomic\_or}}
\label{api:python:async_cond_atomic_or}
\index{async\_cond\_atomic\_or!Python API}
\input{\topdir/client/fragments/cond_atomic_or}

\paragraph{Definition:}
\begin{pythoncode}
def async_cond_atomic_or(self, spacename, key, predicates, attributes)
\end{pythoncode}

\paragraph{Parameters:}
\begin{itemize}[noitemsep]
\item \code{spacename}\\
The name of an existing space.

\item \code{key}\\
The key for the operation where \code{key} is a bytestring and \code{key\_sz}
specifies the number of bytes in \code{key}.

\item \code{predicates}\\
The predicates to check against.  \code{checks} is an object with the individual
predicates as properties.

\item \code{attributes}\\
The attributes to modify and their respective values.

\end{itemize}

\paragraph{Returns:}
A Deferred object with a \code{wait} method that returns True if the operation
succeeded, or False if any of the provided predicates failed.  Raises an
exception on error.


\paragraph{See also:}  This is the asynchronous form of \code{cond\_atomic\_or}.

%%%%%%%%%%%%%%%%%%%% group_atomic_or %%%%%%%%%%%%%%%%%%%%
\pagebreak
\subsubsection{\code{group\_atomic\_or}}
\label{api:python:group_atomic_or}
\index{group\_atomic\_or!Python API}
Store the bitwise OR of the existing value and the specified number for each
object in \code{space} that matches \code{checks}.

This operation will only affect objects that match the provided \code{checks}.
Objects that do not match \code{checks} will be unaffected by the group call.
Each object that matches \code{checks} will be atomically updated with the check
on the object.  HyperDex guarantees that no object will be altered if the
\code{checks} do not pass at the time of the write.  Objects that are updated
concurrently with the group call may or may not be updated; however, regardless
of any other concurrent operations, the preceding guarantee will always hold.



\paragraph{Definition:}
\begin{pythoncode}
def group_atomic_or(self, spacename, predicates, attributes)
\end{pythoncode}

\paragraph{Parameters:}
\begin{itemize}[noitemsep]
\item \code{spacename}\\
The name of an existing space.

\item \code{predicates}\\
The predicates to check against.  \code{checks} is an object with the individual
predicates as properties.

\item \code{attributes}\\
The attributes to modify and their respective values.

\end{itemize}

\paragraph{Returns:}
A count of the number of objects, and a \code{client.Error} object indicating
the status of the operation.


\pagebreak
\subsubsection{\code{async\_group\_atomic\_or}}
\label{api:python:async_group_atomic_or}
\index{async\_group\_atomic\_or!Python API}
Store the bitwise OR of the existing value and the specified number for each
object in \code{space} that matches \code{checks}.

This operation will only affect objects that match the provided \code{checks}.
Objects that do not match \code{checks} will be unaffected by the group call.
Each object that matches \code{checks} will be atomically updated with the check
on the object.  HyperDex guarantees that no object will be altered if the
\code{checks} do not pass at the time of the write.  Objects that are updated
concurrently with the group call may or may not be updated; however, regardless
of any other concurrent operations, the preceding guarantee will always hold.



\paragraph{Definition:}
\begin{pythoncode}
def async_group_atomic_or(self, spacename, predicates, attributes)
\end{pythoncode}

\paragraph{Parameters:}
\begin{itemize}[noitemsep]
\item \code{spacename}\\
The name of an existing space.

\item \code{predicates}\\
The predicates to check against.  \code{checks} is an object with the individual
predicates as properties.

\item \code{attributes}\\
The attributes to modify and their respective values.

\end{itemize}

\paragraph{Returns:}
This asynchronous operation returns a \code{Deferred} object with a
\code{waitForIt} method which blocks and returns a number indicating the number
of objects counted.

On error, this function will raise a \code{HyperDexClientException} describing
the error.


\paragraph{See also:}  This is the asynchronous form of \code{group\_atomic\_or}.

%%%%%%%%%%%%%%%%%%%% atomic_xor %%%%%%%%%%%%%%%%%%%%
\pagebreak
\subsubsection{\code{atomic\_xor}}
\label{api:python:atomic_xor}
\index{atomic\_xor!Python API}
Store the bitwise XOR of the existing value and the specified number for each
attribute.
This operation requires a pre-existing object in order to complete successfully.
If no object exists, the operation will fail with \code{NOTFOUND}.



\paragraph{Definition:}
\begin{pythoncode}
def atomic_xor(self, spacename, key, attributes)
\end{pythoncode}

\paragraph{Parameters:}
\begin{itemize}[noitemsep]
\item \code{spacename}\\
The name of an existing space.

\item \code{key}\\
The key for the operation where \code{key} is a bytestring and \code{key\_sz}
specifies the number of bytes in \code{key}.

\item \code{attributes}\\
The attributes to modify and their respective values.

\end{itemize}

\paragraph{Returns:}
True if the operation succeeded, False if any of the provided predicates failed.
Raises an exception on error.


\pagebreak
\subsubsection{\code{async\_atomic\_xor}}
\label{api:python:async_atomic_xor}
\index{async\_atomic\_xor!Python API}
Store the bitwise XOR of the existing value and the specified number for each
attribute.
This operation requires a pre-existing object in order to complete successfully.
If no object exists, the operation will fail with \code{NOTFOUND}.



\paragraph{Definition:}
\begin{pythoncode}
def async_atomic_xor(self, spacename, key, attributes)
\end{pythoncode}

\paragraph{Parameters:}
\begin{itemize}[noitemsep]
\item \code{spacename}\\
The name of an existing space.

\item \code{key}\\
The key for the operation where \code{key} is a bytestring and \code{key\_sz}
specifies the number of bytes in \code{key}.

\item \code{attributes}\\
The attributes to modify and their respective values.

\end{itemize}

\paragraph{Returns:}
A Deferred object with a \code{wait} method that returns True if the operation
succeeded, or False if any of the provided predicates failed.  Raises an
exception on error.


\paragraph{See also:}  This is the asynchronous form of \code{atomic\_xor}.

%%%%%%%%%%%%%%%%%%%% cond_atomic_xor %%%%%%%%%%%%%%%%%%%%
\pagebreak
\subsubsection{\code{cond\_atomic\_xor}}
\label{api:python:cond_atomic_xor}
\index{cond\_atomic\_xor!Python API}
Conditionally store the bitwise XOR of the existing value and the specified
number for each attribute.

%%% Generated below here
\paragraph{Behavior:}
\begin{itemize}[noitemsep]
This operation requires a pre-existing object in order to complete successfully.
If no object exists, the operation will fail with \code{NOTFOUND}.

This operation will succeed if and only if the predicates specified by
\code{checks} hold on the pre-existing object.  If any of the predicates are not
true for the existing object, then the operation will have no effect and fail
with \code{CMPFAIL}.

All checks are atomic with the write.  HyperDex guarantees that no other
operation will come between validating the checks, and writing the new version
of the object.

\end{itemize}


\paragraph{Definition:}
\begin{pythoncode}
def cond_atomic_xor(self, spacename, key, predicates, attributes)
\end{pythoncode}

\paragraph{Parameters:}
\begin{itemize}[noitemsep]
\item \code{spacename}\\
The name of an existing space.

\item \code{key}\\
The key for the operation where \code{key} is a bytestring and \code{key\_sz}
specifies the number of bytes in \code{key}.

\item \code{predicates}\\
The predicates to check against.  \code{checks} is an object with the individual
predicates as properties.

\item \code{attributes}\\
The attributes to modify and their respective values.

\end{itemize}

\paragraph{Returns:}
True if the operation succeeded, False if any of the provided predicates failed.
Raises an exception on error.


\pagebreak
\subsubsection{\code{async\_cond\_atomic\_xor}}
\label{api:python:async_cond_atomic_xor}
\index{async\_cond\_atomic\_xor!Python API}
Conditionally store the bitwise XOR of the existing value and the specified
number for each attribute.

%%% Generated below here
\paragraph{Behavior:}
\begin{itemize}[noitemsep]
This operation requires a pre-existing object in order to complete successfully.
If no object exists, the operation will fail with \code{NOTFOUND}.

This operation will succeed if and only if the predicates specified by
\code{checks} hold on the pre-existing object.  If any of the predicates are not
true for the existing object, then the operation will have no effect and fail
with \code{CMPFAIL}.

All checks are atomic with the write.  HyperDex guarantees that no other
operation will come between validating the checks, and writing the new version
of the object.

\end{itemize}


\paragraph{Definition:}
\begin{pythoncode}
def async_cond_atomic_xor(self, spacename, key, predicates, attributes)
\end{pythoncode}

\paragraph{Parameters:}
\begin{itemize}[noitemsep]
\item \code{spacename}\\
The name of an existing space.

\item \code{key}\\
The key for the operation where \code{key} is a bytestring and \code{key\_sz}
specifies the number of bytes in \code{key}.

\item \code{predicates}\\
The predicates to check against.  \code{checks} is an object with the individual
predicates as properties.

\item \code{attributes}\\
The attributes to modify and their respective values.

\end{itemize}

\paragraph{Returns:}
A Deferred object with a \code{wait} method that returns True if the operation
succeeded, or False if any of the provided predicates failed.  Raises an
exception on error.


\paragraph{See also:}  This is the asynchronous form of \code{cond\_atomic\_xor}.

%%%%%%%%%%%%%%%%%%%% group_atomic_xor %%%%%%%%%%%%%%%%%%%%
\pagebreak
\subsubsection{\code{group\_atomic\_xor}}
\label{api:python:group_atomic_xor}
\index{group\_atomic\_xor!Python API}
\input{\topdir/client/fragments/group_atomic_xor}

\paragraph{Definition:}
\begin{pythoncode}
def group_atomic_xor(self, spacename, predicates, attributes)
\end{pythoncode}

\paragraph{Parameters:}
\begin{itemize}[noitemsep]
\item \code{spacename}\\
The name of an existing space.

\item \code{predicates}\\
The predicates to check against.  \code{checks} is an object with the individual
predicates as properties.

\item \code{attributes}\\
The attributes to modify and their respective values.

\end{itemize}

\paragraph{Returns:}
A count of the number of objects, and a \code{client.Error} object indicating
the status of the operation.


\pagebreak
\subsubsection{\code{async\_group\_atomic\_xor}}
\label{api:python:async_group_atomic_xor}
\index{async\_group\_atomic\_xor!Python API}
\input{\topdir/client/fragments/group_atomic_xor}

\paragraph{Definition:}
\begin{pythoncode}
def async_group_atomic_xor(self, spacename, predicates, attributes)
\end{pythoncode}

\paragraph{Parameters:}
\begin{itemize}[noitemsep]
\item \code{spacename}\\
The name of an existing space.

\item \code{predicates}\\
The predicates to check against.  \code{checks} is an object with the individual
predicates as properties.

\item \code{attributes}\\
The attributes to modify and their respective values.

\end{itemize}

\paragraph{Returns:}
This asynchronous operation returns a \code{Deferred} object with a
\code{waitForIt} method which blocks and returns a number indicating the number
of objects counted.

On error, this function will raise a \code{HyperDexClientException} describing
the error.


\paragraph{See also:}  This is the asynchronous form of \code{group\_atomic\_xor}.

%%%%%%%%%%%%%%%%%%%% atomic_min %%%%%%%%%%%%%%%%%%%%
\pagebreak
\subsubsection{\code{atomic\_min}}
\label{api:python:atomic_min}
\index{atomic\_min!Python API}
Store the minimum of the existing value and the provided value for each
attribute.
This operation requires a pre-existing object in order to complete successfully.
If no object exists, the operation will fail with \code{NOTFOUND}.



\paragraph{Definition:}
\begin{pythoncode}
def atomic_min(self, spacename, key, attributes)
\end{pythoncode}

\paragraph{Parameters:}
\begin{itemize}[noitemsep]
\item \code{spacename}\\
The name of an existing space.

\item \code{key}\\
The key for the operation where \code{key} is a bytestring and \code{key\_sz}
specifies the number of bytes in \code{key}.

\item \code{attributes}\\
The attributes to modify and their respective values.

\end{itemize}

\paragraph{Returns:}
True if the operation succeeded, False if any of the provided predicates failed.
Raises an exception on error.


\pagebreak
\subsubsection{\code{async\_atomic\_min}}
\label{api:python:async_atomic_min}
\index{async\_atomic\_min!Python API}
Store the minimum of the existing value and the provided value for each
attribute.
This operation requires a pre-existing object in order to complete successfully.
If no object exists, the operation will fail with \code{NOTFOUND}.



\paragraph{Definition:}
\begin{pythoncode}
def async_atomic_min(self, spacename, key, attributes)
\end{pythoncode}

\paragraph{Parameters:}
\begin{itemize}[noitemsep]
\item \code{spacename}\\
The name of an existing space.

\item \code{key}\\
The key for the operation where \code{key} is a bytestring and \code{key\_sz}
specifies the number of bytes in \code{key}.

\item \code{attributes}\\
The attributes to modify and their respective values.

\end{itemize}

\paragraph{Returns:}
A Deferred object with a \code{wait} method that returns True if the operation
succeeded, or False if any of the provided predicates failed.  Raises an
exception on error.


\paragraph{See also:}  This is the asynchronous form of \code{atomic\_min}.

%%%%%%%%%%%%%%%%%%%% cond_atomic_min %%%%%%%%%%%%%%%%%%%%
\pagebreak
\subsubsection{\code{cond\_atomic\_min}}
\label{api:python:cond_atomic_min}
\index{cond\_atomic\_min!Python API}
\input{\topdir/client/fragments/cond_atomic_min}

\paragraph{Definition:}
\begin{pythoncode}
def cond_atomic_min(self, spacename, key, predicates, attributes)
\end{pythoncode}

\paragraph{Parameters:}
\begin{itemize}[noitemsep]
\item \code{spacename}\\
The name of an existing space.

\item \code{key}\\
The key for the operation where \code{key} is a bytestring and \code{key\_sz}
specifies the number of bytes in \code{key}.

\item \code{predicates}\\
The predicates to check against.  \code{checks} is an object with the individual
predicates as properties.

\item \code{attributes}\\
The attributes to modify and their respective values.

\end{itemize}

\paragraph{Returns:}
True if the operation succeeded, False if any of the provided predicates failed.
Raises an exception on error.


\pagebreak
\subsubsection{\code{async\_cond\_atomic\_min}}
\label{api:python:async_cond_atomic_min}
\index{async\_cond\_atomic\_min!Python API}
\input{\topdir/client/fragments/cond_atomic_min}

\paragraph{Definition:}
\begin{pythoncode}
def async_cond_atomic_min(self, spacename, key, predicates, attributes)
\end{pythoncode}

\paragraph{Parameters:}
\begin{itemize}[noitemsep]
\item \code{spacename}\\
The name of an existing space.

\item \code{key}\\
The key for the operation where \code{key} is a bytestring and \code{key\_sz}
specifies the number of bytes in \code{key}.

\item \code{predicates}\\
The predicates to check against.  \code{checks} is an object with the individual
predicates as properties.

\item \code{attributes}\\
The attributes to modify and their respective values.

\end{itemize}

\paragraph{Returns:}
A Deferred object with a \code{wait} method that returns True if the operation
succeeded, or False if any of the provided predicates failed.  Raises an
exception on error.


\paragraph{See also:}  This is the asynchronous form of \code{cond\_atomic\_min}.

%%%%%%%%%%%%%%%%%%%% group_atomic_min %%%%%%%%%%%%%%%%%%%%
\pagebreak
\subsubsection{\code{group\_atomic\_min}}
\label{api:python:group_atomic_min}
\index{group\_atomic\_min!Python API}
Store the minimum of the existing value and the provided value for each
object in \code{space} that matches \code{checks}.

This operation will only affect objects that match the provided \code{checks}.
Objects that do not match \code{checks} will be unaffected by the group call.
Each object that matches \code{checks} will be atomically updated with the check
on the object.  HyperDex guarantees that no object will be altered if the
\code{checks} do not pass at the time of the write.  Objects that are updated
concurrently with the group call may or may not be updated; however, regardless
of any other concurrent operations, the preceding guarantee will always hold.



\paragraph{Definition:}
\begin{pythoncode}
def group_atomic_min(self, spacename, predicates, attributes)
\end{pythoncode}

\paragraph{Parameters:}
\begin{itemize}[noitemsep]
\item \code{spacename}\\
The name of an existing space.

\item \code{predicates}\\
The predicates to check against.  \code{checks} is an object with the individual
predicates as properties.

\item \code{attributes}\\
The attributes to modify and their respective values.

\end{itemize}

\paragraph{Returns:}
A count of the number of objects, and a \code{client.Error} object indicating
the status of the operation.


\pagebreak
\subsubsection{\code{async\_group\_atomic\_min}}
\label{api:python:async_group_atomic_min}
\index{async\_group\_atomic\_min!Python API}
Store the minimum of the existing value and the provided value for each
object in \code{space} that matches \code{checks}.

This operation will only affect objects that match the provided \code{checks}.
Objects that do not match \code{checks} will be unaffected by the group call.
Each object that matches \code{checks} will be atomically updated with the check
on the object.  HyperDex guarantees that no object will be altered if the
\code{checks} do not pass at the time of the write.  Objects that are updated
concurrently with the group call may or may not be updated; however, regardless
of any other concurrent operations, the preceding guarantee will always hold.



\paragraph{Definition:}
\begin{pythoncode}
def async_group_atomic_min(self, spacename, predicates, attributes)
\end{pythoncode}

\paragraph{Parameters:}
\begin{itemize}[noitemsep]
\item \code{spacename}\\
The name of an existing space.

\item \code{predicates}\\
The predicates to check against.  \code{checks} is an object with the individual
predicates as properties.

\item \code{attributes}\\
The attributes to modify and their respective values.

\end{itemize}

\paragraph{Returns:}
This asynchronous operation returns a \code{Deferred} object with a
\code{waitForIt} method which blocks and returns a number indicating the number
of objects counted.

On error, this function will raise a \code{HyperDexClientException} describing
the error.


\paragraph{See also:}  This is the asynchronous form of \code{group\_atomic\_min}.

%%%%%%%%%%%%%%%%%%%% atomic_max %%%%%%%%%%%%%%%%%%%%
\pagebreak
\subsubsection{\code{atomic\_max}}
\label{api:python:atomic_max}
\index{atomic\_max!Python API}
\input{\topdir/client/fragments/atomic_max}

\paragraph{Definition:}
\begin{pythoncode}
def atomic_max(self, spacename, key, attributes)
\end{pythoncode}

\paragraph{Parameters:}
\begin{itemize}[noitemsep]
\item \code{spacename}\\
The name of an existing space.

\item \code{key}\\
The key for the operation where \code{key} is a bytestring and \code{key\_sz}
specifies the number of bytes in \code{key}.

\item \code{attributes}\\
The attributes to modify and their respective values.

\end{itemize}

\paragraph{Returns:}
True if the operation succeeded, False if any of the provided predicates failed.
Raises an exception on error.


\pagebreak
\subsubsection{\code{async\_atomic\_max}}
\label{api:python:async_atomic_max}
\index{async\_atomic\_max!Python API}
\input{\topdir/client/fragments/atomic_max}

\paragraph{Definition:}
\begin{pythoncode}
def async_atomic_max(self, spacename, key, attributes)
\end{pythoncode}

\paragraph{Parameters:}
\begin{itemize}[noitemsep]
\item \code{spacename}\\
The name of an existing space.

\item \code{key}\\
The key for the operation where \code{key} is a bytestring and \code{key\_sz}
specifies the number of bytes in \code{key}.

\item \code{attributes}\\
The attributes to modify and their respective values.

\end{itemize}

\paragraph{Returns:}
A Deferred object with a \code{wait} method that returns True if the operation
succeeded, or False if any of the provided predicates failed.  Raises an
exception on error.


\paragraph{See also:}  This is the asynchronous form of \code{atomic\_max}.

%%%%%%%%%%%%%%%%%%%% cond_atomic_max %%%%%%%%%%%%%%%%%%%%
\pagebreak
\subsubsection{\code{cond\_atomic\_max}}
\label{api:python:cond_atomic_max}
\index{cond\_atomic\_max!Python API}
Store the maximum of the existing value and the provided value for each
attribute if and only if \code{checks} hold on the object.
This operation requires a pre-existing object in order to complete successfully.
If no object exists, the operation will fail with \code{NOTFOUND}.


This operation will succeed if and only if the predicates specified by
\code{checks} hold on the pre-existing object.  If any of the predicates are not
true for the existing object, then the operation will have no effect and fail
with \code{CMPFAIL}.

All checks are atomic with the write.  HyperDex guarantees that no other
operation will come between validating the checks, and writing the new version
of the object.



\paragraph{Definition:}
\begin{pythoncode}
def cond_atomic_max(self, spacename, key, predicates, attributes)
\end{pythoncode}

\paragraph{Parameters:}
\begin{itemize}[noitemsep]
\item \code{spacename}\\
The name of an existing space.

\item \code{key}\\
The key for the operation where \code{key} is a bytestring and \code{key\_sz}
specifies the number of bytes in \code{key}.

\item \code{predicates}\\
The predicates to check against.  \code{checks} is an object with the individual
predicates as properties.

\item \code{attributes}\\
The attributes to modify and their respective values.

\end{itemize}

\paragraph{Returns:}
True if the operation succeeded, False if any of the provided predicates failed.
Raises an exception on error.


\pagebreak
\subsubsection{\code{async\_cond\_atomic\_max}}
\label{api:python:async_cond_atomic_max}
\index{async\_cond\_atomic\_max!Python API}
Store the maximum of the existing value and the provided value for each
attribute if and only if \code{checks} hold on the object.
This operation requires a pre-existing object in order to complete successfully.
If no object exists, the operation will fail with \code{NOTFOUND}.


This operation will succeed if and only if the predicates specified by
\code{checks} hold on the pre-existing object.  If any of the predicates are not
true for the existing object, then the operation will have no effect and fail
with \code{CMPFAIL}.

All checks are atomic with the write.  HyperDex guarantees that no other
operation will come between validating the checks, and writing the new version
of the object.



\paragraph{Definition:}
\begin{pythoncode}
def async_cond_atomic_max(self, spacename, key, predicates, attributes)
\end{pythoncode}

\paragraph{Parameters:}
\begin{itemize}[noitemsep]
\item \code{spacename}\\
The name of an existing space.

\item \code{key}\\
The key for the operation where \code{key} is a bytestring and \code{key\_sz}
specifies the number of bytes in \code{key}.

\item \code{predicates}\\
The predicates to check against.  \code{checks} is an object with the individual
predicates as properties.

\item \code{attributes}\\
The attributes to modify and their respective values.

\end{itemize}

\paragraph{Returns:}
A Deferred object with a \code{wait} method that returns True if the operation
succeeded, or False if any of the provided predicates failed.  Raises an
exception on error.


\paragraph{See also:}  This is the asynchronous form of \code{cond\_atomic\_max}.

%%%%%%%%%%%%%%%%%%%% group_atomic_max %%%%%%%%%%%%%%%%%%%%
\pagebreak
\subsubsection{\code{group\_atomic\_max}}
\label{api:python:group_atomic_max}
\index{group\_atomic\_max!Python API}
\input{\topdir/client/fragments/group_atomic_max}

\paragraph{Definition:}
\begin{pythoncode}
def group_atomic_max(self, spacename, predicates, attributes)
\end{pythoncode}

\paragraph{Parameters:}
\begin{itemize}[noitemsep]
\item \code{spacename}\\
The name of an existing space.

\item \code{predicates}\\
The predicates to check against.  \code{checks} is an object with the individual
predicates as properties.

\item \code{attributes}\\
The attributes to modify and their respective values.

\end{itemize}

\paragraph{Returns:}
A count of the number of objects, and a \code{client.Error} object indicating
the status of the operation.


\pagebreak
\subsubsection{\code{async\_group\_atomic\_max}}
\label{api:python:async_group_atomic_max}
\index{async\_group\_atomic\_max!Python API}
\input{\topdir/client/fragments/group_atomic_max}

\paragraph{Definition:}
\begin{pythoncode}
def async_group_atomic_max(self, spacename, predicates, attributes)
\end{pythoncode}

\paragraph{Parameters:}
\begin{itemize}[noitemsep]
\item \code{spacename}\\
The name of an existing space.

\item \code{predicates}\\
The predicates to check against.  \code{checks} is an object with the individual
predicates as properties.

\item \code{attributes}\\
The attributes to modify and their respective values.

\end{itemize}

\paragraph{Returns:}
This asynchronous operation returns a \code{Deferred} object with a
\code{waitForIt} method which blocks and returns a number indicating the number
of objects counted.

On error, this function will raise a \code{HyperDexClientException} describing
the error.


\paragraph{See also:}  This is the asynchronous form of \code{group\_atomic\_max}.

%%%%%%%%%%%%%%%%%%%% string_prepend %%%%%%%%%%%%%%%%%%%%
\pagebreak
\subsubsection{\code{string\_prepend}}
\label{api:python:string_prepend}
\index{string\_prepend!Python API}
Prepend the specified string to the existing value for each attribute.

%%% Generated below here
\paragraph{Behavior:}
\begin{itemize}[noitemsep]
This operation requires a pre-existing object in order to complete successfully.
If no object exists, the operation will fail with \code{NOTFOUND}.

\end{itemize}


\paragraph{Definition:}
\begin{pythoncode}
def string_prepend(self, spacename, key, attributes)
\end{pythoncode}

\paragraph{Parameters:}
\begin{itemize}[noitemsep]
\item \code{spacename}\\
The name of an existing space.

\item \code{key}\\
The key for the operation where \code{key} is a bytestring and \code{key\_sz}
specifies the number of bytes in \code{key}.

\item \code{attributes}\\
The attributes to modify and their respective values.

\end{itemize}

\paragraph{Returns:}
True if the operation succeeded, False if any of the provided predicates failed.
Raises an exception on error.


\pagebreak
\subsubsection{\code{async\_string\_prepend}}
\label{api:python:async_string_prepend}
\index{async\_string\_prepend!Python API}
Prepend the specified string to the existing value for each attribute.

%%% Generated below here
\paragraph{Behavior:}
\begin{itemize}[noitemsep]
This operation requires a pre-existing object in order to complete successfully.
If no object exists, the operation will fail with \code{NOTFOUND}.

\end{itemize}


\paragraph{Definition:}
\begin{pythoncode}
def async_string_prepend(self, spacename, key, attributes)
\end{pythoncode}

\paragraph{Parameters:}
\begin{itemize}[noitemsep]
\item \code{spacename}\\
The name of an existing space.

\item \code{key}\\
The key for the operation where \code{key} is a bytestring and \code{key\_sz}
specifies the number of bytes in \code{key}.

\item \code{attributes}\\
The attributes to modify and their respective values.

\end{itemize}

\paragraph{Returns:}
A Deferred object with a \code{wait} method that returns True if the operation
succeeded, or False if any of the provided predicates failed.  Raises an
exception on error.


\paragraph{See also:}  This is the asynchronous form of \code{string\_prepend}.

%%%%%%%%%%%%%%%%%%%% cond_string_prepend %%%%%%%%%%%%%%%%%%%%
\pagebreak
\subsubsection{\code{cond\_string\_prepend}}
\label{api:python:cond_string_prepend}
\index{cond\_string\_prepend!Python API}
Conditionally prepend the specified string to the existing value for each
attribute.

%%% Generated below here
\paragraph{Behavior:}
\begin{itemize}[noitemsep]
This operation requires a pre-existing object in order to complete successfully.
If no object exists, the operation will fail with \code{NOTFOUND}.

This operation will succeed if and only if the predicates specified by
\code{checks} hold on the pre-existing object.  If any of the predicates are not
true for the existing object, then the operation will have no effect and fail
with \code{CMPFAIL}.

All checks are atomic with the write.  HyperDex guarantees that no other
operation will come between validating the checks, and writing the new version
of the object.

\end{itemize}


\paragraph{Definition:}
\begin{pythoncode}
def cond_string_prepend(self, spacename, key, predicates, attributes)
\end{pythoncode}

\paragraph{Parameters:}
\begin{itemize}[noitemsep]
\item \code{spacename}\\
The name of an existing space.

\item \code{key}\\
The key for the operation where \code{key} is a bytestring and \code{key\_sz}
specifies the number of bytes in \code{key}.

\item \code{predicates}\\
The predicates to check against.  \code{checks} is an object with the individual
predicates as properties.

\item \code{attributes}\\
The attributes to modify and their respective values.

\end{itemize}

\paragraph{Returns:}
True if the operation succeeded, False if any of the provided predicates failed.
Raises an exception on error.


\pagebreak
\subsubsection{\code{async\_cond\_string\_prepend}}
\label{api:python:async_cond_string_prepend}
\index{async\_cond\_string\_prepend!Python API}
Conditionally prepend the specified string to the existing value for each
attribute.

%%% Generated below here
\paragraph{Behavior:}
\begin{itemize}[noitemsep]
This operation requires a pre-existing object in order to complete successfully.
If no object exists, the operation will fail with \code{NOTFOUND}.

This operation will succeed if and only if the predicates specified by
\code{checks} hold on the pre-existing object.  If any of the predicates are not
true for the existing object, then the operation will have no effect and fail
with \code{CMPFAIL}.

All checks are atomic with the write.  HyperDex guarantees that no other
operation will come between validating the checks, and writing the new version
of the object.

\end{itemize}


\paragraph{Definition:}
\begin{pythoncode}
def async_cond_string_prepend(self, spacename, key, predicates, attributes)
\end{pythoncode}

\paragraph{Parameters:}
\begin{itemize}[noitemsep]
\item \code{spacename}\\
The name of an existing space.

\item \code{key}\\
The key for the operation where \code{key} is a bytestring and \code{key\_sz}
specifies the number of bytes in \code{key}.

\item \code{predicates}\\
The predicates to check against.  \code{checks} is an object with the individual
predicates as properties.

\item \code{attributes}\\
The attributes to modify and their respective values.

\end{itemize}

\paragraph{Returns:}
A Deferred object with a \code{wait} method that returns True if the operation
succeeded, or False if any of the provided predicates failed.  Raises an
exception on error.


\paragraph{See also:}  This is the asynchronous form of \code{cond\_string\_prepend}.

%%%%%%%%%%%%%%%%%%%% group_string_prepend %%%%%%%%%%%%%%%%%%%%
\pagebreak
\subsubsection{\code{group\_string\_prepend}}
\label{api:python:group_string_prepend}
\index{group\_string\_prepend!Python API}
Prepend the specified string to the existing value for each object in
\code{space} that matches \code{checks}.

This operation will only affect objects that match the provided \code{checks}.
Objects that do not match \code{checks} will be unaffected by the group call.
Each object that matches \code{checks} will be atomically updated with the check
on the object.  HyperDex guarantees that no object will be altered if the
\code{checks} do not pass at the time of the write.  Objects that are updated
concurrently with the group call may or may not be updated; however, regardless
of any other concurrent operations, the preceding guarantee will always hold.



\paragraph{Definition:}
\begin{pythoncode}
def group_string_prepend(self, spacename, predicates, attributes)
\end{pythoncode}

\paragraph{Parameters:}
\begin{itemize}[noitemsep]
\item \code{spacename}\\
The name of an existing space.

\item \code{predicates}\\
The predicates to check against.  \code{checks} is an object with the individual
predicates as properties.

\item \code{attributes}\\
The attributes to modify and their respective values.

\end{itemize}

\paragraph{Returns:}
A count of the number of objects, and a \code{client.Error} object indicating
the status of the operation.


\pagebreak
\subsubsection{\code{async\_group\_string\_prepend}}
\label{api:python:async_group_string_prepend}
\index{async\_group\_string\_prepend!Python API}
Prepend the specified string to the existing value for each object in
\code{space} that matches \code{checks}.

This operation will only affect objects that match the provided \code{checks}.
Objects that do not match \code{checks} will be unaffected by the group call.
Each object that matches \code{checks} will be atomically updated with the check
on the object.  HyperDex guarantees that no object will be altered if the
\code{checks} do not pass at the time of the write.  Objects that are updated
concurrently with the group call may or may not be updated; however, regardless
of any other concurrent operations, the preceding guarantee will always hold.



\paragraph{Definition:}
\begin{pythoncode}
def async_group_string_prepend(self, spacename, predicates, attributes)
\end{pythoncode}

\paragraph{Parameters:}
\begin{itemize}[noitemsep]
\item \code{spacename}\\
The name of an existing space.

\item \code{predicates}\\
The predicates to check against.  \code{checks} is an object with the individual
predicates as properties.

\item \code{attributes}\\
The attributes to modify and their respective values.

\end{itemize}

\paragraph{Returns:}
This asynchronous operation returns a \code{Deferred} object with a
\code{waitForIt} method which blocks and returns a number indicating the number
of objects counted.

On error, this function will raise a \code{HyperDexClientException} describing
the error.


\paragraph{See also:}  This is the asynchronous form of \code{group\_string\_prepend}.

%%%%%%%%%%%%%%%%%%%% string_append %%%%%%%%%%%%%%%%%%%%
\pagebreak
\subsubsection{\code{string\_append}}
\label{api:python:string_append}
\index{string\_append!Python API}
Append the specified string to the existing value for each attribute.
This operation requires a pre-existing object in order to complete successfully.
If no object exists, the operation will fail with \code{NOTFOUND}.



\paragraph{Definition:}
\begin{pythoncode}
def string_append(self, spacename, key, attributes)
\end{pythoncode}

\paragraph{Parameters:}
\begin{itemize}[noitemsep]
\item \code{spacename}\\
The name of an existing space.

\item \code{key}\\
The key for the operation where \code{key} is a bytestring and \code{key\_sz}
specifies the number of bytes in \code{key}.

\item \code{attributes}\\
The attributes to modify and their respective values.

\end{itemize}

\paragraph{Returns:}
True if the operation succeeded, False if any of the provided predicates failed.
Raises an exception on error.


\pagebreak
\subsubsection{\code{async\_string\_append}}
\label{api:python:async_string_append}
\index{async\_string\_append!Python API}
Append the specified string to the existing value for each attribute.
This operation requires a pre-existing object in order to complete successfully.
If no object exists, the operation will fail with \code{NOTFOUND}.



\paragraph{Definition:}
\begin{pythoncode}
def async_string_append(self, spacename, key, attributes)
\end{pythoncode}

\paragraph{Parameters:}
\begin{itemize}[noitemsep]
\item \code{spacename}\\
The name of an existing space.

\item \code{key}\\
The key for the operation where \code{key} is a bytestring and \code{key\_sz}
specifies the number of bytes in \code{key}.

\item \code{attributes}\\
The attributes to modify and their respective values.

\end{itemize}

\paragraph{Returns:}
A Deferred object with a \code{wait} method that returns True if the operation
succeeded, or False if any of the provided predicates failed.  Raises an
exception on error.


\paragraph{See also:}  This is the asynchronous form of \code{string\_append}.

%%%%%%%%%%%%%%%%%%%% cond_string_append %%%%%%%%%%%%%%%%%%%%
\pagebreak
\subsubsection{\code{cond\_string\_append}}
\label{api:python:cond_string_append}
\index{cond\_string\_append!Python API}
Append the specified string to the existing value for each attribute if and only
if \code{checks} hold on the object.
This operation requires a pre-existing object in order to complete successfully.
If no object exists, the operation will fail with \code{NOTFOUND}.


This operation will succeed if and only if the predicates specified by
\code{checks} hold on the pre-existing object.  If any of the predicates are not
true for the existing object, then the operation will have no effect and fail
with \code{CMPFAIL}.

All checks are atomic with the write.  HyperDex guarantees that no other
operation will come between validating the checks, and writing the new version
of the object.



\paragraph{Definition:}
\begin{pythoncode}
def cond_string_append(self, spacename, key, predicates, attributes)
\end{pythoncode}

\paragraph{Parameters:}
\begin{itemize}[noitemsep]
\item \code{spacename}\\
The name of an existing space.

\item \code{key}\\
The key for the operation where \code{key} is a bytestring and \code{key\_sz}
specifies the number of bytes in \code{key}.

\item \code{predicates}\\
The predicates to check against.  \code{checks} is an object with the individual
predicates as properties.

\item \code{attributes}\\
The attributes to modify and their respective values.

\end{itemize}

\paragraph{Returns:}
True if the operation succeeded, False if any of the provided predicates failed.
Raises an exception on error.


\pagebreak
\subsubsection{\code{async\_cond\_string\_append}}
\label{api:python:async_cond_string_append}
\index{async\_cond\_string\_append!Python API}
Append the specified string to the existing value for each attribute if and only
if \code{checks} hold on the object.
This operation requires a pre-existing object in order to complete successfully.
If no object exists, the operation will fail with \code{NOTFOUND}.


This operation will succeed if and only if the predicates specified by
\code{checks} hold on the pre-existing object.  If any of the predicates are not
true for the existing object, then the operation will have no effect and fail
with \code{CMPFAIL}.

All checks are atomic with the write.  HyperDex guarantees that no other
operation will come between validating the checks, and writing the new version
of the object.



\paragraph{Definition:}
\begin{pythoncode}
def async_cond_string_append(self, spacename, key, predicates, attributes)
\end{pythoncode}

\paragraph{Parameters:}
\begin{itemize}[noitemsep]
\item \code{spacename}\\
The name of an existing space.

\item \code{key}\\
The key for the operation where \code{key} is a bytestring and \code{key\_sz}
specifies the number of bytes in \code{key}.

\item \code{predicates}\\
The predicates to check against.  \code{checks} is an object with the individual
predicates as properties.

\item \code{attributes}\\
The attributes to modify and their respective values.

\end{itemize}

\paragraph{Returns:}
A Deferred object with a \code{wait} method that returns True if the operation
succeeded, or False if any of the provided predicates failed.  Raises an
exception on error.


\paragraph{See also:}  This is the asynchronous form of \code{cond\_string\_append}.

%%%%%%%%%%%%%%%%%%%% group_string_append %%%%%%%%%%%%%%%%%%%%
\pagebreak
\subsubsection{\code{group\_string\_append}}
\label{api:python:group_string_append}
\index{group\_string\_append!Python API}
\input{\topdir/client/fragments/group_string_append}

\paragraph{Definition:}
\begin{pythoncode}
def group_string_append(self, spacename, predicates, attributes)
\end{pythoncode}

\paragraph{Parameters:}
\begin{itemize}[noitemsep]
\item \code{spacename}\\
The name of an existing space.

\item \code{predicates}\\
The predicates to check against.  \code{checks} is an object with the individual
predicates as properties.

\item \code{attributes}\\
The attributes to modify and their respective values.

\end{itemize}

\paragraph{Returns:}
A count of the number of objects, and a \code{client.Error} object indicating
the status of the operation.


\pagebreak
\subsubsection{\code{async\_group\_string\_append}}
\label{api:python:async_group_string_append}
\index{async\_group\_string\_append!Python API}
\input{\topdir/client/fragments/group_string_append}

\paragraph{Definition:}
\begin{pythoncode}
def async_group_string_append(self, spacename, predicates, attributes)
\end{pythoncode}

\paragraph{Parameters:}
\begin{itemize}[noitemsep]
\item \code{spacename}\\
The name of an existing space.

\item \code{predicates}\\
The predicates to check against.  \code{checks} is an object with the individual
predicates as properties.

\item \code{attributes}\\
The attributes to modify and their respective values.

\end{itemize}

\paragraph{Returns:}
This asynchronous operation returns a \code{Deferred} object with a
\code{waitForIt} method which blocks and returns a number indicating the number
of objects counted.

On error, this function will raise a \code{HyperDexClientException} describing
the error.


\paragraph{See also:}  This is the asynchronous form of \code{group\_string\_append}.

%%%%%%%%%%%%%%%%%%%% string_ltrim %%%%%%%%%%%%%%%%%%%%
\pagebreak
\subsubsection{\code{string\_ltrim}}
\label{api:python:string_ltrim}
\index{string\_ltrim!Python API}
Remove the specified number of bytes from the start of the string.
This operation requires a pre-existing object in order to complete successfully.
If no object exists, the operation will fail with \code{NOTFOUND}.



\paragraph{Definition:}
\begin{pythoncode}
def string_ltrim(self, spacename, key, attributes)
\end{pythoncode}

\paragraph{Parameters:}
\begin{itemize}[noitemsep]
\item \code{spacename}\\
The name of an existing space.

\item \code{key}\\
The key for the operation where \code{key} is a bytestring and \code{key\_sz}
specifies the number of bytes in \code{key}.

\item \code{attributes}\\
The attributes to modify and their respective values.

\end{itemize}

\paragraph{Returns:}
True if the operation succeeded, False if any of the provided predicates failed.
Raises an exception on error.


\pagebreak
\subsubsection{\code{async\_string\_ltrim}}
\label{api:python:async_string_ltrim}
\index{async\_string\_ltrim!Python API}
Remove the specified number of bytes from the start of the string.
This operation requires a pre-existing object in order to complete successfully.
If no object exists, the operation will fail with \code{NOTFOUND}.



\paragraph{Definition:}
\begin{pythoncode}
def async_string_ltrim(self, spacename, key, attributes)
\end{pythoncode}

\paragraph{Parameters:}
\begin{itemize}[noitemsep]
\item \code{spacename}\\
The name of an existing space.

\item \code{key}\\
The key for the operation where \code{key} is a bytestring and \code{key\_sz}
specifies the number of bytes in \code{key}.

\item \code{attributes}\\
The attributes to modify and their respective values.

\end{itemize}

\paragraph{Returns:}
A Deferred object with a \code{wait} method that returns True if the operation
succeeded, or False if any of the provided predicates failed.  Raises an
exception on error.


\paragraph{See also:}  This is the asynchronous form of \code{string\_ltrim}.

%%%%%%%%%%%%%%%%%%%% cond_string_ltrim %%%%%%%%%%%%%%%%%%%%
\pagebreak
\subsubsection{\code{cond\_string\_ltrim}}
\label{api:python:cond_string_ltrim}
\index{cond\_string\_ltrim!Python API}
\input{\topdir/client/fragments/cond_string_ltrim}

\paragraph{Definition:}
\begin{pythoncode}
def cond_string_ltrim(self, spacename, key, predicates, attributes)
\end{pythoncode}

\paragraph{Parameters:}
\begin{itemize}[noitemsep]
\item \code{spacename}\\
The name of an existing space.

\item \code{key}\\
The key for the operation where \code{key} is a bytestring and \code{key\_sz}
specifies the number of bytes in \code{key}.

\item \code{predicates}\\
The predicates to check against.  \code{checks} is an object with the individual
predicates as properties.

\item \code{attributes}\\
The attributes to modify and their respective values.

\end{itemize}

\paragraph{Returns:}
True if the operation succeeded, False if any of the provided predicates failed.
Raises an exception on error.


\pagebreak
\subsubsection{\code{async\_cond\_string\_ltrim}}
\label{api:python:async_cond_string_ltrim}
\index{async\_cond\_string\_ltrim!Python API}
\input{\topdir/client/fragments/cond_string_ltrim}

\paragraph{Definition:}
\begin{pythoncode}
def async_cond_string_ltrim(self, spacename, key, predicates, attributes)
\end{pythoncode}

\paragraph{Parameters:}
\begin{itemize}[noitemsep]
\item \code{spacename}\\
The name of an existing space.

\item \code{key}\\
The key for the operation where \code{key} is a bytestring and \code{key\_sz}
specifies the number of bytes in \code{key}.

\item \code{predicates}\\
The predicates to check against.  \code{checks} is an object with the individual
predicates as properties.

\item \code{attributes}\\
The attributes to modify and their respective values.

\end{itemize}

\paragraph{Returns:}
A Deferred object with a \code{wait} method that returns True if the operation
succeeded, or False if any of the provided predicates failed.  Raises an
exception on error.


\paragraph{See also:}  This is the asynchronous form of \code{cond\_string\_ltrim}.

%%%%%%%%%%%%%%%%%%%% group_string_ltrim %%%%%%%%%%%%%%%%%%%%
\pagebreak
\subsubsection{\code{group\_string\_ltrim}}
\label{api:python:group_string_ltrim}
\index{group\_string\_ltrim!Python API}
Remove the specified number of bytes from the start of the string for for each
object in \code{space} that matches \code{checks}.

This operation will only affect objects that match the provided \code{checks}.
Objects that do not match \code{checks} will be unaffected by the group call.
Each object that matches \code{checks} will be atomically updated with the check
on the object.  HyperDex guarantees that no object will be altered if the
\code{checks} do not pass at the time of the write.  Objects that are updated
concurrently with the group call may or may not be updated; however, regardless
of any other concurrent operations, the preceding guarantee will always hold.



\paragraph{Definition:}
\begin{pythoncode}
def group_string_ltrim(self, spacename, predicates, attributes)
\end{pythoncode}

\paragraph{Parameters:}
\begin{itemize}[noitemsep]
\item \code{spacename}\\
The name of an existing space.

\item \code{predicates}\\
The predicates to check against.  \code{checks} is an object with the individual
predicates as properties.

\item \code{attributes}\\
The attributes to modify and their respective values.

\end{itemize}

\paragraph{Returns:}
A count of the number of objects, and a \code{client.Error} object indicating
the status of the operation.


\pagebreak
\subsubsection{\code{async\_group\_string\_ltrim}}
\label{api:python:async_group_string_ltrim}
\index{async\_group\_string\_ltrim!Python API}
Remove the specified number of bytes from the start of the string for for each
object in \code{space} that matches \code{checks}.

This operation will only affect objects that match the provided \code{checks}.
Objects that do not match \code{checks} will be unaffected by the group call.
Each object that matches \code{checks} will be atomically updated with the check
on the object.  HyperDex guarantees that no object will be altered if the
\code{checks} do not pass at the time of the write.  Objects that are updated
concurrently with the group call may or may not be updated; however, regardless
of any other concurrent operations, the preceding guarantee will always hold.



\paragraph{Definition:}
\begin{pythoncode}
def async_group_string_ltrim(self, spacename, predicates, attributes)
\end{pythoncode}

\paragraph{Parameters:}
\begin{itemize}[noitemsep]
\item \code{spacename}\\
The name of an existing space.

\item \code{predicates}\\
The predicates to check against.  \code{checks} is an object with the individual
predicates as properties.

\item \code{attributes}\\
The attributes to modify and their respective values.

\end{itemize}

\paragraph{Returns:}
This asynchronous operation returns a \code{Deferred} object with a
\code{waitForIt} method which blocks and returns a number indicating the number
of objects counted.

On error, this function will raise a \code{HyperDexClientException} describing
the error.


\paragraph{See also:}  This is the asynchronous form of \code{group\_string\_ltrim}.

%%%%%%%%%%%%%%%%%%%% string_rtrim %%%%%%%%%%%%%%%%%%%%
\pagebreak
\subsubsection{\code{string\_rtrim}}
\label{api:python:string_rtrim}
\index{string\_rtrim!Python API}
Remove the specified number of bytes from the end of the string.
This operation requires a pre-existing object in order to complete successfully.
If no object exists, the operation will fail with \code{NOTFOUND}.



\paragraph{Definition:}
\begin{pythoncode}
def string_rtrim(self, spacename, key, attributes)
\end{pythoncode}

\paragraph{Parameters:}
\begin{itemize}[noitemsep]
\item \code{spacename}\\
The name of an existing space.

\item \code{key}\\
The key for the operation where \code{key} is a bytestring and \code{key\_sz}
specifies the number of bytes in \code{key}.

\item \code{attributes}\\
The attributes to modify and their respective values.

\end{itemize}

\paragraph{Returns:}
True if the operation succeeded, False if any of the provided predicates failed.
Raises an exception on error.


\pagebreak
\subsubsection{\code{async\_string\_rtrim}}
\label{api:python:async_string_rtrim}
\index{async\_string\_rtrim!Python API}
Remove the specified number of bytes from the end of the string.
This operation requires a pre-existing object in order to complete successfully.
If no object exists, the operation will fail with \code{NOTFOUND}.



\paragraph{Definition:}
\begin{pythoncode}
def async_string_rtrim(self, spacename, key, attributes)
\end{pythoncode}

\paragraph{Parameters:}
\begin{itemize}[noitemsep]
\item \code{spacename}\\
The name of an existing space.

\item \code{key}\\
The key for the operation where \code{key} is a bytestring and \code{key\_sz}
specifies the number of bytes in \code{key}.

\item \code{attributes}\\
The attributes to modify and their respective values.

\end{itemize}

\paragraph{Returns:}
A Deferred object with a \code{wait} method that returns True if the operation
succeeded, or False if any of the provided predicates failed.  Raises an
exception on error.


\paragraph{See also:}  This is the asynchronous form of \code{string\_rtrim}.

%%%%%%%%%%%%%%%%%%%% cond_string_rtrim %%%%%%%%%%%%%%%%%%%%
\pagebreak
\subsubsection{\code{cond\_string\_rtrim}}
\label{api:python:cond_string_rtrim}
\index{cond\_string\_rtrim!Python API}
Remove the specified number of bytes from the end of the string if and only if
\code{checks} hold on the object.
This operation requires a pre-existing object in order to complete successfully.
If no object exists, the operation will fail with \code{NOTFOUND}.


This operation will succeed if and only if the predicates specified by
\code{checks} hold on the pre-existing object.  If any of the predicates are not
true for the existing object, then the operation will have no effect and fail
with \code{CMPFAIL}.

All checks are atomic with the write.  HyperDex guarantees that no other
operation will come between validating the checks, and writing the new version
of the object.



\paragraph{Definition:}
\begin{pythoncode}
def cond_string_rtrim(self, spacename, key, predicates, attributes)
\end{pythoncode}

\paragraph{Parameters:}
\begin{itemize}[noitemsep]
\item \code{spacename}\\
The name of an existing space.

\item \code{key}\\
The key for the operation where \code{key} is a bytestring and \code{key\_sz}
specifies the number of bytes in \code{key}.

\item \code{predicates}\\
The predicates to check against.  \code{checks} is an object with the individual
predicates as properties.

\item \code{attributes}\\
The attributes to modify and their respective values.

\end{itemize}

\paragraph{Returns:}
True if the operation succeeded, False if any of the provided predicates failed.
Raises an exception on error.


\pagebreak
\subsubsection{\code{async\_cond\_string\_rtrim}}
\label{api:python:async_cond_string_rtrim}
\index{async\_cond\_string\_rtrim!Python API}
Remove the specified number of bytes from the end of the string if and only if
\code{checks} hold on the object.
This operation requires a pre-existing object in order to complete successfully.
If no object exists, the operation will fail with \code{NOTFOUND}.


This operation will succeed if and only if the predicates specified by
\code{checks} hold on the pre-existing object.  If any of the predicates are not
true for the existing object, then the operation will have no effect and fail
with \code{CMPFAIL}.

All checks are atomic with the write.  HyperDex guarantees that no other
operation will come between validating the checks, and writing the new version
of the object.



\paragraph{Definition:}
\begin{pythoncode}
def async_cond_string_rtrim(self, spacename, key, predicates, attributes)
\end{pythoncode}

\paragraph{Parameters:}
\begin{itemize}[noitemsep]
\item \code{spacename}\\
The name of an existing space.

\item \code{key}\\
The key for the operation where \code{key} is a bytestring and \code{key\_sz}
specifies the number of bytes in \code{key}.

\item \code{predicates}\\
The predicates to check against.  \code{checks} is an object with the individual
predicates as properties.

\item \code{attributes}\\
The attributes to modify and their respective values.

\end{itemize}

\paragraph{Returns:}
A Deferred object with a \code{wait} method that returns True if the operation
succeeded, or False if any of the provided predicates failed.  Raises an
exception on error.


\paragraph{See also:}  This is the asynchronous form of \code{cond\_string\_rtrim}.

%%%%%%%%%%%%%%%%%%%% group_string_rtrim %%%%%%%%%%%%%%%%%%%%
\pagebreak
\subsubsection{\code{group\_string\_rtrim}}
\label{api:python:group_string_rtrim}
\index{group\_string\_rtrim!Python API}
Remove the specified number of bytes from the end of the string for for each
object in \code{space} that matches \code{checks}.

This operation will only affect objects that match the provided \code{checks}.
Objects that do not match \code{checks} will be unaffected by the group call.
Each object that matches \code{checks} will be atomically updated with the check
on the object.  HyperDex guarantees that no object will be altered if the
\code{checks} do not pass at the time of the write.  Objects that are updated
concurrently with the group call may or may not be updated; however, regardless
of any other concurrent operations, the preceding guarantee will always hold.



\paragraph{Definition:}
\begin{pythoncode}
def group_string_rtrim(self, spacename, predicates, attributes)
\end{pythoncode}

\paragraph{Parameters:}
\begin{itemize}[noitemsep]
\item \code{spacename}\\
The name of an existing space.

\item \code{predicates}\\
The predicates to check against.  \code{checks} is an object with the individual
predicates as properties.

\item \code{attributes}\\
The attributes to modify and their respective values.

\end{itemize}

\paragraph{Returns:}
A count of the number of objects, and a \code{client.Error} object indicating
the status of the operation.


\pagebreak
\subsubsection{\code{async\_group\_string\_rtrim}}
\label{api:python:async_group_string_rtrim}
\index{async\_group\_string\_rtrim!Python API}
Remove the specified number of bytes from the end of the string for for each
object in \code{space} that matches \code{checks}.

This operation will only affect objects that match the provided \code{checks}.
Objects that do not match \code{checks} will be unaffected by the group call.
Each object that matches \code{checks} will be atomically updated with the check
on the object.  HyperDex guarantees that no object will be altered if the
\code{checks} do not pass at the time of the write.  Objects that are updated
concurrently with the group call may or may not be updated; however, regardless
of any other concurrent operations, the preceding guarantee will always hold.



\paragraph{Definition:}
\begin{pythoncode}
def async_group_string_rtrim(self, spacename, predicates, attributes)
\end{pythoncode}

\paragraph{Parameters:}
\begin{itemize}[noitemsep]
\item \code{spacename}\\
The name of an existing space.

\item \code{predicates}\\
The predicates to check against.  \code{checks} is an object with the individual
predicates as properties.

\item \code{attributes}\\
The attributes to modify and their respective values.

\end{itemize}

\paragraph{Returns:}
This asynchronous operation returns a \code{Deferred} object with a
\code{waitForIt} method which blocks and returns a number indicating the number
of objects counted.

On error, this function will raise a \code{HyperDexClientException} describing
the error.


\paragraph{See also:}  This is the asynchronous form of \code{group\_string\_rtrim}.

%%%%%%%%%%%%%%%%%%%% list_lpush %%%%%%%%%%%%%%%%%%%%
\pagebreak
\subsubsection{\code{list\_lpush}}
\label{api:python:list_lpush}
\index{list\_lpush!Python API}
Push the specified value onto the front of the list for each attribute.

%%% Generated below here
\paragraph{Behavior:}
\begin{itemize}[noitemsep]
This operation requires a pre-existing object in order to complete successfully.
If no object exists, the operation will fail with \code{NOTFOUND}.

\end{itemize}


\paragraph{Definition:}
\begin{pythoncode}
def list_lpush(self, spacename, key, attributes)
\end{pythoncode}

\paragraph{Parameters:}
\begin{itemize}[noitemsep]
\item \code{spacename}\\
The name of an existing space.

\item \code{key}\\
The key for the operation where \code{key} is a bytestring and \code{key\_sz}
specifies the number of bytes in \code{key}.

\item \code{attributes}\\
The attributes to modify and their respective values.

\end{itemize}

\paragraph{Returns:}
True if the operation succeeded, False if any of the provided predicates failed.
Raises an exception on error.


\pagebreak
\subsubsection{\code{async\_list\_lpush}}
\label{api:python:async_list_lpush}
\index{async\_list\_lpush!Python API}
Push the specified value onto the front of the list for each attribute.

%%% Generated below here
\paragraph{Behavior:}
\begin{itemize}[noitemsep]
This operation requires a pre-existing object in order to complete successfully.
If no object exists, the operation will fail with \code{NOTFOUND}.

\end{itemize}


\paragraph{Definition:}
\begin{pythoncode}
def async_list_lpush(self, spacename, key, attributes)
\end{pythoncode}

\paragraph{Parameters:}
\begin{itemize}[noitemsep]
\item \code{spacename}\\
The name of an existing space.

\item \code{key}\\
The key for the operation where \code{key} is a bytestring and \code{key\_sz}
specifies the number of bytes in \code{key}.

\item \code{attributes}\\
The attributes to modify and their respective values.

\end{itemize}

\paragraph{Returns:}
A Deferred object with a \code{wait} method that returns True if the operation
succeeded, or False if any of the provided predicates failed.  Raises an
exception on error.


\paragraph{See also:}  This is the asynchronous form of \code{list\_lpush}.

%%%%%%%%%%%%%%%%%%%% cond_list_lpush %%%%%%%%%%%%%%%%%%%%
\pagebreak
\subsubsection{\code{cond\_list\_lpush}}
\label{api:python:cond_list_lpush}
\index{cond\_list\_lpush!Python API}
Condtitionally push the specified value onto the front of the list for each
attribute.

%%% Generated below here
\paragraph{Behavior:}
\begin{itemize}[noitemsep]
This operation requires a pre-existing object in order to complete successfully.
If no object exists, the operation will fail with \code{NOTFOUND}.

This operation will succeed if and only if the predicates specified by
\code{checks} hold on the pre-existing object.  If any of the predicates are not
true for the existing object, then the operation will have no effect and fail
with \code{CMPFAIL}.

All checks are atomic with the write.  HyperDex guarantees that no other
operation will come between validating the checks, and writing the new version
of the object.

\end{itemize}


\paragraph{Definition:}
\begin{pythoncode}
def cond_list_lpush(self, spacename, key, predicates, attributes)
\end{pythoncode}

\paragraph{Parameters:}
\begin{itemize}[noitemsep]
\item \code{spacename}\\
The name of an existing space.

\item \code{key}\\
The key for the operation where \code{key} is a bytestring and \code{key\_sz}
specifies the number of bytes in \code{key}.

\item \code{predicates}\\
The predicates to check against.  \code{checks} is an object with the individual
predicates as properties.

\item \code{attributes}\\
The attributes to modify and their respective values.

\end{itemize}

\paragraph{Returns:}
True if the operation succeeded, False if any of the provided predicates failed.
Raises an exception on error.


\pagebreak
\subsubsection{\code{async\_cond\_list\_lpush}}
\label{api:python:async_cond_list_lpush}
\index{async\_cond\_list\_lpush!Python API}
Condtitionally push the specified value onto the front of the list for each
attribute.

%%% Generated below here
\paragraph{Behavior:}
\begin{itemize}[noitemsep]
This operation requires a pre-existing object in order to complete successfully.
If no object exists, the operation will fail with \code{NOTFOUND}.

This operation will succeed if and only if the predicates specified by
\code{checks} hold on the pre-existing object.  If any of the predicates are not
true for the existing object, then the operation will have no effect and fail
with \code{CMPFAIL}.

All checks are atomic with the write.  HyperDex guarantees that no other
operation will come between validating the checks, and writing the new version
of the object.

\end{itemize}


\paragraph{Definition:}
\begin{pythoncode}
def async_cond_list_lpush(self, spacename, key, predicates, attributes)
\end{pythoncode}

\paragraph{Parameters:}
\begin{itemize}[noitemsep]
\item \code{spacename}\\
The name of an existing space.

\item \code{key}\\
The key for the operation where \code{key} is a bytestring and \code{key\_sz}
specifies the number of bytes in \code{key}.

\item \code{predicates}\\
The predicates to check against.  \code{checks} is an object with the individual
predicates as properties.

\item \code{attributes}\\
The attributes to modify and their respective values.

\end{itemize}

\paragraph{Returns:}
A Deferred object with a \code{wait} method that returns True if the operation
succeeded, or False if any of the provided predicates failed.  Raises an
exception on error.


\paragraph{See also:}  This is the asynchronous form of \code{cond\_list\_lpush}.

%%%%%%%%%%%%%%%%%%%% group_list_lpush %%%%%%%%%%%%%%%%%%%%
\pagebreak
\subsubsection{\code{group\_list\_lpush}}
\label{api:python:group_list_lpush}
\index{group\_list\_lpush!Python API}
Push the specified value onto the front of the list for each object in
\code{space} that matches \code{checks}.

This operation will only affect objects that match the provided \code{checks}.
Objects that do not match \code{checks} will be unaffected by the group call.
Each object that matches \code{checks} will be atomically updated with the check
on the object.  HyperDex guarantees that no object will be altered if the
\code{checks} do not pass at the time of the write.  Objects that are updated
concurrently with the group call may or may not be updated; however, regardless
of any other concurrent operations, the preceding guarantee will always hold.



\paragraph{Definition:}
\begin{pythoncode}
def group_list_lpush(self, spacename, predicates, attributes)
\end{pythoncode}

\paragraph{Parameters:}
\begin{itemize}[noitemsep]
\item \code{spacename}\\
The name of an existing space.

\item \code{predicates}\\
The predicates to check against.  \code{checks} is an object with the individual
predicates as properties.

\item \code{attributes}\\
The attributes to modify and their respective values.

\end{itemize}

\paragraph{Returns:}
A count of the number of objects, and a \code{client.Error} object indicating
the status of the operation.


\pagebreak
\subsubsection{\code{async\_group\_list\_lpush}}
\label{api:python:async_group_list_lpush}
\index{async\_group\_list\_lpush!Python API}
Push the specified value onto the front of the list for each object in
\code{space} that matches \code{checks}.

This operation will only affect objects that match the provided \code{checks}.
Objects that do not match \code{checks} will be unaffected by the group call.
Each object that matches \code{checks} will be atomically updated with the check
on the object.  HyperDex guarantees that no object will be altered if the
\code{checks} do not pass at the time of the write.  Objects that are updated
concurrently with the group call may or may not be updated; however, regardless
of any other concurrent operations, the preceding guarantee will always hold.



\paragraph{Definition:}
\begin{pythoncode}
def async_group_list_lpush(self, spacename, predicates, attributes)
\end{pythoncode}

\paragraph{Parameters:}
\begin{itemize}[noitemsep]
\item \code{spacename}\\
The name of an existing space.

\item \code{predicates}\\
The predicates to check against.  \code{checks} is an object with the individual
predicates as properties.

\item \code{attributes}\\
The attributes to modify and their respective values.

\end{itemize}

\paragraph{Returns:}
This asynchronous operation returns a \code{Deferred} object with a
\code{waitForIt} method which blocks and returns a number indicating the number
of objects counted.

On error, this function will raise a \code{HyperDexClientException} describing
the error.


\paragraph{See also:}  This is the asynchronous form of \code{group\_list\_lpush}.

%%%%%%%%%%%%%%%%%%%% list_rpush %%%%%%%%%%%%%%%%%%%%
\pagebreak
\subsubsection{\code{list\_rpush}}
\label{api:python:list_rpush}
\index{list\_rpush!Python API}
Push the specified value onto the back of the list for each attribute.

%%% Generated below here
\paragraph{Behavior:}
\begin{itemize}[noitemsep]
This operation requires a pre-existing object in order to complete successfully.
If no object exists, the operation will fail with \code{NOTFOUND}.

\end{itemize}


\paragraph{Definition:}
\begin{pythoncode}
def list_rpush(self, spacename, key, attributes)
\end{pythoncode}

\paragraph{Parameters:}
\begin{itemize}[noitemsep]
\item \code{spacename}\\
The name of an existing space.

\item \code{key}\\
The key for the operation where \code{key} is a bytestring and \code{key\_sz}
specifies the number of bytes in \code{key}.

\item \code{attributes}\\
The attributes to modify and their respective values.

\end{itemize}

\paragraph{Returns:}
True if the operation succeeded, False if any of the provided predicates failed.
Raises an exception on error.


\pagebreak
\subsubsection{\code{async\_list\_rpush}}
\label{api:python:async_list_rpush}
\index{async\_list\_rpush!Python API}
Push the specified value onto the back of the list for each attribute.

%%% Generated below here
\paragraph{Behavior:}
\begin{itemize}[noitemsep]
This operation requires a pre-existing object in order to complete successfully.
If no object exists, the operation will fail with \code{NOTFOUND}.

\end{itemize}


\paragraph{Definition:}
\begin{pythoncode}
def async_list_rpush(self, spacename, key, attributes)
\end{pythoncode}

\paragraph{Parameters:}
\begin{itemize}[noitemsep]
\item \code{spacename}\\
The name of an existing space.

\item \code{key}\\
The key for the operation where \code{key} is a bytestring and \code{key\_sz}
specifies the number of bytes in \code{key}.

\item \code{attributes}\\
The attributes to modify and their respective values.

\end{itemize}

\paragraph{Returns:}
A Deferred object with a \code{wait} method that returns True if the operation
succeeded, or False if any of the provided predicates failed.  Raises an
exception on error.


\paragraph{See also:}  This is the asynchronous form of \code{list\_rpush}.

%%%%%%%%%%%%%%%%%%%% cond_list_rpush %%%%%%%%%%%%%%%%%%%%
\pagebreak
\subsubsection{\code{cond\_list\_rpush}}
\label{api:python:cond_list_rpush}
\index{cond\_list\_rpush!Python API}
Push the specified value onto the back of the list for each attribute if and
only if the \code{checks} hold on the object.
This operation requires a pre-existing object in order to complete successfully.
If no object exists, the operation will fail with \code{NOTFOUND}.


This operation will succeed if and only if the predicates specified by
\code{checks} hold on the pre-existing object.  If any of the predicates are not
true for the existing object, then the operation will have no effect and fail
with \code{CMPFAIL}.

All checks are atomic with the write.  HyperDex guarantees that no other
operation will come between validating the checks, and writing the new version
of the object.



\paragraph{Definition:}
\begin{pythoncode}
def cond_list_rpush(self, spacename, key, predicates, attributes)
\end{pythoncode}

\paragraph{Parameters:}
\begin{itemize}[noitemsep]
\item \code{spacename}\\
The name of an existing space.

\item \code{key}\\
The key for the operation where \code{key} is a bytestring and \code{key\_sz}
specifies the number of bytes in \code{key}.

\item \code{predicates}\\
The predicates to check against.  \code{checks} is an object with the individual
predicates as properties.

\item \code{attributes}\\
The attributes to modify and their respective values.

\end{itemize}

\paragraph{Returns:}
True if the operation succeeded, False if any of the provided predicates failed.
Raises an exception on error.


\pagebreak
\subsubsection{\code{async\_cond\_list\_rpush}}
\label{api:python:async_cond_list_rpush}
\index{async\_cond\_list\_rpush!Python API}
Push the specified value onto the back of the list for each attribute if and
only if the \code{checks} hold on the object.
This operation requires a pre-existing object in order to complete successfully.
If no object exists, the operation will fail with \code{NOTFOUND}.


This operation will succeed if and only if the predicates specified by
\code{checks} hold on the pre-existing object.  If any of the predicates are not
true for the existing object, then the operation will have no effect and fail
with \code{CMPFAIL}.

All checks are atomic with the write.  HyperDex guarantees that no other
operation will come between validating the checks, and writing the new version
of the object.



\paragraph{Definition:}
\begin{pythoncode}
def async_cond_list_rpush(self, spacename, key, predicates, attributes)
\end{pythoncode}

\paragraph{Parameters:}
\begin{itemize}[noitemsep]
\item \code{spacename}\\
The name of an existing space.

\item \code{key}\\
The key for the operation where \code{key} is a bytestring and \code{key\_sz}
specifies the number of bytes in \code{key}.

\item \code{predicates}\\
The predicates to check against.  \code{checks} is an object with the individual
predicates as properties.

\item \code{attributes}\\
The attributes to modify and their respective values.

\end{itemize}

\paragraph{Returns:}
A Deferred object with a \code{wait} method that returns True if the operation
succeeded, or False if any of the provided predicates failed.  Raises an
exception on error.


\paragraph{See also:}  This is the asynchronous form of \code{cond\_list\_rpush}.

%%%%%%%%%%%%%%%%%%%% group_list_rpush %%%%%%%%%%%%%%%%%%%%
\pagebreak
\subsubsection{\code{group\_list\_rpush}}
\label{api:python:group_list_rpush}
\index{group\_list\_rpush!Python API}
\input{\topdir/client/fragments/group_list_rpush}

\paragraph{Definition:}
\begin{pythoncode}
def group_list_rpush(self, spacename, predicates, attributes)
\end{pythoncode}

\paragraph{Parameters:}
\begin{itemize}[noitemsep]
\item \code{spacename}\\
The name of an existing space.

\item \code{predicates}\\
The predicates to check against.  \code{checks} is an object with the individual
predicates as properties.

\item \code{attributes}\\
The attributes to modify and their respective values.

\end{itemize}

\paragraph{Returns:}
A count of the number of objects, and a \code{client.Error} object indicating
the status of the operation.


\pagebreak
\subsubsection{\code{async\_group\_list\_rpush}}
\label{api:python:async_group_list_rpush}
\index{async\_group\_list\_rpush!Python API}
\input{\topdir/client/fragments/group_list_rpush}

\paragraph{Definition:}
\begin{pythoncode}
def async_group_list_rpush(self, spacename, predicates, attributes)
\end{pythoncode}

\paragraph{Parameters:}
\begin{itemize}[noitemsep]
\item \code{spacename}\\
The name of an existing space.

\item \code{predicates}\\
The predicates to check against.  \code{checks} is an object with the individual
predicates as properties.

\item \code{attributes}\\
The attributes to modify and their respective values.

\end{itemize}

\paragraph{Returns:}
This asynchronous operation returns a \code{Deferred} object with a
\code{waitForIt} method which blocks and returns a number indicating the number
of objects counted.

On error, this function will raise a \code{HyperDexClientException} describing
the error.


\paragraph{See also:}  This is the asynchronous form of \code{group\_list\_rpush}.

%%%%%%%%%%%%%%%%%%%% set_add %%%%%%%%%%%%%%%%%%%%
\pagebreak
\subsubsection{\code{set\_add}}
\label{api:python:set_add}
\index{set\_add!Python API}
Add the specified value to the set for each attribute.

%%% Generated below here
\paragraph{Behavior:}
\begin{itemize}[noitemsep]
This operation requires a pre-existing object in order to complete successfully.
If no object exists, the operation will fail with \code{NOTFOUND}.

\end{itemize}


\paragraph{Definition:}
\begin{pythoncode}
def set_add(self, spacename, key, attributes)
\end{pythoncode}

\paragraph{Parameters:}
\begin{itemize}[noitemsep]
\item \code{spacename}\\
The name of an existing space.

\item \code{key}\\
The key for the operation where \code{key} is a bytestring and \code{key\_sz}
specifies the number of bytes in \code{key}.

\item \code{attributes}\\
The attributes to modify and their respective values.

\end{itemize}

\paragraph{Returns:}
True if the operation succeeded, False if any of the provided predicates failed.
Raises an exception on error.


\pagebreak
\subsubsection{\code{async\_set\_add}}
\label{api:python:async_set_add}
\index{async\_set\_add!Python API}
Add the specified value to the set for each attribute.

%%% Generated below here
\paragraph{Behavior:}
\begin{itemize}[noitemsep]
This operation requires a pre-existing object in order to complete successfully.
If no object exists, the operation will fail with \code{NOTFOUND}.

\end{itemize}


\paragraph{Definition:}
\begin{pythoncode}
def async_set_add(self, spacename, key, attributes)
\end{pythoncode}

\paragraph{Parameters:}
\begin{itemize}[noitemsep]
\item \code{spacename}\\
The name of an existing space.

\item \code{key}\\
The key for the operation where \code{key} is a bytestring and \code{key\_sz}
specifies the number of bytes in \code{key}.

\item \code{attributes}\\
The attributes to modify and their respective values.

\end{itemize}

\paragraph{Returns:}
A Deferred object with a \code{wait} method that returns True if the operation
succeeded, or False if any of the provided predicates failed.  Raises an
exception on error.


\paragraph{See also:}  This is the asynchronous form of \code{set\_add}.

%%%%%%%%%%%%%%%%%%%% cond_set_add %%%%%%%%%%%%%%%%%%%%
\pagebreak
\subsubsection{\code{cond\_set\_add}}
\label{api:python:cond_set_add}
\index{cond\_set\_add!Python API}
Conditionally add the specified value to the set for each attribute.

%%% Generated below here
\paragraph{Behavior:}
\begin{itemize}[noitemsep]
This operation requires a pre-existing object in order to complete successfully.
If no object exists, the operation will fail with \code{NOTFOUND}.

This operation will succeed if and only if the predicates specified by
\code{checks} hold on the pre-existing object.  If any of the predicates are not
true for the existing object, then the operation will have no effect and fail
with \code{CMPFAIL}.

All checks are atomic with the write.  HyperDex guarantees that no other
operation will come between validating the checks, and writing the new version
of the object.

\end{itemize}


\paragraph{Definition:}
\begin{pythoncode}
def cond_set_add(self, spacename, key, predicates, attributes)
\end{pythoncode}

\paragraph{Parameters:}
\begin{itemize}[noitemsep]
\item \code{spacename}\\
The name of an existing space.

\item \code{key}\\
The key for the operation where \code{key} is a bytestring and \code{key\_sz}
specifies the number of bytes in \code{key}.

\item \code{predicates}\\
The predicates to check against.  \code{checks} is an object with the individual
predicates as properties.

\item \code{attributes}\\
The attributes to modify and their respective values.

\end{itemize}

\paragraph{Returns:}
True if the operation succeeded, False if any of the provided predicates failed.
Raises an exception on error.


\pagebreak
\subsubsection{\code{async\_cond\_set\_add}}
\label{api:python:async_cond_set_add}
\index{async\_cond\_set\_add!Python API}
Conditionally add the specified value to the set for each attribute.

%%% Generated below here
\paragraph{Behavior:}
\begin{itemize}[noitemsep]
This operation requires a pre-existing object in order to complete successfully.
If no object exists, the operation will fail with \code{NOTFOUND}.

This operation will succeed if and only if the predicates specified by
\code{checks} hold on the pre-existing object.  If any of the predicates are not
true for the existing object, then the operation will have no effect and fail
with \code{CMPFAIL}.

All checks are atomic with the write.  HyperDex guarantees that no other
operation will come between validating the checks, and writing the new version
of the object.

\end{itemize}


\paragraph{Definition:}
\begin{pythoncode}
def async_cond_set_add(self, spacename, key, predicates, attributes)
\end{pythoncode}

\paragraph{Parameters:}
\begin{itemize}[noitemsep]
\item \code{spacename}\\
The name of an existing space.

\item \code{key}\\
The key for the operation where \code{key} is a bytestring and \code{key\_sz}
specifies the number of bytes in \code{key}.

\item \code{predicates}\\
The predicates to check against.  \code{checks} is an object with the individual
predicates as properties.

\item \code{attributes}\\
The attributes to modify and their respective values.

\end{itemize}

\paragraph{Returns:}
A Deferred object with a \code{wait} method that returns True if the operation
succeeded, or False if any of the provided predicates failed.  Raises an
exception on error.


\paragraph{See also:}  This is the asynchronous form of \code{cond\_set\_add}.

%%%%%%%%%%%%%%%%%%%% group_set_add %%%%%%%%%%%%%%%%%%%%
\pagebreak
\subsubsection{\code{group\_set\_add}}
\label{api:python:group_set_add}
\index{group\_set\_add!Python API}
Add the specified value to the set for each object in \code{space} that matches
\code{checks}.

This operation will only affect objects that match the provided \code{checks}.
Objects that do not match \code{checks} will be unaffected by the group call.
Each object that matches \code{checks} will be atomically updated with the check
on the object.  HyperDex guarantees that no object will be altered if the
\code{checks} do not pass at the time of the write.  Objects that are updated
concurrently with the group call may or may not be updated; however, regardless
of any other concurrent operations, the preceding guarantee will always hold.



\paragraph{Definition:}
\begin{pythoncode}
def group_set_add(self, spacename, predicates, attributes)
\end{pythoncode}

\paragraph{Parameters:}
\begin{itemize}[noitemsep]
\item \code{spacename}\\
The name of an existing space.

\item \code{predicates}\\
The predicates to check against.  \code{checks} is an object with the individual
predicates as properties.

\item \code{attributes}\\
The attributes to modify and their respective values.

\end{itemize}

\paragraph{Returns:}
A count of the number of objects, and a \code{client.Error} object indicating
the status of the operation.


\pagebreak
\subsubsection{\code{async\_group\_set\_add}}
\label{api:python:async_group_set_add}
\index{async\_group\_set\_add!Python API}
Add the specified value to the set for each object in \code{space} that matches
\code{checks}.

This operation will only affect objects that match the provided \code{checks}.
Objects that do not match \code{checks} will be unaffected by the group call.
Each object that matches \code{checks} will be atomically updated with the check
on the object.  HyperDex guarantees that no object will be altered if the
\code{checks} do not pass at the time of the write.  Objects that are updated
concurrently with the group call may or may not be updated; however, regardless
of any other concurrent operations, the preceding guarantee will always hold.



\paragraph{Definition:}
\begin{pythoncode}
def async_group_set_add(self, spacename, predicates, attributes)
\end{pythoncode}

\paragraph{Parameters:}
\begin{itemize}[noitemsep]
\item \code{spacename}\\
The name of an existing space.

\item \code{predicates}\\
The predicates to check against.  \code{checks} is an object with the individual
predicates as properties.

\item \code{attributes}\\
The attributes to modify and their respective values.

\end{itemize}

\paragraph{Returns:}
This asynchronous operation returns a \code{Deferred} object with a
\code{waitForIt} method which blocks and returns a number indicating the number
of objects counted.

On error, this function will raise a \code{HyperDexClientException} describing
the error.


\paragraph{See also:}  This is the asynchronous form of \code{group\_set\_add}.

%%%%%%%%%%%%%%%%%%%% set_remove %%%%%%%%%%%%%%%%%%%%
\pagebreak
\subsubsection{\code{set\_remove}}
\label{api:python:set_remove}
\index{set\_remove!Python API}
Remove the specified value from the set.  If the value is not contained within
the set, this operation will do nothing.

%%% Generated below here
\paragraph{Behavior:}
\begin{itemize}[noitemsep]
This operation requires a pre-existing object in order to complete successfully.
If no object exists, the operation will fail with \code{NOTFOUND}.

\end{itemize}


\paragraph{Definition:}
\begin{pythoncode}
def set_remove(self, spacename, key, attributes)
\end{pythoncode}

\paragraph{Parameters:}
\begin{itemize}[noitemsep]
\item \code{spacename}\\
The name of an existing space.

\item \code{key}\\
The key for the operation where \code{key} is a bytestring and \code{key\_sz}
specifies the number of bytes in \code{key}.

\item \code{attributes}\\
The attributes to modify and their respective values.

\end{itemize}

\paragraph{Returns:}
True if the operation succeeded, False if any of the provided predicates failed.
Raises an exception on error.


\pagebreak
\subsubsection{\code{async\_set\_remove}}
\label{api:python:async_set_remove}
\index{async\_set\_remove!Python API}
Remove the specified value from the set.  If the value is not contained within
the set, this operation will do nothing.

%%% Generated below here
\paragraph{Behavior:}
\begin{itemize}[noitemsep]
This operation requires a pre-existing object in order to complete successfully.
If no object exists, the operation will fail with \code{NOTFOUND}.

\end{itemize}


\paragraph{Definition:}
\begin{pythoncode}
def async_set_remove(self, spacename, key, attributes)
\end{pythoncode}

\paragraph{Parameters:}
\begin{itemize}[noitemsep]
\item \code{spacename}\\
The name of an existing space.

\item \code{key}\\
The key for the operation where \code{key} is a bytestring and \code{key\_sz}
specifies the number of bytes in \code{key}.

\item \code{attributes}\\
The attributes to modify and their respective values.

\end{itemize}

\paragraph{Returns:}
A Deferred object with a \code{wait} method that returns True if the operation
succeeded, or False if any of the provided predicates failed.  Raises an
exception on error.


\paragraph{See also:}  This is the asynchronous form of \code{set\_remove}.

%%%%%%%%%%%%%%%%%%%% cond_set_remove %%%%%%%%%%%%%%%%%%%%
\pagebreak
\subsubsection{\code{cond\_set\_remove}}
\label{api:python:cond_set_remove}
\index{cond\_set\_remove!Python API}
Conditionally remove the specified value from the set.  If the value is not
contained within the set, this operation will do nothing.

%%% Generated below here
\paragraph{Behavior:}
\begin{itemize}[noitemsep]
This operation requires a pre-existing object in order to complete successfully.
If no object exists, the operation will fail with \code{NOTFOUND}.

This operation will succeed if and only if the predicates specified by
\code{checks} hold on the pre-existing object.  If any of the predicates are not
true for the existing object, then the operation will have no effect and fail
with \code{CMPFAIL}.

All checks are atomic with the write.  HyperDex guarantees that no other
operation will come between validating the checks, and writing the new version
of the object.

\end{itemize}


\paragraph{Definition:}
\begin{pythoncode}
def cond_set_remove(self, spacename, key, predicates, attributes)
\end{pythoncode}

\paragraph{Parameters:}
\begin{itemize}[noitemsep]
\item \code{spacename}\\
The name of an existing space.

\item \code{key}\\
The key for the operation where \code{key} is a bytestring and \code{key\_sz}
specifies the number of bytes in \code{key}.

\item \code{predicates}\\
The predicates to check against.  \code{checks} is an object with the individual
predicates as properties.

\item \code{attributes}\\
The attributes to modify and their respective values.

\end{itemize}

\paragraph{Returns:}
True if the operation succeeded, False if any of the provided predicates failed.
Raises an exception on error.


\pagebreak
\subsubsection{\code{async\_cond\_set\_remove}}
\label{api:python:async_cond_set_remove}
\index{async\_cond\_set\_remove!Python API}
Conditionally remove the specified value from the set.  If the value is not
contained within the set, this operation will do nothing.

%%% Generated below here
\paragraph{Behavior:}
\begin{itemize}[noitemsep]
This operation requires a pre-existing object in order to complete successfully.
If no object exists, the operation will fail with \code{NOTFOUND}.

This operation will succeed if and only if the predicates specified by
\code{checks} hold on the pre-existing object.  If any of the predicates are not
true for the existing object, then the operation will have no effect and fail
with \code{CMPFAIL}.

All checks are atomic with the write.  HyperDex guarantees that no other
operation will come between validating the checks, and writing the new version
of the object.

\end{itemize}


\paragraph{Definition:}
\begin{pythoncode}
def async_cond_set_remove(self, spacename, key, predicates, attributes)
\end{pythoncode}

\paragraph{Parameters:}
\begin{itemize}[noitemsep]
\item \code{spacename}\\
The name of an existing space.

\item \code{key}\\
The key for the operation where \code{key} is a bytestring and \code{key\_sz}
specifies the number of bytes in \code{key}.

\item \code{predicates}\\
The predicates to check against.  \code{checks} is an object with the individual
predicates as properties.

\item \code{attributes}\\
The attributes to modify and their respective values.

\end{itemize}

\paragraph{Returns:}
A Deferred object with a \code{wait} method that returns True if the operation
succeeded, or False if any of the provided predicates failed.  Raises an
exception on error.


\paragraph{See also:}  This is the asynchronous form of \code{cond\_set\_remove}.

%%%%%%%%%%%%%%%%%%%% group_set_remove %%%%%%%%%%%%%%%%%%%%
\pagebreak
\subsubsection{\code{group\_set\_remove}}
\label{api:python:group_set_remove}
\index{group\_set\_remove!Python API}
Remove the specified value from the set for each object in \code{space} that
matches \code{checks}.  If the value is not contained within the set, this
operation will do nothing.

This operation will only affect objects that match the provided \code{checks}.
Objects that do not match \code{checks} will be unaffected by the group call.
Each object that matches \code{checks} will be atomically updated with the check
on the object.  HyperDex guarantees that no object will be altered if the
\code{checks} do not pass at the time of the write.  Objects that are updated
concurrently with the group call may or may not be updated; however, regardless
of any other concurrent operations, the preceding guarantee will always hold.



\paragraph{Definition:}
\begin{pythoncode}
def group_set_remove(self, spacename, predicates, attributes)
\end{pythoncode}

\paragraph{Parameters:}
\begin{itemize}[noitemsep]
\item \code{spacename}\\
The name of an existing space.

\item \code{predicates}\\
The predicates to check against.  \code{checks} is an object with the individual
predicates as properties.

\item \code{attributes}\\
The attributes to modify and their respective values.

\end{itemize}

\paragraph{Returns:}
A count of the number of objects, and a \code{client.Error} object indicating
the status of the operation.


\pagebreak
\subsubsection{\code{async\_group\_set\_remove}}
\label{api:python:async_group_set_remove}
\index{async\_group\_set\_remove!Python API}
Remove the specified value from the set for each object in \code{space} that
matches \code{checks}.  If the value is not contained within the set, this
operation will do nothing.

This operation will only affect objects that match the provided \code{checks}.
Objects that do not match \code{checks} will be unaffected by the group call.
Each object that matches \code{checks} will be atomically updated with the check
on the object.  HyperDex guarantees that no object will be altered if the
\code{checks} do not pass at the time of the write.  Objects that are updated
concurrently with the group call may or may not be updated; however, regardless
of any other concurrent operations, the preceding guarantee will always hold.



\paragraph{Definition:}
\begin{pythoncode}
def async_group_set_remove(self, spacename, predicates, attributes)
\end{pythoncode}

\paragraph{Parameters:}
\begin{itemize}[noitemsep]
\item \code{spacename}\\
The name of an existing space.

\item \code{predicates}\\
The predicates to check against.  \code{checks} is an object with the individual
predicates as properties.

\item \code{attributes}\\
The attributes to modify and their respective values.

\end{itemize}

\paragraph{Returns:}
This asynchronous operation returns a \code{Deferred} object with a
\code{waitForIt} method which blocks and returns a number indicating the number
of objects counted.

On error, this function will raise a \code{HyperDexClientException} describing
the error.


\paragraph{See also:}  This is the asynchronous form of \code{group\_set\_remove}.

%%%%%%%%%%%%%%%%%%%% set_intersect %%%%%%%%%%%%%%%%%%%%
\pagebreak
\subsubsection{\code{set\_intersect}}
\label{api:python:set_intersect}
\index{set\_intersect!Python API}
Store the intersection of the specified set and the existing value for each
attribute.

%%% Generated below here
\paragraph{Behavior:}
\begin{itemize}[noitemsep]
This operation requires a pre-existing object in order to complete successfully.
If no object exists, the operation will fail with \code{NOTFOUND}.

\end{itemize}


\paragraph{Definition:}
\begin{pythoncode}
def set_intersect(self, spacename, key, attributes)
\end{pythoncode}

\paragraph{Parameters:}
\begin{itemize}[noitemsep]
\item \code{spacename}\\
The name of an existing space.

\item \code{key}\\
The key for the operation where \code{key} is a bytestring and \code{key\_sz}
specifies the number of bytes in \code{key}.

\item \code{attributes}\\
The attributes to modify and their respective values.

\end{itemize}

\paragraph{Returns:}
True if the operation succeeded, False if any of the provided predicates failed.
Raises an exception on error.


\pagebreak
\subsubsection{\code{async\_set\_intersect}}
\label{api:python:async_set_intersect}
\index{async\_set\_intersect!Python API}
Store the intersection of the specified set and the existing value for each
attribute.

%%% Generated below here
\paragraph{Behavior:}
\begin{itemize}[noitemsep]
This operation requires a pre-existing object in order to complete successfully.
If no object exists, the operation will fail with \code{NOTFOUND}.

\end{itemize}


\paragraph{Definition:}
\begin{pythoncode}
def async_set_intersect(self, spacename, key, attributes)
\end{pythoncode}

\paragraph{Parameters:}
\begin{itemize}[noitemsep]
\item \code{spacename}\\
The name of an existing space.

\item \code{key}\\
The key for the operation where \code{key} is a bytestring and \code{key\_sz}
specifies the number of bytes in \code{key}.

\item \code{attributes}\\
The attributes to modify and their respective values.

\end{itemize}

\paragraph{Returns:}
A Deferred object with a \code{wait} method that returns True if the operation
succeeded, or False if any of the provided predicates failed.  Raises an
exception on error.


\paragraph{See also:}  This is the asynchronous form of \code{set\_intersect}.

%%%%%%%%%%%%%%%%%%%% cond_set_intersect %%%%%%%%%%%%%%%%%%%%
\pagebreak
\subsubsection{\code{cond\_set\_intersect}}
\label{api:python:cond_set_intersect}
\index{cond\_set\_intersect!Python API}
\input{\topdir/client/fragments/cond_set_intersect}

\paragraph{Definition:}
\begin{pythoncode}
def cond_set_intersect(self, spacename, key, predicates, attributes)
\end{pythoncode}

\paragraph{Parameters:}
\begin{itemize}[noitemsep]
\item \code{spacename}\\
The name of an existing space.

\item \code{key}\\
The key for the operation where \code{key} is a bytestring and \code{key\_sz}
specifies the number of bytes in \code{key}.

\item \code{predicates}\\
The predicates to check against.  \code{checks} is an object with the individual
predicates as properties.

\item \code{attributes}\\
The attributes to modify and their respective values.

\end{itemize}

\paragraph{Returns:}
True if the operation succeeded, False if any of the provided predicates failed.
Raises an exception on error.


\pagebreak
\subsubsection{\code{async\_cond\_set\_intersect}}
\label{api:python:async_cond_set_intersect}
\index{async\_cond\_set\_intersect!Python API}
\input{\topdir/client/fragments/cond_set_intersect}

\paragraph{Definition:}
\begin{pythoncode}
def async_cond_set_intersect(self, spacename, key, predicates, attributes)
\end{pythoncode}

\paragraph{Parameters:}
\begin{itemize}[noitemsep]
\item \code{spacename}\\
The name of an existing space.

\item \code{key}\\
The key for the operation where \code{key} is a bytestring and \code{key\_sz}
specifies the number of bytes in \code{key}.

\item \code{predicates}\\
The predicates to check against.  \code{checks} is an object with the individual
predicates as properties.

\item \code{attributes}\\
The attributes to modify and their respective values.

\end{itemize}

\paragraph{Returns:}
A Deferred object with a \code{wait} method that returns True if the operation
succeeded, or False if any of the provided predicates failed.  Raises an
exception on error.


\paragraph{See also:}  This is the asynchronous form of \code{cond\_set\_intersect}.

%%%%%%%%%%%%%%%%%%%% group_set_intersect %%%%%%%%%%%%%%%%%%%%
\pagebreak
\subsubsection{\code{group\_set\_intersect}}
\label{api:python:group_set_intersect}
\index{group\_set\_intersect!Python API}
\input{\topdir/client/fragments/group_set_intersect}

\paragraph{Definition:}
\begin{pythoncode}
def group_set_intersect(self, spacename, predicates, attributes)
\end{pythoncode}

\paragraph{Parameters:}
\begin{itemize}[noitemsep]
\item \code{spacename}\\
The name of an existing space.

\item \code{predicates}\\
The predicates to check against.  \code{checks} is an object with the individual
predicates as properties.

\item \code{attributes}\\
The attributes to modify and their respective values.

\end{itemize}

\paragraph{Returns:}
A count of the number of objects, and a \code{client.Error} object indicating
the status of the operation.


\pagebreak
\subsubsection{\code{async\_group\_set\_intersect}}
\label{api:python:async_group_set_intersect}
\index{async\_group\_set\_intersect!Python API}
\input{\topdir/client/fragments/group_set_intersect}

\paragraph{Definition:}
\begin{pythoncode}
def async_group_set_intersect(self, spacename, predicates, attributes)
\end{pythoncode}

\paragraph{Parameters:}
\begin{itemize}[noitemsep]
\item \code{spacename}\\
The name of an existing space.

\item \code{predicates}\\
The predicates to check against.  \code{checks} is an object with the individual
predicates as properties.

\item \code{attributes}\\
The attributes to modify and their respective values.

\end{itemize}

\paragraph{Returns:}
This asynchronous operation returns a \code{Deferred} object with a
\code{waitForIt} method which blocks and returns a number indicating the number
of objects counted.

On error, this function will raise a \code{HyperDexClientException} describing
the error.


\paragraph{See also:}  This is the asynchronous form of \code{group\_set\_intersect}.

%%%%%%%%%%%%%%%%%%%% set_union %%%%%%%%%%%%%%%%%%%%
\pagebreak
\subsubsection{\code{set\_union}}
\label{api:python:set_union}
\index{set\_union!Python API}
Store the union of the specified set and the existing value for each attribute.

%%% Generated below here
\paragraph{Behavior:}
\begin{itemize}[noitemsep]
This operation requires a pre-existing object in order to complete successfully.
If no object exists, the operation will fail with \code{NOTFOUND}.

\end{itemize}


\paragraph{Definition:}
\begin{pythoncode}
def set_union(self, spacename, key, attributes)
\end{pythoncode}

\paragraph{Parameters:}
\begin{itemize}[noitemsep]
\item \code{spacename}\\
The name of an existing space.

\item \code{key}\\
The key for the operation where \code{key} is a bytestring and \code{key\_sz}
specifies the number of bytes in \code{key}.

\item \code{attributes}\\
The attributes to modify and their respective values.

\end{itemize}

\paragraph{Returns:}
True if the operation succeeded, False if any of the provided predicates failed.
Raises an exception on error.


\pagebreak
\subsubsection{\code{async\_set\_union}}
\label{api:python:async_set_union}
\index{async\_set\_union!Python API}
Store the union of the specified set and the existing value for each attribute.

%%% Generated below here
\paragraph{Behavior:}
\begin{itemize}[noitemsep]
This operation requires a pre-existing object in order to complete successfully.
If no object exists, the operation will fail with \code{NOTFOUND}.

\end{itemize}


\paragraph{Definition:}
\begin{pythoncode}
def async_set_union(self, spacename, key, attributes)
\end{pythoncode}

\paragraph{Parameters:}
\begin{itemize}[noitemsep]
\item \code{spacename}\\
The name of an existing space.

\item \code{key}\\
The key for the operation where \code{key} is a bytestring and \code{key\_sz}
specifies the number of bytes in \code{key}.

\item \code{attributes}\\
The attributes to modify and their respective values.

\end{itemize}

\paragraph{Returns:}
A Deferred object with a \code{wait} method that returns True if the operation
succeeded, or False if any of the provided predicates failed.  Raises an
exception on error.


\paragraph{See also:}  This is the asynchronous form of \code{set\_union}.

%%%%%%%%%%%%%%%%%%%% cond_set_union %%%%%%%%%%%%%%%%%%%%
\pagebreak
\subsubsection{\code{cond\_set\_union}}
\label{api:python:cond_set_union}
\index{cond\_set\_union!Python API}
Conditionally store the union of the specified set and the existing value for
each attribute.

%%% Generated below here
\paragraph{Behavior:}
\begin{itemize}[noitemsep]
This operation requires a pre-existing object in order to complete successfully.
If no object exists, the operation will fail with \code{NOTFOUND}.

This operation will succeed if and only if the predicates specified by
\code{checks} hold on the pre-existing object.  If any of the predicates are not
true for the existing object, then the operation will have no effect and fail
with \code{CMPFAIL}.

All checks are atomic with the write.  HyperDex guarantees that no other
operation will come between validating the checks, and writing the new version
of the object.

\end{itemize}


\paragraph{Definition:}
\begin{pythoncode}
def cond_set_union(self, spacename, key, predicates, attributes)
\end{pythoncode}

\paragraph{Parameters:}
\begin{itemize}[noitemsep]
\item \code{spacename}\\
The name of an existing space.

\item \code{key}\\
The key for the operation where \code{key} is a bytestring and \code{key\_sz}
specifies the number of bytes in \code{key}.

\item \code{predicates}\\
The predicates to check against.  \code{checks} is an object with the individual
predicates as properties.

\item \code{attributes}\\
The attributes to modify and their respective values.

\end{itemize}

\paragraph{Returns:}
True if the operation succeeded, False if any of the provided predicates failed.
Raises an exception on error.


\pagebreak
\subsubsection{\code{async\_cond\_set\_union}}
\label{api:python:async_cond_set_union}
\index{async\_cond\_set\_union!Python API}
Conditionally store the union of the specified set and the existing value for
each attribute.

%%% Generated below here
\paragraph{Behavior:}
\begin{itemize}[noitemsep]
This operation requires a pre-existing object in order to complete successfully.
If no object exists, the operation will fail with \code{NOTFOUND}.

This operation will succeed if and only if the predicates specified by
\code{checks} hold on the pre-existing object.  If any of the predicates are not
true for the existing object, then the operation will have no effect and fail
with \code{CMPFAIL}.

All checks are atomic with the write.  HyperDex guarantees that no other
operation will come between validating the checks, and writing the new version
of the object.

\end{itemize}


\paragraph{Definition:}
\begin{pythoncode}
def async_cond_set_union(self, spacename, key, predicates, attributes)
\end{pythoncode}

\paragraph{Parameters:}
\begin{itemize}[noitemsep]
\item \code{spacename}\\
The name of an existing space.

\item \code{key}\\
The key for the operation where \code{key} is a bytestring and \code{key\_sz}
specifies the number of bytes in \code{key}.

\item \code{predicates}\\
The predicates to check against.  \code{checks} is an object with the individual
predicates as properties.

\item \code{attributes}\\
The attributes to modify and their respective values.

\end{itemize}

\paragraph{Returns:}
A Deferred object with a \code{wait} method that returns True if the operation
succeeded, or False if any of the provided predicates failed.  Raises an
exception on error.


\paragraph{See also:}  This is the asynchronous form of \code{cond\_set\_union}.

%%%%%%%%%%%%%%%%%%%% group_set_union %%%%%%%%%%%%%%%%%%%%
\pagebreak
\subsubsection{\code{group\_set\_union}}
\label{api:python:group_set_union}
\index{group\_set\_union!Python API}
Store the union of the specified set and the existing value for each object in
\code{space} that matches \code{checks}.

This operation will only affect objects that match the provided \code{checks}.
Objects that do not match \code{checks} will be unaffected by the group call.
Each object that matches \code{checks} will be atomically updated with the check
on the object.  HyperDex guarantees that no object will be altered if the
\code{checks} do not pass at the time of the write.  Objects that are updated
concurrently with the group call may or may not be updated; however, regardless
of any other concurrent operations, the preceding guarantee will always hold.



\paragraph{Definition:}
\begin{pythoncode}
def group_set_union(self, spacename, predicates, attributes)
\end{pythoncode}

\paragraph{Parameters:}
\begin{itemize}[noitemsep]
\item \code{spacename}\\
The name of an existing space.

\item \code{predicates}\\
The predicates to check against.  \code{checks} is an object with the individual
predicates as properties.

\item \code{attributes}\\
The attributes to modify and their respective values.

\end{itemize}

\paragraph{Returns:}
A count of the number of objects, and a \code{client.Error} object indicating
the status of the operation.


\pagebreak
\subsubsection{\code{async\_group\_set\_union}}
\label{api:python:async_group_set_union}
\index{async\_group\_set\_union!Python API}
Store the union of the specified set and the existing value for each object in
\code{space} that matches \code{checks}.

This operation will only affect objects that match the provided \code{checks}.
Objects that do not match \code{checks} will be unaffected by the group call.
Each object that matches \code{checks} will be atomically updated with the check
on the object.  HyperDex guarantees that no object will be altered if the
\code{checks} do not pass at the time of the write.  Objects that are updated
concurrently with the group call may or may not be updated; however, regardless
of any other concurrent operations, the preceding guarantee will always hold.



\paragraph{Definition:}
\begin{pythoncode}
def async_group_set_union(self, spacename, predicates, attributes)
\end{pythoncode}

\paragraph{Parameters:}
\begin{itemize}[noitemsep]
\item \code{spacename}\\
The name of an existing space.

\item \code{predicates}\\
The predicates to check against.  \code{checks} is an object with the individual
predicates as properties.

\item \code{attributes}\\
The attributes to modify and their respective values.

\end{itemize}

\paragraph{Returns:}
This asynchronous operation returns a \code{Deferred} object with a
\code{waitForIt} method which blocks and returns a number indicating the number
of objects counted.

On error, this function will raise a \code{HyperDexClientException} describing
the error.


\paragraph{See also:}  This is the asynchronous form of \code{group\_set\_union}.

%%%%%%%%%%%%%%%%%%%% document_rename %%%%%%%%%%%%%%%%%%%%
\pagebreak
\subsubsection{\code{document\_rename}}
\label{api:python:document_rename}
\index{document\_rename!Python API}
Move a field within a document from one name to another.
This operation requires a pre-existing object in order to complete successfully.
If no object exists, the operation will fail with \code{NOTFOUND}.



\paragraph{Definition:}
\begin{pythoncode}
def document_rename(self, spacename, key, attributes)
\end{pythoncode}

\paragraph{Parameters:}
\begin{itemize}[noitemsep]
\item \code{spacename}\\
The name of an existing space.

\item \code{key}\\
The key for the operation where \code{key} is a bytestring and \code{key\_sz}
specifies the number of bytes in \code{key}.

\item \code{attributes}\\
The attributes to modify and their respective values.

\end{itemize}

\paragraph{Returns:}
True if the operation succeeded, False if any of the provided predicates failed.
Raises an exception on error.


\pagebreak
\subsubsection{\code{async\_document\_rename}}
\label{api:python:async_document_rename}
\index{async\_document\_rename!Python API}
Move a field within a document from one name to another.
This operation requires a pre-existing object in order to complete successfully.
If no object exists, the operation will fail with \code{NOTFOUND}.



\paragraph{Definition:}
\begin{pythoncode}
def async_document_rename(self, spacename, key, attributes)
\end{pythoncode}

\paragraph{Parameters:}
\begin{itemize}[noitemsep]
\item \code{spacename}\\
The name of an existing space.

\item \code{key}\\
The key for the operation where \code{key} is a bytestring and \code{key\_sz}
specifies the number of bytes in \code{key}.

\item \code{attributes}\\
The attributes to modify and their respective values.

\end{itemize}

\paragraph{Returns:}
A Deferred object with a \code{wait} method that returns True if the operation
succeeded, or False if any of the provided predicates failed.  Raises an
exception on error.


\paragraph{See also:}  This is the asynchronous form of \code{document\_rename}.

%%%%%%%%%%%%%%%%%%%% cond_document_rename %%%%%%%%%%%%%%%%%%%%
\pagebreak
\subsubsection{\code{cond\_document\_rename}}
\label{api:python:cond_document_rename}
\index{cond\_document\_rename!Python API}
Move a field within a document from one name to another if and only if the
\code{checks} hold on the object.
This operation requires a pre-existing object in order to complete successfully.
If no object exists, the operation will fail with \code{NOTFOUND}.


This operation will succeed if and only if the predicates specified by
\code{checks} hold on the pre-existing object.  If any of the predicates are not
true for the existing object, then the operation will have no effect and fail
with \code{CMPFAIL}.

All checks are atomic with the write.  HyperDex guarantees that no other
operation will come between validating the checks, and writing the new version
of the object.



\paragraph{Definition:}
\begin{pythoncode}
def cond_document_rename(self, spacename, key, predicates, attributes)
\end{pythoncode}

\paragraph{Parameters:}
\begin{itemize}[noitemsep]
\item \code{spacename}\\
The name of an existing space.

\item \code{key}\\
The key for the operation where \code{key} is a bytestring and \code{key\_sz}
specifies the number of bytes in \code{key}.

\item \code{predicates}\\
The predicates to check against.  \code{checks} is an object with the individual
predicates as properties.

\item \code{attributes}\\
The attributes to modify and their respective values.

\end{itemize}

\paragraph{Returns:}
True if the operation succeeded, False if any of the provided predicates failed.
Raises an exception on error.


\pagebreak
\subsubsection{\code{async\_cond\_document\_rename}}
\label{api:python:async_cond_document_rename}
\index{async\_cond\_document\_rename!Python API}
Move a field within a document from one name to another if and only if the
\code{checks} hold on the object.
This operation requires a pre-existing object in order to complete successfully.
If no object exists, the operation will fail with \code{NOTFOUND}.


This operation will succeed if and only if the predicates specified by
\code{checks} hold on the pre-existing object.  If any of the predicates are not
true for the existing object, then the operation will have no effect and fail
with \code{CMPFAIL}.

All checks are atomic with the write.  HyperDex guarantees that no other
operation will come between validating the checks, and writing the new version
of the object.



\paragraph{Definition:}
\begin{pythoncode}
def async_cond_document_rename(self, spacename, key, predicates, attributes)
\end{pythoncode}

\paragraph{Parameters:}
\begin{itemize}[noitemsep]
\item \code{spacename}\\
The name of an existing space.

\item \code{key}\\
The key for the operation where \code{key} is a bytestring and \code{key\_sz}
specifies the number of bytes in \code{key}.

\item \code{predicates}\\
The predicates to check against.  \code{checks} is an object with the individual
predicates as properties.

\item \code{attributes}\\
The attributes to modify and their respective values.

\end{itemize}

\paragraph{Returns:}
A Deferred object with a \code{wait} method that returns True if the operation
succeeded, or False if any of the provided predicates failed.  Raises an
exception on error.


\paragraph{See also:}  This is the asynchronous form of \code{cond\_document\_rename}.

%%%%%%%%%%%%%%%%%%%% group_document_rename %%%%%%%%%%%%%%%%%%%%
\pagebreak
\subsubsection{\code{group\_document\_rename}}
\label{api:python:group_document_rename}
\index{group\_document\_rename!Python API}
Move a field within a document from one name to another for each object in
\code{space} that matches \code{checks}.

This operation will only affect objects that match the provided \code{checks}.
Objects that do not match \code{checks} will be unaffected by the group call.
Each object that matches \code{checks} will be atomically updated with the check
on the object.  HyperDex guarantees that no object will be altered if the
\code{checks} do not pass at the time of the write.  Objects that are updated
concurrently with the group call may or may not be updated; however, regardless
of any other concurrent operations, the preceding guarantee will always hold.



\paragraph{Definition:}
\begin{pythoncode}
def group_document_rename(self, spacename, predicates, attributes)
\end{pythoncode}

\paragraph{Parameters:}
\begin{itemize}[noitemsep]
\item \code{spacename}\\
The name of an existing space.

\item \code{predicates}\\
The predicates to check against.  \code{checks} is an object with the individual
predicates as properties.

\item \code{attributes}\\
The attributes to modify and their respective values.

\end{itemize}

\paragraph{Returns:}
A count of the number of objects, and a \code{client.Error} object indicating
the status of the operation.


\pagebreak
\subsubsection{\code{async\_group\_document\_rename}}
\label{api:python:async_group_document_rename}
\index{async\_group\_document\_rename!Python API}
Move a field within a document from one name to another for each object in
\code{space} that matches \code{checks}.

This operation will only affect objects that match the provided \code{checks}.
Objects that do not match \code{checks} will be unaffected by the group call.
Each object that matches \code{checks} will be atomically updated with the check
on the object.  HyperDex guarantees that no object will be altered if the
\code{checks} do not pass at the time of the write.  Objects that are updated
concurrently with the group call may or may not be updated; however, regardless
of any other concurrent operations, the preceding guarantee will always hold.



\paragraph{Definition:}
\begin{pythoncode}
def async_group_document_rename(self, spacename, predicates, attributes)
\end{pythoncode}

\paragraph{Parameters:}
\begin{itemize}[noitemsep]
\item \code{spacename}\\
The name of an existing space.

\item \code{predicates}\\
The predicates to check against.  \code{checks} is an object with the individual
predicates as properties.

\item \code{attributes}\\
The attributes to modify and their respective values.

\end{itemize}

\paragraph{Returns:}
This asynchronous operation returns a \code{Deferred} object with a
\code{waitForIt} method which blocks and returns a number indicating the number
of objects counted.

On error, this function will raise a \code{HyperDexClientException} describing
the error.


\paragraph{See also:}  This is the asynchronous form of \code{group\_document\_rename}.

%%%%%%%%%%%%%%%%%%%% document_unset %%%%%%%%%%%%%%%%%%%%
\pagebreak
\subsubsection{\code{document\_unset}}
\label{api:python:document_unset}
\index{document\_unset!Python API}
Remove a field or object from a document.
This operation requires a pre-existing object in order to complete successfully.
If no object exists, the operation will fail with \code{NOTFOUND}.



\paragraph{Definition:}
\begin{pythoncode}
def document_unset(self, spacename, key, attributes)
\end{pythoncode}

\paragraph{Parameters:}
\begin{itemize}[noitemsep]
\item \code{spacename}\\
The name of an existing space.

\item \code{key}\\
The key for the operation where \code{key} is a bytestring and \code{key\_sz}
specifies the number of bytes in \code{key}.

\item \code{attributes}\\
The attributes to modify and their respective values.

\end{itemize}

\paragraph{Returns:}
True if the operation succeeded, False if any of the provided predicates failed.
Raises an exception on error.


\pagebreak
\subsubsection{\code{async\_document\_unset}}
\label{api:python:async_document_unset}
\index{async\_document\_unset!Python API}
Remove a field or object from a document.
This operation requires a pre-existing object in order to complete successfully.
If no object exists, the operation will fail with \code{NOTFOUND}.



\paragraph{Definition:}
\begin{pythoncode}
def async_document_unset(self, spacename, key, attributes)
\end{pythoncode}

\paragraph{Parameters:}
\begin{itemize}[noitemsep]
\item \code{spacename}\\
The name of an existing space.

\item \code{key}\\
The key for the operation where \code{key} is a bytestring and \code{key\_sz}
specifies the number of bytes in \code{key}.

\item \code{attributes}\\
The attributes to modify and their respective values.

\end{itemize}

\paragraph{Returns:}
A Deferred object with a \code{wait} method that returns True if the operation
succeeded, or False if any of the provided predicates failed.  Raises an
exception on error.


\paragraph{See also:}  This is the asynchronous form of \code{document\_unset}.

%%%%%%%%%%%%%%%%%%%% cond_document_unset %%%%%%%%%%%%%%%%%%%%
\pagebreak
\subsubsection{\code{cond\_document\_unset}}
\label{api:python:cond_document_unset}
\index{cond\_document\_unset!Python API}
Remove a field or object from a document if and only if the \code{checks} hold
on the object.
This operation requires a pre-existing object in order to complete successfully.
If no object exists, the operation will fail with \code{NOTFOUND}.


This operation will succeed if and only if the predicates specified by
\code{checks} hold on the pre-existing object.  If any of the predicates are not
true for the existing object, then the operation will have no effect and fail
with \code{CMPFAIL}.

All checks are atomic with the write.  HyperDex guarantees that no other
operation will come between validating the checks, and writing the new version
of the object.



\paragraph{Definition:}
\begin{pythoncode}
def cond_document_unset(self, spacename, key, predicates, attributes)
\end{pythoncode}

\paragraph{Parameters:}
\begin{itemize}[noitemsep]
\item \code{spacename}\\
The name of an existing space.

\item \code{key}\\
The key for the operation where \code{key} is a bytestring and \code{key\_sz}
specifies the number of bytes in \code{key}.

\item \code{predicates}\\
The predicates to check against.  \code{checks} is an object with the individual
predicates as properties.

\item \code{attributes}\\
The attributes to modify and their respective values.

\end{itemize}

\paragraph{Returns:}
True if the operation succeeded, False if any of the provided predicates failed.
Raises an exception on error.


\pagebreak
\subsubsection{\code{async\_cond\_document\_unset}}
\label{api:python:async_cond_document_unset}
\index{async\_cond\_document\_unset!Python API}
Remove a field or object from a document if and only if the \code{checks} hold
on the object.
This operation requires a pre-existing object in order to complete successfully.
If no object exists, the operation will fail with \code{NOTFOUND}.


This operation will succeed if and only if the predicates specified by
\code{checks} hold on the pre-existing object.  If any of the predicates are not
true for the existing object, then the operation will have no effect and fail
with \code{CMPFAIL}.

All checks are atomic with the write.  HyperDex guarantees that no other
operation will come between validating the checks, and writing the new version
of the object.



\paragraph{Definition:}
\begin{pythoncode}
def async_cond_document_unset(self, spacename, key, predicates, attributes)
\end{pythoncode}

\paragraph{Parameters:}
\begin{itemize}[noitemsep]
\item \code{spacename}\\
The name of an existing space.

\item \code{key}\\
The key for the operation where \code{key} is a bytestring and \code{key\_sz}
specifies the number of bytes in \code{key}.

\item \code{predicates}\\
The predicates to check against.  \code{checks} is an object with the individual
predicates as properties.

\item \code{attributes}\\
The attributes to modify and their respective values.

\end{itemize}

\paragraph{Returns:}
A Deferred object with a \code{wait} method that returns True if the operation
succeeded, or False if any of the provided predicates failed.  Raises an
exception on error.


\paragraph{See also:}  This is the asynchronous form of \code{cond\_document\_unset}.

%%%%%%%%%%%%%%%%%%%% group_document_unset %%%%%%%%%%%%%%%%%%%%
\pagebreak
\subsubsection{\code{group\_document\_unset}}
\label{api:python:group_document_unset}
\index{group\_document\_unset!Python API}
Remove a field or object from a document for each object in \code{space} that
matches \code{checks}.

This operation will only affect objects that match the provided \code{checks}.
Objects that do not match \code{checks} will be unaffected by the group call.
Each object that matches \code{checks} will be atomically updated with the check
on the object.  HyperDex guarantees that no object will be altered if the
\code{checks} do not pass at the time of the write.  Objects that are updated
concurrently with the group call may or may not be updated; however, regardless
of any other concurrent operations, the preceding guarantee will always hold.



\paragraph{Definition:}
\begin{pythoncode}
def group_document_unset(self, spacename, predicates, attributes)
\end{pythoncode}

\paragraph{Parameters:}
\begin{itemize}[noitemsep]
\item \code{spacename}\\
The name of an existing space.

\item \code{predicates}\\
The predicates to check against.  \code{checks} is an object with the individual
predicates as properties.

\item \code{attributes}\\
The attributes to modify and their respective values.

\end{itemize}

\paragraph{Returns:}
A count of the number of objects, and a \code{client.Error} object indicating
the status of the operation.


\pagebreak
\subsubsection{\code{async\_group\_document\_unset}}
\label{api:python:async_group_document_unset}
\index{async\_group\_document\_unset!Python API}
Remove a field or object from a document for each object in \code{space} that
matches \code{checks}.

This operation will only affect objects that match the provided \code{checks}.
Objects that do not match \code{checks} will be unaffected by the group call.
Each object that matches \code{checks} will be atomically updated with the check
on the object.  HyperDex guarantees that no object will be altered if the
\code{checks} do not pass at the time of the write.  Objects that are updated
concurrently with the group call may or may not be updated; however, regardless
of any other concurrent operations, the preceding guarantee will always hold.



\paragraph{Definition:}
\begin{pythoncode}
def async_group_document_unset(self, spacename, predicates, attributes)
\end{pythoncode}

\paragraph{Parameters:}
\begin{itemize}[noitemsep]
\item \code{spacename}\\
The name of an existing space.

\item \code{predicates}\\
The predicates to check against.  \code{checks} is an object with the individual
predicates as properties.

\item \code{attributes}\\
The attributes to modify and their respective values.

\end{itemize}

\paragraph{Returns:}
This asynchronous operation returns a \code{Deferred} object with a
\code{waitForIt} method which blocks and returns a number indicating the number
of objects counted.

On error, this function will raise a \code{HyperDexClientException} describing
the error.


\paragraph{See also:}  This is the asynchronous form of \code{group\_document\_unset}.

%%%%%%%%%%%%%%%%%%%% map_add %%%%%%%%%%%%%%%%%%%%
\pagebreak
\subsubsection{\code{map\_add}}
\label{api:python:map_add}
\index{map\_add!Python API}
Insert a key-value pair into the map specified by each map-attribute.

%%% Generated below here
\paragraph{Behavior:}
\begin{itemize}[noitemsep]
This operation requires a pre-existing object in order to complete successfully.
If no object exists, the operation will fail with \code{NOTFOUND}.

\end{itemize}


\paragraph{Definition:}
\begin{pythoncode}
def map_add(self, spacename, key, mapattributes)
\end{pythoncode}

\paragraph{Parameters:}
\begin{itemize}[noitemsep]
\item \code{spacename}\\
The name of an existing space.

\item \code{key}\\
The key for the operation where \code{key} is a bytestring and \code{key\_sz}
specifies the number of bytes in \code{key}.

\item \code{mapattributes}\\
A dictionary of map attributes to modify and their respective key-value pairs.

\end{itemize}

\paragraph{Returns:}
True if the operation succeeded, False if any of the provided predicates failed.
Raises an exception on error.


\pagebreak
\subsubsection{\code{async\_map\_add}}
\label{api:python:async_map_add}
\index{async\_map\_add!Python API}
Insert a key-value pair into the map specified by each map-attribute.

%%% Generated below here
\paragraph{Behavior:}
\begin{itemize}[noitemsep]
This operation requires a pre-existing object in order to complete successfully.
If no object exists, the operation will fail with \code{NOTFOUND}.

\end{itemize}


\paragraph{Definition:}
\begin{pythoncode}
def async_map_add(self, spacename, key, mapattributes)
\end{pythoncode}

\paragraph{Parameters:}
\begin{itemize}[noitemsep]
\item \code{spacename}\\
The name of an existing space.

\item \code{key}\\
The key for the operation where \code{key} is a bytestring and \code{key\_sz}
specifies the number of bytes in \code{key}.

\item \code{mapattributes}\\
A dictionary of map attributes to modify and their respective key-value pairs.

\end{itemize}

\paragraph{Returns:}
A Deferred object with a \code{wait} method that returns True if the operation
succeeded, or False if any of the provided predicates failed.  Raises an
exception on error.


\paragraph{See also:}  This is the asynchronous form of \code{map\_add}.

%%%%%%%%%%%%%%%%%%%% cond_map_add %%%%%%%%%%%%%%%%%%%%
\pagebreak
\subsubsection{\code{cond\_map\_add}}
\label{api:python:cond_map_add}
\index{cond\_map\_add!Python API}
Conditionally insert a key-value pair into the map specified by each
map-attribute.

%%% Generated below here
\paragraph{Behavior:}
\begin{itemize}[noitemsep]
This operation requires a pre-existing object in order to complete successfully.
If no object exists, the operation will fail with \code{NOTFOUND}.

This operation will succeed if and only if the predicates specified by
\code{checks} hold on the pre-existing object.  If any of the predicates are not
true for the existing object, then the operation will have no effect and fail
with \code{CMPFAIL}.

All checks are atomic with the write.  HyperDex guarantees that no other
operation will come between validating the checks, and writing the new version
of the object.

\end{itemize}


\paragraph{Definition:}
\begin{pythoncode}
def cond_map_add(self, spacename, key, predicates, mapattributes)
\end{pythoncode}

\paragraph{Parameters:}
\begin{itemize}[noitemsep]
\item \code{spacename}\\
The name of an existing space.

\item \code{key}\\
The key for the operation where \code{key} is a bytestring and \code{key\_sz}
specifies the number of bytes in \code{key}.

\item \code{predicates}\\
The predicates to check against.  \code{checks} is an object with the individual
predicates as properties.

\item \code{mapattributes}\\
A dictionary of map attributes to modify and their respective key-value pairs.

\end{itemize}

\paragraph{Returns:}
True if the operation succeeded, False if any of the provided predicates failed.
Raises an exception on error.


\pagebreak
\subsubsection{\code{async\_cond\_map\_add}}
\label{api:python:async_cond_map_add}
\index{async\_cond\_map\_add!Python API}
Conditionally insert a key-value pair into the map specified by each
map-attribute.

%%% Generated below here
\paragraph{Behavior:}
\begin{itemize}[noitemsep]
This operation requires a pre-existing object in order to complete successfully.
If no object exists, the operation will fail with \code{NOTFOUND}.

This operation will succeed if and only if the predicates specified by
\code{checks} hold on the pre-existing object.  If any of the predicates are not
true for the existing object, then the operation will have no effect and fail
with \code{CMPFAIL}.

All checks are atomic with the write.  HyperDex guarantees that no other
operation will come between validating the checks, and writing the new version
of the object.

\end{itemize}


\paragraph{Definition:}
\begin{pythoncode}
def async_cond_map_add(self, spacename, key, predicates, mapattributes)
\end{pythoncode}

\paragraph{Parameters:}
\begin{itemize}[noitemsep]
\item \code{spacename}\\
The name of an existing space.

\item \code{key}\\
The key for the operation where \code{key} is a bytestring and \code{key\_sz}
specifies the number of bytes in \code{key}.

\item \code{predicates}\\
The predicates to check against.  \code{checks} is an object with the individual
predicates as properties.

\item \code{mapattributes}\\
A dictionary of map attributes to modify and their respective key-value pairs.

\end{itemize}

\paragraph{Returns:}
A Deferred object with a \code{wait} method that returns True if the operation
succeeded, or False if any of the provided predicates failed.  Raises an
exception on error.


\paragraph{See also:}  This is the asynchronous form of \code{cond\_map\_add}.

%%%%%%%%%%%%%%%%%%%% group_map_add %%%%%%%%%%%%%%%%%%%%
\pagebreak
\subsubsection{\code{group\_map\_add}}
\label{api:python:group_map_add}
\index{group\_map\_add!Python API}
Insert a key-value pair into the map specified by each map-attribute for each
object in \code{space} that matches \code{checks}.

This operation will only affect objects that match the provided \code{checks}.
Objects that do not match \code{checks} will be unaffected by the group call.
Each object that matches \code{checks} will be atomically updated with the check
on the object.  HyperDex guarantees that no object will be altered if the
\code{checks} do not pass at the time of the write.  Objects that are updated
concurrently with the group call may or may not be updated; however, regardless
of any other concurrent operations, the preceding guarantee will always hold.



\paragraph{Definition:}
\begin{pythoncode}
def group_map_add(self, spacename, predicates, mapattributes)
\end{pythoncode}

\paragraph{Parameters:}
\begin{itemize}[noitemsep]
\item \code{spacename}\\
The name of an existing space.

\item \code{predicates}\\
The predicates to check against.  \code{checks} is an object with the individual
predicates as properties.

\item \code{mapattributes}\\
A dictionary of map attributes to modify and their respective key-value pairs.

\end{itemize}

\paragraph{Returns:}
A count of the number of objects, and a \code{client.Error} object indicating
the status of the operation.


\pagebreak
\subsubsection{\code{async\_group\_map\_add}}
\label{api:python:async_group_map_add}
\index{async\_group\_map\_add!Python API}
Insert a key-value pair into the map specified by each map-attribute for each
object in \code{space} that matches \code{checks}.

This operation will only affect objects that match the provided \code{checks}.
Objects that do not match \code{checks} will be unaffected by the group call.
Each object that matches \code{checks} will be atomically updated with the check
on the object.  HyperDex guarantees that no object will be altered if the
\code{checks} do not pass at the time of the write.  Objects that are updated
concurrently with the group call may or may not be updated; however, regardless
of any other concurrent operations, the preceding guarantee will always hold.



\paragraph{Definition:}
\begin{pythoncode}
def async_group_map_add(self, spacename, predicates, mapattributes)
\end{pythoncode}

\paragraph{Parameters:}
\begin{itemize}[noitemsep]
\item \code{spacename}\\
The name of an existing space.

\item \code{predicates}\\
The predicates to check against.  \code{checks} is an object with the individual
predicates as properties.

\item \code{mapattributes}\\
A dictionary of map attributes to modify and their respective key-value pairs.

\end{itemize}

\paragraph{Returns:}
This asynchronous operation returns a \code{Deferred} object with a
\code{waitForIt} method which blocks and returns a number indicating the number
of objects counted.

On error, this function will raise a \code{HyperDexClientException} describing
the error.


\paragraph{See also:}  This is the asynchronous form of \code{group\_map\_add}.

%%%%%%%%%%%%%%%%%%%% map_remove %%%%%%%%%%%%%%%%%%%%
\pagebreak
\subsubsection{\code{map\_remove}}
\label{api:python:map_remove}
\index{map\_remove!Python API}
Remove a key-value pair from the map specified by each attribute.  If there is
no pair with the specified key within the map, this operation will do nothing.
This operation requires a pre-existing object in order to complete successfully.
If no object exists, the operation will fail with \code{NOTFOUND}.



\paragraph{Definition:}
\begin{pythoncode}
def map_remove(self, spacename, key, attributes)
\end{pythoncode}

\paragraph{Parameters:}
\begin{itemize}[noitemsep]
\item \code{spacename}\\
The name of an existing space.

\item \code{key}\\
The key for the operation where \code{key} is a bytestring and \code{key\_sz}
specifies the number of bytes in \code{key}.

\item \code{attributes}\\
The attributes to modify and their respective values.

\end{itemize}

\paragraph{Returns:}
True if the operation succeeded, False if any of the provided predicates failed.
Raises an exception on error.


\pagebreak
\subsubsection{\code{async\_map\_remove}}
\label{api:python:async_map_remove}
\index{async\_map\_remove!Python API}
Remove a key-value pair from the map specified by each attribute.  If there is
no pair with the specified key within the map, this operation will do nothing.
This operation requires a pre-existing object in order to complete successfully.
If no object exists, the operation will fail with \code{NOTFOUND}.



\paragraph{Definition:}
\begin{pythoncode}
def async_map_remove(self, spacename, key, attributes)
\end{pythoncode}

\paragraph{Parameters:}
\begin{itemize}[noitemsep]
\item \code{spacename}\\
The name of an existing space.

\item \code{key}\\
The key for the operation where \code{key} is a bytestring and \code{key\_sz}
specifies the number of bytes in \code{key}.

\item \code{attributes}\\
The attributes to modify and their respective values.

\end{itemize}

\paragraph{Returns:}
A Deferred object with a \code{wait} method that returns True if the operation
succeeded, or False if any of the provided predicates failed.  Raises an
exception on error.


\paragraph{See also:}  This is the asynchronous form of \code{map\_remove}.

%%%%%%%%%%%%%%%%%%%% cond_map_remove %%%%%%%%%%%%%%%%%%%%
\pagebreak
\subsubsection{\code{cond\_map\_remove}}
\label{api:python:cond_map_remove}
\index{cond\_map\_remove!Python API}
Remove a key-value pair from the map specified by each attribute if and only if
\code{checks} hold on the object.  If there is no pair with the specified key
within the map, this operation will do nothing.
This operation requires a pre-existing object in order to complete successfully.
If no object exists, the operation will fail with \code{NOTFOUND}.


This operation will succeed if and only if the predicates specified by
\code{checks} hold on the pre-existing object.  If any of the predicates are not
true for the existing object, then the operation will have no effect and fail
with \code{CMPFAIL}.

All checks are atomic with the write.  HyperDex guarantees that no other
operation will come between validating the checks, and writing the new version
of the object.



\paragraph{Definition:}
\begin{pythoncode}
def cond_map_remove(self, spacename, key, predicates, attributes)
\end{pythoncode}

\paragraph{Parameters:}
\begin{itemize}[noitemsep]
\item \code{spacename}\\
The name of an existing space.

\item \code{key}\\
The key for the operation where \code{key} is a bytestring and \code{key\_sz}
specifies the number of bytes in \code{key}.

\item \code{predicates}\\
The predicates to check against.  \code{checks} is an object with the individual
predicates as properties.

\item \code{attributes}\\
The attributes to modify and their respective values.

\end{itemize}

\paragraph{Returns:}
True if the operation succeeded, False if any of the provided predicates failed.
Raises an exception on error.


\pagebreak
\subsubsection{\code{async\_cond\_map\_remove}}
\label{api:python:async_cond_map_remove}
\index{async\_cond\_map\_remove!Python API}
Remove a key-value pair from the map specified by each attribute if and only if
\code{checks} hold on the object.  If there is no pair with the specified key
within the map, this operation will do nothing.
This operation requires a pre-existing object in order to complete successfully.
If no object exists, the operation will fail with \code{NOTFOUND}.


This operation will succeed if and only if the predicates specified by
\code{checks} hold on the pre-existing object.  If any of the predicates are not
true for the existing object, then the operation will have no effect and fail
with \code{CMPFAIL}.

All checks are atomic with the write.  HyperDex guarantees that no other
operation will come between validating the checks, and writing the new version
of the object.



\paragraph{Definition:}
\begin{pythoncode}
def async_cond_map_remove(self, spacename, key, predicates, attributes)
\end{pythoncode}

\paragraph{Parameters:}
\begin{itemize}[noitemsep]
\item \code{spacename}\\
The name of an existing space.

\item \code{key}\\
The key for the operation where \code{key} is a bytestring and \code{key\_sz}
specifies the number of bytes in \code{key}.

\item \code{predicates}\\
The predicates to check against.  \code{checks} is an object with the individual
predicates as properties.

\item \code{attributes}\\
The attributes to modify and their respective values.

\end{itemize}

\paragraph{Returns:}
A Deferred object with a \code{wait} method that returns True if the operation
succeeded, or False if any of the provided predicates failed.  Raises an
exception on error.


\paragraph{See also:}  This is the asynchronous form of \code{cond\_map\_remove}.

%%%%%%%%%%%%%%%%%%%% group_map_remove %%%%%%%%%%%%%%%%%%%%
\pagebreak
\subsubsection{\code{group\_map\_remove}}
\label{api:python:group_map_remove}
\index{group\_map\_remove!Python API}
Remove a key-value pair from the map specified by each attribute for each object
in \code{space} that matches \code{checks}.  If there is no pair with the
specified key within the map, this operation will do nothing.

This operation will only affect objects that match the provided \code{checks}.
Objects that do not match \code{checks} will be unaffected by the group call.
Each object that matches \code{checks} will be atomically updated with the check
on the object.  HyperDex guarantees that no object will be altered if the
\code{checks} do not pass at the time of the write.  Objects that are updated
concurrently with the group call may or may not be updated; however, regardless
of any other concurrent operations, the preceding guarantee will always hold.



\paragraph{Definition:}
\begin{pythoncode}
def group_map_remove(self, spacename, predicates, attributes)
\end{pythoncode}

\paragraph{Parameters:}
\begin{itemize}[noitemsep]
\item \code{spacename}\\
The name of an existing space.

\item \code{predicates}\\
The predicates to check against.  \code{checks} is an object with the individual
predicates as properties.

\item \code{attributes}\\
The attributes to modify and their respective values.

\end{itemize}

\paragraph{Returns:}
A count of the number of objects, and a \code{client.Error} object indicating
the status of the operation.


\pagebreak
\subsubsection{\code{async\_group\_map\_remove}}
\label{api:python:async_group_map_remove}
\index{async\_group\_map\_remove!Python API}
Remove a key-value pair from the map specified by each attribute for each object
in \code{space} that matches \code{checks}.  If there is no pair with the
specified key within the map, this operation will do nothing.

This operation will only affect objects that match the provided \code{checks}.
Objects that do not match \code{checks} will be unaffected by the group call.
Each object that matches \code{checks} will be atomically updated with the check
on the object.  HyperDex guarantees that no object will be altered if the
\code{checks} do not pass at the time of the write.  Objects that are updated
concurrently with the group call may or may not be updated; however, regardless
of any other concurrent operations, the preceding guarantee will always hold.



\paragraph{Definition:}
\begin{pythoncode}
def async_group_map_remove(self, spacename, predicates, attributes)
\end{pythoncode}

\paragraph{Parameters:}
\begin{itemize}[noitemsep]
\item \code{spacename}\\
The name of an existing space.

\item \code{predicates}\\
The predicates to check against.  \code{checks} is an object with the individual
predicates as properties.

\item \code{attributes}\\
The attributes to modify and their respective values.

\end{itemize}

\paragraph{Returns:}
This asynchronous operation returns a \code{Deferred} object with a
\code{waitForIt} method which blocks and returns a number indicating the number
of objects counted.

On error, this function will raise a \code{HyperDexClientException} describing
the error.


\paragraph{See also:}  This is the asynchronous form of \code{group\_map\_remove}.

%%%%%%%%%%%%%%%%%%%% map_atomic_add %%%%%%%%%%%%%%%%%%%%
\pagebreak
\subsubsection{\code{map\_atomic\_add}}
\label{api:python:map_atomic_add}
\index{map\_atomic\_add!Python API}
Add the specified number to the value of a key-value pair within each map.

%%% Generated below here
\paragraph{Behavior:}
\begin{itemize}[noitemsep]
This operation requires a pre-existing object in order to complete successfully.
If no object exists, the operation will fail with \code{NOTFOUND}.

\item This operation mutates the value of a key-value pair in a map.  This call
    is similar to the equivalent call without the \code{map\_} prefix, but
    operates on the value of a pair in a map, instead of on an attribute's
    value.  If there is no pair with the specified map key, a new pair will be
    created and initialized to its default value.  If this is undesirable, it
    may be avoided by using a conditional operation that requires that the map
    contain the key in question.

\end{itemize}


\paragraph{Definition:}
\begin{pythoncode}
def map_atomic_add(self, spacename, key, mapattributes)
\end{pythoncode}

\paragraph{Parameters:}
\begin{itemize}[noitemsep]
\item \code{spacename}\\
The name of an existing space.

\item \code{key}\\
The key for the operation where \code{key} is a bytestring and \code{key\_sz}
specifies the number of bytes in \code{key}.

\item \code{mapattributes}\\
A dictionary of map attributes to modify and their respective key-value pairs.

\end{itemize}

\paragraph{Returns:}
True if the operation succeeded, False if any of the provided predicates failed.
Raises an exception on error.


\pagebreak
\subsubsection{\code{async\_map\_atomic\_add}}
\label{api:python:async_map_atomic_add}
\index{async\_map\_atomic\_add!Python API}
Add the specified number to the value of a key-value pair within each map.

%%% Generated below here
\paragraph{Behavior:}
\begin{itemize}[noitemsep]
This operation requires a pre-existing object in order to complete successfully.
If no object exists, the operation will fail with \code{NOTFOUND}.

\item This operation mutates the value of a key-value pair in a map.  This call
    is similar to the equivalent call without the \code{map\_} prefix, but
    operates on the value of a pair in a map, instead of on an attribute's
    value.  If there is no pair with the specified map key, a new pair will be
    created and initialized to its default value.  If this is undesirable, it
    may be avoided by using a conditional operation that requires that the map
    contain the key in question.

\end{itemize}


\paragraph{Definition:}
\begin{pythoncode}
def async_map_atomic_add(self, spacename, key, mapattributes)
\end{pythoncode}

\paragraph{Parameters:}
\begin{itemize}[noitemsep]
\item \code{spacename}\\
The name of an existing space.

\item \code{key}\\
The key for the operation where \code{key} is a bytestring and \code{key\_sz}
specifies the number of bytes in \code{key}.

\item \code{mapattributes}\\
A dictionary of map attributes to modify and their respective key-value pairs.

\end{itemize}

\paragraph{Returns:}
A Deferred object with a \code{wait} method that returns True if the operation
succeeded, or False if any of the provided predicates failed.  Raises an
exception on error.


\paragraph{See also:}  This is the asynchronous form of \code{map\_atomic\_add}.

%%%%%%%%%%%%%%%%%%%% cond_map_atomic_add %%%%%%%%%%%%%%%%%%%%
\pagebreak
\subsubsection{\code{cond\_map\_atomic\_add}}
\label{api:python:cond_map_atomic_add}
\index{cond\_map\_atomic\_add!Python API}
Conditionally add the specified number to the value of a key-value pair within
each map.

%%% Generated below here
\paragraph{Behavior:}
\begin{itemize}[noitemsep]
This operation requires a pre-existing object in order to complete successfully.
If no object exists, the operation will fail with \code{NOTFOUND}.

This operation will succeed if and only if the predicates specified by
\code{checks} hold on the pre-existing object.  If any of the predicates are not
true for the existing object, then the operation will have no effect and fail
with \code{CMPFAIL}.

All checks are atomic with the write.  HyperDex guarantees that no other
operation will come between validating the checks, and writing the new version
of the object.

\item This operation mutates the value of a key-value pair in a map.  This call
    is similar to the equivalent call without the \code{map\_} prefix, but
    operates on the value of a pair in a map, instead of on an attribute's
    value.  If there is no pair with the specified map key, a new pair will be
    created and initialized to its default value.  If this is undesirable, it
    may be avoided by using a conditional operation that requires that the map
    contain the key in question.

\end{itemize}


\paragraph{Definition:}
\begin{pythoncode}
def cond_map_atomic_add(self, spacename, key, predicates, mapattributes)
\end{pythoncode}

\paragraph{Parameters:}
\begin{itemize}[noitemsep]
\item \code{spacename}\\
The name of an existing space.

\item \code{key}\\
The key for the operation where \code{key} is a bytestring and \code{key\_sz}
specifies the number of bytes in \code{key}.

\item \code{predicates}\\
The predicates to check against.  \code{checks} is an object with the individual
predicates as properties.

\item \code{mapattributes}\\
A dictionary of map attributes to modify and their respective key-value pairs.

\end{itemize}

\paragraph{Returns:}
True if the operation succeeded, False if any of the provided predicates failed.
Raises an exception on error.


\pagebreak
\subsubsection{\code{async\_cond\_map\_atomic\_add}}
\label{api:python:async_cond_map_atomic_add}
\index{async\_cond\_map\_atomic\_add!Python API}
Conditionally add the specified number to the value of a key-value pair within
each map.

%%% Generated below here
\paragraph{Behavior:}
\begin{itemize}[noitemsep]
This operation requires a pre-existing object in order to complete successfully.
If no object exists, the operation will fail with \code{NOTFOUND}.

This operation will succeed if and only if the predicates specified by
\code{checks} hold on the pre-existing object.  If any of the predicates are not
true for the existing object, then the operation will have no effect and fail
with \code{CMPFAIL}.

All checks are atomic with the write.  HyperDex guarantees that no other
operation will come between validating the checks, and writing the new version
of the object.

\item This operation mutates the value of a key-value pair in a map.  This call
    is similar to the equivalent call without the \code{map\_} prefix, but
    operates on the value of a pair in a map, instead of on an attribute's
    value.  If there is no pair with the specified map key, a new pair will be
    created and initialized to its default value.  If this is undesirable, it
    may be avoided by using a conditional operation that requires that the map
    contain the key in question.

\end{itemize}


\paragraph{Definition:}
\begin{pythoncode}
def async_cond_map_atomic_add(self, spacename, key, predicates, mapattributes)
\end{pythoncode}

\paragraph{Parameters:}
\begin{itemize}[noitemsep]
\item \code{spacename}\\
The name of an existing space.

\item \code{key}\\
The key for the operation where \code{key} is a bytestring and \code{key\_sz}
specifies the number of bytes in \code{key}.

\item \code{predicates}\\
The predicates to check against.  \code{checks} is an object with the individual
predicates as properties.

\item \code{mapattributes}\\
A dictionary of map attributes to modify and their respective key-value pairs.

\end{itemize}

\paragraph{Returns:}
A Deferred object with a \code{wait} method that returns True if the operation
succeeded, or False if any of the provided predicates failed.  Raises an
exception on error.


\paragraph{See also:}  This is the asynchronous form of \code{cond\_map\_atomic\_add}.

%%%%%%%%%%%%%%%%%%%% group_map_atomic_add %%%%%%%%%%%%%%%%%%%%
\pagebreak
\subsubsection{\code{group\_map\_atomic\_add}}
\label{api:python:group_map_atomic_add}
\index{group\_map\_atomic\_add!Python API}
Add the specified number to the value of a key-value pair within each map for
each object in \code{space} that matches \code{checks}.

This operation will only affect objects that match the provided \code{checks}.
Objects that do not match \code{checks} will be unaffected by the group call.
Each object that matches \code{checks} will be atomically updated with the check
on the object.  HyperDex guarantees that no object will be altered if the
\code{checks} do not pass at the time of the write.  Objects that are updated
concurrently with the group call may or may not be updated; however, regardless
of any other concurrent operations, the preceding guarantee will always hold.



\paragraph{Definition:}
\begin{pythoncode}
def group_map_atomic_add(self, spacename, predicates, mapattributes)
\end{pythoncode}

\paragraph{Parameters:}
\begin{itemize}[noitemsep]
\item \code{spacename}\\
The name of an existing space.

\item \code{predicates}\\
The predicates to check against.  \code{checks} is an object with the individual
predicates as properties.

\item \code{mapattributes}\\
A dictionary of map attributes to modify and their respective key-value pairs.

\end{itemize}

\paragraph{Returns:}
A count of the number of objects, and a \code{client.Error} object indicating
the status of the operation.


\pagebreak
\subsubsection{\code{async\_group\_map\_atomic\_add}}
\label{api:python:async_group_map_atomic_add}
\index{async\_group\_map\_atomic\_add!Python API}
Add the specified number to the value of a key-value pair within each map for
each object in \code{space} that matches \code{checks}.

This operation will only affect objects that match the provided \code{checks}.
Objects that do not match \code{checks} will be unaffected by the group call.
Each object that matches \code{checks} will be atomically updated with the check
on the object.  HyperDex guarantees that no object will be altered if the
\code{checks} do not pass at the time of the write.  Objects that are updated
concurrently with the group call may or may not be updated; however, regardless
of any other concurrent operations, the preceding guarantee will always hold.



\paragraph{Definition:}
\begin{pythoncode}
def async_group_map_atomic_add(self, spacename, predicates, mapattributes)
\end{pythoncode}

\paragraph{Parameters:}
\begin{itemize}[noitemsep]
\item \code{spacename}\\
The name of an existing space.

\item \code{predicates}\\
The predicates to check against.  \code{checks} is an object with the individual
predicates as properties.

\item \code{mapattributes}\\
A dictionary of map attributes to modify and their respective key-value pairs.

\end{itemize}

\paragraph{Returns:}
This asynchronous operation returns a \code{Deferred} object with a
\code{waitForIt} method which blocks and returns a number indicating the number
of objects counted.

On error, this function will raise a \code{HyperDexClientException} describing
the error.


\paragraph{See also:}  This is the asynchronous form of \code{group\_map\_atomic\_add}.

%%%%%%%%%%%%%%%%%%%% map_atomic_sub %%%%%%%%%%%%%%%%%%%%
\pagebreak
\subsubsection{\code{map\_atomic\_sub}}
\label{api:python:map_atomic_sub}
\index{map\_atomic\_sub!Python API}
Subtract the specified number from the value of a key-value pair within each
map.
This operation requires a pre-existing object in order to complete successfully.
If no object exists, the operation will fail with \code{NOTFOUND}.



\paragraph{Definition:}
\begin{pythoncode}
def map_atomic_sub(self, spacename, key, mapattributes)
\end{pythoncode}

\paragraph{Parameters:}
\begin{itemize}[noitemsep]
\item \code{spacename}\\
The name of an existing space.

\item \code{key}\\
The key for the operation where \code{key} is a bytestring and \code{key\_sz}
specifies the number of bytes in \code{key}.

\item \code{mapattributes}\\
A dictionary of map attributes to modify and their respective key-value pairs.

\end{itemize}

\paragraph{Returns:}
True if the operation succeeded, False if any of the provided predicates failed.
Raises an exception on error.


\pagebreak
\subsubsection{\code{async\_map\_atomic\_sub}}
\label{api:python:async_map_atomic_sub}
\index{async\_map\_atomic\_sub!Python API}
Subtract the specified number from the value of a key-value pair within each
map.
This operation requires a pre-existing object in order to complete successfully.
If no object exists, the operation will fail with \code{NOTFOUND}.



\paragraph{Definition:}
\begin{pythoncode}
def async_map_atomic_sub(self, spacename, key, mapattributes)
\end{pythoncode}

\paragraph{Parameters:}
\begin{itemize}[noitemsep]
\item \code{spacename}\\
The name of an existing space.

\item \code{key}\\
The key for the operation where \code{key} is a bytestring and \code{key\_sz}
specifies the number of bytes in \code{key}.

\item \code{mapattributes}\\
A dictionary of map attributes to modify and their respective key-value pairs.

\end{itemize}

\paragraph{Returns:}
A Deferred object with a \code{wait} method that returns True if the operation
succeeded, or False if any of the provided predicates failed.  Raises an
exception on error.


\paragraph{See also:}  This is the asynchronous form of \code{map\_atomic\_sub}.

%%%%%%%%%%%%%%%%%%%% cond_map_atomic_sub %%%%%%%%%%%%%%%%%%%%
\pagebreak
\subsubsection{\code{cond\_map\_atomic\_sub}}
\label{api:python:cond_map_atomic_sub}
\index{cond\_map\_atomic\_sub!Python API}
Subtract the specified number from the value of a key-value pair within each
map if and only if the \code{checks} hold on the object.
This operation requires a pre-existing object in order to complete successfully.
If no object exists, the operation will fail with \code{NOTFOUND}.


This operation will succeed if and only if the predicates specified by
\code{checks} hold on the pre-existing object.  If any of the predicates are not
true for the existing object, then the operation will have no effect and fail
with \code{CMPFAIL}.

All checks are atomic with the write.  HyperDex guarantees that no other
operation will come between validating the checks, and writing the new version
of the object.



\paragraph{Definition:}
\begin{pythoncode}
def cond_map_atomic_sub(self, spacename, key, predicates, mapattributes)
\end{pythoncode}

\paragraph{Parameters:}
\begin{itemize}[noitemsep]
\item \code{spacename}\\
The name of an existing space.

\item \code{key}\\
The key for the operation where \code{key} is a bytestring and \code{key\_sz}
specifies the number of bytes in \code{key}.

\item \code{predicates}\\
The predicates to check against.  \code{checks} is an object with the individual
predicates as properties.

\item \code{mapattributes}\\
A dictionary of map attributes to modify and their respective key-value pairs.

\end{itemize}

\paragraph{Returns:}
True if the operation succeeded, False if any of the provided predicates failed.
Raises an exception on error.


\pagebreak
\subsubsection{\code{async\_cond\_map\_atomic\_sub}}
\label{api:python:async_cond_map_atomic_sub}
\index{async\_cond\_map\_atomic\_sub!Python API}
Subtract the specified number from the value of a key-value pair within each
map if and only if the \code{checks} hold on the object.
This operation requires a pre-existing object in order to complete successfully.
If no object exists, the operation will fail with \code{NOTFOUND}.


This operation will succeed if and only if the predicates specified by
\code{checks} hold on the pre-existing object.  If any of the predicates are not
true for the existing object, then the operation will have no effect and fail
with \code{CMPFAIL}.

All checks are atomic with the write.  HyperDex guarantees that no other
operation will come between validating the checks, and writing the new version
of the object.



\paragraph{Definition:}
\begin{pythoncode}
def async_cond_map_atomic_sub(self, spacename, key, predicates, mapattributes)
\end{pythoncode}

\paragraph{Parameters:}
\begin{itemize}[noitemsep]
\item \code{spacename}\\
The name of an existing space.

\item \code{key}\\
The key for the operation where \code{key} is a bytestring and \code{key\_sz}
specifies the number of bytes in \code{key}.

\item \code{predicates}\\
The predicates to check against.  \code{checks} is an object with the individual
predicates as properties.

\item \code{mapattributes}\\
A dictionary of map attributes to modify and their respective key-value pairs.

\end{itemize}

\paragraph{Returns:}
A Deferred object with a \code{wait} method that returns True if the operation
succeeded, or False if any of the provided predicates failed.  Raises an
exception on error.


\paragraph{See also:}  This is the asynchronous form of \code{cond\_map\_atomic\_sub}.

%%%%%%%%%%%%%%%%%%%% group_map_atomic_sub %%%%%%%%%%%%%%%%%%%%
\pagebreak
\subsubsection{\code{group\_map\_atomic\_sub}}
\label{api:python:group_map_atomic_sub}
\index{group\_map\_atomic\_sub!Python API}
Subtract the specified number from the value of a key-value pair within each
map for each object in \code{space} that matches \code{checks}.

This operation will only affect objects that match the provided \code{checks}.
Objects that do not match \code{checks} will be unaffected by the group call.
Each object that matches \code{checks} will be atomically updated with the check
on the object.  HyperDex guarantees that no object will be altered if the
\code{checks} do not pass at the time of the write.  Objects that are updated
concurrently with the group call may or may not be updated; however, regardless
of any other concurrent operations, the preceding guarantee will always hold.



\paragraph{Definition:}
\begin{pythoncode}
def group_map_atomic_sub(self, spacename, predicates, mapattributes)
\end{pythoncode}

\paragraph{Parameters:}
\begin{itemize}[noitemsep]
\item \code{spacename}\\
The name of an existing space.

\item \code{predicates}\\
The predicates to check against.  \code{checks} is an object with the individual
predicates as properties.

\item \code{mapattributes}\\
A dictionary of map attributes to modify and their respective key-value pairs.

\end{itemize}

\paragraph{Returns:}
A count of the number of objects, and a \code{client.Error} object indicating
the status of the operation.


\pagebreak
\subsubsection{\code{async\_group\_map\_atomic\_sub}}
\label{api:python:async_group_map_atomic_sub}
\index{async\_group\_map\_atomic\_sub!Python API}
Subtract the specified number from the value of a key-value pair within each
map for each object in \code{space} that matches \code{checks}.

This operation will only affect objects that match the provided \code{checks}.
Objects that do not match \code{checks} will be unaffected by the group call.
Each object that matches \code{checks} will be atomically updated with the check
on the object.  HyperDex guarantees that no object will be altered if the
\code{checks} do not pass at the time of the write.  Objects that are updated
concurrently with the group call may or may not be updated; however, regardless
of any other concurrent operations, the preceding guarantee will always hold.



\paragraph{Definition:}
\begin{pythoncode}
def async_group_map_atomic_sub(self, spacename, predicates, mapattributes)
\end{pythoncode}

\paragraph{Parameters:}
\begin{itemize}[noitemsep]
\item \code{spacename}\\
The name of an existing space.

\item \code{predicates}\\
The predicates to check against.  \code{checks} is an object with the individual
predicates as properties.

\item \code{mapattributes}\\
A dictionary of map attributes to modify and their respective key-value pairs.

\end{itemize}

\paragraph{Returns:}
This asynchronous operation returns a \code{Deferred} object with a
\code{waitForIt} method which blocks and returns a number indicating the number
of objects counted.

On error, this function will raise a \code{HyperDexClientException} describing
the error.


\paragraph{See also:}  This is the asynchronous form of \code{group\_map\_atomic\_sub}.

%%%%%%%%%%%%%%%%%%%% map_atomic_mul %%%%%%%%%%%%%%%%%%%%
\pagebreak
\subsubsection{\code{map\_atomic\_mul}}
\label{api:python:map_atomic_mul}
\index{map\_atomic\_mul!Python API}
\input{\topdir/client/fragments/map_atomic_mul}

\paragraph{Definition:}
\begin{pythoncode}
def map_atomic_mul(self, spacename, key, mapattributes)
\end{pythoncode}

\paragraph{Parameters:}
\begin{itemize}[noitemsep]
\item \code{spacename}\\
The name of an existing space.

\item \code{key}\\
The key for the operation where \code{key} is a bytestring and \code{key\_sz}
specifies the number of bytes in \code{key}.

\item \code{mapattributes}\\
A dictionary of map attributes to modify and their respective key-value pairs.

\end{itemize}

\paragraph{Returns:}
True if the operation succeeded, False if any of the provided predicates failed.
Raises an exception on error.


\pagebreak
\subsubsection{\code{async\_map\_atomic\_mul}}
\label{api:python:async_map_atomic_mul}
\index{async\_map\_atomic\_mul!Python API}
\input{\topdir/client/fragments/map_atomic_mul}

\paragraph{Definition:}
\begin{pythoncode}
def async_map_atomic_mul(self, spacename, key, mapattributes)
\end{pythoncode}

\paragraph{Parameters:}
\begin{itemize}[noitemsep]
\item \code{spacename}\\
The name of an existing space.

\item \code{key}\\
The key for the operation where \code{key} is a bytestring and \code{key\_sz}
specifies the number of bytes in \code{key}.

\item \code{mapattributes}\\
A dictionary of map attributes to modify and their respective key-value pairs.

\end{itemize}

\paragraph{Returns:}
A Deferred object with a \code{wait} method that returns True if the operation
succeeded, or False if any of the provided predicates failed.  Raises an
exception on error.


\paragraph{See also:}  This is the asynchronous form of \code{map\_atomic\_mul}.

%%%%%%%%%%%%%%%%%%%% cond_map_atomic_mul %%%%%%%%%%%%%%%%%%%%
\pagebreak
\subsubsection{\code{cond\_map\_atomic\_mul}}
\label{api:python:cond_map_atomic_mul}
\index{cond\_map\_atomic\_mul!Python API}
\input{\topdir/client/fragments/cond_map_atomic_mul}

\paragraph{Definition:}
\begin{pythoncode}
def cond_map_atomic_mul(self, spacename, key, predicates, mapattributes)
\end{pythoncode}

\paragraph{Parameters:}
\begin{itemize}[noitemsep]
\item \code{spacename}\\
The name of an existing space.

\item \code{key}\\
The key for the operation where \code{key} is a bytestring and \code{key\_sz}
specifies the number of bytes in \code{key}.

\item \code{predicates}\\
The predicates to check against.  \code{checks} is an object with the individual
predicates as properties.

\item \code{mapattributes}\\
A dictionary of map attributes to modify and their respective key-value pairs.

\end{itemize}

\paragraph{Returns:}
True if the operation succeeded, False if any of the provided predicates failed.
Raises an exception on error.


\pagebreak
\subsubsection{\code{async\_cond\_map\_atomic\_mul}}
\label{api:python:async_cond_map_atomic_mul}
\index{async\_cond\_map\_atomic\_mul!Python API}
\input{\topdir/client/fragments/cond_map_atomic_mul}

\paragraph{Definition:}
\begin{pythoncode}
def async_cond_map_atomic_mul(self, spacename, key, predicates, mapattributes)
\end{pythoncode}

\paragraph{Parameters:}
\begin{itemize}[noitemsep]
\item \code{spacename}\\
The name of an existing space.

\item \code{key}\\
The key for the operation where \code{key} is a bytestring and \code{key\_sz}
specifies the number of bytes in \code{key}.

\item \code{predicates}\\
The predicates to check against.  \code{checks} is an object with the individual
predicates as properties.

\item \code{mapattributes}\\
A dictionary of map attributes to modify and their respective key-value pairs.

\end{itemize}

\paragraph{Returns:}
A Deferred object with a \code{wait} method that returns True if the operation
succeeded, or False if any of the provided predicates failed.  Raises an
exception on error.


\paragraph{See also:}  This is the asynchronous form of \code{cond\_map\_atomic\_mul}.

%%%%%%%%%%%%%%%%%%%% group_map_atomic_mul %%%%%%%%%%%%%%%%%%%%
\pagebreak
\subsubsection{\code{group\_map\_atomic\_mul}}
\label{api:python:group_map_atomic_mul}
\index{group\_map\_atomic\_mul!Python API}
\input{\topdir/client/fragments/group_map_atomic_mul}

\paragraph{Definition:}
\begin{pythoncode}
def group_map_atomic_mul(self, spacename, predicates, mapattributes)
\end{pythoncode}

\paragraph{Parameters:}
\begin{itemize}[noitemsep]
\item \code{spacename}\\
The name of an existing space.

\item \code{predicates}\\
The predicates to check against.  \code{checks} is an object with the individual
predicates as properties.

\item \code{mapattributes}\\
A dictionary of map attributes to modify and their respective key-value pairs.

\end{itemize}

\paragraph{Returns:}
A count of the number of objects, and a \code{client.Error} object indicating
the status of the operation.


\pagebreak
\subsubsection{\code{async\_group\_map\_atomic\_mul}}
\label{api:python:async_group_map_atomic_mul}
\index{async\_group\_map\_atomic\_mul!Python API}
\input{\topdir/client/fragments/group_map_atomic_mul}

\paragraph{Definition:}
\begin{pythoncode}
def async_group_map_atomic_mul(self, spacename, predicates, mapattributes)
\end{pythoncode}

\paragraph{Parameters:}
\begin{itemize}[noitemsep]
\item \code{spacename}\\
The name of an existing space.

\item \code{predicates}\\
The predicates to check against.  \code{checks} is an object with the individual
predicates as properties.

\item \code{mapattributes}\\
A dictionary of map attributes to modify and their respective key-value pairs.

\end{itemize}

\paragraph{Returns:}
This asynchronous operation returns a \code{Deferred} object with a
\code{waitForIt} method which blocks and returns a number indicating the number
of objects counted.

On error, this function will raise a \code{HyperDexClientException} describing
the error.


\paragraph{See also:}  This is the asynchronous form of \code{group\_map\_atomic\_mul}.

%%%%%%%%%%%%%%%%%%%% map_atomic_div %%%%%%%%%%%%%%%%%%%%
\pagebreak
\subsubsection{\code{map\_atomic\_div}}
\label{api:python:map_atomic_div}
\index{map\_atomic\_div!Python API}
Divide the value of each key-value pair by the specified number for each map.

%%% Generated below here
\paragraph{Behavior:}
\begin{itemize}[noitemsep]
This operation requires a pre-existing object in order to complete successfully.
If no object exists, the operation will fail with \code{NOTFOUND}.

\item This operation mutates the value of a key-value pair in a map.  This call
    is similar to the equivalent call without the \code{map\_} prefix, but
    operates on the value of a pair in a map, instead of on an attribute's
    value.  If there is no pair with the specified map key, a new pair will be
    created and initialized to its default value.  If this is undesirable, it
    may be avoided by using a conditional operation that requires that the map
    contain the key in question.

\end{itemize}


\paragraph{Definition:}
\begin{pythoncode}
def map_atomic_div(self, spacename, key, mapattributes)
\end{pythoncode}

\paragraph{Parameters:}
\begin{itemize}[noitemsep]
\item \code{spacename}\\
The name of an existing space.

\item \code{key}\\
The key for the operation where \code{key} is a bytestring and \code{key\_sz}
specifies the number of bytes in \code{key}.

\item \code{mapattributes}\\
A dictionary of map attributes to modify and their respective key-value pairs.

\end{itemize}

\paragraph{Returns:}
True if the operation succeeded, False if any of the provided predicates failed.
Raises an exception on error.


\pagebreak
\subsubsection{\code{async\_map\_atomic\_div}}
\label{api:python:async_map_atomic_div}
\index{async\_map\_atomic\_div!Python API}
Divide the value of each key-value pair by the specified number for each map.

%%% Generated below here
\paragraph{Behavior:}
\begin{itemize}[noitemsep]
This operation requires a pre-existing object in order to complete successfully.
If no object exists, the operation will fail with \code{NOTFOUND}.

\item This operation mutates the value of a key-value pair in a map.  This call
    is similar to the equivalent call without the \code{map\_} prefix, but
    operates on the value of a pair in a map, instead of on an attribute's
    value.  If there is no pair with the specified map key, a new pair will be
    created and initialized to its default value.  If this is undesirable, it
    may be avoided by using a conditional operation that requires that the map
    contain the key in question.

\end{itemize}


\paragraph{Definition:}
\begin{pythoncode}
def async_map_atomic_div(self, spacename, key, mapattributes)
\end{pythoncode}

\paragraph{Parameters:}
\begin{itemize}[noitemsep]
\item \code{spacename}\\
The name of an existing space.

\item \code{key}\\
The key for the operation where \code{key} is a bytestring and \code{key\_sz}
specifies the number of bytes in \code{key}.

\item \code{mapattributes}\\
A dictionary of map attributes to modify and their respective key-value pairs.

\end{itemize}

\paragraph{Returns:}
A Deferred object with a \code{wait} method that returns True if the operation
succeeded, or False if any of the provided predicates failed.  Raises an
exception on error.


\paragraph{See also:}  This is the asynchronous form of \code{map\_atomic\_div}.

%%%%%%%%%%%%%%%%%%%% cond_map_atomic_div %%%%%%%%%%%%%%%%%%%%
\pagebreak
\subsubsection{\code{cond\_map\_atomic\_div}}
\label{api:python:cond_map_atomic_div}
\index{cond\_map\_atomic\_div!Python API}
Divide the value of each key-value pair by the specified number for each map if
and only if the \code{checks} hold on the object.
This operation requires a pre-existing object in order to complete successfully.
If no object exists, the operation will fail with \code{NOTFOUND}.


This operation will succeed if and only if the predicates specified by
\code{checks} hold on the pre-existing object.  If any of the predicates are not
true for the existing object, then the operation will have no effect and fail
with \code{CMPFAIL}.

All checks are atomic with the write.  HyperDex guarantees that no other
operation will come between validating the checks, and writing the new version
of the object.



\paragraph{Definition:}
\begin{pythoncode}
def cond_map_atomic_div(self, spacename, key, predicates, mapattributes)
\end{pythoncode}

\paragraph{Parameters:}
\begin{itemize}[noitemsep]
\item \code{spacename}\\
The name of an existing space.

\item \code{key}\\
The key for the operation where \code{key} is a bytestring and \code{key\_sz}
specifies the number of bytes in \code{key}.

\item \code{predicates}\\
The predicates to check against.  \code{checks} is an object with the individual
predicates as properties.

\item \code{mapattributes}\\
A dictionary of map attributes to modify and their respective key-value pairs.

\end{itemize}

\paragraph{Returns:}
True if the operation succeeded, False if any of the provided predicates failed.
Raises an exception on error.


\pagebreak
\subsubsection{\code{async\_cond\_map\_atomic\_div}}
\label{api:python:async_cond_map_atomic_div}
\index{async\_cond\_map\_atomic\_div!Python API}
Divide the value of each key-value pair by the specified number for each map if
and only if the \code{checks} hold on the object.
This operation requires a pre-existing object in order to complete successfully.
If no object exists, the operation will fail with \code{NOTFOUND}.


This operation will succeed if and only if the predicates specified by
\code{checks} hold on the pre-existing object.  If any of the predicates are not
true for the existing object, then the operation will have no effect and fail
with \code{CMPFAIL}.

All checks are atomic with the write.  HyperDex guarantees that no other
operation will come between validating the checks, and writing the new version
of the object.



\paragraph{Definition:}
\begin{pythoncode}
def async_cond_map_atomic_div(self, spacename, key, predicates, mapattributes)
\end{pythoncode}

\paragraph{Parameters:}
\begin{itemize}[noitemsep]
\item \code{spacename}\\
The name of an existing space.

\item \code{key}\\
The key for the operation where \code{key} is a bytestring and \code{key\_sz}
specifies the number of bytes in \code{key}.

\item \code{predicates}\\
The predicates to check against.  \code{checks} is an object with the individual
predicates as properties.

\item \code{mapattributes}\\
A dictionary of map attributes to modify and their respective key-value pairs.

\end{itemize}

\paragraph{Returns:}
A Deferred object with a \code{wait} method that returns True if the operation
succeeded, or False if any of the provided predicates failed.  Raises an
exception on error.


\paragraph{See also:}  This is the asynchronous form of \code{cond\_map\_atomic\_div}.

%%%%%%%%%%%%%%%%%%%% group_map_atomic_div %%%%%%%%%%%%%%%%%%%%
\pagebreak
\subsubsection{\code{group\_map\_atomic\_div}}
\label{api:python:group_map_atomic_div}
\index{group\_map\_atomic\_div!Python API}
Divide the value of each key-value pair by the specified number for each object
in \code{space} that matches \code{checks}.

This operation will only affect objects that match the provided \code{checks}.
Objects that do not match \code{checks} will be unaffected by the group call.
Each object that matches \code{checks} will be atomically updated with the check
on the object.  HyperDex guarantees that no object will be altered if the
\code{checks} do not pass at the time of the write.  Objects that are updated
concurrently with the group call may or may not be updated; however, regardless
of any other concurrent operations, the preceding guarantee will always hold.



\paragraph{Definition:}
\begin{pythoncode}
def group_map_atomic_div(self, spacename, predicates, mapattributes)
\end{pythoncode}

\paragraph{Parameters:}
\begin{itemize}[noitemsep]
\item \code{spacename}\\
The name of an existing space.

\item \code{predicates}\\
The predicates to check against.  \code{checks} is an object with the individual
predicates as properties.

\item \code{mapattributes}\\
A dictionary of map attributes to modify and their respective key-value pairs.

\end{itemize}

\paragraph{Returns:}
A count of the number of objects, and a \code{client.Error} object indicating
the status of the operation.


\pagebreak
\subsubsection{\code{async\_group\_map\_atomic\_div}}
\label{api:python:async_group_map_atomic_div}
\index{async\_group\_map\_atomic\_div!Python API}
Divide the value of each key-value pair by the specified number for each object
in \code{space} that matches \code{checks}.

This operation will only affect objects that match the provided \code{checks}.
Objects that do not match \code{checks} will be unaffected by the group call.
Each object that matches \code{checks} will be atomically updated with the check
on the object.  HyperDex guarantees that no object will be altered if the
\code{checks} do not pass at the time of the write.  Objects that are updated
concurrently with the group call may or may not be updated; however, regardless
of any other concurrent operations, the preceding guarantee will always hold.



\paragraph{Definition:}
\begin{pythoncode}
def async_group_map_atomic_div(self, spacename, predicates, mapattributes)
\end{pythoncode}

\paragraph{Parameters:}
\begin{itemize}[noitemsep]
\item \code{spacename}\\
The name of an existing space.

\item \code{predicates}\\
The predicates to check against.  \code{checks} is an object with the individual
predicates as properties.

\item \code{mapattributes}\\
A dictionary of map attributes to modify and their respective key-value pairs.

\end{itemize}

\paragraph{Returns:}
This asynchronous operation returns a \code{Deferred} object with a
\code{waitForIt} method which blocks and returns a number indicating the number
of objects counted.

On error, this function will raise a \code{HyperDexClientException} describing
the error.


\paragraph{See also:}  This is the asynchronous form of \code{group\_map\_atomic\_div}.

%%%%%%%%%%%%%%%%%%%% map_atomic_mod %%%%%%%%%%%%%%%%%%%%
\pagebreak
\subsubsection{\code{map\_atomic\_mod}}
\label{api:python:map_atomic_mod}
\index{map\_atomic\_mod!Python API}
Store the value of the key-value pair modulo the specified number for each map.
This operation requires a pre-existing object in order to complete successfully.
If no object exists, the operation will fail with \code{NOTFOUND}.



\paragraph{Definition:}
\begin{pythoncode}
def map_atomic_mod(self, spacename, key, mapattributes)
\end{pythoncode}

\paragraph{Parameters:}
\begin{itemize}[noitemsep]
\item \code{spacename}\\
The name of an existing space.

\item \code{key}\\
The key for the operation where \code{key} is a bytestring and \code{key\_sz}
specifies the number of bytes in \code{key}.

\item \code{mapattributes}\\
A dictionary of map attributes to modify and their respective key-value pairs.

\end{itemize}

\paragraph{Returns:}
True if the operation succeeded, False if any of the provided predicates failed.
Raises an exception on error.


\pagebreak
\subsubsection{\code{async\_map\_atomic\_mod}}
\label{api:python:async_map_atomic_mod}
\index{async\_map\_atomic\_mod!Python API}
Store the value of the key-value pair modulo the specified number for each map.
This operation requires a pre-existing object in order to complete successfully.
If no object exists, the operation will fail with \code{NOTFOUND}.



\paragraph{Definition:}
\begin{pythoncode}
def async_map_atomic_mod(self, spacename, key, mapattributes)
\end{pythoncode}

\paragraph{Parameters:}
\begin{itemize}[noitemsep]
\item \code{spacename}\\
The name of an existing space.

\item \code{key}\\
The key for the operation where \code{key} is a bytestring and \code{key\_sz}
specifies the number of bytes in \code{key}.

\item \code{mapattributes}\\
A dictionary of map attributes to modify and their respective key-value pairs.

\end{itemize}

\paragraph{Returns:}
A Deferred object with a \code{wait} method that returns True if the operation
succeeded, or False if any of the provided predicates failed.  Raises an
exception on error.


\paragraph{See also:}  This is the asynchronous form of \code{map\_atomic\_mod}.

%%%%%%%%%%%%%%%%%%%% cond_map_atomic_mod %%%%%%%%%%%%%%%%%%%%
\pagebreak
\subsubsection{\code{cond\_map\_atomic\_mod}}
\label{api:python:cond_map_atomic_mod}
\index{cond\_map\_atomic\_mod!Python API}
Conditionally store the value of the key-value pair modulo the specified number
for each map.

%%% Generated below here
\paragraph{Behavior:}
\begin{itemize}[noitemsep]
This operation requires a pre-existing object in order to complete successfully.
If no object exists, the operation will fail with \code{NOTFOUND}.

This operation will succeed if and only if the predicates specified by
\code{checks} hold on the pre-existing object.  If any of the predicates are not
true for the existing object, then the operation will have no effect and fail
with \code{CMPFAIL}.

All checks are atomic with the write.  HyperDex guarantees that no other
operation will come between validating the checks, and writing the new version
of the object.

\item This operation mutates the value of a key-value pair in a map.  This call
    is similar to the equivalent call without the \code{map\_} prefix, but
    operates on the value of a pair in a map, instead of on an attribute's
    value.  If there is no pair with the specified map key, a new pair will be
    created and initialized to its default value.  If this is undesirable, it
    may be avoided by using a conditional operation that requires that the map
    contain the key in question.

\end{itemize}


\paragraph{Definition:}
\begin{pythoncode}
def cond_map_atomic_mod(self, spacename, key, predicates, mapattributes)
\end{pythoncode}

\paragraph{Parameters:}
\begin{itemize}[noitemsep]
\item \code{spacename}\\
The name of an existing space.

\item \code{key}\\
The key for the operation where \code{key} is a bytestring and \code{key\_sz}
specifies the number of bytes in \code{key}.

\item \code{predicates}\\
The predicates to check against.  \code{checks} is an object with the individual
predicates as properties.

\item \code{mapattributes}\\
A dictionary of map attributes to modify and their respective key-value pairs.

\end{itemize}

\paragraph{Returns:}
True if the operation succeeded, False if any of the provided predicates failed.
Raises an exception on error.


\pagebreak
\subsubsection{\code{async\_cond\_map\_atomic\_mod}}
\label{api:python:async_cond_map_atomic_mod}
\index{async\_cond\_map\_atomic\_mod!Python API}
Conditionally store the value of the key-value pair modulo the specified number
for each map.

%%% Generated below here
\paragraph{Behavior:}
\begin{itemize}[noitemsep]
This operation requires a pre-existing object in order to complete successfully.
If no object exists, the operation will fail with \code{NOTFOUND}.

This operation will succeed if and only if the predicates specified by
\code{checks} hold on the pre-existing object.  If any of the predicates are not
true for the existing object, then the operation will have no effect and fail
with \code{CMPFAIL}.

All checks are atomic with the write.  HyperDex guarantees that no other
operation will come between validating the checks, and writing the new version
of the object.

\item This operation mutates the value of a key-value pair in a map.  This call
    is similar to the equivalent call without the \code{map\_} prefix, but
    operates on the value of a pair in a map, instead of on an attribute's
    value.  If there is no pair with the specified map key, a new pair will be
    created and initialized to its default value.  If this is undesirable, it
    may be avoided by using a conditional operation that requires that the map
    contain the key in question.

\end{itemize}


\paragraph{Definition:}
\begin{pythoncode}
def async_cond_map_atomic_mod(self, spacename, key, predicates, mapattributes)
\end{pythoncode}

\paragraph{Parameters:}
\begin{itemize}[noitemsep]
\item \code{spacename}\\
The name of an existing space.

\item \code{key}\\
The key for the operation where \code{key} is a bytestring and \code{key\_sz}
specifies the number of bytes in \code{key}.

\item \code{predicates}\\
The predicates to check against.  \code{checks} is an object with the individual
predicates as properties.

\item \code{mapattributes}\\
A dictionary of map attributes to modify and their respective key-value pairs.

\end{itemize}

\paragraph{Returns:}
A Deferred object with a \code{wait} method that returns True if the operation
succeeded, or False if any of the provided predicates failed.  Raises an
exception on error.


\paragraph{See also:}  This is the asynchronous form of \code{cond\_map\_atomic\_mod}.

%%%%%%%%%%%%%%%%%%%% group_map_atomic_mod %%%%%%%%%%%%%%%%%%%%
\pagebreak
\subsubsection{\code{group\_map\_atomic\_mod}}
\label{api:python:group_map_atomic_mod}
\index{group\_map\_atomic\_mod!Python API}
Store the value of the key-value pair modulo the specified number for each
object in \code{space} that matches \code{checks}.

This operation will only affect objects that match the provided \code{checks}.
Objects that do not match \code{checks} will be unaffected by the group call.
Each object that matches \code{checks} will be atomically updated with the check
on the object.  HyperDex guarantees that no object will be altered if the
\code{checks} do not pass at the time of the write.  Objects that are updated
concurrently with the group call may or may not be updated; however, regardless
of any other concurrent operations, the preceding guarantee will always hold.



\paragraph{Definition:}
\begin{pythoncode}
def group_map_atomic_mod(self, spacename, predicates, mapattributes)
\end{pythoncode}

\paragraph{Parameters:}
\begin{itemize}[noitemsep]
\item \code{spacename}\\
The name of an existing space.

\item \code{predicates}\\
The predicates to check against.  \code{checks} is an object with the individual
predicates as properties.

\item \code{mapattributes}\\
A dictionary of map attributes to modify and their respective key-value pairs.

\end{itemize}

\paragraph{Returns:}
A count of the number of objects, and a \code{client.Error} object indicating
the status of the operation.


\pagebreak
\subsubsection{\code{async\_group\_map\_atomic\_mod}}
\label{api:python:async_group_map_atomic_mod}
\index{async\_group\_map\_atomic\_mod!Python API}
Store the value of the key-value pair modulo the specified number for each
object in \code{space} that matches \code{checks}.

This operation will only affect objects that match the provided \code{checks}.
Objects that do not match \code{checks} will be unaffected by the group call.
Each object that matches \code{checks} will be atomically updated with the check
on the object.  HyperDex guarantees that no object will be altered if the
\code{checks} do not pass at the time of the write.  Objects that are updated
concurrently with the group call may or may not be updated; however, regardless
of any other concurrent operations, the preceding guarantee will always hold.



\paragraph{Definition:}
\begin{pythoncode}
def async_group_map_atomic_mod(self, spacename, predicates, mapattributes)
\end{pythoncode}

\paragraph{Parameters:}
\begin{itemize}[noitemsep]
\item \code{spacename}\\
The name of an existing space.

\item \code{predicates}\\
The predicates to check against.  \code{checks} is an object with the individual
predicates as properties.

\item \code{mapattributes}\\
A dictionary of map attributes to modify and their respective key-value pairs.

\end{itemize}

\paragraph{Returns:}
This asynchronous operation returns a \code{Deferred} object with a
\code{waitForIt} method which blocks and returns a number indicating the number
of objects counted.

On error, this function will raise a \code{HyperDexClientException} describing
the error.


\paragraph{See also:}  This is the asynchronous form of \code{group\_map\_atomic\_mod}.

%%%%%%%%%%%%%%%%%%%% map_atomic_and %%%%%%%%%%%%%%%%%%%%
\pagebreak
\subsubsection{\code{map\_atomic\_and}}
\label{api:python:map_atomic_and}
\index{map\_atomic\_and!Python API}
Store the bitwise AND of the value of the key-value pair and the specified
number for each map.
This operation requires a pre-existing object in order to complete successfully.
If no object exists, the operation will fail with \code{NOTFOUND}.



\paragraph{Definition:}
\begin{pythoncode}
def map_atomic_and(self, spacename, key, mapattributes)
\end{pythoncode}

\paragraph{Parameters:}
\begin{itemize}[noitemsep]
\item \code{spacename}\\
The name of an existing space.

\item \code{key}\\
The key for the operation where \code{key} is a bytestring and \code{key\_sz}
specifies the number of bytes in \code{key}.

\item \code{mapattributes}\\
A dictionary of map attributes to modify and their respective key-value pairs.

\end{itemize}

\paragraph{Returns:}
True if the operation succeeded, False if any of the provided predicates failed.
Raises an exception on error.


\pagebreak
\subsubsection{\code{async\_map\_atomic\_and}}
\label{api:python:async_map_atomic_and}
\index{async\_map\_atomic\_and!Python API}
Store the bitwise AND of the value of the key-value pair and the specified
number for each map.
This operation requires a pre-existing object in order to complete successfully.
If no object exists, the operation will fail with \code{NOTFOUND}.



\paragraph{Definition:}
\begin{pythoncode}
def async_map_atomic_and(self, spacename, key, mapattributes)
\end{pythoncode}

\paragraph{Parameters:}
\begin{itemize}[noitemsep]
\item \code{spacename}\\
The name of an existing space.

\item \code{key}\\
The key for the operation where \code{key} is a bytestring and \code{key\_sz}
specifies the number of bytes in \code{key}.

\item \code{mapattributes}\\
A dictionary of map attributes to modify and their respective key-value pairs.

\end{itemize}

\paragraph{Returns:}
A Deferred object with a \code{wait} method that returns True if the operation
succeeded, or False if any of the provided predicates failed.  Raises an
exception on error.


\paragraph{See also:}  This is the asynchronous form of \code{map\_atomic\_and}.

%%%%%%%%%%%%%%%%%%%% cond_map_atomic_and %%%%%%%%%%%%%%%%%%%%
\pagebreak
\subsubsection{\code{cond\_map\_atomic\_and}}
\label{api:python:cond_map_atomic_and}
\index{cond\_map\_atomic\_and!Python API}
\input{\topdir/client/fragments/cond_map_atomic_and}

\paragraph{Definition:}
\begin{pythoncode}
def cond_map_atomic_and(self, spacename, key, predicates, mapattributes)
\end{pythoncode}

\paragraph{Parameters:}
\begin{itemize}[noitemsep]
\item \code{spacename}\\
The name of an existing space.

\item \code{key}\\
The key for the operation where \code{key} is a bytestring and \code{key\_sz}
specifies the number of bytes in \code{key}.

\item \code{predicates}\\
The predicates to check against.  \code{checks} is an object with the individual
predicates as properties.

\item \code{mapattributes}\\
A dictionary of map attributes to modify and their respective key-value pairs.

\end{itemize}

\paragraph{Returns:}
True if the operation succeeded, False if any of the provided predicates failed.
Raises an exception on error.


\pagebreak
\subsubsection{\code{async\_cond\_map\_atomic\_and}}
\label{api:python:async_cond_map_atomic_and}
\index{async\_cond\_map\_atomic\_and!Python API}
\input{\topdir/client/fragments/cond_map_atomic_and}

\paragraph{Definition:}
\begin{pythoncode}
def async_cond_map_atomic_and(self, spacename, key, predicates, mapattributes)
\end{pythoncode}

\paragraph{Parameters:}
\begin{itemize}[noitemsep]
\item \code{spacename}\\
The name of an existing space.

\item \code{key}\\
The key for the operation where \code{key} is a bytestring and \code{key\_sz}
specifies the number of bytes in \code{key}.

\item \code{predicates}\\
The predicates to check against.  \code{checks} is an object with the individual
predicates as properties.

\item \code{mapattributes}\\
A dictionary of map attributes to modify and their respective key-value pairs.

\end{itemize}

\paragraph{Returns:}
A Deferred object with a \code{wait} method that returns True if the operation
succeeded, or False if any of the provided predicates failed.  Raises an
exception on error.


\paragraph{See also:}  This is the asynchronous form of \code{cond\_map\_atomic\_and}.

%%%%%%%%%%%%%%%%%%%% group_map_atomic_and %%%%%%%%%%%%%%%%%%%%
\pagebreak
\subsubsection{\code{group\_map\_atomic\_and}}
\label{api:python:group_map_atomic_and}
\index{group\_map\_atomic\_and!Python API}
Store the bitwise AND of the value of the key-value pair and the specified
number for each map attribute for each object in \code{space} that matches
\code{checks}.

This operation will only affect objects that match the provided \code{checks}.
Objects that do not match \code{checks} will be unaffected by the group call.
Each object that matches \code{checks} will be atomically updated with the check
on the object.  HyperDex guarantees that no object will be altered if the
\code{checks} do not pass at the time of the write.  Objects that are updated
concurrently with the group call may or may not be updated; however, regardless
of any other concurrent operations, the preceding guarantee will always hold.



\paragraph{Definition:}
\begin{pythoncode}
def group_map_atomic_and(self, spacename, predicates, mapattributes)
\end{pythoncode}

\paragraph{Parameters:}
\begin{itemize}[noitemsep]
\item \code{spacename}\\
The name of an existing space.

\item \code{predicates}\\
The predicates to check against.  \code{checks} is an object with the individual
predicates as properties.

\item \code{mapattributes}\\
A dictionary of map attributes to modify and their respective key-value pairs.

\end{itemize}

\paragraph{Returns:}
A count of the number of objects, and a \code{client.Error} object indicating
the status of the operation.


\pagebreak
\subsubsection{\code{async\_group\_map\_atomic\_and}}
\label{api:python:async_group_map_atomic_and}
\index{async\_group\_map\_atomic\_and!Python API}
Store the bitwise AND of the value of the key-value pair and the specified
number for each map attribute for each object in \code{space} that matches
\code{checks}.

This operation will only affect objects that match the provided \code{checks}.
Objects that do not match \code{checks} will be unaffected by the group call.
Each object that matches \code{checks} will be atomically updated with the check
on the object.  HyperDex guarantees that no object will be altered if the
\code{checks} do not pass at the time of the write.  Objects that are updated
concurrently with the group call may or may not be updated; however, regardless
of any other concurrent operations, the preceding guarantee will always hold.



\paragraph{Definition:}
\begin{pythoncode}
def async_group_map_atomic_and(self, spacename, predicates, mapattributes)
\end{pythoncode}

\paragraph{Parameters:}
\begin{itemize}[noitemsep]
\item \code{spacename}\\
The name of an existing space.

\item \code{predicates}\\
The predicates to check against.  \code{checks} is an object with the individual
predicates as properties.

\item \code{mapattributes}\\
A dictionary of map attributes to modify and their respective key-value pairs.

\end{itemize}

\paragraph{Returns:}
This asynchronous operation returns a \code{Deferred} object with a
\code{waitForIt} method which blocks and returns a number indicating the number
of objects counted.

On error, this function will raise a \code{HyperDexClientException} describing
the error.


\paragraph{See also:}  This is the asynchronous form of \code{group\_map\_atomic\_and}.

%%%%%%%%%%%%%%%%%%%% map_atomic_or %%%%%%%%%%%%%%%%%%%%
\pagebreak
\subsubsection{\code{map\_atomic\_or}}
\label{api:python:map_atomic_or}
\index{map\_atomic\_or!Python API}
Store the bitwise OR of the value of the key-value pair and the specified number
for each map.

%%% Generated below here
\paragraph{Behavior:}
\begin{itemize}[noitemsep]
This operation requires a pre-existing object in order to complete successfully.
If no object exists, the operation will fail with \code{NOTFOUND}.

\item This operation mutates the value of a key-value pair in a map.  This call
    is similar to the equivalent call without the \code{map\_} prefix, but
    operates on the value of a pair in a map, instead of on an attribute's
    value.  If there is no pair with the specified map key, a new pair will be
    created and initialized to its default value.  If this is undesirable, it
    may be avoided by using a conditional operation that requires that the map
    contain the key in question.

\end{itemize}


\paragraph{Definition:}
\begin{pythoncode}
def map_atomic_or(self, spacename, key, mapattributes)
\end{pythoncode}

\paragraph{Parameters:}
\begin{itemize}[noitemsep]
\item \code{spacename}\\
The name of an existing space.

\item \code{key}\\
The key for the operation where \code{key} is a bytestring and \code{key\_sz}
specifies the number of bytes in \code{key}.

\item \code{mapattributes}\\
A dictionary of map attributes to modify and their respective key-value pairs.

\end{itemize}

\paragraph{Returns:}
True if the operation succeeded, False if any of the provided predicates failed.
Raises an exception on error.


\pagebreak
\subsubsection{\code{async\_map\_atomic\_or}}
\label{api:python:async_map_atomic_or}
\index{async\_map\_atomic\_or!Python API}
Store the bitwise OR of the value of the key-value pair and the specified number
for each map.

%%% Generated below here
\paragraph{Behavior:}
\begin{itemize}[noitemsep]
This operation requires a pre-existing object in order to complete successfully.
If no object exists, the operation will fail with \code{NOTFOUND}.

\item This operation mutates the value of a key-value pair in a map.  This call
    is similar to the equivalent call without the \code{map\_} prefix, but
    operates on the value of a pair in a map, instead of on an attribute's
    value.  If there is no pair with the specified map key, a new pair will be
    created and initialized to its default value.  If this is undesirable, it
    may be avoided by using a conditional operation that requires that the map
    contain the key in question.

\end{itemize}


\paragraph{Definition:}
\begin{pythoncode}
def async_map_atomic_or(self, spacename, key, mapattributes)
\end{pythoncode}

\paragraph{Parameters:}
\begin{itemize}[noitemsep]
\item \code{spacename}\\
The name of an existing space.

\item \code{key}\\
The key for the operation where \code{key} is a bytestring and \code{key\_sz}
specifies the number of bytes in \code{key}.

\item \code{mapattributes}\\
A dictionary of map attributes to modify and their respective key-value pairs.

\end{itemize}

\paragraph{Returns:}
A Deferred object with a \code{wait} method that returns True if the operation
succeeded, or False if any of the provided predicates failed.  Raises an
exception on error.


\paragraph{See also:}  This is the asynchronous form of \code{map\_atomic\_or}.

%%%%%%%%%%%%%%%%%%%% cond_map_atomic_or %%%%%%%%%%%%%%%%%%%%
\pagebreak
\subsubsection{\code{cond\_map\_atomic\_or}}
\label{api:python:cond_map_atomic_or}
\index{cond\_map\_atomic\_or!Python API}
\input{\topdir/client/fragments/cond_map_atomic_or}

\paragraph{Definition:}
\begin{pythoncode}
def cond_map_atomic_or(self, spacename, key, predicates, mapattributes)
\end{pythoncode}

\paragraph{Parameters:}
\begin{itemize}[noitemsep]
\item \code{spacename}\\
The name of an existing space.

\item \code{key}\\
The key for the operation where \code{key} is a bytestring and \code{key\_sz}
specifies the number of bytes in \code{key}.

\item \code{predicates}\\
The predicates to check against.  \code{checks} is an object with the individual
predicates as properties.

\item \code{mapattributes}\\
A dictionary of map attributes to modify and their respective key-value pairs.

\end{itemize}

\paragraph{Returns:}
True if the operation succeeded, False if any of the provided predicates failed.
Raises an exception on error.


\pagebreak
\subsubsection{\code{async\_cond\_map\_atomic\_or}}
\label{api:python:async_cond_map_atomic_or}
\index{async\_cond\_map\_atomic\_or!Python API}
\input{\topdir/client/fragments/cond_map_atomic_or}

\paragraph{Definition:}
\begin{pythoncode}
def async_cond_map_atomic_or(self, spacename, key, predicates, mapattributes)
\end{pythoncode}

\paragraph{Parameters:}
\begin{itemize}[noitemsep]
\item \code{spacename}\\
The name of an existing space.

\item \code{key}\\
The key for the operation where \code{key} is a bytestring and \code{key\_sz}
specifies the number of bytes in \code{key}.

\item \code{predicates}\\
The predicates to check against.  \code{checks} is an object with the individual
predicates as properties.

\item \code{mapattributes}\\
A dictionary of map attributes to modify and their respective key-value pairs.

\end{itemize}

\paragraph{Returns:}
A Deferred object with a \code{wait} method that returns True if the operation
succeeded, or False if any of the provided predicates failed.  Raises an
exception on error.


\paragraph{See also:}  This is the asynchronous form of \code{cond\_map\_atomic\_or}.

%%%%%%%%%%%%%%%%%%%% group_map_atomic_or %%%%%%%%%%%%%%%%%%%%
\pagebreak
\subsubsection{\code{group\_map\_atomic\_or}}
\label{api:python:group_map_atomic_or}
\index{group\_map\_atomic\_or!Python API}
\input{\topdir/client/fragments/group_map_atomic_or}

\paragraph{Definition:}
\begin{pythoncode}
def group_map_atomic_or(self, spacename, predicates, mapattributes)
\end{pythoncode}

\paragraph{Parameters:}
\begin{itemize}[noitemsep]
\item \code{spacename}\\
The name of an existing space.

\item \code{predicates}\\
The predicates to check against.  \code{checks} is an object with the individual
predicates as properties.

\item \code{mapattributes}\\
A dictionary of map attributes to modify and their respective key-value pairs.

\end{itemize}

\paragraph{Returns:}
A count of the number of objects, and a \code{client.Error} object indicating
the status of the operation.


\pagebreak
\subsubsection{\code{async\_group\_map\_atomic\_or}}
\label{api:python:async_group_map_atomic_or}
\index{async\_group\_map\_atomic\_or!Python API}
\input{\topdir/client/fragments/group_map_atomic_or}

\paragraph{Definition:}
\begin{pythoncode}
def async_group_map_atomic_or(self, spacename, predicates, mapattributes)
\end{pythoncode}

\paragraph{Parameters:}
\begin{itemize}[noitemsep]
\item \code{spacename}\\
The name of an existing space.

\item \code{predicates}\\
The predicates to check against.  \code{checks} is an object with the individual
predicates as properties.

\item \code{mapattributes}\\
A dictionary of map attributes to modify and their respective key-value pairs.

\end{itemize}

\paragraph{Returns:}
This asynchronous operation returns a \code{Deferred} object with a
\code{waitForIt} method which blocks and returns a number indicating the number
of objects counted.

On error, this function will raise a \code{HyperDexClientException} describing
the error.


\paragraph{See also:}  This is the asynchronous form of \code{group\_map\_atomic\_or}.

%%%%%%%%%%%%%%%%%%%% map_atomic_xor %%%%%%%%%%%%%%%%%%%%
\pagebreak
\subsubsection{\code{map\_atomic\_xor}}
\label{api:python:map_atomic_xor}
\index{map\_atomic\_xor!Python API}
Store the bitwise XOR of the value of the key-value pair and the specified
number for each map attribute.
This operation requires a pre-existing object in order to complete successfully.
If no object exists, the operation will fail with \code{NOTFOUND}.



\paragraph{Definition:}
\begin{pythoncode}
def map_atomic_xor(self, spacename, key, mapattributes)
\end{pythoncode}

\paragraph{Parameters:}
\begin{itemize}[noitemsep]
\item \code{spacename}\\
The name of an existing space.

\item \code{key}\\
The key for the operation where \code{key} is a bytestring and \code{key\_sz}
specifies the number of bytes in \code{key}.

\item \code{mapattributes}\\
A dictionary of map attributes to modify and their respective key-value pairs.

\end{itemize}

\paragraph{Returns:}
True if the operation succeeded, False if any of the provided predicates failed.
Raises an exception on error.


\pagebreak
\subsubsection{\code{async\_map\_atomic\_xor}}
\label{api:python:async_map_atomic_xor}
\index{async\_map\_atomic\_xor!Python API}
Store the bitwise XOR of the value of the key-value pair and the specified
number for each map attribute.
This operation requires a pre-existing object in order to complete successfully.
If no object exists, the operation will fail with \code{NOTFOUND}.



\paragraph{Definition:}
\begin{pythoncode}
def async_map_atomic_xor(self, spacename, key, mapattributes)
\end{pythoncode}

\paragraph{Parameters:}
\begin{itemize}[noitemsep]
\item \code{spacename}\\
The name of an existing space.

\item \code{key}\\
The key for the operation where \code{key} is a bytestring and \code{key\_sz}
specifies the number of bytes in \code{key}.

\item \code{mapattributes}\\
A dictionary of map attributes to modify and their respective key-value pairs.

\end{itemize}

\paragraph{Returns:}
A Deferred object with a \code{wait} method that returns True if the operation
succeeded, or False if any of the provided predicates failed.  Raises an
exception on error.


\paragraph{See also:}  This is the asynchronous form of \code{map\_atomic\_xor}.

%%%%%%%%%%%%%%%%%%%% cond_map_atomic_xor %%%%%%%%%%%%%%%%%%%%
\pagebreak
\subsubsection{\code{cond\_map\_atomic\_xor}}
\label{api:python:cond_map_atomic_xor}
\index{cond\_map\_atomic\_xor!Python API}
Conditionally store the bitwise XOR of the value of the key-value pair and the
specified number for each map.

%%% Generated below here
\paragraph{Behavior:}
\begin{itemize}[noitemsep]
This operation requires a pre-existing object in order to complete successfully.
If no object exists, the operation will fail with \code{NOTFOUND}.

This operation will succeed if and only if the predicates specified by
\code{checks} hold on the pre-existing object.  If any of the predicates are not
true for the existing object, then the operation will have no effect and fail
with \code{CMPFAIL}.

All checks are atomic with the write.  HyperDex guarantees that no other
operation will come between validating the checks, and writing the new version
of the object.

\item This operation mutates the value of a key-value pair in a map.  This call
    is similar to the equivalent call without the \code{map\_} prefix, but
    operates on the value of a pair in a map, instead of on an attribute's
    value.  If there is no pair with the specified map key, a new pair will be
    created and initialized to its default value.  If this is undesirable, it
    may be avoided by using a conditional operation that requires that the map
    contain the key in question.

\end{itemize}


\paragraph{Definition:}
\begin{pythoncode}
def cond_map_atomic_xor(self, spacename, key, predicates, mapattributes)
\end{pythoncode}

\paragraph{Parameters:}
\begin{itemize}[noitemsep]
\item \code{spacename}\\
The name of an existing space.

\item \code{key}\\
The key for the operation where \code{key} is a bytestring and \code{key\_sz}
specifies the number of bytes in \code{key}.

\item \code{predicates}\\
The predicates to check against.  \code{checks} is an object with the individual
predicates as properties.

\item \code{mapattributes}\\
A dictionary of map attributes to modify and their respective key-value pairs.

\end{itemize}

\paragraph{Returns:}
True if the operation succeeded, False if any of the provided predicates failed.
Raises an exception on error.


\pagebreak
\subsubsection{\code{async\_cond\_map\_atomic\_xor}}
\label{api:python:async_cond_map_atomic_xor}
\index{async\_cond\_map\_atomic\_xor!Python API}
Conditionally store the bitwise XOR of the value of the key-value pair and the
specified number for each map.

%%% Generated below here
\paragraph{Behavior:}
\begin{itemize}[noitemsep]
This operation requires a pre-existing object in order to complete successfully.
If no object exists, the operation will fail with \code{NOTFOUND}.

This operation will succeed if and only if the predicates specified by
\code{checks} hold on the pre-existing object.  If any of the predicates are not
true for the existing object, then the operation will have no effect and fail
with \code{CMPFAIL}.

All checks are atomic with the write.  HyperDex guarantees that no other
operation will come between validating the checks, and writing the new version
of the object.

\item This operation mutates the value of a key-value pair in a map.  This call
    is similar to the equivalent call without the \code{map\_} prefix, but
    operates on the value of a pair in a map, instead of on an attribute's
    value.  If there is no pair with the specified map key, a new pair will be
    created and initialized to its default value.  If this is undesirable, it
    may be avoided by using a conditional operation that requires that the map
    contain the key in question.

\end{itemize}


\paragraph{Definition:}
\begin{pythoncode}
def async_cond_map_atomic_xor(self, spacename, key, predicates, mapattributes)
\end{pythoncode}

\paragraph{Parameters:}
\begin{itemize}[noitemsep]
\item \code{spacename}\\
The name of an existing space.

\item \code{key}\\
The key for the operation where \code{key} is a bytestring and \code{key\_sz}
specifies the number of bytes in \code{key}.

\item \code{predicates}\\
The predicates to check against.  \code{checks} is an object with the individual
predicates as properties.

\item \code{mapattributes}\\
A dictionary of map attributes to modify and their respective key-value pairs.

\end{itemize}

\paragraph{Returns:}
A Deferred object with a \code{wait} method that returns True if the operation
succeeded, or False if any of the provided predicates failed.  Raises an
exception on error.


\paragraph{See also:}  This is the asynchronous form of \code{cond\_map\_atomic\_xor}.

%%%%%%%%%%%%%%%%%%%% group_map_atomic_xor %%%%%%%%%%%%%%%%%%%%
\pagebreak
\subsubsection{\code{group\_map\_atomic\_xor}}
\label{api:python:group_map_atomic_xor}
\index{group\_map\_atomic\_xor!Python API}
\input{\topdir/client/fragments/group_map_atomic_xor}

\paragraph{Definition:}
\begin{pythoncode}
def group_map_atomic_xor(self, spacename, predicates, mapattributes)
\end{pythoncode}

\paragraph{Parameters:}
\begin{itemize}[noitemsep]
\item \code{spacename}\\
The name of an existing space.

\item \code{predicates}\\
The predicates to check against.  \code{checks} is an object with the individual
predicates as properties.

\item \code{mapattributes}\\
A dictionary of map attributes to modify and their respective key-value pairs.

\end{itemize}

\paragraph{Returns:}
A count of the number of objects, and a \code{client.Error} object indicating
the status of the operation.


\pagebreak
\subsubsection{\code{async\_group\_map\_atomic\_xor}}
\label{api:python:async_group_map_atomic_xor}
\index{async\_group\_map\_atomic\_xor!Python API}
\input{\topdir/client/fragments/group_map_atomic_xor}

\paragraph{Definition:}
\begin{pythoncode}
def async_group_map_atomic_xor(self, spacename, predicates, mapattributes)
\end{pythoncode}

\paragraph{Parameters:}
\begin{itemize}[noitemsep]
\item \code{spacename}\\
The name of an existing space.

\item \code{predicates}\\
The predicates to check against.  \code{checks} is an object with the individual
predicates as properties.

\item \code{mapattributes}\\
A dictionary of map attributes to modify and their respective key-value pairs.

\end{itemize}

\paragraph{Returns:}
This asynchronous operation returns a \code{Deferred} object with a
\code{waitForIt} method which blocks and returns a number indicating the number
of objects counted.

On error, this function will raise a \code{HyperDexClientException} describing
the error.


\paragraph{See also:}  This is the asynchronous form of \code{group\_map\_atomic\_xor}.

%%%%%%%%%%%%%%%%%%%% map_string_prepend %%%%%%%%%%%%%%%%%%%%
\pagebreak
\subsubsection{\code{map\_string\_prepend}}
\label{api:python:map_string_prepend}
\index{map\_string\_prepend!Python API}
Prepend the specified string to the value of the key-value pair for each map.

%%% Generated below here
\paragraph{Behavior:}
\begin{itemize}[noitemsep]
This operation requires a pre-existing object in order to complete successfully.
If no object exists, the operation will fail with \code{NOTFOUND}.

\item This operation mutates the value of a key-value pair in a map.  This call
    is similar to the equivalent call without the \code{map\_} prefix, but
    operates on the value of a pair in a map, instead of on an attribute's
    value.  If there is no pair with the specified map key, a new pair will be
    created and initialized to its default value.  If this is undesirable, it
    may be avoided by using a conditional operation that requires that the map
    contain the key in question.

\end{itemize}


\paragraph{Definition:}
\begin{pythoncode}
def map_string_prepend(self, spacename, key, mapattributes)
\end{pythoncode}

\paragraph{Parameters:}
\begin{itemize}[noitemsep]
\item \code{spacename}\\
The name of an existing space.

\item \code{key}\\
The key for the operation where \code{key} is a bytestring and \code{key\_sz}
specifies the number of bytes in \code{key}.

\item \code{mapattributes}\\
A dictionary of map attributes to modify and their respective key-value pairs.

\end{itemize}

\paragraph{Returns:}
True if the operation succeeded, False if any of the provided predicates failed.
Raises an exception on error.


\pagebreak
\subsubsection{\code{async\_map\_string\_prepend}}
\label{api:python:async_map_string_prepend}
\index{async\_map\_string\_prepend!Python API}
Prepend the specified string to the value of the key-value pair for each map.

%%% Generated below here
\paragraph{Behavior:}
\begin{itemize}[noitemsep]
This operation requires a pre-existing object in order to complete successfully.
If no object exists, the operation will fail with \code{NOTFOUND}.

\item This operation mutates the value of a key-value pair in a map.  This call
    is similar to the equivalent call without the \code{map\_} prefix, but
    operates on the value of a pair in a map, instead of on an attribute's
    value.  If there is no pair with the specified map key, a new pair will be
    created and initialized to its default value.  If this is undesirable, it
    may be avoided by using a conditional operation that requires that the map
    contain the key in question.

\end{itemize}


\paragraph{Definition:}
\begin{pythoncode}
def async_map_string_prepend(self, spacename, key, mapattributes)
\end{pythoncode}

\paragraph{Parameters:}
\begin{itemize}[noitemsep]
\item \code{spacename}\\
The name of an existing space.

\item \code{key}\\
The key for the operation where \code{key} is a bytestring and \code{key\_sz}
specifies the number of bytes in \code{key}.

\item \code{mapattributes}\\
A dictionary of map attributes to modify and their respective key-value pairs.

\end{itemize}

\paragraph{Returns:}
A Deferred object with a \code{wait} method that returns True if the operation
succeeded, or False if any of the provided predicates failed.  Raises an
exception on error.


\paragraph{See also:}  This is the asynchronous form of \code{map\_string\_prepend}.

%%%%%%%%%%%%%%%%%%%% cond_map_string_prepend %%%%%%%%%%%%%%%%%%%%
\pagebreak
\subsubsection{\code{cond\_map\_string\_prepend}}
\label{api:python:cond_map_string_prepend}
\index{cond\_map\_string\_prepend!Python API}
\input{\topdir/client/fragments/cond_map_string_prepend}

\paragraph{Definition:}
\begin{pythoncode}
def cond_map_string_prepend(self, spacename, key, predicates, mapattributes)
\end{pythoncode}

\paragraph{Parameters:}
\begin{itemize}[noitemsep]
\item \code{spacename}\\
The name of an existing space.

\item \code{key}\\
The key for the operation where \code{key} is a bytestring and \code{key\_sz}
specifies the number of bytes in \code{key}.

\item \code{predicates}\\
The predicates to check against.  \code{checks} is an object with the individual
predicates as properties.

\item \code{mapattributes}\\
A dictionary of map attributes to modify and their respective key-value pairs.

\end{itemize}

\paragraph{Returns:}
True if the operation succeeded, False if any of the provided predicates failed.
Raises an exception on error.


\pagebreak
\subsubsection{\code{async\_cond\_map\_string\_prepend}}
\label{api:python:async_cond_map_string_prepend}
\index{async\_cond\_map\_string\_prepend!Python API}
\input{\topdir/client/fragments/cond_map_string_prepend}

\paragraph{Definition:}
\begin{pythoncode}
def async_cond_map_string_prepend(self, spacename, key, predicates, mapattributes)
\end{pythoncode}

\paragraph{Parameters:}
\begin{itemize}[noitemsep]
\item \code{spacename}\\
The name of an existing space.

\item \code{key}\\
The key for the operation where \code{key} is a bytestring and \code{key\_sz}
specifies the number of bytes in \code{key}.

\item \code{predicates}\\
The predicates to check against.  \code{checks} is an object with the individual
predicates as properties.

\item \code{mapattributes}\\
A dictionary of map attributes to modify and their respective key-value pairs.

\end{itemize}

\paragraph{Returns:}
A Deferred object with a \code{wait} method that returns True if the operation
succeeded, or False if any of the provided predicates failed.  Raises an
exception on error.


\paragraph{See also:}  This is the asynchronous form of \code{cond\_map\_string\_prepend}.

%%%%%%%%%%%%%%%%%%%% group_map_string_prepend %%%%%%%%%%%%%%%%%%%%
\pagebreak
\subsubsection{\code{group\_map\_string\_prepend}}
\label{api:python:group_map_string_prepend}
\index{group\_map\_string\_prepend!Python API}
Prepend the specified string to the value of the key-value pair for each map
attribute for each object in \code{space} that matches \code{checks}.

This operation will only affect objects that match the provided \code{checks}.
Objects that do not match \code{checks} will be unaffected by the group call.
Each object that matches \code{checks} will be atomically updated with the check
on the object.  HyperDex guarantees that no object will be altered if the
\code{checks} do not pass at the time of the write.  Objects that are updated
concurrently with the group call may or may not be updated; however, regardless
of any other concurrent operations, the preceding guarantee will always hold.



\paragraph{Definition:}
\begin{pythoncode}
def group_map_string_prepend(self, spacename, predicates, mapattributes)
\end{pythoncode}

\paragraph{Parameters:}
\begin{itemize}[noitemsep]
\item \code{spacename}\\
The name of an existing space.

\item \code{predicates}\\
The predicates to check against.  \code{checks} is an object with the individual
predicates as properties.

\item \code{mapattributes}\\
A dictionary of map attributes to modify and their respective key-value pairs.

\end{itemize}

\paragraph{Returns:}
A count of the number of objects, and a \code{client.Error} object indicating
the status of the operation.


\pagebreak
\subsubsection{\code{async\_group\_map\_string\_prepend}}
\label{api:python:async_group_map_string_prepend}
\index{async\_group\_map\_string\_prepend!Python API}
Prepend the specified string to the value of the key-value pair for each map
attribute for each object in \code{space} that matches \code{checks}.

This operation will only affect objects that match the provided \code{checks}.
Objects that do not match \code{checks} will be unaffected by the group call.
Each object that matches \code{checks} will be atomically updated with the check
on the object.  HyperDex guarantees that no object will be altered if the
\code{checks} do not pass at the time of the write.  Objects that are updated
concurrently with the group call may or may not be updated; however, regardless
of any other concurrent operations, the preceding guarantee will always hold.



\paragraph{Definition:}
\begin{pythoncode}
def async_group_map_string_prepend(self, spacename, predicates, mapattributes)
\end{pythoncode}

\paragraph{Parameters:}
\begin{itemize}[noitemsep]
\item \code{spacename}\\
The name of an existing space.

\item \code{predicates}\\
The predicates to check against.  \code{checks} is an object with the individual
predicates as properties.

\item \code{mapattributes}\\
A dictionary of map attributes to modify and their respective key-value pairs.

\end{itemize}

\paragraph{Returns:}
This asynchronous operation returns a \code{Deferred} object with a
\code{waitForIt} method which blocks and returns a number indicating the number
of objects counted.

On error, this function will raise a \code{HyperDexClientException} describing
the error.


\paragraph{See also:}  This is the asynchronous form of \code{group\_map\_string\_prepend}.

%%%%%%%%%%%%%%%%%%%% map_string_append %%%%%%%%%%%%%%%%%%%%
\pagebreak
\subsubsection{\code{map\_string\_append}}
\label{api:python:map_string_append}
\index{map\_string\_append!Python API}
Append the specified string to the value of the key-value pair for each map
attribute.
This operation requires a pre-existing object in order to complete successfully.
If no object exists, the operation will fail with \code{NOTFOUND}.



\paragraph{Definition:}
\begin{pythoncode}
def map_string_append(self, spacename, key, mapattributes)
\end{pythoncode}

\paragraph{Parameters:}
\begin{itemize}[noitemsep]
\item \code{spacename}\\
The name of an existing space.

\item \code{key}\\
The key for the operation where \code{key} is a bytestring and \code{key\_sz}
specifies the number of bytes in \code{key}.

\item \code{mapattributes}\\
A dictionary of map attributes to modify and their respective key-value pairs.

\end{itemize}

\paragraph{Returns:}
True if the operation succeeded, False if any of the provided predicates failed.
Raises an exception on error.


\pagebreak
\subsubsection{\code{async\_map\_string\_append}}
\label{api:python:async_map_string_append}
\index{async\_map\_string\_append!Python API}
Append the specified string to the value of the key-value pair for each map
attribute.
This operation requires a pre-existing object in order to complete successfully.
If no object exists, the operation will fail with \code{NOTFOUND}.



\paragraph{Definition:}
\begin{pythoncode}
def async_map_string_append(self, spacename, key, mapattributes)
\end{pythoncode}

\paragraph{Parameters:}
\begin{itemize}[noitemsep]
\item \code{spacename}\\
The name of an existing space.

\item \code{key}\\
The key for the operation where \code{key} is a bytestring and \code{key\_sz}
specifies the number of bytes in \code{key}.

\item \code{mapattributes}\\
A dictionary of map attributes to modify and their respective key-value pairs.

\end{itemize}

\paragraph{Returns:}
A Deferred object with a \code{wait} method that returns True if the operation
succeeded, or False if any of the provided predicates failed.  Raises an
exception on error.


\paragraph{See also:}  This is the asynchronous form of \code{map\_string\_append}.

%%%%%%%%%%%%%%%%%%%% cond_map_string_append %%%%%%%%%%%%%%%%%%%%
\pagebreak
\subsubsection{\code{cond\_map\_string\_append}}
\label{api:python:cond_map_string_append}
\index{cond\_map\_string\_append!Python API}
Append the specified string to the value of the key-value pair for each map
attribute if and only if the \code{checks} hold on the object.
This operation requires a pre-existing object in order to complete successfully.
If no object exists, the operation will fail with \code{NOTFOUND}.


This operation will succeed if and only if the predicates specified by
\code{checks} hold on the pre-existing object.  If any of the predicates are not
true for the existing object, then the operation will have no effect and fail
with \code{CMPFAIL}.

All checks are atomic with the write.  HyperDex guarantees that no other
operation will come between validating the checks, and writing the new version
of the object.



\paragraph{Definition:}
\begin{pythoncode}
def cond_map_string_append(self, spacename, key, predicates, mapattributes)
\end{pythoncode}

\paragraph{Parameters:}
\begin{itemize}[noitemsep]
\item \code{spacename}\\
The name of an existing space.

\item \code{key}\\
The key for the operation where \code{key} is a bytestring and \code{key\_sz}
specifies the number of bytes in \code{key}.

\item \code{predicates}\\
The predicates to check against.  \code{checks} is an object with the individual
predicates as properties.

\item \code{mapattributes}\\
A dictionary of map attributes to modify and their respective key-value pairs.

\end{itemize}

\paragraph{Returns:}
True if the operation succeeded, False if any of the provided predicates failed.
Raises an exception on error.


\pagebreak
\subsubsection{\code{async\_cond\_map\_string\_append}}
\label{api:python:async_cond_map_string_append}
\index{async\_cond\_map\_string\_append!Python API}
Append the specified string to the value of the key-value pair for each map
attribute if and only if the \code{checks} hold on the object.
This operation requires a pre-existing object in order to complete successfully.
If no object exists, the operation will fail with \code{NOTFOUND}.


This operation will succeed if and only if the predicates specified by
\code{checks} hold on the pre-existing object.  If any of the predicates are not
true for the existing object, then the operation will have no effect and fail
with \code{CMPFAIL}.

All checks are atomic with the write.  HyperDex guarantees that no other
operation will come between validating the checks, and writing the new version
of the object.



\paragraph{Definition:}
\begin{pythoncode}
def async_cond_map_string_append(self, spacename, key, predicates, mapattributes)
\end{pythoncode}

\paragraph{Parameters:}
\begin{itemize}[noitemsep]
\item \code{spacename}\\
The name of an existing space.

\item \code{key}\\
The key for the operation where \code{key} is a bytestring and \code{key\_sz}
specifies the number of bytes in \code{key}.

\item \code{predicates}\\
The predicates to check against.  \code{checks} is an object with the individual
predicates as properties.

\item \code{mapattributes}\\
A dictionary of map attributes to modify and their respective key-value pairs.

\end{itemize}

\paragraph{Returns:}
A Deferred object with a \code{wait} method that returns True if the operation
succeeded, or False if any of the provided predicates failed.  Raises an
exception on error.


\paragraph{See also:}  This is the asynchronous form of \code{cond\_map\_string\_append}.

%%%%%%%%%%%%%%%%%%%% group_map_string_append %%%%%%%%%%%%%%%%%%%%
\pagebreak
\subsubsection{\code{group\_map\_string\_append}}
\label{api:python:group_map_string_append}
\index{group\_map\_string\_append!Python API}
\input{\topdir/client/fragments/group_map_string_append}

\paragraph{Definition:}
\begin{pythoncode}
def group_map_string_append(self, spacename, predicates, mapattributes)
\end{pythoncode}

\paragraph{Parameters:}
\begin{itemize}[noitemsep]
\item \code{spacename}\\
The name of an existing space.

\item \code{predicates}\\
The predicates to check against.  \code{checks} is an object with the individual
predicates as properties.

\item \code{mapattributes}\\
A dictionary of map attributes to modify and their respective key-value pairs.

\end{itemize}

\paragraph{Returns:}
A count of the number of objects, and a \code{client.Error} object indicating
the status of the operation.


\pagebreak
\subsubsection{\code{async\_group\_map\_string\_append}}
\label{api:python:async_group_map_string_append}
\index{async\_group\_map\_string\_append!Python API}
\input{\topdir/client/fragments/group_map_string_append}

\paragraph{Definition:}
\begin{pythoncode}
def async_group_map_string_append(self, spacename, predicates, mapattributes)
\end{pythoncode}

\paragraph{Parameters:}
\begin{itemize}[noitemsep]
\item \code{spacename}\\
The name of an existing space.

\item \code{predicates}\\
The predicates to check against.  \code{checks} is an object with the individual
predicates as properties.

\item \code{mapattributes}\\
A dictionary of map attributes to modify and their respective key-value pairs.

\end{itemize}

\paragraph{Returns:}
This asynchronous operation returns a \code{Deferred} object with a
\code{waitForIt} method which blocks and returns a number indicating the number
of objects counted.

On error, this function will raise a \code{HyperDexClientException} describing
the error.


\paragraph{See also:}  This is the asynchronous form of \code{group\_map\_string\_append}.

%%%%%%%%%%%%%%%%%%%% map_atomic_min %%%%%%%%%%%%%%%%%%%%
\pagebreak
\subsubsection{\code{map\_atomic\_min}}
\label{api:python:map_atomic_min}
\index{map\_atomic\_min!Python API}
\input{\topdir/client/fragments/map_atomic_min}

\paragraph{Definition:}
\begin{pythoncode}
def map_atomic_min(self, spacename, key, mapattributes)
\end{pythoncode}

\paragraph{Parameters:}
\begin{itemize}[noitemsep]
\item \code{spacename}\\
The name of an existing space.

\item \code{key}\\
The key for the operation where \code{key} is a bytestring and \code{key\_sz}
specifies the number of bytes in \code{key}.

\item \code{mapattributes}\\
A dictionary of map attributes to modify and their respective key-value pairs.

\end{itemize}

\paragraph{Returns:}
True if the operation succeeded, False if any of the provided predicates failed.
Raises an exception on error.


\pagebreak
\subsubsection{\code{async\_map\_atomic\_min}}
\label{api:python:async_map_atomic_min}
\index{async\_map\_atomic\_min!Python API}
\input{\topdir/client/fragments/map_atomic_min}

\paragraph{Definition:}
\begin{pythoncode}
def async_map_atomic_min(self, spacename, key, mapattributes)
\end{pythoncode}

\paragraph{Parameters:}
\begin{itemize}[noitemsep]
\item \code{spacename}\\
The name of an existing space.

\item \code{key}\\
The key for the operation where \code{key} is a bytestring and \code{key\_sz}
specifies the number of bytes in \code{key}.

\item \code{mapattributes}\\
A dictionary of map attributes to modify and their respective key-value pairs.

\end{itemize}

\paragraph{Returns:}
A Deferred object with a \code{wait} method that returns True if the operation
succeeded, or False if any of the provided predicates failed.  Raises an
exception on error.


\paragraph{See also:}  This is the asynchronous form of \code{map\_atomic\_min}.

%%%%%%%%%%%%%%%%%%%% cond_map_atomic_min %%%%%%%%%%%%%%%%%%%%
\pagebreak
\subsubsection{\code{cond\_map\_atomic\_min}}
\label{api:python:cond_map_atomic_min}
\index{cond\_map\_atomic\_min!Python API}
\input{\topdir/client/fragments/cond_map_atomic_min}

\paragraph{Definition:}
\begin{pythoncode}
def cond_map_atomic_min(self, spacename, key, predicates, mapattributes)
\end{pythoncode}

\paragraph{Parameters:}
\begin{itemize}[noitemsep]
\item \code{spacename}\\
The name of an existing space.

\item \code{key}\\
The key for the operation where \code{key} is a bytestring and \code{key\_sz}
specifies the number of bytes in \code{key}.

\item \code{predicates}\\
The predicates to check against.  \code{checks} is an object with the individual
predicates as properties.

\item \code{mapattributes}\\
A dictionary of map attributes to modify and their respective key-value pairs.

\end{itemize}

\paragraph{Returns:}
True if the operation succeeded, False if any of the provided predicates failed.
Raises an exception on error.


\pagebreak
\subsubsection{\code{async\_cond\_map\_atomic\_min}}
\label{api:python:async_cond_map_atomic_min}
\index{async\_cond\_map\_atomic\_min!Python API}
\input{\topdir/client/fragments/cond_map_atomic_min}

\paragraph{Definition:}
\begin{pythoncode}
def async_cond_map_atomic_min(self, spacename, key, predicates, mapattributes)
\end{pythoncode}

\paragraph{Parameters:}
\begin{itemize}[noitemsep]
\item \code{spacename}\\
The name of an existing space.

\item \code{key}\\
The key for the operation where \code{key} is a bytestring and \code{key\_sz}
specifies the number of bytes in \code{key}.

\item \code{predicates}\\
The predicates to check against.  \code{checks} is an object with the individual
predicates as properties.

\item \code{mapattributes}\\
A dictionary of map attributes to modify and their respective key-value pairs.

\end{itemize}

\paragraph{Returns:}
A Deferred object with a \code{wait} method that returns True if the operation
succeeded, or False if any of the provided predicates failed.  Raises an
exception on error.


\paragraph{See also:}  This is the asynchronous form of \code{cond\_map\_atomic\_min}.

%%%%%%%%%%%%%%%%%%%% group_map_atomic_min %%%%%%%%%%%%%%%%%%%%
\pagebreak
\subsubsection{\code{group\_map\_atomic\_min}}
\label{api:python:group_map_atomic_min}
\index{group\_map\_atomic\_min!Python API}
Take the minium of the specified value and existing value for each key-value
pair for each object in \code{space} that matches \code{checks}.

This operation will only affect objects that match the provided \code{checks}.
Objects that do not match \code{checks} will be unaffected by the group call.
Each object that matches \code{checks} will be atomically updated with the check
on the object.  HyperDex guarantees that no object will be altered if the
\code{checks} do not pass at the time of the write.  Objects that are updated
concurrently with the group call may or may not be updated; however, regardless
of any other concurrent operations, the preceding guarantee will always hold.



\paragraph{Definition:}
\begin{pythoncode}
def group_map_atomic_min(self, spacename, predicates, mapattributes)
\end{pythoncode}

\paragraph{Parameters:}
\begin{itemize}[noitemsep]
\item \code{spacename}\\
The name of an existing space.

\item \code{predicates}\\
The predicates to check against.  \code{checks} is an object with the individual
predicates as properties.

\item \code{mapattributes}\\
A dictionary of map attributes to modify and their respective key-value pairs.

\end{itemize}

\paragraph{Returns:}
A count of the number of objects, and a \code{client.Error} object indicating
the status of the operation.


\pagebreak
\subsubsection{\code{async\_group\_map\_atomic\_min}}
\label{api:python:async_group_map_atomic_min}
\index{async\_group\_map\_atomic\_min!Python API}
Take the minium of the specified value and existing value for each key-value
pair for each object in \code{space} that matches \code{checks}.

This operation will only affect objects that match the provided \code{checks}.
Objects that do not match \code{checks} will be unaffected by the group call.
Each object that matches \code{checks} will be atomically updated with the check
on the object.  HyperDex guarantees that no object will be altered if the
\code{checks} do not pass at the time of the write.  Objects that are updated
concurrently with the group call may or may not be updated; however, regardless
of any other concurrent operations, the preceding guarantee will always hold.



\paragraph{Definition:}
\begin{pythoncode}
def async_group_map_atomic_min(self, spacename, predicates, mapattributes)
\end{pythoncode}

\paragraph{Parameters:}
\begin{itemize}[noitemsep]
\item \code{spacename}\\
The name of an existing space.

\item \code{predicates}\\
The predicates to check against.  \code{checks} is an object with the individual
predicates as properties.

\item \code{mapattributes}\\
A dictionary of map attributes to modify and their respective key-value pairs.

\end{itemize}

\paragraph{Returns:}
This asynchronous operation returns a \code{Deferred} object with a
\code{waitForIt} method which blocks and returns a number indicating the number
of objects counted.

On error, this function will raise a \code{HyperDexClientException} describing
the error.


\paragraph{See also:}  This is the asynchronous form of \code{group\_map\_atomic\_min}.

%%%%%%%%%%%%%%%%%%%% map_atomic_max %%%%%%%%%%%%%%%%%%%%
\pagebreak
\subsubsection{\code{map\_atomic\_max}}
\label{api:python:map_atomic_max}
\index{map\_atomic\_max!Python API}
Take the maximum of the specified value and existing value for each key-value
pair.
This operation requires a pre-existing object in order to complete successfully.
If no object exists, the operation will fail with \code{NOTFOUND}.



\paragraph{Definition:}
\begin{pythoncode}
def map_atomic_max(self, spacename, key, mapattributes)
\end{pythoncode}

\paragraph{Parameters:}
\begin{itemize}[noitemsep]
\item \code{spacename}\\
The name of an existing space.

\item \code{key}\\
The key for the operation where \code{key} is a bytestring and \code{key\_sz}
specifies the number of bytes in \code{key}.

\item \code{mapattributes}\\
A dictionary of map attributes to modify and their respective key-value pairs.

\end{itemize}

\paragraph{Returns:}
True if the operation succeeded, False if any of the provided predicates failed.
Raises an exception on error.


\pagebreak
\subsubsection{\code{async\_map\_atomic\_max}}
\label{api:python:async_map_atomic_max}
\index{async\_map\_atomic\_max!Python API}
Take the maximum of the specified value and existing value for each key-value
pair.
This operation requires a pre-existing object in order to complete successfully.
If no object exists, the operation will fail with \code{NOTFOUND}.



\paragraph{Definition:}
\begin{pythoncode}
def async_map_atomic_max(self, spacename, key, mapattributes)
\end{pythoncode}

\paragraph{Parameters:}
\begin{itemize}[noitemsep]
\item \code{spacename}\\
The name of an existing space.

\item \code{key}\\
The key for the operation where \code{key} is a bytestring and \code{key\_sz}
specifies the number of bytes in \code{key}.

\item \code{mapattributes}\\
A dictionary of map attributes to modify and their respective key-value pairs.

\end{itemize}

\paragraph{Returns:}
A Deferred object with a \code{wait} method that returns True if the operation
succeeded, or False if any of the provided predicates failed.  Raises an
exception on error.


\paragraph{See also:}  This is the asynchronous form of \code{map\_atomic\_max}.

%%%%%%%%%%%%%%%%%%%% cond_map_atomic_max %%%%%%%%%%%%%%%%%%%%
\pagebreak
\subsubsection{\code{cond\_map\_atomic\_max}}
\label{api:python:cond_map_atomic_max}
\index{cond\_map\_atomic\_max!Python API}
Take the maximum of the specified value and existing value for each key-value
pair if and only if the \code{checks} hold on the object.
This operation requires a pre-existing object in order to complete successfully.
If no object exists, the operation will fail with \code{NOTFOUND}.


This operation will succeed if and only if the predicates specified by
\code{checks} hold on the pre-existing object.  If any of the predicates are not
true for the existing object, then the operation will have no effect and fail
with \code{CMPFAIL}.

All checks are atomic with the write.  HyperDex guarantees that no other
operation will come between validating the checks, and writing the new version
of the object.



\paragraph{Definition:}
\begin{pythoncode}
def cond_map_atomic_max(self, spacename, key, predicates, mapattributes)
\end{pythoncode}

\paragraph{Parameters:}
\begin{itemize}[noitemsep]
\item \code{spacename}\\
The name of an existing space.

\item \code{key}\\
The key for the operation where \code{key} is a bytestring and \code{key\_sz}
specifies the number of bytes in \code{key}.

\item \code{predicates}\\
The predicates to check against.  \code{checks} is an object with the individual
predicates as properties.

\item \code{mapattributes}\\
A dictionary of map attributes to modify and their respective key-value pairs.

\end{itemize}

\paragraph{Returns:}
True if the operation succeeded, False if any of the provided predicates failed.
Raises an exception on error.


\pagebreak
\subsubsection{\code{async\_cond\_map\_atomic\_max}}
\label{api:python:async_cond_map_atomic_max}
\index{async\_cond\_map\_atomic\_max!Python API}
Take the maximum of the specified value and existing value for each key-value
pair if and only if the \code{checks} hold on the object.
This operation requires a pre-existing object in order to complete successfully.
If no object exists, the operation will fail with \code{NOTFOUND}.


This operation will succeed if and only if the predicates specified by
\code{checks} hold on the pre-existing object.  If any of the predicates are not
true for the existing object, then the operation will have no effect and fail
with \code{CMPFAIL}.

All checks are atomic with the write.  HyperDex guarantees that no other
operation will come between validating the checks, and writing the new version
of the object.



\paragraph{Definition:}
\begin{pythoncode}
def async_cond_map_atomic_max(self, spacename, key, predicates, mapattributes)
\end{pythoncode}

\paragraph{Parameters:}
\begin{itemize}[noitemsep]
\item \code{spacename}\\
The name of an existing space.

\item \code{key}\\
The key for the operation where \code{key} is a bytestring and \code{key\_sz}
specifies the number of bytes in \code{key}.

\item \code{predicates}\\
The predicates to check against.  \code{checks} is an object with the individual
predicates as properties.

\item \code{mapattributes}\\
A dictionary of map attributes to modify and their respective key-value pairs.

\end{itemize}

\paragraph{Returns:}
A Deferred object with a \code{wait} method that returns True if the operation
succeeded, or False if any of the provided predicates failed.  Raises an
exception on error.


\paragraph{See also:}  This is the asynchronous form of \code{cond\_map\_atomic\_max}.

%%%%%%%%%%%%%%%%%%%% group_map_atomic_max %%%%%%%%%%%%%%%%%%%%
\pagebreak
\subsubsection{\code{group\_map\_atomic\_max}}
\label{api:python:group_map_atomic_max}
\index{group\_map\_atomic\_max!Python API}
\input{\topdir/client/fragments/group_map_atomic_max}

\paragraph{Definition:}
\begin{pythoncode}
def group_map_atomic_max(self, spacename, predicates, mapattributes)
\end{pythoncode}

\paragraph{Parameters:}
\begin{itemize}[noitemsep]
\item \code{spacename}\\
The name of an existing space.

\item \code{predicates}\\
The predicates to check against.  \code{checks} is an object with the individual
predicates as properties.

\item \code{mapattributes}\\
A dictionary of map attributes to modify and their respective key-value pairs.

\end{itemize}

\paragraph{Returns:}
A count of the number of objects, and a \code{client.Error} object indicating
the status of the operation.


\pagebreak
\subsubsection{\code{async\_group\_map\_atomic\_max}}
\label{api:python:async_group_map_atomic_max}
\index{async\_group\_map\_atomic\_max!Python API}
\input{\topdir/client/fragments/group_map_atomic_max}

\paragraph{Definition:}
\begin{pythoncode}
def async_group_map_atomic_max(self, spacename, predicates, mapattributes)
\end{pythoncode}

\paragraph{Parameters:}
\begin{itemize}[noitemsep]
\item \code{spacename}\\
The name of an existing space.

\item \code{predicates}\\
The predicates to check against.  \code{checks} is an object with the individual
predicates as properties.

\item \code{mapattributes}\\
A dictionary of map attributes to modify and their respective key-value pairs.

\end{itemize}

\paragraph{Returns:}
This asynchronous operation returns a \code{Deferred} object with a
\code{waitForIt} method which blocks and returns a number indicating the number
of objects counted.

On error, this function will raise a \code{HyperDexClientException} describing
the error.


\paragraph{See also:}  This is the asynchronous form of \code{group\_map\_atomic\_max}.

%%%%%%%%%%%%%%%%%%%% search %%%%%%%%%%%%%%%%%%%%
\pagebreak
\subsubsection{\code{search}}
\label{api:python:search}
\index{search!Python API}
Return all objects that match the specified \code{checks}.

\paragraph{Behavior:}
\begin{itemize}[noitemsep]
This operation behaves as an iterator and may return multiple objects from the
single call.

\item This operation return to the user the requested object(s).

\end{itemize}


\paragraph{Definition:}
\begin{pythoncode}
def search(self, spacename, predicates)
\end{pythoncode}

\paragraph{Parameters:}
\begin{itemize}[noitemsep]
\item \code{spacename}\\
The name of the space as a string.

\item \code{predicates}\\
A set of predicates to check against.  \code{checks} is a map from the
attributes' names to the predicates to check.

\end{itemize}

\paragraph{Returns:}
Two channels, one for returning objects that match the search, and one for
returning errors encountered during the search.


%%%%%%%%%%%%%%%%%%%% sorted_search %%%%%%%%%%%%%%%%%%%%
\pagebreak
\subsubsection{\code{sorted\_search}}
\label{api:python:sorted_search}
\index{sorted\_search!Python API}
Return all objects that match the specified \code{checks}, sorted according to
\code{attr}.
This operation behaves as an iterator and may return multiple objects from the
single call.



\paragraph{Definition:}
\begin{pythoncode}
def sorted_search(self, spacename, predicates, sortby, limit, maxmin)
\end{pythoncode}

\paragraph{Parameters:}
\begin{itemize}[noitemsep]
\item \code{spacename}\\
The name of the space as a string.

\item \code{predicates}\\
A set of predicates to check against.  \code{checks} is a map from the
attributes' names to the predicates to check.

\item \code{sortby}\\
The attribute to sort by.

\item \code{limit}\\
The number of results to return.

\item \code{maxmin}\\
Maximize (\code{'max'}) or minimize (\code{'min'}).

\end{itemize}

\paragraph{Returns:}
Two channels, one for returning objects that match the search, and one for
returning errors encountered during the search.


%%%%%%%%%%%%%%%%%%%% count %%%%%%%%%%%%%%%%%%%%
\pagebreak
\subsubsection{\code{count}}
\label{api:python:count}
\index{count!Python API}
Count the number of objects that match the specified \code{checks}.

\paragraph{Behavior:}
\begin{itemize}[noitemsep]
\item This will return the number of objects counted by the search.  If an error
    occurs during the count, the count will reflect a partial count.  The real
    count will be higher than the returned value.  Some languages will throw an
    exception rather than return the partial count.
\end{itemize}


\paragraph{Definition:}
\begin{pythoncode}
def count(self, spacename, predicates)
\end{pythoncode}

\paragraph{Parameters:}
\begin{itemize}[noitemsep]
\item \code{spacename}\\
The name of an existing space.

\item \code{predicates}\\
The predicates to check against.  \code{checks} is an object with the individual
predicates as properties.

\end{itemize}

\paragraph{Returns:}
A count of the number of objects, and a \code{client.Error} object indicating
the status of the operation.


\pagebreak
\subsubsection{\code{async\_count}}
\label{api:python:async_count}
\index{async\_count!Python API}
Count the number of objects that match the specified \code{checks}.

\paragraph{Behavior:}
\begin{itemize}[noitemsep]
\item This will return the number of objects counted by the search.  If an error
    occurs during the count, the count will reflect a partial count.  The real
    count will be higher than the returned value.  Some languages will throw an
    exception rather than return the partial count.
\end{itemize}


\paragraph{Definition:}
\begin{pythoncode}
def async_count(self, spacename, predicates)
\end{pythoncode}

\paragraph{Parameters:}
\begin{itemize}[noitemsep]
\item \code{spacename}\\
The name of an existing space.

\item \code{predicates}\\
The predicates to check against.  \code{checks} is an object with the individual
predicates as properties.

\end{itemize}

\paragraph{Returns:}
This asynchronous operation returns a \code{Deferred} object with a
\code{waitForIt} method which blocks and returns a number indicating the number
of objects counted.

On error, this function will raise a \code{HyperDexClientException} describing
the error.


\paragraph{See also:}  This is the asynchronous form of \code{count}.

\pagebreak

\subsection{Working with Signals}
\label{sec:api:java:signals}

The HyperDex client library is signal-safe.  Should a signal interrupt the
client during a blocking operation, it will raise a
\code{HyperDexClientException} with status \code{HYPERDEX\_CLIENT\_INTERRUPTED}.

\subsection{Working with Threads}
\label{sec:api:Java:threads}

The Java package is fully reentrant.  Instances of
\code{HyperDex::Client::Client} and their associated state may be accessed from
multiple threads, provided that the application employs its own synchronization
that provides mutual exclusion.

Put simply, a multi-threaded application should protect each \code{Client}
instance with a mutex or lock to ensure correct operation.
