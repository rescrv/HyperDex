\chapter{Ruby API}
\label{chap:api:ruby}

\section{Client Library}
\label{sec:api:ruby:client}

HyperDex provides ruby bindings under the module \code{HyperDex::Client}.  This
library wraps the HyperDex C Client library and enables use of native Ruby data
types.

This library was brought up-to-date following the 1.0.rc5 release.

\subsection{Building the HyperDex Ruby Binding}
\label{sec:api:ruby:building}

The HyperDex Ruby Binding must be requested at configure time as it is not
automatically built.  You can ensure that the Ruby bindings are always built by
providing the \code{--enable-ruby-bindings} option to \code{./configure} like
so:

\begin{consolecode}
% ./configure --enable-client --enable-ruby-bindings
\end{consolecode}

\subsection{Using Ruby Within Your Application}
\label{sec:api:ruby:using}

All client operation are defined in the \code{HyperDex::Ruby} module.  You can
access this in your program with:

\begin{rubycode}
require 'hyperdex'
\end{rubycode}

\subsection{Hello World}
\label{sec:api:ruby:hello-world}

The following is a minimal application that stores the value "Hello World" and
then immediately retrieves the value:

\inputminted{ruby}{\topdir/api/ruby/hello-world.rb}

You can run this example with:

\begin{consolecode}
% ruby hello-world.rb
put "Hello World!"
got:
{:v=>"Hello World!"}
\end{consolecode}

Right away, there are several points worth noting in this example:

\begin{itemize}
\item Every operation is synchronous.  The PUT and GET operations run to
completion by default.

\item Ruby types are automatically converted to HyperDex types.  There's no need
to specify information such as the length of each string, as one would do with
the C API.

\item Ruby symbols are permitted wherever a string may be used.  By convention,
space names and attribute names are specified using Ruby symbols, e.g.
\code{:kv} and \code{:v}.  Of course, you may use strings for these parameters
too.
\end{itemize}

\subsection{Asynchronous Operations}
\label{sec:api:ruby:async-ops}

For convenience, the Ruby bindings treat every operation as synchronous.  This
enables you to write short scripts without concern for asynchronous operations.
Most operations come with an asynchronous form, denoted by the \code{async\_}
prefix.  For example, the above Hello World example could be rewritten in
asynchronous fashion as such:

\inputminted{ruby}{\topdir/api/ruby/hello-world-async-wait.rb}

This enables applications to issue multiple requests simultaneously and wait for
their completion in an application-specific order.  It's also possible to use
the \code{loop} method on the client object to wait for the next request to
complete:

\inputminted{ruby}{\topdir/api/ruby/hello-world-async-loop.rb}

\subsection{Data Structures}
\label{sec:api:ruby:data-structures}

The Ruby bindings automatically manage conversion of data types from Ruby to
HyperDex types, enabling applications to be written in idiomatic Ruby.

\subsubsection{Examples}
\label{sec:api:ruby:examples}

This section shows examples of Ruby data structures that are recognized by
HyperDex.  The examples here are for illustration purposes and are not
exhaustive.

\paragraph{Strings}

The HyperDex client recognizes Ruby's strings and symbols and automatically
converts them to HyperDex strings.  For example, the following two calls are
equivalent and have the same effect:

\begin{rubycode}
c.put("kv", "somekey", {"v" => "somevalue"})
c.put(:kv, :somekey, {:v => :somevalue})
\end{rubycode}

The recommended convention is to use symbols for space and attribute names, and
strings for keys and values like so:

\begin{rubycode}
c.put(:kv, "somekey", {:v => "somevalue"})
\end{rubycode}

\paragraph{Integers}

The HyperDex client recognizes Ruby's integers, longs, and fixnums and
automatically converts them to HyperDex integers.  For example:

\begin{rubycode}
c.put(:kv, "somekey", {:v => 42})
\end{rubycode}

\paragraph{Floats}

The HyperDex client recognizes Ruby's floating point numbers and automatically
converts them to HyperDex floats.  For example:

\begin{rubycode}
c.put(:kv, "somekey", {:v => 3.1415})
\end{rubycode}

\paragraph{Lists}

The HyperDex client recognizes Ruby lists and automatically converts them to
HyperDex lists.  For example:

\begin{rubycode}
c.put(:kv, "somekey", {:v1 => ["a", "b", "c"]})
c.put(:kv, "somekey", {:v2 => [1, 2, 3]})
c.put(:kv, "somekey", {:v3 => [1.0, 0.5, 0.25]})
\end{rubycode}

\paragraph{Sets}

The HyperDex client recognizes Ruby sets and automaticaly converts them to
HyperDex sets.  For example:

\begin{rubycode}
require 'set'
c.put(:kv, "somekey", {:v1 => (Set.new ["a", "b", "c"])})
c.put(:kv, "somekey", {:v2 => (Set.new [1, 2, 3])})
c.put(:kv, "somekey", {:v3 => (Set.new [1.0, 0.5, 0.25])})
\end{rubycode}

Note that you'll have to include the set module from the standard library.

\paragraph{Maps}

The HyperDex client recognizes Ruby hashes and automatically converts them to
HyperDex maps.  For example:

\begin{rubycode}
c.put(:kv, "somekey", {:v1 => {"k" => "v"}})
c.put(:kv, "somekey", {:v2 => {1 => 2}})
c.put(:kv, "somekey", {:v3 => {3.14 => 0.125}})
c.put(:kv, "somekey", {:v3 => {"a" => 1}})
\end{rubycode}

\subsection{Attributes}
\label{sec:api:ruby:attributes}

Attributes in Ruby are specified in the form of a hash from attribute names to
their values.  As you can see in the examples above, attributes are specified in
the form:

\begin{rubycode}
{:name => "value"}
\end{rubycode}

\subsection{Map Attributes}
\label{sec:api:ruby:map-attributes}

Map attributes in Ruby are specified in the form of a nested hash.  The outer
hash key specifies the name, while the inner hash key-value pair specifies the
key-value pair of the map.  For example:

\begin{rubycode}
{:name => {"key" => "value"}}
\end{rubycode}

\subsection{Predicates}
\label{sec:api:ruby:predicates}

Predicates in Ruby are specified in the form of a hash from attribute names to
their predicates.  In the simple case, the predicate is just a value to be
compared against:

\begin{rubycode}
{:v => "value"}
\end{rubycode}

This is the same as saying:

\begin{rubycode}
{:v => HyperDex::Client::Equals.new('value')}
\end{rubycode}

The Ruby bindings support the full range of predicates supported by HyperDex
itself.  For example:

\begin{rubycode}
{:v => HyperDex::Client::LessEqual.new(5)}
{:v => HyperDex::Client::GreaterEqual.new(5)}
{:v => HyperDex::Client::Range.new(5, 10)}
{:v => HyperDex::Client::Regex.new('^s.*')}
{:v => HyperDex::Client::LengthEquals.new(5)}
{:v => HyperDex::Client::LengthLessEqual.new(5)}
{:v => HyperDex::Client::LengthGreaterEqual.new(5)}
{:v => HyperDex::Client::Contains.new('value')}
\end{rubycode}

\subsection{Error Handling}
\label{sec:api:ruby:error-handling}

All error handling within the Ruby bindings is done via the
\code{begin}/\code{rescue} mechanism of Ruby.  Errors will be raised by the
library and should be handled by your application.  For example, if we were
trying to store an integer (5) as attribute \code{:v}, where \code{:v} is
actually a string, we'd generate an error.

\begin{rubycode}
begin
    puts c.put(:kv, :my_key, {:v => 5})
rescue HyperDex::Client::HyperDexClientException => e
    puts e.status
    puts e.symbol
    puts e
end
\end{rubycode}

Errors of type \code{HyperDexClientException} will contain both a message
indicating what went wrong, as well as the underlying \code{enum
hyperdex\_client\_returncode}.  The member \code{status} indicates the numeric
value of this enum, while \code{symbol} returns the enum as a string.  The above
code will fail with the following output:

\begin{verbatim}
8525
HYPERDEX_CLIENT_WRONGTYPE
invalid attribute "v": attribute has the wrong type
\end{verbatim}

\subsection{Operations}
\label{sec:api:ruby:ops}

% Copyright (c) 2013-2014, Cornell University
% All rights reserved.
%
% Redistribution and use in source and binary forms, with or without
% modification, are permitted provided that the following conditions are met:
%
%     * Redistributions of source code must retain the above copyright notice,
%       this list of conditions and the following disclaimer.
%     * Redistributions in binary form must reproduce the above copyright
%       notice, this list of conditions and the following disclaimer in the
%       documentation and/or other materials provided with the distribution.
%     * Neither the name of HyperDex nor the names of its contributors may be
%       used to endorse or promote products derived from this software without
%       specific prior written permission.
%
% THIS SOFTWARE IS PROVIDED BY THE COPYRIGHT HOLDERS AND CONTRIBUTORS "AS IS"
% AND ANY EXPRESS OR IMPLIED WARRANTIES, INCLUDING, BUT NOT LIMITED TO, THE
% IMPLIED WARRANTIES OF MERCHANTABILITY AND FITNESS FOR A PARTICULAR PURPOSE ARE
% DISCLAIMED. IN NO EVENT SHALL THE COPYRIGHT OWNER OR CONTRIBUTORS BE LIABLE
% FOR ANY DIRECT, INDIRECT, INCIDENTAL, SPECIAL, EXEMPLARY, OR CONSEQUENTIAL
% DAMAGES (INCLUDING, BUT NOT LIMITED TO, PROCUREMENT OF SUBSTITUTE GOODS OR
% SERVICES; LOSS OF USE, DATA, OR PROFITS; OR BUSINESS INTERRUPTION) HOWEVER
% CAUSED AND ON ANY THEORY OF LIABILITY, WHETHER IN CONTRACT, STRICT LIABILITY,
% OR TORT (INCLUDING NEGLIGENCE OR OTHERWISE) ARISING IN ANY WAY OUT OF THE USE
% OF THIS SOFTWARE, EVEN IF ADVISED OF THE POSSIBILITY OF SUCH DAMAGE.

% This LaTeX file is generated by bindings/ruby.py

%%%%%%%%%%%%%%%%%%%% get %%%%%%%%%%%%%%%%%%%%
\pagebreak
\subsubsection{\code{get}}
\label{api:ruby:get}
\index{get!Ruby API}
Get an object by key.

\paragraph{Behavior:}
\begin{itemize}[noitemsep]
\item This operation return to the user the requested object(s).

\end{itemize}


\paragraph{Definition:}
\begin{rubycode}
get(spacename, key)
\end{rubycode}

\paragraph{Parameters:}
\begin{itemize}[noitemsep]
\item \code{spacename}\\
The name of the space as a string or symbol.

\item \code{key}\\
The key for the operation where \code{key} is a Javascript value.

\end{itemize}

\paragraph{Returns:}
Object if found, nil if not found.  Raises exception on error.


\pagebreak
\subsubsection{\code{async\_get}}
\label{api:ruby:async_get}
\index{async\_get!Ruby API}
Get an object by key.

\paragraph{Behavior:}
\begin{itemize}[noitemsep]
\item This operation return to the user the requested object(s).

\end{itemize}


\paragraph{Definition:}
\begin{rubycode}
async_get(spacename, key)
\end{rubycode}

\paragraph{Parameters:}
\begin{itemize}[noitemsep]
\item \code{spacename}\\
The name of the space as a string or symbol.

\item \code{key}\\
The key for the operation where \code{key} is a Javascript value.

\end{itemize}

\paragraph{Returns:}
A Deferred object with a \code{wait} method that returns the object if found,
nil if not found.  Raises exception on error.


\paragraph{See also:}  This is the asynchronous form of \code{get}.

%%%%%%%%%%%%%%%%%%%% put %%%%%%%%%%%%%%%%%%%%
\pagebreak
\subsubsection{\code{put}}
\label{api:ruby:put}
\index{put!Ruby API}
Store or update an object by key.  Existing values will be overwritten with the
values specified by \code{attrs}.  Values not specified by \code{attrs} will be
preserved.

%%% Generated below here
\paragraph{Behavior:}
\begin{itemize}[noitemsep]
\item An existing object will be updated by the operation.  If no object does
    exists, a new object will be created, with attributes initialized to their
    default values.

\end{itemize}


\paragraph{Definition:}
\begin{rubycode}
put(spacename, key, attributes)
\end{rubycode}

\paragraph{Parameters:}
\begin{itemize}[noitemsep]
\item \code{spacename}\\
The name of the space as a string or symbol.

\item \code{key}\\
The key for the operation where \code{key} is a Javascript value.

\item \code{attributes}\\
The set of attributes to modify and their respective values.  \code{attrs}
points to an array of length \code{attrs\_sz}.

\end{itemize}

\paragraph{Returns:}
True if the operation succeeded.  False if any provided predicates failed.
Raises an exception on error.


\pagebreak
\subsubsection{\code{async\_put}}
\label{api:ruby:async_put}
\index{async\_put!Ruby API}
Store or update an object by key.  Existing values will be overwritten with the
values specified by \code{attrs}.  Values not specified by \code{attrs} will be
preserved.

%%% Generated below here
\paragraph{Behavior:}
\begin{itemize}[noitemsep]
\item An existing object will be updated by the operation.  If no object does
    exists, a new object will be created, with attributes initialized to their
    default values.

\end{itemize}


\paragraph{Definition:}
\begin{rubycode}
async_put(spacename, key, attributes)
\end{rubycode}

\paragraph{Parameters:}
\begin{itemize}[noitemsep]
\item \code{spacename}\\
The name of the space as a string or symbol.

\item \code{key}\\
The key for the operation where \code{key} is a Javascript value.

\item \code{attributes}\\
The set of attributes to modify and their respective values.  \code{attrs}
points to an array of length \code{attrs\_sz}.

\end{itemize}

\paragraph{Returns:}
A Deferred object with a \code{wait} method that returns True if the operation
succeeded or False if any provided predicates failed.  Raises an exception on
error.


\paragraph{See also:}  This is the asynchronous form of \code{put}.

%%%%%%%%%%%%%%%%%%%% cond_put %%%%%%%%%%%%%%%%%%%%
\pagebreak
\subsubsection{\code{cond\_put}}
\label{api:ruby:cond_put}
\index{cond\_put!Ruby API}
Conditionally store or update an object by key.  Existing values will be
overwritten with the values specified by \code{attrs}.  Values not specified by
\code{attrs} will be preserved.

%%% Generated below here
\paragraph{Behavior:}
\begin{itemize}[noitemsep]
\item This operation manipulates an existing object.  If no object exists, the
    operation will fail with \code{NOTFOUND}.

\item This operation will succeed if and only if the predicates specified by
    \code{checks} hold on the pre-existing object.  If any of the predicates are
    not true for the existing object, then the operation will have no effect and
    fail with \code{CMPFAIL}.

    All checks are atomic with the write.  HyperDex guarantees that no other
    operation will come between validating the checks, and writing the new
    version of the object..

\end{itemize}


\paragraph{Definition:}
\begin{rubycode}
cond_put(spacename, key, predicates, attributes)
\end{rubycode}

\paragraph{Parameters:}
\begin{itemize}[noitemsep]
\item \code{spacename}\\
The name of the space as a string or symbol.

\item \code{key}\\
The key for the operation where \code{key} is a Javascript value.

\item \code{predicates}\\
A set of predicates to check against.  \code{checks} points to an array of
length \code{checks\_sz}.

\item \code{attributes}\\
The set of attributes to modify and their respective values.  \code{attrs}
points to an array of length \code{attrs\_sz}.

\end{itemize}

\paragraph{Returns:}
True if the operation succeeded.  False if any provided predicates failed.
Raises an exception on error.


\pagebreak
\subsubsection{\code{async\_cond\_put}}
\label{api:ruby:async_cond_put}
\index{async\_cond\_put!Ruby API}
Conditionally store or update an object by key.  Existing values will be
overwritten with the values specified by \code{attrs}.  Values not specified by
\code{attrs} will be preserved.

%%% Generated below here
\paragraph{Behavior:}
\begin{itemize}[noitemsep]
\item This operation manipulates an existing object.  If no object exists, the
    operation will fail with \code{NOTFOUND}.

\item This operation will succeed if and only if the predicates specified by
    \code{checks} hold on the pre-existing object.  If any of the predicates are
    not true for the existing object, then the operation will have no effect and
    fail with \code{CMPFAIL}.

    All checks are atomic with the write.  HyperDex guarantees that no other
    operation will come between validating the checks, and writing the new
    version of the object..

\end{itemize}


\paragraph{Definition:}
\begin{rubycode}
async_cond_put(spacename, key, predicates, attributes)
\end{rubycode}

\paragraph{Parameters:}
\begin{itemize}[noitemsep]
\item \code{spacename}\\
The name of the space as a string or symbol.

\item \code{key}\\
The key for the operation where \code{key} is a Javascript value.

\item \code{predicates}\\
A set of predicates to check against.  \code{checks} points to an array of
length \code{checks\_sz}.

\item \code{attributes}\\
The set of attributes to modify and their respective values.  \code{attrs}
points to an array of length \code{attrs\_sz}.

\end{itemize}

\paragraph{Returns:}
A Deferred object with a \code{wait} method that returns True if the operation
succeeded or False if any provided predicates failed.  Raises an exception on
error.


\paragraph{See also:}  This is the asynchronous form of \code{cond\_put}.

%%%%%%%%%%%%%%%%%%%% put_if_not_exist %%%%%%%%%%%%%%%%%%%%
\pagebreak
\subsubsection{\code{put\_if\_not\_exist}}
\label{api:ruby:put_if_not_exist}
\index{put\_if\_not\_exist!Ruby API}
Store a new object by key.  Values not specified by \code{attrs} will be
initialized to their defaults.

%%% Generated below here
\paragraph{Behavior:}
\begin{itemize}[noitemsep]
\item This operation creates a new object.  If an object exists, the operation
    will fail with \code{CMPFAIL}.

\end{itemize}


\paragraph{Definition:}
\begin{rubycode}
put_if_not_exist(spacename, key, attributes)
\end{rubycode}

\paragraph{Parameters:}
\begin{itemize}[noitemsep]
\item \code{spacename}\\
The name of the space as a string or symbol.

\item \code{key}\\
The key for the operation where \code{key} is a Javascript value.

\item \code{attributes}\\
The set of attributes to modify and their respective values.  \code{attrs}
points to an array of length \code{attrs\_sz}.

\end{itemize}

\paragraph{Returns:}
True if the operation succeeded.  False if any provided predicates failed.
Raises an exception on error.


\pagebreak
\subsubsection{\code{async\_put\_if\_not\_exist}}
\label{api:ruby:async_put_if_not_exist}
\index{async\_put\_if\_not\_exist!Ruby API}
Store a new object by key.  Values not specified by \code{attrs} will be
initialized to their defaults.

%%% Generated below here
\paragraph{Behavior:}
\begin{itemize}[noitemsep]
\item This operation creates a new object.  If an object exists, the operation
    will fail with \code{CMPFAIL}.

\end{itemize}


\paragraph{Definition:}
\begin{rubycode}
async_put_if_not_exist(spacename, key, attributes)
\end{rubycode}

\paragraph{Parameters:}
\begin{itemize}[noitemsep]
\item \code{spacename}\\
The name of the space as a string or symbol.

\item \code{key}\\
The key for the operation where \code{key} is a Javascript value.

\item \code{attributes}\\
The set of attributes to modify and their respective values.  \code{attrs}
points to an array of length \code{attrs\_sz}.

\end{itemize}

\paragraph{Returns:}
A Deferred object with a \code{wait} method that returns True if the operation
succeeded or False if any provided predicates failed.  Raises an exception on
error.


\paragraph{See also:}  This is the asynchronous form of \code{put\_if\_not\_exist}.

%%%%%%%%%%%%%%%%%%%% del %%%%%%%%%%%%%%%%%%%%
\pagebreak
\subsubsection{\code{del}}
\label{api:ruby:del}
\index{del!Ruby API}
Delete an object by key.

%%% Generated below here
\paragraph{Behavior:}
\begin{itemize}[noitemsep]
\item An existing object stored under the given key will be erased.  If no
    object exists, the operation will fail and report \code{NOTFOUND}.

\end{itemize}


\paragraph{Definition:}
\begin{rubycode}
del(spacename, key)
\end{rubycode}

\paragraph{Parameters:}
\begin{itemize}[noitemsep]
\item \code{spacename}\\
The name of the space as a string or symbol.

\item \code{key}\\
The key for the operation where \code{key} is a Javascript value.

\end{itemize}

\paragraph{Returns:}
True if the operation succeeded.  False if any provided predicates failed.
Raises an exception on error.


\pagebreak
\subsubsection{\code{async\_del}}
\label{api:ruby:async_del}
\index{async\_del!Ruby API}
Delete an object by key.

%%% Generated below here
\paragraph{Behavior:}
\begin{itemize}[noitemsep]
\item An existing object stored under the given key will be erased.  If no
    object exists, the operation will fail and report \code{NOTFOUND}.

\end{itemize}


\paragraph{Definition:}
\begin{rubycode}
async_del(spacename, key)
\end{rubycode}

\paragraph{Parameters:}
\begin{itemize}[noitemsep]
\item \code{spacename}\\
The name of the space as a string or symbol.

\item \code{key}\\
The key for the operation where \code{key} is a Javascript value.

\end{itemize}

\paragraph{Returns:}
A Deferred object with a \code{wait} method that returns True if the operation
succeeded or False if any provided predicates failed.  Raises an exception on
error.


\paragraph{See also:}  This is the asynchronous form of \code{del}.

%%%%%%%%%%%%%%%%%%%% cond_del %%%%%%%%%%%%%%%%%%%%
\pagebreak
\subsubsection{\code{cond\_del}}
\label{api:ruby:cond_del}
\index{cond\_del!Ruby API}
Conditionally delete an object by key.

%%% Generated below here
\paragraph{Behavior:}
\begin{itemize}[noitemsep]
\item An existing object stored under the given key will be erased.  If no
    object exists, the operation will fail and report \code{NOTFOUND}.

\item This operation will succeed if and only if the predicates specified by
    \code{checks} hold on the pre-existing object.  If any of the predicates are
    not true for the existing object, then the operation will have no effect and
    fail with \code{CMPFAIL}.

    All checks are atomic with the write.  HyperDex guarantees that no other
    operation will come between validating the checks, and writing the new
    version of the object..

\end{itemize}


\paragraph{Definition:}
\begin{rubycode}
cond_del(spacename, key, predicates)
\end{rubycode}

\paragraph{Parameters:}
\begin{itemize}[noitemsep]
\item \code{spacename}\\
The name of the space as a string or symbol.

\item \code{key}\\
The key for the operation where \code{key} is a Javascript value.

\item \code{predicates}\\
A set of predicates to check against.  \code{checks} points to an array of
length \code{checks\_sz}.

\end{itemize}

\paragraph{Returns:}
True if the operation succeeded.  False if any provided predicates failed.
Raises an exception on error.


\pagebreak
\subsubsection{\code{async\_cond\_del}}
\label{api:ruby:async_cond_del}
\index{async\_cond\_del!Ruby API}
Conditionally delete an object by key.

%%% Generated below here
\paragraph{Behavior:}
\begin{itemize}[noitemsep]
\item An existing object stored under the given key will be erased.  If no
    object exists, the operation will fail and report \code{NOTFOUND}.

\item This operation will succeed if and only if the predicates specified by
    \code{checks} hold on the pre-existing object.  If any of the predicates are
    not true for the existing object, then the operation will have no effect and
    fail with \code{CMPFAIL}.

    All checks are atomic with the write.  HyperDex guarantees that no other
    operation will come between validating the checks, and writing the new
    version of the object..

\end{itemize}


\paragraph{Definition:}
\begin{rubycode}
async_cond_del(spacename, key, predicates)
\end{rubycode}

\paragraph{Parameters:}
\begin{itemize}[noitemsep]
\item \code{spacename}\\
The name of the space as a string or symbol.

\item \code{key}\\
The key for the operation where \code{key} is a Javascript value.

\item \code{predicates}\\
A set of predicates to check against.  \code{checks} points to an array of
length \code{checks\_sz}.

\end{itemize}

\paragraph{Returns:}
A Deferred object with a \code{wait} method that returns True if the operation
succeeded or False if any provided predicates failed.  Raises an exception on
error.


\paragraph{See also:}  This is the asynchronous form of \code{cond\_del}.

%%%%%%%%%%%%%%%%%%%% atomic_add %%%%%%%%%%%%%%%%%%%%
\pagebreak
\subsubsection{\code{atomic\_add}}
\label{api:ruby:atomic_add}
\index{atomic\_add!Ruby API}
Add the specified number to the existing value for each attribute.

%%% Generated below here
\paragraph{Behavior:}
\begin{itemize}[noitemsep]
\item This operation manipulates an existing object.  If no object exists, the
    operation will fail with \code{NOTFOUND}.

\end{itemize}


\paragraph{Definition:}
\begin{rubycode}
atomic_add(spacename, key, attributes)
\end{rubycode}

\paragraph{Parameters:}
\begin{itemize}[noitemsep]
\item \code{spacename}\\
The name of the space as a string or symbol.

\item \code{key}\\
The key for the operation where \code{key} is a Javascript value.

\item \code{attributes}\\
The set of attributes to modify and their respective values.  \code{attrs}
points to an array of length \code{attrs\_sz}.

\end{itemize}

\paragraph{Returns:}
True if the operation succeeded.  False if any provided predicates failed.
Raises an exception on error.


\pagebreak
\subsubsection{\code{async\_atomic\_add}}
\label{api:ruby:async_atomic_add}
\index{async\_atomic\_add!Ruby API}
Add the specified number to the existing value for each attribute.

%%% Generated below here
\paragraph{Behavior:}
\begin{itemize}[noitemsep]
\item This operation manipulates an existing object.  If no object exists, the
    operation will fail with \code{NOTFOUND}.

\end{itemize}


\paragraph{Definition:}
\begin{rubycode}
async_atomic_add(spacename, key, attributes)
\end{rubycode}

\paragraph{Parameters:}
\begin{itemize}[noitemsep]
\item \code{spacename}\\
The name of the space as a string or symbol.

\item \code{key}\\
The key for the operation where \code{key} is a Javascript value.

\item \code{attributes}\\
The set of attributes to modify and their respective values.  \code{attrs}
points to an array of length \code{attrs\_sz}.

\end{itemize}

\paragraph{Returns:}
A Deferred object with a \code{wait} method that returns True if the operation
succeeded or False if any provided predicates failed.  Raises an exception on
error.


\paragraph{See also:}  This is the asynchronous form of \code{atomic\_add}.

%%%%%%%%%%%%%%%%%%%% cond_atomic_add %%%%%%%%%%%%%%%%%%%%
\pagebreak
\subsubsection{\code{cond\_atomic\_add}}
\label{api:ruby:cond_atomic_add}
\index{cond\_atomic\_add!Ruby API}
Conditionally add the specified number to the existing value for each attribute.

%%% Generated below here
\paragraph{Behavior:}
\begin{itemize}[noitemsep]
\item This operation manipulates an existing object.  If no object exists, the
    operation will fail with \code{NOTFOUND}.

\item This operation will succeed if and only if the predicates specified by
    \code{checks} hold on the pre-existing object.  If any of the predicates are
    not true for the existing object, then the operation will have no effect and
    fail with \code{CMPFAIL}.

    All checks are atomic with the write.  HyperDex guarantees that no other
    operation will come between validating the checks, and writing the new
    version of the object..

\end{itemize}


\paragraph{Definition:}
\begin{rubycode}
cond_atomic_add(spacename, key, predicates, attributes)
\end{rubycode}

\paragraph{Parameters:}
\begin{itemize}[noitemsep]
\item \code{spacename}\\
The name of the space as a string or symbol.

\item \code{key}\\
The key for the operation where \code{key} is a Javascript value.

\item \code{predicates}\\
A set of predicates to check against.  \code{checks} points to an array of
length \code{checks\_sz}.

\item \code{attributes}\\
The set of attributes to modify and their respective values.  \code{attrs}
points to an array of length \code{attrs\_sz}.

\end{itemize}

\paragraph{Returns:}
True if the operation succeeded.  False if any provided predicates failed.
Raises an exception on error.


\pagebreak
\subsubsection{\code{async\_cond\_atomic\_add}}
\label{api:ruby:async_cond_atomic_add}
\index{async\_cond\_atomic\_add!Ruby API}
Conditionally add the specified number to the existing value for each attribute.

%%% Generated below here
\paragraph{Behavior:}
\begin{itemize}[noitemsep]
\item This operation manipulates an existing object.  If no object exists, the
    operation will fail with \code{NOTFOUND}.

\item This operation will succeed if and only if the predicates specified by
    \code{checks} hold on the pre-existing object.  If any of the predicates are
    not true for the existing object, then the operation will have no effect and
    fail with \code{CMPFAIL}.

    All checks are atomic with the write.  HyperDex guarantees that no other
    operation will come between validating the checks, and writing the new
    version of the object..

\end{itemize}


\paragraph{Definition:}
\begin{rubycode}
async_cond_atomic_add(spacename, key, predicates, attributes)
\end{rubycode}

\paragraph{Parameters:}
\begin{itemize}[noitemsep]
\item \code{spacename}\\
The name of the space as a string or symbol.

\item \code{key}\\
The key for the operation where \code{key} is a Javascript value.

\item \code{predicates}\\
A set of predicates to check against.  \code{checks} points to an array of
length \code{checks\_sz}.

\item \code{attributes}\\
The set of attributes to modify and their respective values.  \code{attrs}
points to an array of length \code{attrs\_sz}.

\end{itemize}

\paragraph{Returns:}
A Deferred object with a \code{wait} method that returns True if the operation
succeeded or False if any provided predicates failed.  Raises an exception on
error.


\paragraph{See also:}  This is the asynchronous form of \code{cond\_atomic\_add}.

%%%%%%%%%%%%%%%%%%%% atomic_sub %%%%%%%%%%%%%%%%%%%%
\pagebreak
\subsubsection{\code{atomic\_sub}}
\label{api:ruby:atomic_sub}
\index{atomic\_sub!Ruby API}
Subtract the specified number from the existing value for each attribute.

%%% Generated below here
\paragraph{Behavior:}
\begin{itemize}[noitemsep]
\item This operation manipulates an existing object.  If no object exists, the
    operation will fail with \code{NOTFOUND}.

\end{itemize}


\paragraph{Definition:}
\begin{rubycode}
atomic_sub(spacename, key, attributes)
\end{rubycode}

\paragraph{Parameters:}
\begin{itemize}[noitemsep]
\item \code{spacename}\\
The name of the space as a string or symbol.

\item \code{key}\\
The key for the operation where \code{key} is a Javascript value.

\item \code{attributes}\\
The set of attributes to modify and their respective values.  \code{attrs}
points to an array of length \code{attrs\_sz}.

\end{itemize}

\paragraph{Returns:}
True if the operation succeeded.  False if any provided predicates failed.
Raises an exception on error.


\pagebreak
\subsubsection{\code{async\_atomic\_sub}}
\label{api:ruby:async_atomic_sub}
\index{async\_atomic\_sub!Ruby API}
Subtract the specified number from the existing value for each attribute.

%%% Generated below here
\paragraph{Behavior:}
\begin{itemize}[noitemsep]
\item This operation manipulates an existing object.  If no object exists, the
    operation will fail with \code{NOTFOUND}.

\end{itemize}


\paragraph{Definition:}
\begin{rubycode}
async_atomic_sub(spacename, key, attributes)
\end{rubycode}

\paragraph{Parameters:}
\begin{itemize}[noitemsep]
\item \code{spacename}\\
The name of the space as a string or symbol.

\item \code{key}\\
The key for the operation where \code{key} is a Javascript value.

\item \code{attributes}\\
The set of attributes to modify and their respective values.  \code{attrs}
points to an array of length \code{attrs\_sz}.

\end{itemize}

\paragraph{Returns:}
A Deferred object with a \code{wait} method that returns True if the operation
succeeded or False if any provided predicates failed.  Raises an exception on
error.


\paragraph{See also:}  This is the asynchronous form of \code{atomic\_sub}.

%%%%%%%%%%%%%%%%%%%% cond_atomic_sub %%%%%%%%%%%%%%%%%%%%
\pagebreak
\subsubsection{\code{cond\_atomic\_sub}}
\label{api:ruby:cond_atomic_sub}
\index{cond\_atomic\_sub!Ruby API}
Conditionally subtract the specified number from the existing value for each attribute.

%%% Generated below here
\paragraph{Behavior:}
\begin{itemize}[noitemsep]
\item This operation manipulates an existing object.  If no object exists, the
    operation will fail with \code{NOTFOUND}.

\item This operation will succeed if and only if the predicates specified by
    \code{checks} hold on the pre-existing object.  If any of the predicates are
    not true for the existing object, then the operation will have no effect and
    fail with \code{CMPFAIL}.

    All checks are atomic with the write.  HyperDex guarantees that no other
    operation will come between validating the checks, and writing the new
    version of the object..

\end{itemize}


\paragraph{Definition:}
\begin{rubycode}
cond_atomic_sub(spacename, key, predicates, attributes)
\end{rubycode}

\paragraph{Parameters:}
\begin{itemize}[noitemsep]
\item \code{spacename}\\
The name of the space as a string or symbol.

\item \code{key}\\
The key for the operation where \code{key} is a Javascript value.

\item \code{predicates}\\
A set of predicates to check against.  \code{checks} points to an array of
length \code{checks\_sz}.

\item \code{attributes}\\
The set of attributes to modify and their respective values.  \code{attrs}
points to an array of length \code{attrs\_sz}.

\end{itemize}

\paragraph{Returns:}
True if the operation succeeded.  False if any provided predicates failed.
Raises an exception on error.


\pagebreak
\subsubsection{\code{async\_cond\_atomic\_sub}}
\label{api:ruby:async_cond_atomic_sub}
\index{async\_cond\_atomic\_sub!Ruby API}
Conditionally subtract the specified number from the existing value for each attribute.

%%% Generated below here
\paragraph{Behavior:}
\begin{itemize}[noitemsep]
\item This operation manipulates an existing object.  If no object exists, the
    operation will fail with \code{NOTFOUND}.

\item This operation will succeed if and only if the predicates specified by
    \code{checks} hold on the pre-existing object.  If any of the predicates are
    not true for the existing object, then the operation will have no effect and
    fail with \code{CMPFAIL}.

    All checks are atomic with the write.  HyperDex guarantees that no other
    operation will come between validating the checks, and writing the new
    version of the object..

\end{itemize}


\paragraph{Definition:}
\begin{rubycode}
async_cond_atomic_sub(spacename, key, predicates, attributes)
\end{rubycode}

\paragraph{Parameters:}
\begin{itemize}[noitemsep]
\item \code{spacename}\\
The name of the space as a string or symbol.

\item \code{key}\\
The key for the operation where \code{key} is a Javascript value.

\item \code{predicates}\\
A set of predicates to check against.  \code{checks} points to an array of
length \code{checks\_sz}.

\item \code{attributes}\\
The set of attributes to modify and their respective values.  \code{attrs}
points to an array of length \code{attrs\_sz}.

\end{itemize}

\paragraph{Returns:}
A Deferred object with a \code{wait} method that returns True if the operation
succeeded or False if any provided predicates failed.  Raises an exception on
error.


\paragraph{See also:}  This is the asynchronous form of \code{cond\_atomic\_sub}.

%%%%%%%%%%%%%%%%%%%% atomic_mul %%%%%%%%%%%%%%%%%%%%
\pagebreak
\subsubsection{\code{atomic\_mul}}
\label{api:ruby:atomic_mul}
\index{atomic\_mul!Ruby API}
Multiply the existing value by the specified number for each attribute.

%%% Generated below here
\paragraph{Behavior:}
\begin{itemize}[noitemsep]
\item This operation manipulates an existing object.  If no object exists, the
    operation will fail with \code{NOTFOUND}.

\end{itemize}


\paragraph{Definition:}
\begin{rubycode}
atomic_mul(spacename, key, attributes)
\end{rubycode}

\paragraph{Parameters:}
\begin{itemize}[noitemsep]
\item \code{spacename}\\
The name of the space as a string or symbol.

\item \code{key}\\
The key for the operation where \code{key} is a Javascript value.

\item \code{attributes}\\
The set of attributes to modify and their respective values.  \code{attrs}
points to an array of length \code{attrs\_sz}.

\end{itemize}

\paragraph{Returns:}
True if the operation succeeded.  False if any provided predicates failed.
Raises an exception on error.


\pagebreak
\subsubsection{\code{async\_atomic\_mul}}
\label{api:ruby:async_atomic_mul}
\index{async\_atomic\_mul!Ruby API}
Multiply the existing value by the specified number for each attribute.

%%% Generated below here
\paragraph{Behavior:}
\begin{itemize}[noitemsep]
\item This operation manipulates an existing object.  If no object exists, the
    operation will fail with \code{NOTFOUND}.

\end{itemize}


\paragraph{Definition:}
\begin{rubycode}
async_atomic_mul(spacename, key, attributes)
\end{rubycode}

\paragraph{Parameters:}
\begin{itemize}[noitemsep]
\item \code{spacename}\\
The name of the space as a string or symbol.

\item \code{key}\\
The key for the operation where \code{key} is a Javascript value.

\item \code{attributes}\\
The set of attributes to modify and their respective values.  \code{attrs}
points to an array of length \code{attrs\_sz}.

\end{itemize}

\paragraph{Returns:}
A Deferred object with a \code{wait} method that returns True if the operation
succeeded or False if any provided predicates failed.  Raises an exception on
error.


\paragraph{See also:}  This is the asynchronous form of \code{atomic\_mul}.

%%%%%%%%%%%%%%%%%%%% cond_atomic_mul %%%%%%%%%%%%%%%%%%%%
\pagebreak
\subsubsection{\code{cond\_atomic\_mul}}
\label{api:ruby:cond_atomic_mul}
\index{cond\_atomic\_mul!Ruby API}
Conditionally multiply the existing value by the specified number for each
attribute.

%%% Generated below here
\paragraph{Behavior:}
\begin{itemize}[noitemsep]
\item This operation manipulates an existing object.  If no object exists, the
    operation will fail with \code{NOTFOUND}.

\item This operation will succeed if and only if the predicates specified by
    \code{checks} hold on the pre-existing object.  If any of the predicates are
    not true for the existing object, then the operation will have no effect and
    fail with \code{CMPFAIL}.

    All checks are atomic with the write.  HyperDex guarantees that no other
    operation will come between validating the checks, and writing the new
    version of the object..

\end{itemize}


\paragraph{Definition:}
\begin{rubycode}
cond_atomic_mul(spacename, key, predicates, attributes)
\end{rubycode}

\paragraph{Parameters:}
\begin{itemize}[noitemsep]
\item \code{spacename}\\
The name of the space as a string or symbol.

\item \code{key}\\
The key for the operation where \code{key} is a Javascript value.

\item \code{predicates}\\
A set of predicates to check against.  \code{checks} points to an array of
length \code{checks\_sz}.

\item \code{attributes}\\
The set of attributes to modify and their respective values.  \code{attrs}
points to an array of length \code{attrs\_sz}.

\end{itemize}

\paragraph{Returns:}
True if the operation succeeded.  False if any provided predicates failed.
Raises an exception on error.


\pagebreak
\subsubsection{\code{async\_cond\_atomic\_mul}}
\label{api:ruby:async_cond_atomic_mul}
\index{async\_cond\_atomic\_mul!Ruby API}
Conditionally multiply the existing value by the specified number for each
attribute.

%%% Generated below here
\paragraph{Behavior:}
\begin{itemize}[noitemsep]
\item This operation manipulates an existing object.  If no object exists, the
    operation will fail with \code{NOTFOUND}.

\item This operation will succeed if and only if the predicates specified by
    \code{checks} hold on the pre-existing object.  If any of the predicates are
    not true for the existing object, then the operation will have no effect and
    fail with \code{CMPFAIL}.

    All checks are atomic with the write.  HyperDex guarantees that no other
    operation will come between validating the checks, and writing the new
    version of the object..

\end{itemize}


\paragraph{Definition:}
\begin{rubycode}
async_cond_atomic_mul(spacename, key, predicates, attributes)
\end{rubycode}

\paragraph{Parameters:}
\begin{itemize}[noitemsep]
\item \code{spacename}\\
The name of the space as a string or symbol.

\item \code{key}\\
The key for the operation where \code{key} is a Javascript value.

\item \code{predicates}\\
A set of predicates to check against.  \code{checks} points to an array of
length \code{checks\_sz}.

\item \code{attributes}\\
The set of attributes to modify and their respective values.  \code{attrs}
points to an array of length \code{attrs\_sz}.

\end{itemize}

\paragraph{Returns:}
A Deferred object with a \code{wait} method that returns True if the operation
succeeded or False if any provided predicates failed.  Raises an exception on
error.


\paragraph{See also:}  This is the asynchronous form of \code{cond\_atomic\_mul}.

%%%%%%%%%%%%%%%%%%%% atomic_div %%%%%%%%%%%%%%%%%%%%
\pagebreak
\subsubsection{\code{atomic\_div}}
\label{api:ruby:atomic_div}
\index{atomic\_div!Ruby API}
Divide the existing value by the specified number for each attribute.

%%% Generated below here
\paragraph{Behavior:}
\begin{itemize}[noitemsep]
\item This operation manipulates an existing object.  If no object exists, the
    operation will fail with \code{NOTFOUND}.

\end{itemize}


\paragraph{Definition:}
\begin{rubycode}
atomic_div(spacename, key, attributes)
\end{rubycode}

\paragraph{Parameters:}
\begin{itemize}[noitemsep]
\item \code{spacename}\\
The name of the space as a string or symbol.

\item \code{key}\\
The key for the operation where \code{key} is a Javascript value.

\item \code{attributes}\\
The set of attributes to modify and their respective values.  \code{attrs}
points to an array of length \code{attrs\_sz}.

\end{itemize}

\paragraph{Returns:}
True if the operation succeeded.  False if any provided predicates failed.
Raises an exception on error.


\pagebreak
\subsubsection{\code{async\_atomic\_div}}
\label{api:ruby:async_atomic_div}
\index{async\_atomic\_div!Ruby API}
Divide the existing value by the specified number for each attribute.

%%% Generated below here
\paragraph{Behavior:}
\begin{itemize}[noitemsep]
\item This operation manipulates an existing object.  If no object exists, the
    operation will fail with \code{NOTFOUND}.

\end{itemize}


\paragraph{Definition:}
\begin{rubycode}
async_atomic_div(spacename, key, attributes)
\end{rubycode}

\paragraph{Parameters:}
\begin{itemize}[noitemsep]
\item \code{spacename}\\
The name of the space as a string or symbol.

\item \code{key}\\
The key for the operation where \code{key} is a Javascript value.

\item \code{attributes}\\
The set of attributes to modify and their respective values.  \code{attrs}
points to an array of length \code{attrs\_sz}.

\end{itemize}

\paragraph{Returns:}
A Deferred object with a \code{wait} method that returns True if the operation
succeeded or False if any provided predicates failed.  Raises an exception on
error.


\paragraph{See also:}  This is the asynchronous form of \code{atomic\_div}.

%%%%%%%%%%%%%%%%%%%% cond_atomic_div %%%%%%%%%%%%%%%%%%%%
\pagebreak
\subsubsection{\code{cond\_atomic\_div}}
\label{api:ruby:cond_atomic_div}
\index{cond\_atomic\_div!Ruby API}
Conditionally divide the existing value by the specified number for each
attribute.

%%% Generated below here
\paragraph{Behavior:}
\begin{itemize}[noitemsep]
\item This operation manipulates an existing object.  If no object exists, the
    operation will fail with \code{NOTFOUND}.

\item This operation will succeed if and only if the predicates specified by
    \code{checks} hold on the pre-existing object.  If any of the predicates are
    not true for the existing object, then the operation will have no effect and
    fail with \code{CMPFAIL}.

    All checks are atomic with the write.  HyperDex guarantees that no other
    operation will come between validating the checks, and writing the new
    version of the object..

\end{itemize}


\paragraph{Definition:}
\begin{rubycode}
cond_atomic_div(spacename, key, predicates, attributes)
\end{rubycode}

\paragraph{Parameters:}
\begin{itemize}[noitemsep]
\item \code{spacename}\\
The name of the space as a string or symbol.

\item \code{key}\\
The key for the operation where \code{key} is a Javascript value.

\item \code{predicates}\\
A set of predicates to check against.  \code{checks} points to an array of
length \code{checks\_sz}.

\item \code{attributes}\\
The set of attributes to modify and their respective values.  \code{attrs}
points to an array of length \code{attrs\_sz}.

\end{itemize}

\paragraph{Returns:}
True if the operation succeeded.  False if any provided predicates failed.
Raises an exception on error.


\pagebreak
\subsubsection{\code{async\_cond\_atomic\_div}}
\label{api:ruby:async_cond_atomic_div}
\index{async\_cond\_atomic\_div!Ruby API}
Conditionally divide the existing value by the specified number for each
attribute.

%%% Generated below here
\paragraph{Behavior:}
\begin{itemize}[noitemsep]
\item This operation manipulates an existing object.  If no object exists, the
    operation will fail with \code{NOTFOUND}.

\item This operation will succeed if and only if the predicates specified by
    \code{checks} hold on the pre-existing object.  If any of the predicates are
    not true for the existing object, then the operation will have no effect and
    fail with \code{CMPFAIL}.

    All checks are atomic with the write.  HyperDex guarantees that no other
    operation will come between validating the checks, and writing the new
    version of the object..

\end{itemize}


\paragraph{Definition:}
\begin{rubycode}
async_cond_atomic_div(spacename, key, predicates, attributes)
\end{rubycode}

\paragraph{Parameters:}
\begin{itemize}[noitemsep]
\item \code{spacename}\\
The name of the space as a string or symbol.

\item \code{key}\\
The key for the operation where \code{key} is a Javascript value.

\item \code{predicates}\\
A set of predicates to check against.  \code{checks} points to an array of
length \code{checks\_sz}.

\item \code{attributes}\\
The set of attributes to modify and their respective values.  \code{attrs}
points to an array of length \code{attrs\_sz}.

\end{itemize}

\paragraph{Returns:}
A Deferred object with a \code{wait} method that returns True if the operation
succeeded or False if any provided predicates failed.  Raises an exception on
error.


\paragraph{See also:}  This is the asynchronous form of \code{cond\_atomic\_div}.

%%%%%%%%%%%%%%%%%%%% atomic_mod %%%%%%%%%%%%%%%%%%%%
\pagebreak
\subsubsection{\code{atomic\_mod}}
\label{api:ruby:atomic_mod}
\index{atomic\_mod!Ruby API}
Store the existing value modulo the specified number for each attribute.

%%% Generated below here
\paragraph{Behavior:}
\begin{itemize}[noitemsep]
\item This operation manipulates an existing object.  If no object exists, the
    operation will fail with \code{NOTFOUND}.

\end{itemize}


\paragraph{Definition:}
\begin{rubycode}
atomic_mod(spacename, key, attributes)
\end{rubycode}

\paragraph{Parameters:}
\begin{itemize}[noitemsep]
\item \code{spacename}\\
The name of the space as a string or symbol.

\item \code{key}\\
The key for the operation where \code{key} is a Javascript value.

\item \code{attributes}\\
The set of attributes to modify and their respective values.  \code{attrs}
points to an array of length \code{attrs\_sz}.

\end{itemize}

\paragraph{Returns:}
True if the operation succeeded.  False if any provided predicates failed.
Raises an exception on error.


\pagebreak
\subsubsection{\code{async\_atomic\_mod}}
\label{api:ruby:async_atomic_mod}
\index{async\_atomic\_mod!Ruby API}
Store the existing value modulo the specified number for each attribute.

%%% Generated below here
\paragraph{Behavior:}
\begin{itemize}[noitemsep]
\item This operation manipulates an existing object.  If no object exists, the
    operation will fail with \code{NOTFOUND}.

\end{itemize}


\paragraph{Definition:}
\begin{rubycode}
async_atomic_mod(spacename, key, attributes)
\end{rubycode}

\paragraph{Parameters:}
\begin{itemize}[noitemsep]
\item \code{spacename}\\
The name of the space as a string or symbol.

\item \code{key}\\
The key for the operation where \code{key} is a Javascript value.

\item \code{attributes}\\
The set of attributes to modify and their respective values.  \code{attrs}
points to an array of length \code{attrs\_sz}.

\end{itemize}

\paragraph{Returns:}
A Deferred object with a \code{wait} method that returns True if the operation
succeeded or False if any provided predicates failed.  Raises an exception on
error.


\paragraph{See also:}  This is the asynchronous form of \code{atomic\_mod}.

%%%%%%%%%%%%%%%%%%%% cond_atomic_mod %%%%%%%%%%%%%%%%%%%%
\pagebreak
\subsubsection{\code{cond\_atomic\_mod}}
\label{api:ruby:cond_atomic_mod}
\index{cond\_atomic\_mod!Ruby API}
Conditionally store the existing value modulo the specified number for each
attribute.

%%% Generated below here
\paragraph{Behavior:}
\begin{itemize}[noitemsep]
\item This operation manipulates an existing object.  If no object exists, the
    operation will fail with \code{NOTFOUND}.

\item This operation will succeed if and only if the predicates specified by
    \code{checks} hold on the pre-existing object.  If any of the predicates are
    not true for the existing object, then the operation will have no effect and
    fail with \code{CMPFAIL}.

    All checks are atomic with the write.  HyperDex guarantees that no other
    operation will come between validating the checks, and writing the new
    version of the object..

\end{itemize}


\paragraph{Definition:}
\begin{rubycode}
cond_atomic_mod(spacename, key, predicates, attributes)
\end{rubycode}

\paragraph{Parameters:}
\begin{itemize}[noitemsep]
\item \code{spacename}\\
The name of the space as a string or symbol.

\item \code{key}\\
The key for the operation where \code{key} is a Javascript value.

\item \code{predicates}\\
A set of predicates to check against.  \code{checks} points to an array of
length \code{checks\_sz}.

\item \code{attributes}\\
The set of attributes to modify and their respective values.  \code{attrs}
points to an array of length \code{attrs\_sz}.

\end{itemize}

\paragraph{Returns:}
True if the operation succeeded.  False if any provided predicates failed.
Raises an exception on error.


\pagebreak
\subsubsection{\code{async\_cond\_atomic\_mod}}
\label{api:ruby:async_cond_atomic_mod}
\index{async\_cond\_atomic\_mod!Ruby API}
Conditionally store the existing value modulo the specified number for each
attribute.

%%% Generated below here
\paragraph{Behavior:}
\begin{itemize}[noitemsep]
\item This operation manipulates an existing object.  If no object exists, the
    operation will fail with \code{NOTFOUND}.

\item This operation will succeed if and only if the predicates specified by
    \code{checks} hold on the pre-existing object.  If any of the predicates are
    not true for the existing object, then the operation will have no effect and
    fail with \code{CMPFAIL}.

    All checks are atomic with the write.  HyperDex guarantees that no other
    operation will come between validating the checks, and writing the new
    version of the object..

\end{itemize}


\paragraph{Definition:}
\begin{rubycode}
async_cond_atomic_mod(spacename, key, predicates, attributes)
\end{rubycode}

\paragraph{Parameters:}
\begin{itemize}[noitemsep]
\item \code{spacename}\\
The name of the space as a string or symbol.

\item \code{key}\\
The key for the operation where \code{key} is a Javascript value.

\item \code{predicates}\\
A set of predicates to check against.  \code{checks} points to an array of
length \code{checks\_sz}.

\item \code{attributes}\\
The set of attributes to modify and their respective values.  \code{attrs}
points to an array of length \code{attrs\_sz}.

\end{itemize}

\paragraph{Returns:}
A Deferred object with a \code{wait} method that returns True if the operation
succeeded or False if any provided predicates failed.  Raises an exception on
error.


\paragraph{See also:}  This is the asynchronous form of \code{cond\_atomic\_mod}.

%%%%%%%%%%%%%%%%%%%% atomic_and %%%%%%%%%%%%%%%%%%%%
\pagebreak
\subsubsection{\code{atomic\_and}}
\label{api:ruby:atomic_and}
\index{atomic\_and!Ruby API}
Store the bitwise AND of the existing value and the specified number for
each attribute.

%%% Generated below here
\paragraph{Behavior:}
\begin{itemize}[noitemsep]
\item This operation manipulates an existing object.  If no object exists, the
    operation will fail with \code{NOTFOUND}.

\end{itemize}


\paragraph{Definition:}
\begin{rubycode}
atomic_and(spacename, key, attributes)
\end{rubycode}

\paragraph{Parameters:}
\begin{itemize}[noitemsep]
\item \code{spacename}\\
The name of the space as a string or symbol.

\item \code{key}\\
The key for the operation where \code{key} is a Javascript value.

\item \code{attributes}\\
The set of attributes to modify and their respective values.  \code{attrs}
points to an array of length \code{attrs\_sz}.

\end{itemize}

\paragraph{Returns:}
True if the operation succeeded.  False if any provided predicates failed.
Raises an exception on error.


\pagebreak
\subsubsection{\code{async\_atomic\_and}}
\label{api:ruby:async_atomic_and}
\index{async\_atomic\_and!Ruby API}
Store the bitwise AND of the existing value and the specified number for
each attribute.

%%% Generated below here
\paragraph{Behavior:}
\begin{itemize}[noitemsep]
\item This operation manipulates an existing object.  If no object exists, the
    operation will fail with \code{NOTFOUND}.

\end{itemize}


\paragraph{Definition:}
\begin{rubycode}
async_atomic_and(spacename, key, attributes)
\end{rubycode}

\paragraph{Parameters:}
\begin{itemize}[noitemsep]
\item \code{spacename}\\
The name of the space as a string or symbol.

\item \code{key}\\
The key for the operation where \code{key} is a Javascript value.

\item \code{attributes}\\
The set of attributes to modify and their respective values.  \code{attrs}
points to an array of length \code{attrs\_sz}.

\end{itemize}

\paragraph{Returns:}
A Deferred object with a \code{wait} method that returns True if the operation
succeeded or False if any provided predicates failed.  Raises an exception on
error.


\paragraph{See also:}  This is the asynchronous form of \code{atomic\_and}.

%%%%%%%%%%%%%%%%%%%% cond_atomic_and %%%%%%%%%%%%%%%%%%%%
\pagebreak
\subsubsection{\code{cond\_atomic\_and}}
\label{api:ruby:cond_atomic_and}
\index{cond\_atomic\_and!Ruby API}
Conditionally store the bitwise AND of the existing value and the specified
number for each attribute.

%%% Generated below here
\paragraph{Behavior:}
\begin{itemize}[noitemsep]
\item This operation manipulates an existing object.  If no object exists, the
    operation will fail with \code{NOTFOUND}.

\item This operation will succeed if and only if the predicates specified by
    \code{checks} hold on the pre-existing object.  If any of the predicates are
    not true for the existing object, then the operation will have no effect and
    fail with \code{CMPFAIL}.

    All checks are atomic with the write.  HyperDex guarantees that no other
    operation will come between validating the checks, and writing the new
    version of the object..

\end{itemize}


\paragraph{Definition:}
\begin{rubycode}
cond_atomic_and(spacename, key, predicates, attributes)
\end{rubycode}

\paragraph{Parameters:}
\begin{itemize}[noitemsep]
\item \code{spacename}\\
The name of the space as a string or symbol.

\item \code{key}\\
The key for the operation where \code{key} is a Javascript value.

\item \code{predicates}\\
A set of predicates to check against.  \code{checks} points to an array of
length \code{checks\_sz}.

\item \code{attributes}\\
The set of attributes to modify and their respective values.  \code{attrs}
points to an array of length \code{attrs\_sz}.

\end{itemize}

\paragraph{Returns:}
True if the operation succeeded.  False if any provided predicates failed.
Raises an exception on error.


\pagebreak
\subsubsection{\code{async\_cond\_atomic\_and}}
\label{api:ruby:async_cond_atomic_and}
\index{async\_cond\_atomic\_and!Ruby API}
Conditionally store the bitwise AND of the existing value and the specified
number for each attribute.

%%% Generated below here
\paragraph{Behavior:}
\begin{itemize}[noitemsep]
\item This operation manipulates an existing object.  If no object exists, the
    operation will fail with \code{NOTFOUND}.

\item This operation will succeed if and only if the predicates specified by
    \code{checks} hold on the pre-existing object.  If any of the predicates are
    not true for the existing object, then the operation will have no effect and
    fail with \code{CMPFAIL}.

    All checks are atomic with the write.  HyperDex guarantees that no other
    operation will come between validating the checks, and writing the new
    version of the object..

\end{itemize}


\paragraph{Definition:}
\begin{rubycode}
async_cond_atomic_and(spacename, key, predicates, attributes)
\end{rubycode}

\paragraph{Parameters:}
\begin{itemize}[noitemsep]
\item \code{spacename}\\
The name of the space as a string or symbol.

\item \code{key}\\
The key for the operation where \code{key} is a Javascript value.

\item \code{predicates}\\
A set of predicates to check against.  \code{checks} points to an array of
length \code{checks\_sz}.

\item \code{attributes}\\
The set of attributes to modify and their respective values.  \code{attrs}
points to an array of length \code{attrs\_sz}.

\end{itemize}

\paragraph{Returns:}
A Deferred object with a \code{wait} method that returns True if the operation
succeeded or False if any provided predicates failed.  Raises an exception on
error.


\paragraph{See also:}  This is the asynchronous form of \code{cond\_atomic\_and}.

%%%%%%%%%%%%%%%%%%%% atomic_or %%%%%%%%%%%%%%%%%%%%
\pagebreak
\subsubsection{\code{atomic\_or}}
\label{api:ruby:atomic_or}
\index{atomic\_or!Ruby API}
Store the bitwise OR of the existing value and the specified number for each
attribute.

%%% Generated below here
\paragraph{Behavior:}
\begin{itemize}[noitemsep]
\item This operation manipulates an existing object.  If no object exists, the
    operation will fail with \code{NOTFOUND}.

\end{itemize}


\paragraph{Definition:}
\begin{rubycode}
atomic_or(spacename, key, attributes)
\end{rubycode}

\paragraph{Parameters:}
\begin{itemize}[noitemsep]
\item \code{spacename}\\
The name of the space as a string or symbol.

\item \code{key}\\
The key for the operation where \code{key} is a Javascript value.

\item \code{attributes}\\
The set of attributes to modify and their respective values.  \code{attrs}
points to an array of length \code{attrs\_sz}.

\end{itemize}

\paragraph{Returns:}
True if the operation succeeded.  False if any provided predicates failed.
Raises an exception on error.


\pagebreak
\subsubsection{\code{async\_atomic\_or}}
\label{api:ruby:async_atomic_or}
\index{async\_atomic\_or!Ruby API}
Store the bitwise OR of the existing value and the specified number for each
attribute.

%%% Generated below here
\paragraph{Behavior:}
\begin{itemize}[noitemsep]
\item This operation manipulates an existing object.  If no object exists, the
    operation will fail with \code{NOTFOUND}.

\end{itemize}


\paragraph{Definition:}
\begin{rubycode}
async_atomic_or(spacename, key, attributes)
\end{rubycode}

\paragraph{Parameters:}
\begin{itemize}[noitemsep]
\item \code{spacename}\\
The name of the space as a string or symbol.

\item \code{key}\\
The key for the operation where \code{key} is a Javascript value.

\item \code{attributes}\\
The set of attributes to modify and their respective values.  \code{attrs}
points to an array of length \code{attrs\_sz}.

\end{itemize}

\paragraph{Returns:}
A Deferred object with a \code{wait} method that returns True if the operation
succeeded or False if any provided predicates failed.  Raises an exception on
error.


\paragraph{See also:}  This is the asynchronous form of \code{atomic\_or}.

%%%%%%%%%%%%%%%%%%%% cond_atomic_or %%%%%%%%%%%%%%%%%%%%
\pagebreak
\subsubsection{\code{cond\_atomic\_or}}
\label{api:ruby:cond_atomic_or}
\index{cond\_atomic\_or!Ruby API}
Conditionally store the bitwise OR of the existing value and the specified
number for each attribute.

%%% Generated below here
\paragraph{Behavior:}
\begin{itemize}[noitemsep]
\item This operation manipulates an existing object.  If no object exists, the
    operation will fail with \code{NOTFOUND}.

\item This operation will succeed if and only if the predicates specified by
    \code{checks} hold on the pre-existing object.  If any of the predicates are
    not true for the existing object, then the operation will have no effect and
    fail with \code{CMPFAIL}.

    All checks are atomic with the write.  HyperDex guarantees that no other
    operation will come between validating the checks, and writing the new
    version of the object..

\end{itemize}


\paragraph{Definition:}
\begin{rubycode}
cond_atomic_or(spacename, key, predicates, attributes)
\end{rubycode}

\paragraph{Parameters:}
\begin{itemize}[noitemsep]
\item \code{spacename}\\
The name of the space as a string or symbol.

\item \code{key}\\
The key for the operation where \code{key} is a Javascript value.

\item \code{predicates}\\
A set of predicates to check against.  \code{checks} points to an array of
length \code{checks\_sz}.

\item \code{attributes}\\
The set of attributes to modify and their respective values.  \code{attrs}
points to an array of length \code{attrs\_sz}.

\end{itemize}

\paragraph{Returns:}
True if the operation succeeded.  False if any provided predicates failed.
Raises an exception on error.


\pagebreak
\subsubsection{\code{async\_cond\_atomic\_or}}
\label{api:ruby:async_cond_atomic_or}
\index{async\_cond\_atomic\_or!Ruby API}
Conditionally store the bitwise OR of the existing value and the specified
number for each attribute.

%%% Generated below here
\paragraph{Behavior:}
\begin{itemize}[noitemsep]
\item This operation manipulates an existing object.  If no object exists, the
    operation will fail with \code{NOTFOUND}.

\item This operation will succeed if and only if the predicates specified by
    \code{checks} hold on the pre-existing object.  If any of the predicates are
    not true for the existing object, then the operation will have no effect and
    fail with \code{CMPFAIL}.

    All checks are atomic with the write.  HyperDex guarantees that no other
    operation will come between validating the checks, and writing the new
    version of the object..

\end{itemize}


\paragraph{Definition:}
\begin{rubycode}
async_cond_atomic_or(spacename, key, predicates, attributes)
\end{rubycode}

\paragraph{Parameters:}
\begin{itemize}[noitemsep]
\item \code{spacename}\\
The name of the space as a string or symbol.

\item \code{key}\\
The key for the operation where \code{key} is a Javascript value.

\item \code{predicates}\\
A set of predicates to check against.  \code{checks} points to an array of
length \code{checks\_sz}.

\item \code{attributes}\\
The set of attributes to modify and their respective values.  \code{attrs}
points to an array of length \code{attrs\_sz}.

\end{itemize}

\paragraph{Returns:}
A Deferred object with a \code{wait} method that returns True if the operation
succeeded or False if any provided predicates failed.  Raises an exception on
error.


\paragraph{See also:}  This is the asynchronous form of \code{cond\_atomic\_or}.

%%%%%%%%%%%%%%%%%%%% atomic_xor %%%%%%%%%%%%%%%%%%%%
\pagebreak
\subsubsection{\code{atomic\_xor}}
\label{api:ruby:atomic_xor}
\index{atomic\_xor!Ruby API}
Store the bitwise XOR of the existing value and the specified number for each
attribute.

%%% Generated below here
\paragraph{Behavior:}
\begin{itemize}[noitemsep]
\item This operation manipulates an existing object.  If no object exists, the
    operation will fail with \code{NOTFOUND}.

\end{itemize}


\paragraph{Definition:}
\begin{rubycode}
atomic_xor(spacename, key, attributes)
\end{rubycode}

\paragraph{Parameters:}
\begin{itemize}[noitemsep]
\item \code{spacename}\\
The name of the space as a string or symbol.

\item \code{key}\\
The key for the operation where \code{key} is a Javascript value.

\item \code{attributes}\\
The set of attributes to modify and their respective values.  \code{attrs}
points to an array of length \code{attrs\_sz}.

\end{itemize}

\paragraph{Returns:}
True if the operation succeeded.  False if any provided predicates failed.
Raises an exception on error.


\pagebreak
\subsubsection{\code{async\_atomic\_xor}}
\label{api:ruby:async_atomic_xor}
\index{async\_atomic\_xor!Ruby API}
Store the bitwise XOR of the existing value and the specified number for each
attribute.

%%% Generated below here
\paragraph{Behavior:}
\begin{itemize}[noitemsep]
\item This operation manipulates an existing object.  If no object exists, the
    operation will fail with \code{NOTFOUND}.

\end{itemize}


\paragraph{Definition:}
\begin{rubycode}
async_atomic_xor(spacename, key, attributes)
\end{rubycode}

\paragraph{Parameters:}
\begin{itemize}[noitemsep]
\item \code{spacename}\\
The name of the space as a string or symbol.

\item \code{key}\\
The key for the operation where \code{key} is a Javascript value.

\item \code{attributes}\\
The set of attributes to modify and their respective values.  \code{attrs}
points to an array of length \code{attrs\_sz}.

\end{itemize}

\paragraph{Returns:}
A Deferred object with a \code{wait} method that returns True if the operation
succeeded or False if any provided predicates failed.  Raises an exception on
error.


\paragraph{See also:}  This is the asynchronous form of \code{atomic\_xor}.

%%%%%%%%%%%%%%%%%%%% cond_atomic_xor %%%%%%%%%%%%%%%%%%%%
\pagebreak
\subsubsection{\code{cond\_atomic\_xor}}
\label{api:ruby:cond_atomic_xor}
\index{cond\_atomic\_xor!Ruby API}
Conditionally store the bitwise XOR of the existing value and the specified
number for each attribute.

%%% Generated below here
\paragraph{Behavior:}
\begin{itemize}[noitemsep]
\item This operation manipulates an existing object.  If no object exists, the
    operation will fail with \code{NOTFOUND}.

\item This operation will succeed if and only if the predicates specified by
    \code{checks} hold on the pre-existing object.  If any of the predicates are
    not true for the existing object, then the operation will have no effect and
    fail with \code{CMPFAIL}.

    All checks are atomic with the write.  HyperDex guarantees that no other
    operation will come between validating the checks, and writing the new
    version of the object..

\end{itemize}


\paragraph{Definition:}
\begin{rubycode}
cond_atomic_xor(spacename, key, predicates, attributes)
\end{rubycode}

\paragraph{Parameters:}
\begin{itemize}[noitemsep]
\item \code{spacename}\\
The name of the space as a string or symbol.

\item \code{key}\\
The key for the operation where \code{key} is a Javascript value.

\item \code{predicates}\\
A set of predicates to check against.  \code{checks} points to an array of
length \code{checks\_sz}.

\item \code{attributes}\\
The set of attributes to modify and their respective values.  \code{attrs}
points to an array of length \code{attrs\_sz}.

\end{itemize}

\paragraph{Returns:}
True if the operation succeeded.  False if any provided predicates failed.
Raises an exception on error.


\pagebreak
\subsubsection{\code{async\_cond\_atomic\_xor}}
\label{api:ruby:async_cond_atomic_xor}
\index{async\_cond\_atomic\_xor!Ruby API}
Conditionally store the bitwise XOR of the existing value and the specified
number for each attribute.

%%% Generated below here
\paragraph{Behavior:}
\begin{itemize}[noitemsep]
\item This operation manipulates an existing object.  If no object exists, the
    operation will fail with \code{NOTFOUND}.

\item This operation will succeed if and only if the predicates specified by
    \code{checks} hold on the pre-existing object.  If any of the predicates are
    not true for the existing object, then the operation will have no effect and
    fail with \code{CMPFAIL}.

    All checks are atomic with the write.  HyperDex guarantees that no other
    operation will come between validating the checks, and writing the new
    version of the object..

\end{itemize}


\paragraph{Definition:}
\begin{rubycode}
async_cond_atomic_xor(spacename, key, predicates, attributes)
\end{rubycode}

\paragraph{Parameters:}
\begin{itemize}[noitemsep]
\item \code{spacename}\\
The name of the space as a string or symbol.

\item \code{key}\\
The key for the operation where \code{key} is a Javascript value.

\item \code{predicates}\\
A set of predicates to check against.  \code{checks} points to an array of
length \code{checks\_sz}.

\item \code{attributes}\\
The set of attributes to modify and their respective values.  \code{attrs}
points to an array of length \code{attrs\_sz}.

\end{itemize}

\paragraph{Returns:}
A Deferred object with a \code{wait} method that returns True if the operation
succeeded or False if any provided predicates failed.  Raises an exception on
error.


\paragraph{See also:}  This is the asynchronous form of \code{cond\_atomic\_xor}.

%%%%%%%%%%%%%%%%%%%% string_prepend %%%%%%%%%%%%%%%%%%%%
\pagebreak
\subsubsection{\code{string\_prepend}}
\label{api:ruby:string_prepend}
\index{string\_prepend!Ruby API}
Prepend the specified string to the existing value for each attribute.

%%% Generated below here
\paragraph{Behavior:}
\begin{itemize}[noitemsep]
\item This operation manipulates an existing object.  If no object exists, the
    operation will fail with \code{NOTFOUND}.

\end{itemize}


\paragraph{Definition:}
\begin{rubycode}
string_prepend(spacename, key, attributes)
\end{rubycode}

\paragraph{Parameters:}
\begin{itemize}[noitemsep]
\item \code{spacename}\\
The name of the space as a string or symbol.

\item \code{key}\\
The key for the operation where \code{key} is a Javascript value.

\item \code{attributes}\\
The set of attributes to modify and their respective values.  \code{attrs}
points to an array of length \code{attrs\_sz}.

\end{itemize}

\paragraph{Returns:}
True if the operation succeeded.  False if any provided predicates failed.
Raises an exception on error.


\pagebreak
\subsubsection{\code{async\_string\_prepend}}
\label{api:ruby:async_string_prepend}
\index{async\_string\_prepend!Ruby API}
Prepend the specified string to the existing value for each attribute.

%%% Generated below here
\paragraph{Behavior:}
\begin{itemize}[noitemsep]
\item This operation manipulates an existing object.  If no object exists, the
    operation will fail with \code{NOTFOUND}.

\end{itemize}


\paragraph{Definition:}
\begin{rubycode}
async_string_prepend(spacename, key, attributes)
\end{rubycode}

\paragraph{Parameters:}
\begin{itemize}[noitemsep]
\item \code{spacename}\\
The name of the space as a string or symbol.

\item \code{key}\\
The key for the operation where \code{key} is a Javascript value.

\item \code{attributes}\\
The set of attributes to modify and their respective values.  \code{attrs}
points to an array of length \code{attrs\_sz}.

\end{itemize}

\paragraph{Returns:}
A Deferred object with a \code{wait} method that returns True if the operation
succeeded or False if any provided predicates failed.  Raises an exception on
error.


\paragraph{See also:}  This is the asynchronous form of \code{string\_prepend}.

%%%%%%%%%%%%%%%%%%%% cond_string_prepend %%%%%%%%%%%%%%%%%%%%
\pagebreak
\subsubsection{\code{cond\_string\_prepend}}
\label{api:ruby:cond_string_prepend}
\index{cond\_string\_prepend!Ruby API}
Conditionally prepend the specified string to the existing value for each
attribute.

%%% Generated below here
\paragraph{Behavior:}
\begin{itemize}[noitemsep]
\item This operation manipulates an existing object.  If no object exists, the
    operation will fail with \code{NOTFOUND}.

\item This operation will succeed if and only if the predicates specified by
    \code{checks} hold on the pre-existing object.  If any of the predicates are
    not true for the existing object, then the operation will have no effect and
    fail with \code{CMPFAIL}.

    All checks are atomic with the write.  HyperDex guarantees that no other
    operation will come between validating the checks, and writing the new
    version of the object..

\end{itemize}


\paragraph{Definition:}
\begin{rubycode}
cond_string_prepend(spacename, key, predicates, attributes)
\end{rubycode}

\paragraph{Parameters:}
\begin{itemize}[noitemsep]
\item \code{spacename}\\
The name of the space as a string or symbol.

\item \code{key}\\
The key for the operation where \code{key} is a Javascript value.

\item \code{predicates}\\
A set of predicates to check against.  \code{checks} points to an array of
length \code{checks\_sz}.

\item \code{attributes}\\
The set of attributes to modify and their respective values.  \code{attrs}
points to an array of length \code{attrs\_sz}.

\end{itemize}

\paragraph{Returns:}
True if the operation succeeded.  False if any provided predicates failed.
Raises an exception on error.


\pagebreak
\subsubsection{\code{async\_cond\_string\_prepend}}
\label{api:ruby:async_cond_string_prepend}
\index{async\_cond\_string\_prepend!Ruby API}
Conditionally prepend the specified string to the existing value for each
attribute.

%%% Generated below here
\paragraph{Behavior:}
\begin{itemize}[noitemsep]
\item This operation manipulates an existing object.  If no object exists, the
    operation will fail with \code{NOTFOUND}.

\item This operation will succeed if and only if the predicates specified by
    \code{checks} hold on the pre-existing object.  If any of the predicates are
    not true for the existing object, then the operation will have no effect and
    fail with \code{CMPFAIL}.

    All checks are atomic with the write.  HyperDex guarantees that no other
    operation will come between validating the checks, and writing the new
    version of the object..

\end{itemize}


\paragraph{Definition:}
\begin{rubycode}
async_cond_string_prepend(spacename, key, predicates, attributes)
\end{rubycode}

\paragraph{Parameters:}
\begin{itemize}[noitemsep]
\item \code{spacename}\\
The name of the space as a string or symbol.

\item \code{key}\\
The key for the operation where \code{key} is a Javascript value.

\item \code{predicates}\\
A set of predicates to check against.  \code{checks} points to an array of
length \code{checks\_sz}.

\item \code{attributes}\\
The set of attributes to modify and their respective values.  \code{attrs}
points to an array of length \code{attrs\_sz}.

\end{itemize}

\paragraph{Returns:}
A Deferred object with a \code{wait} method that returns True if the operation
succeeded or False if any provided predicates failed.  Raises an exception on
error.


\paragraph{See also:}  This is the asynchronous form of \code{cond\_string\_prepend}.

%%%%%%%%%%%%%%%%%%%% string_append %%%%%%%%%%%%%%%%%%%%
\pagebreak
\subsubsection{\code{string\_append}}
\label{api:ruby:string_append}
\index{string\_append!Ruby API}
Append the specified string to the existing value for each attribute.

%%% Generated below here
\paragraph{Behavior:}
\begin{itemize}[noitemsep]
\item This operation manipulates an existing object.  If no object exists, the
    operation will fail with \code{NOTFOUND}.

\end{itemize}


\paragraph{Definition:}
\begin{rubycode}
string_append(spacename, key, attributes)
\end{rubycode}

\paragraph{Parameters:}
\begin{itemize}[noitemsep]
\item \code{spacename}\\
The name of the space as a string or symbol.

\item \code{key}\\
The key for the operation where \code{key} is a Javascript value.

\item \code{attributes}\\
The set of attributes to modify and their respective values.  \code{attrs}
points to an array of length \code{attrs\_sz}.

\end{itemize}

\paragraph{Returns:}
True if the operation succeeded.  False if any provided predicates failed.
Raises an exception on error.


\pagebreak
\subsubsection{\code{async\_string\_append}}
\label{api:ruby:async_string_append}
\index{async\_string\_append!Ruby API}
Append the specified string to the existing value for each attribute.

%%% Generated below here
\paragraph{Behavior:}
\begin{itemize}[noitemsep]
\item This operation manipulates an existing object.  If no object exists, the
    operation will fail with \code{NOTFOUND}.

\end{itemize}


\paragraph{Definition:}
\begin{rubycode}
async_string_append(spacename, key, attributes)
\end{rubycode}

\paragraph{Parameters:}
\begin{itemize}[noitemsep]
\item \code{spacename}\\
The name of the space as a string or symbol.

\item \code{key}\\
The key for the operation where \code{key} is a Javascript value.

\item \code{attributes}\\
The set of attributes to modify and their respective values.  \code{attrs}
points to an array of length \code{attrs\_sz}.

\end{itemize}

\paragraph{Returns:}
A Deferred object with a \code{wait} method that returns True if the operation
succeeded or False if any provided predicates failed.  Raises an exception on
error.


\paragraph{See also:}  This is the asynchronous form of \code{string\_append}.

%%%%%%%%%%%%%%%%%%%% cond_string_append %%%%%%%%%%%%%%%%%%%%
\pagebreak
\subsubsection{\code{cond\_string\_append}}
\label{api:ruby:cond_string_append}
\index{cond\_string\_append!Ruby API}
Conditionally append the specified string to the existing value for each
attribute.

%%% Generated below here
\paragraph{Behavior:}
\begin{itemize}[noitemsep]
\item This operation manipulates an existing object.  If no object exists, the
    operation will fail with \code{NOTFOUND}.

\item This operation will succeed if and only if the predicates specified by
    \code{checks} hold on the pre-existing object.  If any of the predicates are
    not true for the existing object, then the operation will have no effect and
    fail with \code{CMPFAIL}.

    All checks are atomic with the write.  HyperDex guarantees that no other
    operation will come between validating the checks, and writing the new
    version of the object..

\end{itemize}


\paragraph{Definition:}
\begin{rubycode}
cond_string_append(spacename, key, predicates, attributes)
\end{rubycode}

\paragraph{Parameters:}
\begin{itemize}[noitemsep]
\item \code{spacename}\\
The name of the space as a string or symbol.

\item \code{key}\\
The key for the operation where \code{key} is a Javascript value.

\item \code{predicates}\\
A set of predicates to check against.  \code{checks} points to an array of
length \code{checks\_sz}.

\item \code{attributes}\\
The set of attributes to modify and their respective values.  \code{attrs}
points to an array of length \code{attrs\_sz}.

\end{itemize}

\paragraph{Returns:}
True if the operation succeeded.  False if any provided predicates failed.
Raises an exception on error.


\pagebreak
\subsubsection{\code{async\_cond\_string\_append}}
\label{api:ruby:async_cond_string_append}
\index{async\_cond\_string\_append!Ruby API}
Conditionally append the specified string to the existing value for each
attribute.

%%% Generated below here
\paragraph{Behavior:}
\begin{itemize}[noitemsep]
\item This operation manipulates an existing object.  If no object exists, the
    operation will fail with \code{NOTFOUND}.

\item This operation will succeed if and only if the predicates specified by
    \code{checks} hold on the pre-existing object.  If any of the predicates are
    not true for the existing object, then the operation will have no effect and
    fail with \code{CMPFAIL}.

    All checks are atomic with the write.  HyperDex guarantees that no other
    operation will come between validating the checks, and writing the new
    version of the object..

\end{itemize}


\paragraph{Definition:}
\begin{rubycode}
async_cond_string_append(spacename, key, predicates, attributes)
\end{rubycode}

\paragraph{Parameters:}
\begin{itemize}[noitemsep]
\item \code{spacename}\\
The name of the space as a string or symbol.

\item \code{key}\\
The key for the operation where \code{key} is a Javascript value.

\item \code{predicates}\\
A set of predicates to check against.  \code{checks} points to an array of
length \code{checks\_sz}.

\item \code{attributes}\\
The set of attributes to modify and their respective values.  \code{attrs}
points to an array of length \code{attrs\_sz}.

\end{itemize}

\paragraph{Returns:}
A Deferred object with a \code{wait} method that returns True if the operation
succeeded or False if any provided predicates failed.  Raises an exception on
error.


\paragraph{See also:}  This is the asynchronous form of \code{cond\_string\_append}.

%%%%%%%%%%%%%%%%%%%% list_lpush %%%%%%%%%%%%%%%%%%%%
\pagebreak
\subsubsection{\code{list\_lpush}}
\label{api:ruby:list_lpush}
\index{list\_lpush!Ruby API}
Push the specified value onto the front of the list for each attribute.

%%% Generated below here
\paragraph{Behavior:}
\begin{itemize}[noitemsep]
\item This operation manipulates an existing object.  If no object exists, the
    operation will fail with \code{NOTFOUND}.

\end{itemize}


\paragraph{Definition:}
\begin{rubycode}
list_lpush(spacename, key, attributes)
\end{rubycode}

\paragraph{Parameters:}
\begin{itemize}[noitemsep]
\item \code{spacename}\\
The name of the space as a string or symbol.

\item \code{key}\\
The key for the operation where \code{key} is a Javascript value.

\item \code{attributes}\\
The set of attributes to modify and their respective values.  \code{attrs}
points to an array of length \code{attrs\_sz}.

\end{itemize}

\paragraph{Returns:}
True if the operation succeeded.  False if any provided predicates failed.
Raises an exception on error.


\pagebreak
\subsubsection{\code{async\_list\_lpush}}
\label{api:ruby:async_list_lpush}
\index{async\_list\_lpush!Ruby API}
Push the specified value onto the front of the list for each attribute.

%%% Generated below here
\paragraph{Behavior:}
\begin{itemize}[noitemsep]
\item This operation manipulates an existing object.  If no object exists, the
    operation will fail with \code{NOTFOUND}.

\end{itemize}


\paragraph{Definition:}
\begin{rubycode}
async_list_lpush(spacename, key, attributes)
\end{rubycode}

\paragraph{Parameters:}
\begin{itemize}[noitemsep]
\item \code{spacename}\\
The name of the space as a string or symbol.

\item \code{key}\\
The key for the operation where \code{key} is a Javascript value.

\item \code{attributes}\\
The set of attributes to modify and their respective values.  \code{attrs}
points to an array of length \code{attrs\_sz}.

\end{itemize}

\paragraph{Returns:}
A Deferred object with a \code{wait} method that returns True if the operation
succeeded or False if any provided predicates failed.  Raises an exception on
error.


\paragraph{See also:}  This is the asynchronous form of \code{list\_lpush}.

%%%%%%%%%%%%%%%%%%%% cond_list_lpush %%%%%%%%%%%%%%%%%%%%
\pagebreak
\subsubsection{\code{cond\_list\_lpush}}
\label{api:ruby:cond_list_lpush}
\index{cond\_list\_lpush!Ruby API}
Condtitionally push the specified value onto the front of the list for each
attribute.

%%% Generated below here
\paragraph{Behavior:}
\begin{itemize}[noitemsep]
\item This operation manipulates an existing object.  If no object exists, the
    operation will fail with \code{NOTFOUND}.

\item This operation will succeed if and only if the predicates specified by
    \code{checks} hold on the pre-existing object.  If any of the predicates are
    not true for the existing object, then the operation will have no effect and
    fail with \code{CMPFAIL}.

    All checks are atomic with the write.  HyperDex guarantees that no other
    operation will come between validating the checks, and writing the new
    version of the object..

\end{itemize}


\paragraph{Definition:}
\begin{rubycode}
cond_list_lpush(spacename, key, predicates, attributes)
\end{rubycode}

\paragraph{Parameters:}
\begin{itemize}[noitemsep]
\item \code{spacename}\\
The name of the space as a string or symbol.

\item \code{key}\\
The key for the operation where \code{key} is a Javascript value.

\item \code{predicates}\\
A set of predicates to check against.  \code{checks} points to an array of
length \code{checks\_sz}.

\item \code{attributes}\\
The set of attributes to modify and their respective values.  \code{attrs}
points to an array of length \code{attrs\_sz}.

\end{itemize}

\paragraph{Returns:}
True if the operation succeeded.  False if any provided predicates failed.
Raises an exception on error.


\pagebreak
\subsubsection{\code{async\_cond\_list\_lpush}}
\label{api:ruby:async_cond_list_lpush}
\index{async\_cond\_list\_lpush!Ruby API}
Condtitionally push the specified value onto the front of the list for each
attribute.

%%% Generated below here
\paragraph{Behavior:}
\begin{itemize}[noitemsep]
\item This operation manipulates an existing object.  If no object exists, the
    operation will fail with \code{NOTFOUND}.

\item This operation will succeed if and only if the predicates specified by
    \code{checks} hold on the pre-existing object.  If any of the predicates are
    not true for the existing object, then the operation will have no effect and
    fail with \code{CMPFAIL}.

    All checks are atomic with the write.  HyperDex guarantees that no other
    operation will come between validating the checks, and writing the new
    version of the object..

\end{itemize}


\paragraph{Definition:}
\begin{rubycode}
async_cond_list_lpush(spacename, key, predicates, attributes)
\end{rubycode}

\paragraph{Parameters:}
\begin{itemize}[noitemsep]
\item \code{spacename}\\
The name of the space as a string or symbol.

\item \code{key}\\
The key for the operation where \code{key} is a Javascript value.

\item \code{predicates}\\
A set of predicates to check against.  \code{checks} points to an array of
length \code{checks\_sz}.

\item \code{attributes}\\
The set of attributes to modify and their respective values.  \code{attrs}
points to an array of length \code{attrs\_sz}.

\end{itemize}

\paragraph{Returns:}
A Deferred object with a \code{wait} method that returns True if the operation
succeeded or False if any provided predicates failed.  Raises an exception on
error.


\paragraph{See also:}  This is the asynchronous form of \code{cond\_list\_lpush}.

%%%%%%%%%%%%%%%%%%%% list_rpush %%%%%%%%%%%%%%%%%%%%
\pagebreak
\subsubsection{\code{list\_rpush}}
\label{api:ruby:list_rpush}
\index{list\_rpush!Ruby API}
Push the specified value onto the back of the list for each attribute.

%%% Generated below here
\paragraph{Behavior:}
\begin{itemize}[noitemsep]
\item This operation manipulates an existing object.  If no object exists, the
    operation will fail with \code{NOTFOUND}.

\end{itemize}


\paragraph{Definition:}
\begin{rubycode}
list_rpush(spacename, key, attributes)
\end{rubycode}

\paragraph{Parameters:}
\begin{itemize}[noitemsep]
\item \code{spacename}\\
The name of the space as a string or symbol.

\item \code{key}\\
The key for the operation where \code{key} is a Javascript value.

\item \code{attributes}\\
The set of attributes to modify and their respective values.  \code{attrs}
points to an array of length \code{attrs\_sz}.

\end{itemize}

\paragraph{Returns:}
True if the operation succeeded.  False if any provided predicates failed.
Raises an exception on error.


\pagebreak
\subsubsection{\code{async\_list\_rpush}}
\label{api:ruby:async_list_rpush}
\index{async\_list\_rpush!Ruby API}
Push the specified value onto the back of the list for each attribute.

%%% Generated below here
\paragraph{Behavior:}
\begin{itemize}[noitemsep]
\item This operation manipulates an existing object.  If no object exists, the
    operation will fail with \code{NOTFOUND}.

\end{itemize}


\paragraph{Definition:}
\begin{rubycode}
async_list_rpush(spacename, key, attributes)
\end{rubycode}

\paragraph{Parameters:}
\begin{itemize}[noitemsep]
\item \code{spacename}\\
The name of the space as a string or symbol.

\item \code{key}\\
The key for the operation where \code{key} is a Javascript value.

\item \code{attributes}\\
The set of attributes to modify and their respective values.  \code{attrs}
points to an array of length \code{attrs\_sz}.

\end{itemize}

\paragraph{Returns:}
A Deferred object with a \code{wait} method that returns True if the operation
succeeded or False if any provided predicates failed.  Raises an exception on
error.


\paragraph{See also:}  This is the asynchronous form of \code{list\_rpush}.

%%%%%%%%%%%%%%%%%%%% cond_list_rpush %%%%%%%%%%%%%%%%%%%%
\pagebreak
\subsubsection{\code{cond\_list\_rpush}}
\label{api:ruby:cond_list_rpush}
\index{cond\_list\_rpush!Ruby API}
Conditionally push the specified value onto the back of the list for each
attribute.

%%% Generated below here
\paragraph{Behavior:}
\begin{itemize}[noitemsep]
\item This operation manipulates an existing object.  If no object exists, the
    operation will fail with \code{NOTFOUND}.

\item This operation will succeed if and only if the predicates specified by
    \code{checks} hold on the pre-existing object.  If any of the predicates are
    not true for the existing object, then the operation will have no effect and
    fail with \code{CMPFAIL}.

    All checks are atomic with the write.  HyperDex guarantees that no other
    operation will come between validating the checks, and writing the new
    version of the object..

\end{itemize}


\paragraph{Definition:}
\begin{rubycode}
cond_list_rpush(spacename, key, predicates, attributes)
\end{rubycode}

\paragraph{Parameters:}
\begin{itemize}[noitemsep]
\item \code{spacename}\\
The name of the space as a string or symbol.

\item \code{key}\\
The key for the operation where \code{key} is a Javascript value.

\item \code{predicates}\\
A set of predicates to check against.  \code{checks} points to an array of
length \code{checks\_sz}.

\item \code{attributes}\\
The set of attributes to modify and their respective values.  \code{attrs}
points to an array of length \code{attrs\_sz}.

\end{itemize}

\paragraph{Returns:}
True if the operation succeeded.  False if any provided predicates failed.
Raises an exception on error.


\pagebreak
\subsubsection{\code{async\_cond\_list\_rpush}}
\label{api:ruby:async_cond_list_rpush}
\index{async\_cond\_list\_rpush!Ruby API}
Conditionally push the specified value onto the back of the list for each
attribute.

%%% Generated below here
\paragraph{Behavior:}
\begin{itemize}[noitemsep]
\item This operation manipulates an existing object.  If no object exists, the
    operation will fail with \code{NOTFOUND}.

\item This operation will succeed if and only if the predicates specified by
    \code{checks} hold on the pre-existing object.  If any of the predicates are
    not true for the existing object, then the operation will have no effect and
    fail with \code{CMPFAIL}.

    All checks are atomic with the write.  HyperDex guarantees that no other
    operation will come between validating the checks, and writing the new
    version of the object..

\end{itemize}


\paragraph{Definition:}
\begin{rubycode}
async_cond_list_rpush(spacename, key, predicates, attributes)
\end{rubycode}

\paragraph{Parameters:}
\begin{itemize}[noitemsep]
\item \code{spacename}\\
The name of the space as a string or symbol.

\item \code{key}\\
The key for the operation where \code{key} is a Javascript value.

\item \code{predicates}\\
A set of predicates to check against.  \code{checks} points to an array of
length \code{checks\_sz}.

\item \code{attributes}\\
The set of attributes to modify and their respective values.  \code{attrs}
points to an array of length \code{attrs\_sz}.

\end{itemize}

\paragraph{Returns:}
A Deferred object with a \code{wait} method that returns True if the operation
succeeded or False if any provided predicates failed.  Raises an exception on
error.


\paragraph{See also:}  This is the asynchronous form of \code{cond\_list\_rpush}.

%%%%%%%%%%%%%%%%%%%% set_add %%%%%%%%%%%%%%%%%%%%
\pagebreak
\subsubsection{\code{set\_add}}
\label{api:ruby:set_add}
\index{set\_add!Ruby API}
Add the specified value to the set for each attribute.

%%% Generated below here
\paragraph{Behavior:}
\begin{itemize}[noitemsep]
\item This operation manipulates an existing object.  If no object exists, the
    operation will fail with \code{NOTFOUND}.

\end{itemize}


\paragraph{Definition:}
\begin{rubycode}
set_add(spacename, key, attributes)
\end{rubycode}

\paragraph{Parameters:}
\begin{itemize}[noitemsep]
\item \code{spacename}\\
The name of the space as a string or symbol.

\item \code{key}\\
The key for the operation where \code{key} is a Javascript value.

\item \code{attributes}\\
The set of attributes to modify and their respective values.  \code{attrs}
points to an array of length \code{attrs\_sz}.

\end{itemize}

\paragraph{Returns:}
True if the operation succeeded.  False if any provided predicates failed.
Raises an exception on error.


\pagebreak
\subsubsection{\code{async\_set\_add}}
\label{api:ruby:async_set_add}
\index{async\_set\_add!Ruby API}
Add the specified value to the set for each attribute.

%%% Generated below here
\paragraph{Behavior:}
\begin{itemize}[noitemsep]
\item This operation manipulates an existing object.  If no object exists, the
    operation will fail with \code{NOTFOUND}.

\end{itemize}


\paragraph{Definition:}
\begin{rubycode}
async_set_add(spacename, key, attributes)
\end{rubycode}

\paragraph{Parameters:}
\begin{itemize}[noitemsep]
\item \code{spacename}\\
The name of the space as a string or symbol.

\item \code{key}\\
The key for the operation where \code{key} is a Javascript value.

\item \code{attributes}\\
The set of attributes to modify and their respective values.  \code{attrs}
points to an array of length \code{attrs\_sz}.

\end{itemize}

\paragraph{Returns:}
A Deferred object with a \code{wait} method that returns True if the operation
succeeded or False if any provided predicates failed.  Raises an exception on
error.


\paragraph{See also:}  This is the asynchronous form of \code{set\_add}.

%%%%%%%%%%%%%%%%%%%% cond_set_add %%%%%%%%%%%%%%%%%%%%
\pagebreak
\subsubsection{\code{cond\_set\_add}}
\label{api:ruby:cond_set_add}
\index{cond\_set\_add!Ruby API}
Conditionally add the specified value to the set for each attribute.

%%% Generated below here
\paragraph{Behavior:}
\begin{itemize}[noitemsep]
\item This operation manipulates an existing object.  If no object exists, the
    operation will fail with \code{NOTFOUND}.

\item This operation will succeed if and only if the predicates specified by
    \code{checks} hold on the pre-existing object.  If any of the predicates are
    not true for the existing object, then the operation will have no effect and
    fail with \code{CMPFAIL}.

    All checks are atomic with the write.  HyperDex guarantees that no other
    operation will come between validating the checks, and writing the new
    version of the object..

\end{itemize}


\paragraph{Definition:}
\begin{rubycode}
cond_set_add(spacename, key, predicates, attributes)
\end{rubycode}

\paragraph{Parameters:}
\begin{itemize}[noitemsep]
\item \code{spacename}\\
The name of the space as a string or symbol.

\item \code{key}\\
The key for the operation where \code{key} is a Javascript value.

\item \code{predicates}\\
A set of predicates to check against.  \code{checks} points to an array of
length \code{checks\_sz}.

\item \code{attributes}\\
The set of attributes to modify and their respective values.  \code{attrs}
points to an array of length \code{attrs\_sz}.

\end{itemize}

\paragraph{Returns:}
True if the operation succeeded.  False if any provided predicates failed.
Raises an exception on error.


\pagebreak
\subsubsection{\code{async\_cond\_set\_add}}
\label{api:ruby:async_cond_set_add}
\index{async\_cond\_set\_add!Ruby API}
Conditionally add the specified value to the set for each attribute.

%%% Generated below here
\paragraph{Behavior:}
\begin{itemize}[noitemsep]
\item This operation manipulates an existing object.  If no object exists, the
    operation will fail with \code{NOTFOUND}.

\item This operation will succeed if and only if the predicates specified by
    \code{checks} hold on the pre-existing object.  If any of the predicates are
    not true for the existing object, then the operation will have no effect and
    fail with \code{CMPFAIL}.

    All checks are atomic with the write.  HyperDex guarantees that no other
    operation will come between validating the checks, and writing the new
    version of the object..

\end{itemize}


\paragraph{Definition:}
\begin{rubycode}
async_cond_set_add(spacename, key, predicates, attributes)
\end{rubycode}

\paragraph{Parameters:}
\begin{itemize}[noitemsep]
\item \code{spacename}\\
The name of the space as a string or symbol.

\item \code{key}\\
The key for the operation where \code{key} is a Javascript value.

\item \code{predicates}\\
A set of predicates to check against.  \code{checks} points to an array of
length \code{checks\_sz}.

\item \code{attributes}\\
The set of attributes to modify and their respective values.  \code{attrs}
points to an array of length \code{attrs\_sz}.

\end{itemize}

\paragraph{Returns:}
A Deferred object with a \code{wait} method that returns True if the operation
succeeded or False if any provided predicates failed.  Raises an exception on
error.


\paragraph{See also:}  This is the asynchronous form of \code{cond\_set\_add}.

%%%%%%%%%%%%%%%%%%%% set_remove %%%%%%%%%%%%%%%%%%%%
\pagebreak
\subsubsection{\code{set\_remove}}
\label{api:ruby:set_remove}
\index{set\_remove!Ruby API}
Remove the specified value from the set.  If the value is not contained within
the set, this operation will do nothing.

%%% Generated below here
\paragraph{Behavior:}
\begin{itemize}[noitemsep]
\item This operation manipulates an existing object.  If no object exists, the
    operation will fail with \code{NOTFOUND}.

\end{itemize}


\paragraph{Definition:}
\begin{rubycode}
set_remove(spacename, key, attributes)
\end{rubycode}

\paragraph{Parameters:}
\begin{itemize}[noitemsep]
\item \code{spacename}\\
The name of the space as a string or symbol.

\item \code{key}\\
The key for the operation where \code{key} is a Javascript value.

\item \code{attributes}\\
The set of attributes to modify and their respective values.  \code{attrs}
points to an array of length \code{attrs\_sz}.

\end{itemize}

\paragraph{Returns:}
True if the operation succeeded.  False if any provided predicates failed.
Raises an exception on error.


\pagebreak
\subsubsection{\code{async\_set\_remove}}
\label{api:ruby:async_set_remove}
\index{async\_set\_remove!Ruby API}
Remove the specified value from the set.  If the value is not contained within
the set, this operation will do nothing.

%%% Generated below here
\paragraph{Behavior:}
\begin{itemize}[noitemsep]
\item This operation manipulates an existing object.  If no object exists, the
    operation will fail with \code{NOTFOUND}.

\end{itemize}


\paragraph{Definition:}
\begin{rubycode}
async_set_remove(spacename, key, attributes)
\end{rubycode}

\paragraph{Parameters:}
\begin{itemize}[noitemsep]
\item \code{spacename}\\
The name of the space as a string or symbol.

\item \code{key}\\
The key for the operation where \code{key} is a Javascript value.

\item \code{attributes}\\
The set of attributes to modify and their respective values.  \code{attrs}
points to an array of length \code{attrs\_sz}.

\end{itemize}

\paragraph{Returns:}
A Deferred object with a \code{wait} method that returns True if the operation
succeeded or False if any provided predicates failed.  Raises an exception on
error.


\paragraph{See also:}  This is the asynchronous form of \code{set\_remove}.

%%%%%%%%%%%%%%%%%%%% cond_set_remove %%%%%%%%%%%%%%%%%%%%
\pagebreak
\subsubsection{\code{cond\_set\_remove}}
\label{api:ruby:cond_set_remove}
\index{cond\_set\_remove!Ruby API}
Conditionally remove the specified value from the set.  If the value is not
contained within the set, this operation will do nothing.

%%% Generated below here
\paragraph{Behavior:}
\begin{itemize}[noitemsep]
\item This operation manipulates an existing object.  If no object exists, the
    operation will fail with \code{NOTFOUND}.

\item This operation will succeed if and only if the predicates specified by
    \code{checks} hold on the pre-existing object.  If any of the predicates are
    not true for the existing object, then the operation will have no effect and
    fail with \code{CMPFAIL}.

    All checks are atomic with the write.  HyperDex guarantees that no other
    operation will come between validating the checks, and writing the new
    version of the object..

\end{itemize}


\paragraph{Definition:}
\begin{rubycode}
cond_set_remove(spacename, key, predicates, attributes)
\end{rubycode}

\paragraph{Parameters:}
\begin{itemize}[noitemsep]
\item \code{spacename}\\
The name of the space as a string or symbol.

\item \code{key}\\
The key for the operation where \code{key} is a Javascript value.

\item \code{predicates}\\
A set of predicates to check against.  \code{checks} points to an array of
length \code{checks\_sz}.

\item \code{attributes}\\
The set of attributes to modify and their respective values.  \code{attrs}
points to an array of length \code{attrs\_sz}.

\end{itemize}

\paragraph{Returns:}
True if the operation succeeded.  False if any provided predicates failed.
Raises an exception on error.


\pagebreak
\subsubsection{\code{async\_cond\_set\_remove}}
\label{api:ruby:async_cond_set_remove}
\index{async\_cond\_set\_remove!Ruby API}
Conditionally remove the specified value from the set.  If the value is not
contained within the set, this operation will do nothing.

%%% Generated below here
\paragraph{Behavior:}
\begin{itemize}[noitemsep]
\item This operation manipulates an existing object.  If no object exists, the
    operation will fail with \code{NOTFOUND}.

\item This operation will succeed if and only if the predicates specified by
    \code{checks} hold on the pre-existing object.  If any of the predicates are
    not true for the existing object, then the operation will have no effect and
    fail with \code{CMPFAIL}.

    All checks are atomic with the write.  HyperDex guarantees that no other
    operation will come between validating the checks, and writing the new
    version of the object..

\end{itemize}


\paragraph{Definition:}
\begin{rubycode}
async_cond_set_remove(spacename, key, predicates, attributes)
\end{rubycode}

\paragraph{Parameters:}
\begin{itemize}[noitemsep]
\item \code{spacename}\\
The name of the space as a string or symbol.

\item \code{key}\\
The key for the operation where \code{key} is a Javascript value.

\item \code{predicates}\\
A set of predicates to check against.  \code{checks} points to an array of
length \code{checks\_sz}.

\item \code{attributes}\\
The set of attributes to modify and their respective values.  \code{attrs}
points to an array of length \code{attrs\_sz}.

\end{itemize}

\paragraph{Returns:}
A Deferred object with a \code{wait} method that returns True if the operation
succeeded or False if any provided predicates failed.  Raises an exception on
error.


\paragraph{See also:}  This is the asynchronous form of \code{cond\_set\_remove}.

%%%%%%%%%%%%%%%%%%%% set_intersect %%%%%%%%%%%%%%%%%%%%
\pagebreak
\subsubsection{\code{set\_intersect}}
\label{api:ruby:set_intersect}
\index{set\_intersect!Ruby API}
Store the intersection of the specified set and the existing value for each
attribute.

%%% Generated below here
\paragraph{Behavior:}
\begin{itemize}[noitemsep]
\item This operation manipulates an existing object.  If no object exists, the
    operation will fail with \code{NOTFOUND}.

\end{itemize}


\paragraph{Definition:}
\begin{rubycode}
set_intersect(spacename, key, attributes)
\end{rubycode}

\paragraph{Parameters:}
\begin{itemize}[noitemsep]
\item \code{spacename}\\
The name of the space as a string or symbol.

\item \code{key}\\
The key for the operation where \code{key} is a Javascript value.

\item \code{attributes}\\
The set of attributes to modify and their respective values.  \code{attrs}
points to an array of length \code{attrs\_sz}.

\end{itemize}

\paragraph{Returns:}
True if the operation succeeded.  False if any provided predicates failed.
Raises an exception on error.


\pagebreak
\subsubsection{\code{async\_set\_intersect}}
\label{api:ruby:async_set_intersect}
\index{async\_set\_intersect!Ruby API}
Store the intersection of the specified set and the existing value for each
attribute.

%%% Generated below here
\paragraph{Behavior:}
\begin{itemize}[noitemsep]
\item This operation manipulates an existing object.  If no object exists, the
    operation will fail with \code{NOTFOUND}.

\end{itemize}


\paragraph{Definition:}
\begin{rubycode}
async_set_intersect(spacename, key, attributes)
\end{rubycode}

\paragraph{Parameters:}
\begin{itemize}[noitemsep]
\item \code{spacename}\\
The name of the space as a string or symbol.

\item \code{key}\\
The key for the operation where \code{key} is a Javascript value.

\item \code{attributes}\\
The set of attributes to modify and their respective values.  \code{attrs}
points to an array of length \code{attrs\_sz}.

\end{itemize}

\paragraph{Returns:}
A Deferred object with a \code{wait} method that returns True if the operation
succeeded or False if any provided predicates failed.  Raises an exception on
error.


\paragraph{See also:}  This is the asynchronous form of \code{set\_intersect}.

%%%%%%%%%%%%%%%%%%%% cond_set_intersect %%%%%%%%%%%%%%%%%%%%
\pagebreak
\subsubsection{\code{cond\_set\_intersect}}
\label{api:ruby:cond_set_intersect}
\index{cond\_set\_intersect!Ruby API}
Conditionally store the intersection of the specified set and the existing value
for each attribute.

%%% Generated below here
\paragraph{Behavior:}
\begin{itemize}[noitemsep]
\item This operation manipulates an existing object.  If no object exists, the
    operation will fail with \code{NOTFOUND}.

\item This operation will succeed if and only if the predicates specified by
    \code{checks} hold on the pre-existing object.  If any of the predicates are
    not true for the existing object, then the operation will have no effect and
    fail with \code{CMPFAIL}.

    All checks are atomic with the write.  HyperDex guarantees that no other
    operation will come between validating the checks, and writing the new
    version of the object..

\end{itemize}


\paragraph{Definition:}
\begin{rubycode}
cond_set_intersect(spacename, key, predicates, attributes)
\end{rubycode}

\paragraph{Parameters:}
\begin{itemize}[noitemsep]
\item \code{spacename}\\
The name of the space as a string or symbol.

\item \code{key}\\
The key for the operation where \code{key} is a Javascript value.

\item \code{predicates}\\
A set of predicates to check against.  \code{checks} points to an array of
length \code{checks\_sz}.

\item \code{attributes}\\
The set of attributes to modify and their respective values.  \code{attrs}
points to an array of length \code{attrs\_sz}.

\end{itemize}

\paragraph{Returns:}
True if the operation succeeded.  False if any provided predicates failed.
Raises an exception on error.


\pagebreak
\subsubsection{\code{async\_cond\_set\_intersect}}
\label{api:ruby:async_cond_set_intersect}
\index{async\_cond\_set\_intersect!Ruby API}
Conditionally store the intersection of the specified set and the existing value
for each attribute.

%%% Generated below here
\paragraph{Behavior:}
\begin{itemize}[noitemsep]
\item This operation manipulates an existing object.  If no object exists, the
    operation will fail with \code{NOTFOUND}.

\item This operation will succeed if and only if the predicates specified by
    \code{checks} hold on the pre-existing object.  If any of the predicates are
    not true for the existing object, then the operation will have no effect and
    fail with \code{CMPFAIL}.

    All checks are atomic with the write.  HyperDex guarantees that no other
    operation will come between validating the checks, and writing the new
    version of the object..

\end{itemize}


\paragraph{Definition:}
\begin{rubycode}
async_cond_set_intersect(spacename, key, predicates, attributes)
\end{rubycode}

\paragraph{Parameters:}
\begin{itemize}[noitemsep]
\item \code{spacename}\\
The name of the space as a string or symbol.

\item \code{key}\\
The key for the operation where \code{key} is a Javascript value.

\item \code{predicates}\\
A set of predicates to check against.  \code{checks} points to an array of
length \code{checks\_sz}.

\item \code{attributes}\\
The set of attributes to modify and their respective values.  \code{attrs}
points to an array of length \code{attrs\_sz}.

\end{itemize}

\paragraph{Returns:}
A Deferred object with a \code{wait} method that returns True if the operation
succeeded or False if any provided predicates failed.  Raises an exception on
error.


\paragraph{See also:}  This is the asynchronous form of \code{cond\_set\_intersect}.

%%%%%%%%%%%%%%%%%%%% set_union %%%%%%%%%%%%%%%%%%%%
\pagebreak
\subsubsection{\code{set\_union}}
\label{api:ruby:set_union}
\index{set\_union!Ruby API}
Store the union of the specified set and the existing value for each attribute.

%%% Generated below here
\paragraph{Behavior:}
\begin{itemize}[noitemsep]
\item This operation manipulates an existing object.  If no object exists, the
    operation will fail with \code{NOTFOUND}.

\end{itemize}


\paragraph{Definition:}
\begin{rubycode}
set_union(spacename, key, attributes)
\end{rubycode}

\paragraph{Parameters:}
\begin{itemize}[noitemsep]
\item \code{spacename}\\
The name of the space as a string or symbol.

\item \code{key}\\
The key for the operation where \code{key} is a Javascript value.

\item \code{attributes}\\
The set of attributes to modify and their respective values.  \code{attrs}
points to an array of length \code{attrs\_sz}.

\end{itemize}

\paragraph{Returns:}
True if the operation succeeded.  False if any provided predicates failed.
Raises an exception on error.


\pagebreak
\subsubsection{\code{async\_set\_union}}
\label{api:ruby:async_set_union}
\index{async\_set\_union!Ruby API}
Store the union of the specified set and the existing value for each attribute.

%%% Generated below here
\paragraph{Behavior:}
\begin{itemize}[noitemsep]
\item This operation manipulates an existing object.  If no object exists, the
    operation will fail with \code{NOTFOUND}.

\end{itemize}


\paragraph{Definition:}
\begin{rubycode}
async_set_union(spacename, key, attributes)
\end{rubycode}

\paragraph{Parameters:}
\begin{itemize}[noitemsep]
\item \code{spacename}\\
The name of the space as a string or symbol.

\item \code{key}\\
The key for the operation where \code{key} is a Javascript value.

\item \code{attributes}\\
The set of attributes to modify and their respective values.  \code{attrs}
points to an array of length \code{attrs\_sz}.

\end{itemize}

\paragraph{Returns:}
A Deferred object with a \code{wait} method that returns True if the operation
succeeded or False if any provided predicates failed.  Raises an exception on
error.


\paragraph{See also:}  This is the asynchronous form of \code{set\_union}.

%%%%%%%%%%%%%%%%%%%% cond_set_union %%%%%%%%%%%%%%%%%%%%
\pagebreak
\subsubsection{\code{cond\_set\_union}}
\label{api:ruby:cond_set_union}
\index{cond\_set\_union!Ruby API}
Conditionally store the union of the specified set and the existing value for
each attribute.

%%% Generated below here
\paragraph{Behavior:}
\begin{itemize}[noitemsep]
\item This operation manipulates an existing object.  If no object exists, the
    operation will fail with \code{NOTFOUND}.

\item This operation will succeed if and only if the predicates specified by
    \code{checks} hold on the pre-existing object.  If any of the predicates are
    not true for the existing object, then the operation will have no effect and
    fail with \code{CMPFAIL}.

    All checks are atomic with the write.  HyperDex guarantees that no other
    operation will come between validating the checks, and writing the new
    version of the object..

\end{itemize}


\paragraph{Definition:}
\begin{rubycode}
cond_set_union(spacename, key, predicates, attributes)
\end{rubycode}

\paragraph{Parameters:}
\begin{itemize}[noitemsep]
\item \code{spacename}\\
The name of the space as a string or symbol.

\item \code{key}\\
The key for the operation where \code{key} is a Javascript value.

\item \code{predicates}\\
A set of predicates to check against.  \code{checks} points to an array of
length \code{checks\_sz}.

\item \code{attributes}\\
The set of attributes to modify and their respective values.  \code{attrs}
points to an array of length \code{attrs\_sz}.

\end{itemize}

\paragraph{Returns:}
True if the operation succeeded.  False if any provided predicates failed.
Raises an exception on error.


\pagebreak
\subsubsection{\code{async\_cond\_set\_union}}
\label{api:ruby:async_cond_set_union}
\index{async\_cond\_set\_union!Ruby API}
Conditionally store the union of the specified set and the existing value for
each attribute.

%%% Generated below here
\paragraph{Behavior:}
\begin{itemize}[noitemsep]
\item This operation manipulates an existing object.  If no object exists, the
    operation will fail with \code{NOTFOUND}.

\item This operation will succeed if and only if the predicates specified by
    \code{checks} hold on the pre-existing object.  If any of the predicates are
    not true for the existing object, then the operation will have no effect and
    fail with \code{CMPFAIL}.

    All checks are atomic with the write.  HyperDex guarantees that no other
    operation will come between validating the checks, and writing the new
    version of the object..

\end{itemize}


\paragraph{Definition:}
\begin{rubycode}
async_cond_set_union(spacename, key, predicates, attributes)
\end{rubycode}

\paragraph{Parameters:}
\begin{itemize}[noitemsep]
\item \code{spacename}\\
The name of the space as a string or symbol.

\item \code{key}\\
The key for the operation where \code{key} is a Javascript value.

\item \code{predicates}\\
A set of predicates to check against.  \code{checks} points to an array of
length \code{checks\_sz}.

\item \code{attributes}\\
The set of attributes to modify and their respective values.  \code{attrs}
points to an array of length \code{attrs\_sz}.

\end{itemize}

\paragraph{Returns:}
A Deferred object with a \code{wait} method that returns True if the operation
succeeded or False if any provided predicates failed.  Raises an exception on
error.


\paragraph{See also:}  This is the asynchronous form of \code{cond\_set\_union}.

%%%%%%%%%%%%%%%%%%%% map_add %%%%%%%%%%%%%%%%%%%%
\pagebreak
\subsubsection{\code{map\_add}}
\label{api:ruby:map_add}
\index{map\_add!Ruby API}
Insert a key-value pair into the map specified by each map-attribute.

%%% Generated below here
\paragraph{Behavior:}
\begin{itemize}[noitemsep]
\item This operation manipulates an existing object.  If no object exists, the
    operation will fail with \code{NOTFOUND}.

\end{itemize}


\paragraph{Definition:}
\begin{rubycode}
map_add(spacename, key, mapattributes)
\end{rubycode}

\paragraph{Parameters:}
\begin{itemize}[noitemsep]
\item \code{spacename}\\
The name of the space as a string or symbol.

\item \code{key}\\
The key for the operation where \code{key} is a Javascript value.

\item \code{mapattributes}\\
The set of map attributes to modify and their respective key/values.
\code{mapattrs} points to an array of length \code{mapattrs\_sz}.  Each entry
specify an attribute that is a map and a key within that map.

\end{itemize}

\paragraph{Returns:}
True if the operation succeeded.  False if any provided predicates failed.
Raises an exception on error.


\pagebreak
\subsubsection{\code{async\_map\_add}}
\label{api:ruby:async_map_add}
\index{async\_map\_add!Ruby API}
Insert a key-value pair into the map specified by each map-attribute.

%%% Generated below here
\paragraph{Behavior:}
\begin{itemize}[noitemsep]
\item This operation manipulates an existing object.  If no object exists, the
    operation will fail with \code{NOTFOUND}.

\end{itemize}


\paragraph{Definition:}
\begin{rubycode}
async_map_add(spacename, key, mapattributes)
\end{rubycode}

\paragraph{Parameters:}
\begin{itemize}[noitemsep]
\item \code{spacename}\\
The name of the space as a string or symbol.

\item \code{key}\\
The key for the operation where \code{key} is a Javascript value.

\item \code{mapattributes}\\
The set of map attributes to modify and their respective key/values.
\code{mapattrs} points to an array of length \code{mapattrs\_sz}.  Each entry
specify an attribute that is a map and a key within that map.

\end{itemize}

\paragraph{Returns:}
A Deferred object with a \code{wait} method that returns True if the operation
succeeded or False if any provided predicates failed.  Raises an exception on
error.


\paragraph{See also:}  This is the asynchronous form of \code{map\_add}.

%%%%%%%%%%%%%%%%%%%% cond_map_add %%%%%%%%%%%%%%%%%%%%
\pagebreak
\subsubsection{\code{cond\_map\_add}}
\label{api:ruby:cond_map_add}
\index{cond\_map\_add!Ruby API}
Conditionally insert a key-value pair into the map specified by each
map-attribute.

%%% Generated below here
\paragraph{Behavior:}
\begin{itemize}[noitemsep]
\item This operation manipulates an existing object.  If no object exists, the
    operation will fail with \code{NOTFOUND}.

\item This operation will succeed if and only if the predicates specified by
    \code{checks} hold on the pre-existing object.  If any of the predicates are
    not true for the existing object, then the operation will have no effect and
    fail with \code{CMPFAIL}.

    All checks are atomic with the write.  HyperDex guarantees that no other
    operation will come between validating the checks, and writing the new
    version of the object..

\end{itemize}


\paragraph{Definition:}
\begin{rubycode}
cond_map_add(spacename, key, predicates, mapattributes)
\end{rubycode}

\paragraph{Parameters:}
\begin{itemize}[noitemsep]
\item \code{spacename}\\
The name of the space as a string or symbol.

\item \code{key}\\
The key for the operation where \code{key} is a Javascript value.

\item \code{predicates}\\
A set of predicates to check against.  \code{checks} points to an array of
length \code{checks\_sz}.

\item \code{mapattributes}\\
The set of map attributes to modify and their respective key/values.
\code{mapattrs} points to an array of length \code{mapattrs\_sz}.  Each entry
specify an attribute that is a map and a key within that map.

\end{itemize}

\paragraph{Returns:}
True if the operation succeeded.  False if any provided predicates failed.
Raises an exception on error.


\pagebreak
\subsubsection{\code{async\_cond\_map\_add}}
\label{api:ruby:async_cond_map_add}
\index{async\_cond\_map\_add!Ruby API}
Conditionally insert a key-value pair into the map specified by each
map-attribute.

%%% Generated below here
\paragraph{Behavior:}
\begin{itemize}[noitemsep]
\item This operation manipulates an existing object.  If no object exists, the
    operation will fail with \code{NOTFOUND}.

\item This operation will succeed if and only if the predicates specified by
    \code{checks} hold on the pre-existing object.  If any of the predicates are
    not true for the existing object, then the operation will have no effect and
    fail with \code{CMPFAIL}.

    All checks are atomic with the write.  HyperDex guarantees that no other
    operation will come between validating the checks, and writing the new
    version of the object..

\end{itemize}


\paragraph{Definition:}
\begin{rubycode}
async_cond_map_add(spacename, key, predicates, mapattributes)
\end{rubycode}

\paragraph{Parameters:}
\begin{itemize}[noitemsep]
\item \code{spacename}\\
The name of the space as a string or symbol.

\item \code{key}\\
The key for the operation where \code{key} is a Javascript value.

\item \code{predicates}\\
A set of predicates to check against.  \code{checks} points to an array of
length \code{checks\_sz}.

\item \code{mapattributes}\\
The set of map attributes to modify and their respective key/values.
\code{mapattrs} points to an array of length \code{mapattrs\_sz}.  Each entry
specify an attribute that is a map and a key within that map.

\end{itemize}

\paragraph{Returns:}
A Deferred object with a \code{wait} method that returns True if the operation
succeeded or False if any provided predicates failed.  Raises an exception on
error.


\paragraph{See also:}  This is the asynchronous form of \code{cond\_map\_add}.

%%%%%%%%%%%%%%%%%%%% map_remove %%%%%%%%%%%%%%%%%%%%
\pagebreak
\subsubsection{\code{map\_remove}}
\label{api:ruby:map_remove}
\index{map\_remove!Ruby API}
Remove a key-value pair from the map specified by each attribute.  If there is
no pair with the specified key within the map, this operation will do nothing.

%%% Generated below here
\paragraph{Behavior:}
\begin{itemize}[noitemsep]
\item This operation manipulates an existing object.  If no object exists, the
    operation will fail with \code{NOTFOUND}.

\end{itemize}


\paragraph{Definition:}
\begin{rubycode}
map_remove(spacename, key, attributes)
\end{rubycode}

\paragraph{Parameters:}
\begin{itemize}[noitemsep]
\item \code{spacename}\\
The name of the space as a string or symbol.

\item \code{key}\\
The key for the operation where \code{key} is a Javascript value.

\item \code{attributes}\\
The set of attributes to modify and their respective values.  \code{attrs}
points to an array of length \code{attrs\_sz}.

\end{itemize}

\paragraph{Returns:}
True if the operation succeeded.  False if any provided predicates failed.
Raises an exception on error.


\pagebreak
\subsubsection{\code{async\_map\_remove}}
\label{api:ruby:async_map_remove}
\index{async\_map\_remove!Ruby API}
Remove a key-value pair from the map specified by each attribute.  If there is
no pair with the specified key within the map, this operation will do nothing.

%%% Generated below here
\paragraph{Behavior:}
\begin{itemize}[noitemsep]
\item This operation manipulates an existing object.  If no object exists, the
    operation will fail with \code{NOTFOUND}.

\end{itemize}


\paragraph{Definition:}
\begin{rubycode}
async_map_remove(spacename, key, attributes)
\end{rubycode}

\paragraph{Parameters:}
\begin{itemize}[noitemsep]
\item \code{spacename}\\
The name of the space as a string or symbol.

\item \code{key}\\
The key for the operation where \code{key} is a Javascript value.

\item \code{attributes}\\
The set of attributes to modify and their respective values.  \code{attrs}
points to an array of length \code{attrs\_sz}.

\end{itemize}

\paragraph{Returns:}
A Deferred object with a \code{wait} method that returns True if the operation
succeeded or False if any provided predicates failed.  Raises an exception on
error.


\paragraph{See also:}  This is the asynchronous form of \code{map\_remove}.

%%%%%%%%%%%%%%%%%%%% cond_map_remove %%%%%%%%%%%%%%%%%%%%
\pagebreak
\subsubsection{\code{cond\_map\_remove}}
\label{api:ruby:cond_map_remove}
\index{cond\_map\_remove!Ruby API}
Conditionally remove a key-value pair from the map specified by each attribute.

%%% Generated below here
\paragraph{Behavior:}
\begin{itemize}[noitemsep]
\item This operation manipulates an existing object.  If no object exists, the
    operation will fail with \code{NOTFOUND}.

\item This operation will succeed if and only if the predicates specified by
    \code{checks} hold on the pre-existing object.  If any of the predicates are
    not true for the existing object, then the operation will have no effect and
    fail with \code{CMPFAIL}.

    All checks are atomic with the write.  HyperDex guarantees that no other
    operation will come between validating the checks, and writing the new
    version of the object..

\end{itemize}


\paragraph{Definition:}
\begin{rubycode}
cond_map_remove(spacename, key, predicates, attributes)
\end{rubycode}

\paragraph{Parameters:}
\begin{itemize}[noitemsep]
\item \code{spacename}\\
The name of the space as a string or symbol.

\item \code{key}\\
The key for the operation where \code{key} is a Javascript value.

\item \code{predicates}\\
A set of predicates to check against.  \code{checks} points to an array of
length \code{checks\_sz}.

\item \code{attributes}\\
The set of attributes to modify and their respective values.  \code{attrs}
points to an array of length \code{attrs\_sz}.

\end{itemize}

\paragraph{Returns:}
True if the operation succeeded.  False if any provided predicates failed.
Raises an exception on error.


\pagebreak
\subsubsection{\code{async\_cond\_map\_remove}}
\label{api:ruby:async_cond_map_remove}
\index{async\_cond\_map\_remove!Ruby API}
Conditionally remove a key-value pair from the map specified by each attribute.

%%% Generated below here
\paragraph{Behavior:}
\begin{itemize}[noitemsep]
\item This operation manipulates an existing object.  If no object exists, the
    operation will fail with \code{NOTFOUND}.

\item This operation will succeed if and only if the predicates specified by
    \code{checks} hold on the pre-existing object.  If any of the predicates are
    not true for the existing object, then the operation will have no effect and
    fail with \code{CMPFAIL}.

    All checks are atomic with the write.  HyperDex guarantees that no other
    operation will come between validating the checks, and writing the new
    version of the object..

\end{itemize}


\paragraph{Definition:}
\begin{rubycode}
async_cond_map_remove(spacename, key, predicates, attributes)
\end{rubycode}

\paragraph{Parameters:}
\begin{itemize}[noitemsep]
\item \code{spacename}\\
The name of the space as a string or symbol.

\item \code{key}\\
The key for the operation where \code{key} is a Javascript value.

\item \code{predicates}\\
A set of predicates to check against.  \code{checks} points to an array of
length \code{checks\_sz}.

\item \code{attributes}\\
The set of attributes to modify and their respective values.  \code{attrs}
points to an array of length \code{attrs\_sz}.

\end{itemize}

\paragraph{Returns:}
A Deferred object with a \code{wait} method that returns True if the operation
succeeded or False if any provided predicates failed.  Raises an exception on
error.


\paragraph{See also:}  This is the asynchronous form of \code{cond\_map\_remove}.

%%%%%%%%%%%%%%%%%%%% map_atomic_add %%%%%%%%%%%%%%%%%%%%
\pagebreak
\subsubsection{\code{map\_atomic\_add}}
\label{api:ruby:map_atomic_add}
\index{map\_atomic\_add!Ruby API}
Add the specified number to the value of a key-value pair within each map.

%%% Generated below here
\paragraph{Behavior:}
\begin{itemize}[noitemsep]
\item This operation manipulates an existing object.  If no object exists, the
    operation will fail with \code{NOTFOUND}.

\item This operation mutates the value of a key-value pair in a map.  This call
    is similar to the equivalent call without the \code{map\_} prefix, but
    operates on the value of a pair in a map, instead of on an attribute's
    value.  If there is no pair with the specified map key, a new pair will be
    created and initialized to its default value.  If this is undesirable, it
    may be avoided by using a conditional operation that requires that the map
    contain the key in question.

\end{itemize}


\paragraph{Definition:}
\begin{rubycode}
map_atomic_add(spacename, key, mapattributes)
\end{rubycode}

\paragraph{Parameters:}
\begin{itemize}[noitemsep]
\item \code{spacename}\\
The name of the space as a string or symbol.

\item \code{key}\\
The key for the operation where \code{key} is a Javascript value.

\item \code{mapattributes}\\
The set of map attributes to modify and their respective key/values.
\code{mapattrs} points to an array of length \code{mapattrs\_sz}.  Each entry
specify an attribute that is a map and a key within that map.

\end{itemize}

\paragraph{Returns:}
True if the operation succeeded.  False if any provided predicates failed.
Raises an exception on error.


\pagebreak
\subsubsection{\code{async\_map\_atomic\_add}}
\label{api:ruby:async_map_atomic_add}
\index{async\_map\_atomic\_add!Ruby API}
Add the specified number to the value of a key-value pair within each map.

%%% Generated below here
\paragraph{Behavior:}
\begin{itemize}[noitemsep]
\item This operation manipulates an existing object.  If no object exists, the
    operation will fail with \code{NOTFOUND}.

\item This operation mutates the value of a key-value pair in a map.  This call
    is similar to the equivalent call without the \code{map\_} prefix, but
    operates on the value of a pair in a map, instead of on an attribute's
    value.  If there is no pair with the specified map key, a new pair will be
    created and initialized to its default value.  If this is undesirable, it
    may be avoided by using a conditional operation that requires that the map
    contain the key in question.

\end{itemize}


\paragraph{Definition:}
\begin{rubycode}
async_map_atomic_add(spacename, key, mapattributes)
\end{rubycode}

\paragraph{Parameters:}
\begin{itemize}[noitemsep]
\item \code{spacename}\\
The name of the space as a string or symbol.

\item \code{key}\\
The key for the operation where \code{key} is a Javascript value.

\item \code{mapattributes}\\
The set of map attributes to modify and their respective key/values.
\code{mapattrs} points to an array of length \code{mapattrs\_sz}.  Each entry
specify an attribute that is a map and a key within that map.

\end{itemize}

\paragraph{Returns:}
A Deferred object with a \code{wait} method that returns True if the operation
succeeded or False if any provided predicates failed.  Raises an exception on
error.


\paragraph{See also:}  This is the asynchronous form of \code{map\_atomic\_add}.

%%%%%%%%%%%%%%%%%%%% cond_map_atomic_add %%%%%%%%%%%%%%%%%%%%
\pagebreak
\subsubsection{\code{cond\_map\_atomic\_add}}
\label{api:ruby:cond_map_atomic_add}
\index{cond\_map\_atomic\_add!Ruby API}
Conditionally add the specified number to the value of a key-value pair within
each map.

%%% Generated below here
\paragraph{Behavior:}
\begin{itemize}[noitemsep]
\item This operation manipulates an existing object.  If no object exists, the
    operation will fail with \code{NOTFOUND}.

\item This operation will succeed if and only if the predicates specified by
    \code{checks} hold on the pre-existing object.  If any of the predicates are
    not true for the existing object, then the operation will have no effect and
    fail with \code{CMPFAIL}.

    All checks are atomic with the write.  HyperDex guarantees that no other
    operation will come between validating the checks, and writing the new
    version of the object..

\item This operation mutates the value of a key-value pair in a map.  This call
    is similar to the equivalent call without the \code{map\_} prefix, but
    operates on the value of a pair in a map, instead of on an attribute's
    value.  If there is no pair with the specified map key, a new pair will be
    created and initialized to its default value.  If this is undesirable, it
    may be avoided by using a conditional operation that requires that the map
    contain the key in question.

\end{itemize}


\paragraph{Definition:}
\begin{rubycode}
cond_map_atomic_add(spacename, key, predicates, mapattributes)
\end{rubycode}

\paragraph{Parameters:}
\begin{itemize}[noitemsep]
\item \code{spacename}\\
The name of the space as a string or symbol.

\item \code{key}\\
The key for the operation where \code{key} is a Javascript value.

\item \code{predicates}\\
A set of predicates to check against.  \code{checks} points to an array of
length \code{checks\_sz}.

\item \code{mapattributes}\\
The set of map attributes to modify and their respective key/values.
\code{mapattrs} points to an array of length \code{mapattrs\_sz}.  Each entry
specify an attribute that is a map and a key within that map.

\end{itemize}

\paragraph{Returns:}
True if the operation succeeded.  False if any provided predicates failed.
Raises an exception on error.


\pagebreak
\subsubsection{\code{async\_cond\_map\_atomic\_add}}
\label{api:ruby:async_cond_map_atomic_add}
\index{async\_cond\_map\_atomic\_add!Ruby API}
Conditionally add the specified number to the value of a key-value pair within
each map.

%%% Generated below here
\paragraph{Behavior:}
\begin{itemize}[noitemsep]
\item This operation manipulates an existing object.  If no object exists, the
    operation will fail with \code{NOTFOUND}.

\item This operation will succeed if and only if the predicates specified by
    \code{checks} hold on the pre-existing object.  If any of the predicates are
    not true for the existing object, then the operation will have no effect and
    fail with \code{CMPFAIL}.

    All checks are atomic with the write.  HyperDex guarantees that no other
    operation will come between validating the checks, and writing the new
    version of the object..

\item This operation mutates the value of a key-value pair in a map.  This call
    is similar to the equivalent call without the \code{map\_} prefix, but
    operates on the value of a pair in a map, instead of on an attribute's
    value.  If there is no pair with the specified map key, a new pair will be
    created and initialized to its default value.  If this is undesirable, it
    may be avoided by using a conditional operation that requires that the map
    contain the key in question.

\end{itemize}


\paragraph{Definition:}
\begin{rubycode}
async_cond_map_atomic_add(spacename, key, predicates, mapattributes)
\end{rubycode}

\paragraph{Parameters:}
\begin{itemize}[noitemsep]
\item \code{spacename}\\
The name of the space as a string or symbol.

\item \code{key}\\
The key for the operation where \code{key} is a Javascript value.

\item \code{predicates}\\
A set of predicates to check against.  \code{checks} points to an array of
length \code{checks\_sz}.

\item \code{mapattributes}\\
The set of map attributes to modify and their respective key/values.
\code{mapattrs} points to an array of length \code{mapattrs\_sz}.  Each entry
specify an attribute that is a map and a key within that map.

\end{itemize}

\paragraph{Returns:}
A Deferred object with a \code{wait} method that returns True if the operation
succeeded or False if any provided predicates failed.  Raises an exception on
error.


\paragraph{See also:}  This is the asynchronous form of \code{cond\_map\_atomic\_add}.

%%%%%%%%%%%%%%%%%%%% map_atomic_sub %%%%%%%%%%%%%%%%%%%%
\pagebreak
\subsubsection{\code{map\_atomic\_sub}}
\label{api:ruby:map_atomic_sub}
\index{map\_atomic\_sub!Ruby API}
Subtract the specified number from the value of a key-value pair within each
map.

%%% Generated below here
\paragraph{Behavior:}
\begin{itemize}[noitemsep]
\item This operation manipulates an existing object.  If no object exists, the
    operation will fail with \code{NOTFOUND}.

\item This operation mutates the value of a key-value pair in a map.  This call
    is similar to the equivalent call without the \code{map\_} prefix, but
    operates on the value of a pair in a map, instead of on an attribute's
    value.  If there is no pair with the specified map key, a new pair will be
    created and initialized to its default value.  If this is undesirable, it
    may be avoided by using a conditional operation that requires that the map
    contain the key in question.

\end{itemize}


\paragraph{Definition:}
\begin{rubycode}
map_atomic_sub(spacename, key, mapattributes)
\end{rubycode}

\paragraph{Parameters:}
\begin{itemize}[noitemsep]
\item \code{spacename}\\
The name of the space as a string or symbol.

\item \code{key}\\
The key for the operation where \code{key} is a Javascript value.

\item \code{mapattributes}\\
The set of map attributes to modify and their respective key/values.
\code{mapattrs} points to an array of length \code{mapattrs\_sz}.  Each entry
specify an attribute that is a map and a key within that map.

\end{itemize}

\paragraph{Returns:}
True if the operation succeeded.  False if any provided predicates failed.
Raises an exception on error.


\pagebreak
\subsubsection{\code{async\_map\_atomic\_sub}}
\label{api:ruby:async_map_atomic_sub}
\index{async\_map\_atomic\_sub!Ruby API}
Subtract the specified number from the value of a key-value pair within each
map.

%%% Generated below here
\paragraph{Behavior:}
\begin{itemize}[noitemsep]
\item This operation manipulates an existing object.  If no object exists, the
    operation will fail with \code{NOTFOUND}.

\item This operation mutates the value of a key-value pair in a map.  This call
    is similar to the equivalent call without the \code{map\_} prefix, but
    operates on the value of a pair in a map, instead of on an attribute's
    value.  If there is no pair with the specified map key, a new pair will be
    created and initialized to its default value.  If this is undesirable, it
    may be avoided by using a conditional operation that requires that the map
    contain the key in question.

\end{itemize}


\paragraph{Definition:}
\begin{rubycode}
async_map_atomic_sub(spacename, key, mapattributes)
\end{rubycode}

\paragraph{Parameters:}
\begin{itemize}[noitemsep]
\item \code{spacename}\\
The name of the space as a string or symbol.

\item \code{key}\\
The key for the operation where \code{key} is a Javascript value.

\item \code{mapattributes}\\
The set of map attributes to modify and their respective key/values.
\code{mapattrs} points to an array of length \code{mapattrs\_sz}.  Each entry
specify an attribute that is a map and a key within that map.

\end{itemize}

\paragraph{Returns:}
A Deferred object with a \code{wait} method that returns True if the operation
succeeded or False if any provided predicates failed.  Raises an exception on
error.


\paragraph{See also:}  This is the asynchronous form of \code{map\_atomic\_sub}.

%%%%%%%%%%%%%%%%%%%% cond_map_atomic_sub %%%%%%%%%%%%%%%%%%%%
\pagebreak
\subsubsection{\code{cond\_map\_atomic\_sub}}
\label{api:ruby:cond_map_atomic_sub}
\index{cond\_map\_atomic\_sub!Ruby API}
Subtract the specified number from the value of a key-value pair within each
map.

%%% Generated below here
\paragraph{Behavior:}
\begin{itemize}[noitemsep]
\item This operation manipulates an existing object.  If no object exists, the
    operation will fail with \code{NOTFOUND}.

\item This operation will succeed if and only if the predicates specified by
    \code{checks} hold on the pre-existing object.  If any of the predicates are
    not true for the existing object, then the operation will have no effect and
    fail with \code{CMPFAIL}.

    All checks are atomic with the write.  HyperDex guarantees that no other
    operation will come between validating the checks, and writing the new
    version of the object..

\item This operation mutates the value of a key-value pair in a map.  This call
    is similar to the equivalent call without the \code{map\_} prefix, but
    operates on the value of a pair in a map, instead of on an attribute's
    value.  If there is no pair with the specified map key, a new pair will be
    created and initialized to its default value.  If this is undesirable, it
    may be avoided by using a conditional operation that requires that the map
    contain the key in question.

\end{itemize}


\paragraph{Definition:}
\begin{rubycode}
cond_map_atomic_sub(spacename, key, predicates, mapattributes)
\end{rubycode}

\paragraph{Parameters:}
\begin{itemize}[noitemsep]
\item \code{spacename}\\
The name of the space as a string or symbol.

\item \code{key}\\
The key for the operation where \code{key} is a Javascript value.

\item \code{predicates}\\
A set of predicates to check against.  \code{checks} points to an array of
length \code{checks\_sz}.

\item \code{mapattributes}\\
The set of map attributes to modify and their respective key/values.
\code{mapattrs} points to an array of length \code{mapattrs\_sz}.  Each entry
specify an attribute that is a map and a key within that map.

\end{itemize}

\paragraph{Returns:}
True if the operation succeeded.  False if any provided predicates failed.
Raises an exception on error.


\pagebreak
\subsubsection{\code{async\_cond\_map\_atomic\_sub}}
\label{api:ruby:async_cond_map_atomic_sub}
\index{async\_cond\_map\_atomic\_sub!Ruby API}
Subtract the specified number from the value of a key-value pair within each
map.

%%% Generated below here
\paragraph{Behavior:}
\begin{itemize}[noitemsep]
\item This operation manipulates an existing object.  If no object exists, the
    operation will fail with \code{NOTFOUND}.

\item This operation will succeed if and only if the predicates specified by
    \code{checks} hold on the pre-existing object.  If any of the predicates are
    not true for the existing object, then the operation will have no effect and
    fail with \code{CMPFAIL}.

    All checks are atomic with the write.  HyperDex guarantees that no other
    operation will come between validating the checks, and writing the new
    version of the object..

\item This operation mutates the value of a key-value pair in a map.  This call
    is similar to the equivalent call without the \code{map\_} prefix, but
    operates on the value of a pair in a map, instead of on an attribute's
    value.  If there is no pair with the specified map key, a new pair will be
    created and initialized to its default value.  If this is undesirable, it
    may be avoided by using a conditional operation that requires that the map
    contain the key in question.

\end{itemize}


\paragraph{Definition:}
\begin{rubycode}
async_cond_map_atomic_sub(spacename, key, predicates, mapattributes)
\end{rubycode}

\paragraph{Parameters:}
\begin{itemize}[noitemsep]
\item \code{spacename}\\
The name of the space as a string or symbol.

\item \code{key}\\
The key for the operation where \code{key} is a Javascript value.

\item \code{predicates}\\
A set of predicates to check against.  \code{checks} points to an array of
length \code{checks\_sz}.

\item \code{mapattributes}\\
The set of map attributes to modify and their respective key/values.
\code{mapattrs} points to an array of length \code{mapattrs\_sz}.  Each entry
specify an attribute that is a map and a key within that map.

\end{itemize}

\paragraph{Returns:}
A Deferred object with a \code{wait} method that returns True if the operation
succeeded or False if any provided predicates failed.  Raises an exception on
error.


\paragraph{See also:}  This is the asynchronous form of \code{cond\_map\_atomic\_sub}.

%%%%%%%%%%%%%%%%%%%% map_atomic_mul %%%%%%%%%%%%%%%%%%%%
\pagebreak
\subsubsection{\code{map\_atomic\_mul}}
\label{api:ruby:map_atomic_mul}
\index{map\_atomic\_mul!Ruby API}
Multiply the value of each key-value pair by the specified number for each map.

%%% Generated below here
\paragraph{Behavior:}
\begin{itemize}[noitemsep]
\item This operation manipulates an existing object.  If no object exists, the
    operation will fail with \code{NOTFOUND}.

\item This operation mutates the value of a key-value pair in a map.  This call
    is similar to the equivalent call without the \code{map\_} prefix, but
    operates on the value of a pair in a map, instead of on an attribute's
    value.  If there is no pair with the specified map key, a new pair will be
    created and initialized to its default value.  If this is undesirable, it
    may be avoided by using a conditional operation that requires that the map
    contain the key in question.

\end{itemize}


\paragraph{Definition:}
\begin{rubycode}
map_atomic_mul(spacename, key, mapattributes)
\end{rubycode}

\paragraph{Parameters:}
\begin{itemize}[noitemsep]
\item \code{spacename}\\
The name of the space as a string or symbol.

\item \code{key}\\
The key for the operation where \code{key} is a Javascript value.

\item \code{mapattributes}\\
The set of map attributes to modify and their respective key/values.
\code{mapattrs} points to an array of length \code{mapattrs\_sz}.  Each entry
specify an attribute that is a map and a key within that map.

\end{itemize}

\paragraph{Returns:}
True if the operation succeeded.  False if any provided predicates failed.
Raises an exception on error.


\pagebreak
\subsubsection{\code{async\_map\_atomic\_mul}}
\label{api:ruby:async_map_atomic_mul}
\index{async\_map\_atomic\_mul!Ruby API}
Multiply the value of each key-value pair by the specified number for each map.

%%% Generated below here
\paragraph{Behavior:}
\begin{itemize}[noitemsep]
\item This operation manipulates an existing object.  If no object exists, the
    operation will fail with \code{NOTFOUND}.

\item This operation mutates the value of a key-value pair in a map.  This call
    is similar to the equivalent call without the \code{map\_} prefix, but
    operates on the value of a pair in a map, instead of on an attribute's
    value.  If there is no pair with the specified map key, a new pair will be
    created and initialized to its default value.  If this is undesirable, it
    may be avoided by using a conditional operation that requires that the map
    contain the key in question.

\end{itemize}


\paragraph{Definition:}
\begin{rubycode}
async_map_atomic_mul(spacename, key, mapattributes)
\end{rubycode}

\paragraph{Parameters:}
\begin{itemize}[noitemsep]
\item \code{spacename}\\
The name of the space as a string or symbol.

\item \code{key}\\
The key for the operation where \code{key} is a Javascript value.

\item \code{mapattributes}\\
The set of map attributes to modify and their respective key/values.
\code{mapattrs} points to an array of length \code{mapattrs\_sz}.  Each entry
specify an attribute that is a map and a key within that map.

\end{itemize}

\paragraph{Returns:}
A Deferred object with a \code{wait} method that returns True if the operation
succeeded or False if any provided predicates failed.  Raises an exception on
error.


\paragraph{See also:}  This is the asynchronous form of \code{map\_atomic\_mul}.

%%%%%%%%%%%%%%%%%%%% cond_map_atomic_mul %%%%%%%%%%%%%%%%%%%%
\pagebreak
\subsubsection{\code{cond\_map\_atomic\_mul}}
\label{api:ruby:cond_map_atomic_mul}
\index{cond\_map\_atomic\_mul!Ruby API}
Conditionally multiply the value of each key-value pair by the specified number
for each map.

%%% Generated below here
\paragraph{Behavior:}
\begin{itemize}[noitemsep]
\item This operation manipulates an existing object.  If no object exists, the
    operation will fail with \code{NOTFOUND}.

\item This operation will succeed if and only if the predicates specified by
    \code{checks} hold on the pre-existing object.  If any of the predicates are
    not true for the existing object, then the operation will have no effect and
    fail with \code{CMPFAIL}.

    All checks are atomic with the write.  HyperDex guarantees that no other
    operation will come between validating the checks, and writing the new
    version of the object..

\item This operation mutates the value of a key-value pair in a map.  This call
    is similar to the equivalent call without the \code{map\_} prefix, but
    operates on the value of a pair in a map, instead of on an attribute's
    value.  If there is no pair with the specified map key, a new pair will be
    created and initialized to its default value.  If this is undesirable, it
    may be avoided by using a conditional operation that requires that the map
    contain the key in question.

\end{itemize}


\paragraph{Definition:}
\begin{rubycode}
cond_map_atomic_mul(spacename, key, predicates, mapattributes)
\end{rubycode}

\paragraph{Parameters:}
\begin{itemize}[noitemsep]
\item \code{spacename}\\
The name of the space as a string or symbol.

\item \code{key}\\
The key for the operation where \code{key} is a Javascript value.

\item \code{predicates}\\
A set of predicates to check against.  \code{checks} points to an array of
length \code{checks\_sz}.

\item \code{mapattributes}\\
The set of map attributes to modify and their respective key/values.
\code{mapattrs} points to an array of length \code{mapattrs\_sz}.  Each entry
specify an attribute that is a map and a key within that map.

\end{itemize}

\paragraph{Returns:}
True if the operation succeeded.  False if any provided predicates failed.
Raises an exception on error.


\pagebreak
\subsubsection{\code{async\_cond\_map\_atomic\_mul}}
\label{api:ruby:async_cond_map_atomic_mul}
\index{async\_cond\_map\_atomic\_mul!Ruby API}
Conditionally multiply the value of each key-value pair by the specified number
for each map.

%%% Generated below here
\paragraph{Behavior:}
\begin{itemize}[noitemsep]
\item This operation manipulates an existing object.  If no object exists, the
    operation will fail with \code{NOTFOUND}.

\item This operation will succeed if and only if the predicates specified by
    \code{checks} hold on the pre-existing object.  If any of the predicates are
    not true for the existing object, then the operation will have no effect and
    fail with \code{CMPFAIL}.

    All checks are atomic with the write.  HyperDex guarantees that no other
    operation will come between validating the checks, and writing the new
    version of the object..

\item This operation mutates the value of a key-value pair in a map.  This call
    is similar to the equivalent call without the \code{map\_} prefix, but
    operates on the value of a pair in a map, instead of on an attribute's
    value.  If there is no pair with the specified map key, a new pair will be
    created and initialized to its default value.  If this is undesirable, it
    may be avoided by using a conditional operation that requires that the map
    contain the key in question.

\end{itemize}


\paragraph{Definition:}
\begin{rubycode}
async_cond_map_atomic_mul(spacename, key, predicates, mapattributes)
\end{rubycode}

\paragraph{Parameters:}
\begin{itemize}[noitemsep]
\item \code{spacename}\\
The name of the space as a string or symbol.

\item \code{key}\\
The key for the operation where \code{key} is a Javascript value.

\item \code{predicates}\\
A set of predicates to check against.  \code{checks} points to an array of
length \code{checks\_sz}.

\item \code{mapattributes}\\
The set of map attributes to modify and their respective key/values.
\code{mapattrs} points to an array of length \code{mapattrs\_sz}.  Each entry
specify an attribute that is a map and a key within that map.

\end{itemize}

\paragraph{Returns:}
A Deferred object with a \code{wait} method that returns True if the operation
succeeded or False if any provided predicates failed.  Raises an exception on
error.


\paragraph{See also:}  This is the asynchronous form of \code{cond\_map\_atomic\_mul}.

%%%%%%%%%%%%%%%%%%%% map_atomic_div %%%%%%%%%%%%%%%%%%%%
\pagebreak
\subsubsection{\code{map\_atomic\_div}}
\label{api:ruby:map_atomic_div}
\index{map\_atomic\_div!Ruby API}
Divide the value of each key-value pair by the specified number for each map.

%%% Generated below here
\paragraph{Behavior:}
\begin{itemize}[noitemsep]
\item This operation manipulates an existing object.  If no object exists, the
    operation will fail with \code{NOTFOUND}.

\item This operation mutates the value of a key-value pair in a map.  This call
    is similar to the equivalent call without the \code{map\_} prefix, but
    operates on the value of a pair in a map, instead of on an attribute's
    value.  If there is no pair with the specified map key, a new pair will be
    created and initialized to its default value.  If this is undesirable, it
    may be avoided by using a conditional operation that requires that the map
    contain the key in question.

\end{itemize}


\paragraph{Definition:}
\begin{rubycode}
map_atomic_div(spacename, key, mapattributes)
\end{rubycode}

\paragraph{Parameters:}
\begin{itemize}[noitemsep]
\item \code{spacename}\\
The name of the space as a string or symbol.

\item \code{key}\\
The key for the operation where \code{key} is a Javascript value.

\item \code{mapattributes}\\
The set of map attributes to modify and their respective key/values.
\code{mapattrs} points to an array of length \code{mapattrs\_sz}.  Each entry
specify an attribute that is a map and a key within that map.

\end{itemize}

\paragraph{Returns:}
True if the operation succeeded.  False if any provided predicates failed.
Raises an exception on error.


\pagebreak
\subsubsection{\code{async\_map\_atomic\_div}}
\label{api:ruby:async_map_atomic_div}
\index{async\_map\_atomic\_div!Ruby API}
Divide the value of each key-value pair by the specified number for each map.

%%% Generated below here
\paragraph{Behavior:}
\begin{itemize}[noitemsep]
\item This operation manipulates an existing object.  If no object exists, the
    operation will fail with \code{NOTFOUND}.

\item This operation mutates the value of a key-value pair in a map.  This call
    is similar to the equivalent call without the \code{map\_} prefix, but
    operates on the value of a pair in a map, instead of on an attribute's
    value.  If there is no pair with the specified map key, a new pair will be
    created and initialized to its default value.  If this is undesirable, it
    may be avoided by using a conditional operation that requires that the map
    contain the key in question.

\end{itemize}


\paragraph{Definition:}
\begin{rubycode}
async_map_atomic_div(spacename, key, mapattributes)
\end{rubycode}

\paragraph{Parameters:}
\begin{itemize}[noitemsep]
\item \code{spacename}\\
The name of the space as a string or symbol.

\item \code{key}\\
The key for the operation where \code{key} is a Javascript value.

\item \code{mapattributes}\\
The set of map attributes to modify and their respective key/values.
\code{mapattrs} points to an array of length \code{mapattrs\_sz}.  Each entry
specify an attribute that is a map and a key within that map.

\end{itemize}

\paragraph{Returns:}
A Deferred object with a \code{wait} method that returns True if the operation
succeeded or False if any provided predicates failed.  Raises an exception on
error.


\paragraph{See also:}  This is the asynchronous form of \code{map\_atomic\_div}.

%%%%%%%%%%%%%%%%%%%% cond_map_atomic_div %%%%%%%%%%%%%%%%%%%%
\pagebreak
\subsubsection{\code{cond\_map\_atomic\_div}}
\label{api:ruby:cond_map_atomic_div}
\index{cond\_map\_atomic\_div!Ruby API}
Conditionally divide the value of each key-value pair by the specified number for each map.

%%% Generated below here
\paragraph{Behavior:}
\begin{itemize}[noitemsep]
\item This operation manipulates an existing object.  If no object exists, the
    operation will fail with \code{NOTFOUND}.

\item This operation will succeed if and only if the predicates specified by
    \code{checks} hold on the pre-existing object.  If any of the predicates are
    not true for the existing object, then the operation will have no effect and
    fail with \code{CMPFAIL}.

    All checks are atomic with the write.  HyperDex guarantees that no other
    operation will come between validating the checks, and writing the new
    version of the object..

\item This operation mutates the value of a key-value pair in a map.  This call
    is similar to the equivalent call without the \code{map\_} prefix, but
    operates on the value of a pair in a map, instead of on an attribute's
    value.  If there is no pair with the specified map key, a new pair will be
    created and initialized to its default value.  If this is undesirable, it
    may be avoided by using a conditional operation that requires that the map
    contain the key in question.

\end{itemize}


\paragraph{Definition:}
\begin{rubycode}
cond_map_atomic_div(spacename, key, predicates, mapattributes)
\end{rubycode}

\paragraph{Parameters:}
\begin{itemize}[noitemsep]
\item \code{spacename}\\
The name of the space as a string or symbol.

\item \code{key}\\
The key for the operation where \code{key} is a Javascript value.

\item \code{predicates}\\
A set of predicates to check against.  \code{checks} points to an array of
length \code{checks\_sz}.

\item \code{mapattributes}\\
The set of map attributes to modify and their respective key/values.
\code{mapattrs} points to an array of length \code{mapattrs\_sz}.  Each entry
specify an attribute that is a map and a key within that map.

\end{itemize}

\paragraph{Returns:}
True if the operation succeeded.  False if any provided predicates failed.
Raises an exception on error.


\pagebreak
\subsubsection{\code{async\_cond\_map\_atomic\_div}}
\label{api:ruby:async_cond_map_atomic_div}
\index{async\_cond\_map\_atomic\_div!Ruby API}
Conditionally divide the value of each key-value pair by the specified number for each map.

%%% Generated below here
\paragraph{Behavior:}
\begin{itemize}[noitemsep]
\item This operation manipulates an existing object.  If no object exists, the
    operation will fail with \code{NOTFOUND}.

\item This operation will succeed if and only if the predicates specified by
    \code{checks} hold on the pre-existing object.  If any of the predicates are
    not true for the existing object, then the operation will have no effect and
    fail with \code{CMPFAIL}.

    All checks are atomic with the write.  HyperDex guarantees that no other
    operation will come between validating the checks, and writing the new
    version of the object..

\item This operation mutates the value of a key-value pair in a map.  This call
    is similar to the equivalent call without the \code{map\_} prefix, but
    operates on the value of a pair in a map, instead of on an attribute's
    value.  If there is no pair with the specified map key, a new pair will be
    created and initialized to its default value.  If this is undesirable, it
    may be avoided by using a conditional operation that requires that the map
    contain the key in question.

\end{itemize}


\paragraph{Definition:}
\begin{rubycode}
async_cond_map_atomic_div(spacename, key, predicates, mapattributes)
\end{rubycode}

\paragraph{Parameters:}
\begin{itemize}[noitemsep]
\item \code{spacename}\\
The name of the space as a string or symbol.

\item \code{key}\\
The key for the operation where \code{key} is a Javascript value.

\item \code{predicates}\\
A set of predicates to check against.  \code{checks} points to an array of
length \code{checks\_sz}.

\item \code{mapattributes}\\
The set of map attributes to modify and their respective key/values.
\code{mapattrs} points to an array of length \code{mapattrs\_sz}.  Each entry
specify an attribute that is a map and a key within that map.

\end{itemize}

\paragraph{Returns:}
A Deferred object with a \code{wait} method that returns True if the operation
succeeded or False if any provided predicates failed.  Raises an exception on
error.


\paragraph{See also:}  This is the asynchronous form of \code{cond\_map\_atomic\_div}.

%%%%%%%%%%%%%%%%%%%% map_atomic_mod %%%%%%%%%%%%%%%%%%%%
\pagebreak
\subsubsection{\code{map\_atomic\_mod}}
\label{api:ruby:map_atomic_mod}
\index{map\_atomic\_mod!Ruby API}
Store the value of the key-value pair modulo the specified number for each map.

%%% Generated below here
\paragraph{Behavior:}
\begin{itemize}[noitemsep]
\item This operation manipulates an existing object.  If no object exists, the
    operation will fail with \code{NOTFOUND}.

\item This operation mutates the value of a key-value pair in a map.  This call
    is similar to the equivalent call without the \code{map\_} prefix, but
    operates on the value of a pair in a map, instead of on an attribute's
    value.  If there is no pair with the specified map key, a new pair will be
    created and initialized to its default value.  If this is undesirable, it
    may be avoided by using a conditional operation that requires that the map
    contain the key in question.

\end{itemize}


\paragraph{Definition:}
\begin{rubycode}
map_atomic_mod(spacename, key, mapattributes)
\end{rubycode}

\paragraph{Parameters:}
\begin{itemize}[noitemsep]
\item \code{spacename}\\
The name of the space as a string or symbol.

\item \code{key}\\
The key for the operation where \code{key} is a Javascript value.

\item \code{mapattributes}\\
The set of map attributes to modify and their respective key/values.
\code{mapattrs} points to an array of length \code{mapattrs\_sz}.  Each entry
specify an attribute that is a map and a key within that map.

\end{itemize}

\paragraph{Returns:}
True if the operation succeeded.  False if any provided predicates failed.
Raises an exception on error.


\pagebreak
\subsubsection{\code{async\_map\_atomic\_mod}}
\label{api:ruby:async_map_atomic_mod}
\index{async\_map\_atomic\_mod!Ruby API}
Store the value of the key-value pair modulo the specified number for each map.

%%% Generated below here
\paragraph{Behavior:}
\begin{itemize}[noitemsep]
\item This operation manipulates an existing object.  If no object exists, the
    operation will fail with \code{NOTFOUND}.

\item This operation mutates the value of a key-value pair in a map.  This call
    is similar to the equivalent call without the \code{map\_} prefix, but
    operates on the value of a pair in a map, instead of on an attribute's
    value.  If there is no pair with the specified map key, a new pair will be
    created and initialized to its default value.  If this is undesirable, it
    may be avoided by using a conditional operation that requires that the map
    contain the key in question.

\end{itemize}


\paragraph{Definition:}
\begin{rubycode}
async_map_atomic_mod(spacename, key, mapattributes)
\end{rubycode}

\paragraph{Parameters:}
\begin{itemize}[noitemsep]
\item \code{spacename}\\
The name of the space as a string or symbol.

\item \code{key}\\
The key for the operation where \code{key} is a Javascript value.

\item \code{mapattributes}\\
The set of map attributes to modify and their respective key/values.
\code{mapattrs} points to an array of length \code{mapattrs\_sz}.  Each entry
specify an attribute that is a map and a key within that map.

\end{itemize}

\paragraph{Returns:}
A Deferred object with a \code{wait} method that returns True if the operation
succeeded or False if any provided predicates failed.  Raises an exception on
error.


\paragraph{See also:}  This is the asynchronous form of \code{map\_atomic\_mod}.

%%%%%%%%%%%%%%%%%%%% cond_map_atomic_mod %%%%%%%%%%%%%%%%%%%%
\pagebreak
\subsubsection{\code{cond\_map\_atomic\_mod}}
\label{api:ruby:cond_map_atomic_mod}
\index{cond\_map\_atomic\_mod!Ruby API}
Conditionally store the value of the key-value pair modulo the specified number
for each map.

%%% Generated below here
\paragraph{Behavior:}
\begin{itemize}[noitemsep]
\item This operation manipulates an existing object.  If no object exists, the
    operation will fail with \code{NOTFOUND}.

\item This operation will succeed if and only if the predicates specified by
    \code{checks} hold on the pre-existing object.  If any of the predicates are
    not true for the existing object, then the operation will have no effect and
    fail with \code{CMPFAIL}.

    All checks are atomic with the write.  HyperDex guarantees that no other
    operation will come between validating the checks, and writing the new
    version of the object..

\item This operation mutates the value of a key-value pair in a map.  This call
    is similar to the equivalent call without the \code{map\_} prefix, but
    operates on the value of a pair in a map, instead of on an attribute's
    value.  If there is no pair with the specified map key, a new pair will be
    created and initialized to its default value.  If this is undesirable, it
    may be avoided by using a conditional operation that requires that the map
    contain the key in question.

\end{itemize}


\paragraph{Definition:}
\begin{rubycode}
cond_map_atomic_mod(spacename, key, predicates, mapattributes)
\end{rubycode}

\paragraph{Parameters:}
\begin{itemize}[noitemsep]
\item \code{spacename}\\
The name of the space as a string or symbol.

\item \code{key}\\
The key for the operation where \code{key} is a Javascript value.

\item \code{predicates}\\
A set of predicates to check against.  \code{checks} points to an array of
length \code{checks\_sz}.

\item \code{mapattributes}\\
The set of map attributes to modify and their respective key/values.
\code{mapattrs} points to an array of length \code{mapattrs\_sz}.  Each entry
specify an attribute that is a map and a key within that map.

\end{itemize}

\paragraph{Returns:}
True if the operation succeeded.  False if any provided predicates failed.
Raises an exception on error.


\pagebreak
\subsubsection{\code{async\_cond\_map\_atomic\_mod}}
\label{api:ruby:async_cond_map_atomic_mod}
\index{async\_cond\_map\_atomic\_mod!Ruby API}
Conditionally store the value of the key-value pair modulo the specified number
for each map.

%%% Generated below here
\paragraph{Behavior:}
\begin{itemize}[noitemsep]
\item This operation manipulates an existing object.  If no object exists, the
    operation will fail with \code{NOTFOUND}.

\item This operation will succeed if and only if the predicates specified by
    \code{checks} hold on the pre-existing object.  If any of the predicates are
    not true for the existing object, then the operation will have no effect and
    fail with \code{CMPFAIL}.

    All checks are atomic with the write.  HyperDex guarantees that no other
    operation will come between validating the checks, and writing the new
    version of the object..

\item This operation mutates the value of a key-value pair in a map.  This call
    is similar to the equivalent call without the \code{map\_} prefix, but
    operates on the value of a pair in a map, instead of on an attribute's
    value.  If there is no pair with the specified map key, a new pair will be
    created and initialized to its default value.  If this is undesirable, it
    may be avoided by using a conditional operation that requires that the map
    contain the key in question.

\end{itemize}


\paragraph{Definition:}
\begin{rubycode}
async_cond_map_atomic_mod(spacename, key, predicates, mapattributes)
\end{rubycode}

\paragraph{Parameters:}
\begin{itemize}[noitemsep]
\item \code{spacename}\\
The name of the space as a string or symbol.

\item \code{key}\\
The key for the operation where \code{key} is a Javascript value.

\item \code{predicates}\\
A set of predicates to check against.  \code{checks} points to an array of
length \code{checks\_sz}.

\item \code{mapattributes}\\
The set of map attributes to modify and their respective key/values.
\code{mapattrs} points to an array of length \code{mapattrs\_sz}.  Each entry
specify an attribute that is a map and a key within that map.

\end{itemize}

\paragraph{Returns:}
A Deferred object with a \code{wait} method that returns True if the operation
succeeded or False if any provided predicates failed.  Raises an exception on
error.


\paragraph{See also:}  This is the asynchronous form of \code{cond\_map\_atomic\_mod}.

%%%%%%%%%%%%%%%%%%%% map_atomic_and %%%%%%%%%%%%%%%%%%%%
\pagebreak
\subsubsection{\code{map\_atomic\_and}}
\label{api:ruby:map_atomic_and}
\index{map\_atomic\_and!Ruby API}
Store the bitwise AND of the value of the key-value pair and the specified
number for each map.

%%% Generated below here
\paragraph{Behavior:}
\begin{itemize}[noitemsep]
\item This operation manipulates an existing object.  If no object exists, the
    operation will fail with \code{NOTFOUND}.

\item This operation mutates the value of a key-value pair in a map.  This call
    is similar to the equivalent call without the \code{map\_} prefix, but
    operates on the value of a pair in a map, instead of on an attribute's
    value.  If there is no pair with the specified map key, a new pair will be
    created and initialized to its default value.  If this is undesirable, it
    may be avoided by using a conditional operation that requires that the map
    contain the key in question.

\end{itemize}


\paragraph{Definition:}
\begin{rubycode}
map_atomic_and(spacename, key, mapattributes)
\end{rubycode}

\paragraph{Parameters:}
\begin{itemize}[noitemsep]
\item \code{spacename}\\
The name of the space as a string or symbol.

\item \code{key}\\
The key for the operation where \code{key} is a Javascript value.

\item \code{mapattributes}\\
The set of map attributes to modify and their respective key/values.
\code{mapattrs} points to an array of length \code{mapattrs\_sz}.  Each entry
specify an attribute that is a map and a key within that map.

\end{itemize}

\paragraph{Returns:}
True if the operation succeeded.  False if any provided predicates failed.
Raises an exception on error.


\pagebreak
\subsubsection{\code{async\_map\_atomic\_and}}
\label{api:ruby:async_map_atomic_and}
\index{async\_map\_atomic\_and!Ruby API}
Store the bitwise AND of the value of the key-value pair and the specified
number for each map.

%%% Generated below here
\paragraph{Behavior:}
\begin{itemize}[noitemsep]
\item This operation manipulates an existing object.  If no object exists, the
    operation will fail with \code{NOTFOUND}.

\item This operation mutates the value of a key-value pair in a map.  This call
    is similar to the equivalent call without the \code{map\_} prefix, but
    operates on the value of a pair in a map, instead of on an attribute's
    value.  If there is no pair with the specified map key, a new pair will be
    created and initialized to its default value.  If this is undesirable, it
    may be avoided by using a conditional operation that requires that the map
    contain the key in question.

\end{itemize}


\paragraph{Definition:}
\begin{rubycode}
async_map_atomic_and(spacename, key, mapattributes)
\end{rubycode}

\paragraph{Parameters:}
\begin{itemize}[noitemsep]
\item \code{spacename}\\
The name of the space as a string or symbol.

\item \code{key}\\
The key for the operation where \code{key} is a Javascript value.

\item \code{mapattributes}\\
The set of map attributes to modify and their respective key/values.
\code{mapattrs} points to an array of length \code{mapattrs\_sz}.  Each entry
specify an attribute that is a map and a key within that map.

\end{itemize}

\paragraph{Returns:}
A Deferred object with a \code{wait} method that returns True if the operation
succeeded or False if any provided predicates failed.  Raises an exception on
error.


\paragraph{See also:}  This is the asynchronous form of \code{map\_atomic\_and}.

%%%%%%%%%%%%%%%%%%%% cond_map_atomic_and %%%%%%%%%%%%%%%%%%%%
\pagebreak
\subsubsection{\code{cond\_map\_atomic\_and}}
\label{api:ruby:cond_map_atomic_and}
\index{cond\_map\_atomic\_and!Ruby API}
Conditionally store the bitwise AND of the value of the key-value pair and the
specified number for each map.

%%% Generated below here
\paragraph{Behavior:}
\begin{itemize}[noitemsep]
\item This operation manipulates an existing object.  If no object exists, the
    operation will fail with \code{NOTFOUND}.

\item This operation will succeed if and only if the predicates specified by
    \code{checks} hold on the pre-existing object.  If any of the predicates are
    not true for the existing object, then the operation will have no effect and
    fail with \code{CMPFAIL}.

    All checks are atomic with the write.  HyperDex guarantees that no other
    operation will come between validating the checks, and writing the new
    version of the object..

\item This operation mutates the value of a key-value pair in a map.  This call
    is similar to the equivalent call without the \code{map\_} prefix, but
    operates on the value of a pair in a map, instead of on an attribute's
    value.  If there is no pair with the specified map key, a new pair will be
    created and initialized to its default value.  If this is undesirable, it
    may be avoided by using a conditional operation that requires that the map
    contain the key in question.

\end{itemize}


\paragraph{Definition:}
\begin{rubycode}
cond_map_atomic_and(spacename, key, predicates, mapattributes)
\end{rubycode}

\paragraph{Parameters:}
\begin{itemize}[noitemsep]
\item \code{spacename}\\
The name of the space as a string or symbol.

\item \code{key}\\
The key for the operation where \code{key} is a Javascript value.

\item \code{predicates}\\
A set of predicates to check against.  \code{checks} points to an array of
length \code{checks\_sz}.

\item \code{mapattributes}\\
The set of map attributes to modify and their respective key/values.
\code{mapattrs} points to an array of length \code{mapattrs\_sz}.  Each entry
specify an attribute that is a map and a key within that map.

\end{itemize}

\paragraph{Returns:}
True if the operation succeeded.  False if any provided predicates failed.
Raises an exception on error.


\pagebreak
\subsubsection{\code{async\_cond\_map\_atomic\_and}}
\label{api:ruby:async_cond_map_atomic_and}
\index{async\_cond\_map\_atomic\_and!Ruby API}
Conditionally store the bitwise AND of the value of the key-value pair and the
specified number for each map.

%%% Generated below here
\paragraph{Behavior:}
\begin{itemize}[noitemsep]
\item This operation manipulates an existing object.  If no object exists, the
    operation will fail with \code{NOTFOUND}.

\item This operation will succeed if and only if the predicates specified by
    \code{checks} hold on the pre-existing object.  If any of the predicates are
    not true for the existing object, then the operation will have no effect and
    fail with \code{CMPFAIL}.

    All checks are atomic with the write.  HyperDex guarantees that no other
    operation will come between validating the checks, and writing the new
    version of the object..

\item This operation mutates the value of a key-value pair in a map.  This call
    is similar to the equivalent call without the \code{map\_} prefix, but
    operates on the value of a pair in a map, instead of on an attribute's
    value.  If there is no pair with the specified map key, a new pair will be
    created and initialized to its default value.  If this is undesirable, it
    may be avoided by using a conditional operation that requires that the map
    contain the key in question.

\end{itemize}


\paragraph{Definition:}
\begin{rubycode}
async_cond_map_atomic_and(spacename, key, predicates, mapattributes)
\end{rubycode}

\paragraph{Parameters:}
\begin{itemize}[noitemsep]
\item \code{spacename}\\
The name of the space as a string or symbol.

\item \code{key}\\
The key for the operation where \code{key} is a Javascript value.

\item \code{predicates}\\
A set of predicates to check against.  \code{checks} points to an array of
length \code{checks\_sz}.

\item \code{mapattributes}\\
The set of map attributes to modify and their respective key/values.
\code{mapattrs} points to an array of length \code{mapattrs\_sz}.  Each entry
specify an attribute that is a map and a key within that map.

\end{itemize}

\paragraph{Returns:}
A Deferred object with a \code{wait} method that returns True if the operation
succeeded or False if any provided predicates failed.  Raises an exception on
error.


\paragraph{See also:}  This is the asynchronous form of \code{cond\_map\_atomic\_and}.

%%%%%%%%%%%%%%%%%%%% map_atomic_or %%%%%%%%%%%%%%%%%%%%
\pagebreak
\subsubsection{\code{map\_atomic\_or}}
\label{api:ruby:map_atomic_or}
\index{map\_atomic\_or!Ruby API}
Store the bitwise OR of the value of the key-value pair and the specified number
for each map.

%%% Generated below here
\paragraph{Behavior:}
\begin{itemize}[noitemsep]
\item This operation manipulates an existing object.  If no object exists, the
    operation will fail with \code{NOTFOUND}.

\item This operation mutates the value of a key-value pair in a map.  This call
    is similar to the equivalent call without the \code{map\_} prefix, but
    operates on the value of a pair in a map, instead of on an attribute's
    value.  If there is no pair with the specified map key, a new pair will be
    created and initialized to its default value.  If this is undesirable, it
    may be avoided by using a conditional operation that requires that the map
    contain the key in question.

\end{itemize}


\paragraph{Definition:}
\begin{rubycode}
map_atomic_or(spacename, key, mapattributes)
\end{rubycode}

\paragraph{Parameters:}
\begin{itemize}[noitemsep]
\item \code{spacename}\\
The name of the space as a string or symbol.

\item \code{key}\\
The key for the operation where \code{key} is a Javascript value.

\item \code{mapattributes}\\
The set of map attributes to modify and their respective key/values.
\code{mapattrs} points to an array of length \code{mapattrs\_sz}.  Each entry
specify an attribute that is a map and a key within that map.

\end{itemize}

\paragraph{Returns:}
True if the operation succeeded.  False if any provided predicates failed.
Raises an exception on error.


\pagebreak
\subsubsection{\code{async\_map\_atomic\_or}}
\label{api:ruby:async_map_atomic_or}
\index{async\_map\_atomic\_or!Ruby API}
Store the bitwise OR of the value of the key-value pair and the specified number
for each map.

%%% Generated below here
\paragraph{Behavior:}
\begin{itemize}[noitemsep]
\item This operation manipulates an existing object.  If no object exists, the
    operation will fail with \code{NOTFOUND}.

\item This operation mutates the value of a key-value pair in a map.  This call
    is similar to the equivalent call without the \code{map\_} prefix, but
    operates on the value of a pair in a map, instead of on an attribute's
    value.  If there is no pair with the specified map key, a new pair will be
    created and initialized to its default value.  If this is undesirable, it
    may be avoided by using a conditional operation that requires that the map
    contain the key in question.

\end{itemize}


\paragraph{Definition:}
\begin{rubycode}
async_map_atomic_or(spacename, key, mapattributes)
\end{rubycode}

\paragraph{Parameters:}
\begin{itemize}[noitemsep]
\item \code{spacename}\\
The name of the space as a string or symbol.

\item \code{key}\\
The key for the operation where \code{key} is a Javascript value.

\item \code{mapattributes}\\
The set of map attributes to modify and their respective key/values.
\code{mapattrs} points to an array of length \code{mapattrs\_sz}.  Each entry
specify an attribute that is a map and a key within that map.

\end{itemize}

\paragraph{Returns:}
A Deferred object with a \code{wait} method that returns True if the operation
succeeded or False if any provided predicates failed.  Raises an exception on
error.


\paragraph{See also:}  This is the asynchronous form of \code{map\_atomic\_or}.

%%%%%%%%%%%%%%%%%%%% cond_map_atomic_or %%%%%%%%%%%%%%%%%%%%
\pagebreak
\subsubsection{\code{cond\_map\_atomic\_or}}
\label{api:ruby:cond_map_atomic_or}
\index{cond\_map\_atomic\_or!Ruby API}
Conditionally store the bitwise OR of the value of the key-value pair and the
specified number for each map.

%%% Generated below here
\paragraph{Behavior:}
\begin{itemize}[noitemsep]
\item This operation manipulates an existing object.  If no object exists, the
    operation will fail with \code{NOTFOUND}.

\item This operation will succeed if and only if the predicates specified by
    \code{checks} hold on the pre-existing object.  If any of the predicates are
    not true for the existing object, then the operation will have no effect and
    fail with \code{CMPFAIL}.

    All checks are atomic with the write.  HyperDex guarantees that no other
    operation will come between validating the checks, and writing the new
    version of the object..

\item This operation mutates the value of a key-value pair in a map.  This call
    is similar to the equivalent call without the \code{map\_} prefix, but
    operates on the value of a pair in a map, instead of on an attribute's
    value.  If there is no pair with the specified map key, a new pair will be
    created and initialized to its default value.  If this is undesirable, it
    may be avoided by using a conditional operation that requires that the map
    contain the key in question.

\end{itemize}


\paragraph{Definition:}
\begin{rubycode}
cond_map_atomic_or(spacename, key, predicates, mapattributes)
\end{rubycode}

\paragraph{Parameters:}
\begin{itemize}[noitemsep]
\item \code{spacename}\\
The name of the space as a string or symbol.

\item \code{key}\\
The key for the operation where \code{key} is a Javascript value.

\item \code{predicates}\\
A set of predicates to check against.  \code{checks} points to an array of
length \code{checks\_sz}.

\item \code{mapattributes}\\
The set of map attributes to modify and their respective key/values.
\code{mapattrs} points to an array of length \code{mapattrs\_sz}.  Each entry
specify an attribute that is a map and a key within that map.

\end{itemize}

\paragraph{Returns:}
True if the operation succeeded.  False if any provided predicates failed.
Raises an exception on error.


\pagebreak
\subsubsection{\code{async\_cond\_map\_atomic\_or}}
\label{api:ruby:async_cond_map_atomic_or}
\index{async\_cond\_map\_atomic\_or!Ruby API}
Conditionally store the bitwise OR of the value of the key-value pair and the
specified number for each map.

%%% Generated below here
\paragraph{Behavior:}
\begin{itemize}[noitemsep]
\item This operation manipulates an existing object.  If no object exists, the
    operation will fail with \code{NOTFOUND}.

\item This operation will succeed if and only if the predicates specified by
    \code{checks} hold on the pre-existing object.  If any of the predicates are
    not true for the existing object, then the operation will have no effect and
    fail with \code{CMPFAIL}.

    All checks are atomic with the write.  HyperDex guarantees that no other
    operation will come between validating the checks, and writing the new
    version of the object..

\item This operation mutates the value of a key-value pair in a map.  This call
    is similar to the equivalent call without the \code{map\_} prefix, but
    operates on the value of a pair in a map, instead of on an attribute's
    value.  If there is no pair with the specified map key, a new pair will be
    created and initialized to its default value.  If this is undesirable, it
    may be avoided by using a conditional operation that requires that the map
    contain the key in question.

\end{itemize}


\paragraph{Definition:}
\begin{rubycode}
async_cond_map_atomic_or(spacename, key, predicates, mapattributes)
\end{rubycode}

\paragraph{Parameters:}
\begin{itemize}[noitemsep]
\item \code{spacename}\\
The name of the space as a string or symbol.

\item \code{key}\\
The key for the operation where \code{key} is a Javascript value.

\item \code{predicates}\\
A set of predicates to check against.  \code{checks} points to an array of
length \code{checks\_sz}.

\item \code{mapattributes}\\
The set of map attributes to modify and their respective key/values.
\code{mapattrs} points to an array of length \code{mapattrs\_sz}.  Each entry
specify an attribute that is a map and a key within that map.

\end{itemize}

\paragraph{Returns:}
A Deferred object with a \code{wait} method that returns True if the operation
succeeded or False if any provided predicates failed.  Raises an exception on
error.


\paragraph{See also:}  This is the asynchronous form of \code{cond\_map\_atomic\_or}.

%%%%%%%%%%%%%%%%%%%% map_atomic_xor %%%%%%%%%%%%%%%%%%%%
\pagebreak
\subsubsection{\code{map\_atomic\_xor}}
\label{api:ruby:map_atomic_xor}
\index{map\_atomic\_xor!Ruby API}
Store the bitwise XOR of the value of the key-value pair and the specified
number for each map.

%%% Generated below here
\paragraph{Behavior:}
\begin{itemize}[noitemsep]
\item This operation manipulates an existing object.  If no object exists, the
    operation will fail with \code{NOTFOUND}.

\item This operation mutates the value of a key-value pair in a map.  This call
    is similar to the equivalent call without the \code{map\_} prefix, but
    operates on the value of a pair in a map, instead of on an attribute's
    value.  If there is no pair with the specified map key, a new pair will be
    created and initialized to its default value.  If this is undesirable, it
    may be avoided by using a conditional operation that requires that the map
    contain the key in question.

\end{itemize}


\paragraph{Definition:}
\begin{rubycode}
map_atomic_xor(spacename, key, mapattributes)
\end{rubycode}

\paragraph{Parameters:}
\begin{itemize}[noitemsep]
\item \code{spacename}\\
The name of the space as a string or symbol.

\item \code{key}\\
The key for the operation where \code{key} is a Javascript value.

\item \code{mapattributes}\\
The set of map attributes to modify and their respective key/values.
\code{mapattrs} points to an array of length \code{mapattrs\_sz}.  Each entry
specify an attribute that is a map and a key within that map.

\end{itemize}

\paragraph{Returns:}
True if the operation succeeded.  False if any provided predicates failed.
Raises an exception on error.


\pagebreak
\subsubsection{\code{async\_map\_atomic\_xor}}
\label{api:ruby:async_map_atomic_xor}
\index{async\_map\_atomic\_xor!Ruby API}
Store the bitwise XOR of the value of the key-value pair and the specified
number for each map.

%%% Generated below here
\paragraph{Behavior:}
\begin{itemize}[noitemsep]
\item This operation manipulates an existing object.  If no object exists, the
    operation will fail with \code{NOTFOUND}.

\item This operation mutates the value of a key-value pair in a map.  This call
    is similar to the equivalent call without the \code{map\_} prefix, but
    operates on the value of a pair in a map, instead of on an attribute's
    value.  If there is no pair with the specified map key, a new pair will be
    created and initialized to its default value.  If this is undesirable, it
    may be avoided by using a conditional operation that requires that the map
    contain the key in question.

\end{itemize}


\paragraph{Definition:}
\begin{rubycode}
async_map_atomic_xor(spacename, key, mapattributes)
\end{rubycode}

\paragraph{Parameters:}
\begin{itemize}[noitemsep]
\item \code{spacename}\\
The name of the space as a string or symbol.

\item \code{key}\\
The key for the operation where \code{key} is a Javascript value.

\item \code{mapattributes}\\
The set of map attributes to modify and their respective key/values.
\code{mapattrs} points to an array of length \code{mapattrs\_sz}.  Each entry
specify an attribute that is a map and a key within that map.

\end{itemize}

\paragraph{Returns:}
A Deferred object with a \code{wait} method that returns True if the operation
succeeded or False if any provided predicates failed.  Raises an exception on
error.


\paragraph{See also:}  This is the asynchronous form of \code{map\_atomic\_xor}.

%%%%%%%%%%%%%%%%%%%% cond_map_atomic_xor %%%%%%%%%%%%%%%%%%%%
\pagebreak
\subsubsection{\code{cond\_map\_atomic\_xor}}
\label{api:ruby:cond_map_atomic_xor}
\index{cond\_map\_atomic\_xor!Ruby API}
Conditionally store the bitwise XOR of the value of the key-value pair and the
specified number for each map.

%%% Generated below here
\paragraph{Behavior:}
\begin{itemize}[noitemsep]
\item This operation manipulates an existing object.  If no object exists, the
    operation will fail with \code{NOTFOUND}.

\item This operation will succeed if and only if the predicates specified by
    \code{checks} hold on the pre-existing object.  If any of the predicates are
    not true for the existing object, then the operation will have no effect and
    fail with \code{CMPFAIL}.

    All checks are atomic with the write.  HyperDex guarantees that no other
    operation will come between validating the checks, and writing the new
    version of the object..

\item This operation mutates the value of a key-value pair in a map.  This call
    is similar to the equivalent call without the \code{map\_} prefix, but
    operates on the value of a pair in a map, instead of on an attribute's
    value.  If there is no pair with the specified map key, a new pair will be
    created and initialized to its default value.  If this is undesirable, it
    may be avoided by using a conditional operation that requires that the map
    contain the key in question.

\end{itemize}


\paragraph{Definition:}
\begin{rubycode}
cond_map_atomic_xor(spacename, key, predicates, mapattributes)
\end{rubycode}

\paragraph{Parameters:}
\begin{itemize}[noitemsep]
\item \code{spacename}\\
The name of the space as a string or symbol.

\item \code{key}\\
The key for the operation where \code{key} is a Javascript value.

\item \code{predicates}\\
A set of predicates to check against.  \code{checks} points to an array of
length \code{checks\_sz}.

\item \code{mapattributes}\\
The set of map attributes to modify and their respective key/values.
\code{mapattrs} points to an array of length \code{mapattrs\_sz}.  Each entry
specify an attribute that is a map and a key within that map.

\end{itemize}

\paragraph{Returns:}
True if the operation succeeded.  False if any provided predicates failed.
Raises an exception on error.


\pagebreak
\subsubsection{\code{async\_cond\_map\_atomic\_xor}}
\label{api:ruby:async_cond_map_atomic_xor}
\index{async\_cond\_map\_atomic\_xor!Ruby API}
Conditionally store the bitwise XOR of the value of the key-value pair and the
specified number for each map.

%%% Generated below here
\paragraph{Behavior:}
\begin{itemize}[noitemsep]
\item This operation manipulates an existing object.  If no object exists, the
    operation will fail with \code{NOTFOUND}.

\item This operation will succeed if and only if the predicates specified by
    \code{checks} hold on the pre-existing object.  If any of the predicates are
    not true for the existing object, then the operation will have no effect and
    fail with \code{CMPFAIL}.

    All checks are atomic with the write.  HyperDex guarantees that no other
    operation will come between validating the checks, and writing the new
    version of the object..

\item This operation mutates the value of a key-value pair in a map.  This call
    is similar to the equivalent call without the \code{map\_} prefix, but
    operates on the value of a pair in a map, instead of on an attribute's
    value.  If there is no pair with the specified map key, a new pair will be
    created and initialized to its default value.  If this is undesirable, it
    may be avoided by using a conditional operation that requires that the map
    contain the key in question.

\end{itemize}


\paragraph{Definition:}
\begin{rubycode}
async_cond_map_atomic_xor(spacename, key, predicates, mapattributes)
\end{rubycode}

\paragraph{Parameters:}
\begin{itemize}[noitemsep]
\item \code{spacename}\\
The name of the space as a string or symbol.

\item \code{key}\\
The key for the operation where \code{key} is a Javascript value.

\item \code{predicates}\\
A set of predicates to check against.  \code{checks} points to an array of
length \code{checks\_sz}.

\item \code{mapattributes}\\
The set of map attributes to modify and their respective key/values.
\code{mapattrs} points to an array of length \code{mapattrs\_sz}.  Each entry
specify an attribute that is a map and a key within that map.

\end{itemize}

\paragraph{Returns:}
A Deferred object with a \code{wait} method that returns True if the operation
succeeded or False if any provided predicates failed.  Raises an exception on
error.


\paragraph{See also:}  This is the asynchronous form of \code{cond\_map\_atomic\_xor}.

%%%%%%%%%%%%%%%%%%%% map_string_prepend %%%%%%%%%%%%%%%%%%%%
\pagebreak
\subsubsection{\code{map\_string\_prepend}}
\label{api:ruby:map_string_prepend}
\index{map\_string\_prepend!Ruby API}
Prepend the specified string to the value of the key-value pair for each map.

%%% Generated below here
\paragraph{Behavior:}
\begin{itemize}[noitemsep]
\item This operation manipulates an existing object.  If no object exists, the
    operation will fail with \code{NOTFOUND}.

\item This operation mutates the value of a key-value pair in a map.  This call
    is similar to the equivalent call without the \code{map\_} prefix, but
    operates on the value of a pair in a map, instead of on an attribute's
    value.  If there is no pair with the specified map key, a new pair will be
    created and initialized to its default value.  If this is undesirable, it
    may be avoided by using a conditional operation that requires that the map
    contain the key in question.

\end{itemize}


\paragraph{Definition:}
\begin{rubycode}
map_string_prepend(spacename, key, mapattributes)
\end{rubycode}

\paragraph{Parameters:}
\begin{itemize}[noitemsep]
\item \code{spacename}\\
The name of the space as a string or symbol.

\item \code{key}\\
The key for the operation where \code{key} is a Javascript value.

\item \code{mapattributes}\\
The set of map attributes to modify and their respective key/values.
\code{mapattrs} points to an array of length \code{mapattrs\_sz}.  Each entry
specify an attribute that is a map and a key within that map.

\end{itemize}

\paragraph{Returns:}
True if the operation succeeded.  False if any provided predicates failed.
Raises an exception on error.


\pagebreak
\subsubsection{\code{async\_map\_string\_prepend}}
\label{api:ruby:async_map_string_prepend}
\index{async\_map\_string\_prepend!Ruby API}
Prepend the specified string to the value of the key-value pair for each map.

%%% Generated below here
\paragraph{Behavior:}
\begin{itemize}[noitemsep]
\item This operation manipulates an existing object.  If no object exists, the
    operation will fail with \code{NOTFOUND}.

\item This operation mutates the value of a key-value pair in a map.  This call
    is similar to the equivalent call without the \code{map\_} prefix, but
    operates on the value of a pair in a map, instead of on an attribute's
    value.  If there is no pair with the specified map key, a new pair will be
    created and initialized to its default value.  If this is undesirable, it
    may be avoided by using a conditional operation that requires that the map
    contain the key in question.

\end{itemize}


\paragraph{Definition:}
\begin{rubycode}
async_map_string_prepend(spacename, key, mapattributes)
\end{rubycode}

\paragraph{Parameters:}
\begin{itemize}[noitemsep]
\item \code{spacename}\\
The name of the space as a string or symbol.

\item \code{key}\\
The key for the operation where \code{key} is a Javascript value.

\item \code{mapattributes}\\
The set of map attributes to modify and their respective key/values.
\code{mapattrs} points to an array of length \code{mapattrs\_sz}.  Each entry
specify an attribute that is a map and a key within that map.

\end{itemize}

\paragraph{Returns:}
A Deferred object with a \code{wait} method that returns True if the operation
succeeded or False if any provided predicates failed.  Raises an exception on
error.


\paragraph{See also:}  This is the asynchronous form of \code{map\_string\_prepend}.

%%%%%%%%%%%%%%%%%%%% cond_map_string_prepend %%%%%%%%%%%%%%%%%%%%
\pagebreak
\subsubsection{\code{cond\_map\_string\_prepend}}
\label{api:ruby:cond_map_string_prepend}
\index{cond\_map\_string\_prepend!Ruby API}
Conditionally prepend the specified string to the value of the key-value pair
for each map.

%%% Generated below here
\paragraph{Behavior:}
\begin{itemize}[noitemsep]
\item This operation manipulates an existing object.  If no object exists, the
    operation will fail with \code{NOTFOUND}.

\item This operation will succeed if and only if the predicates specified by
    \code{checks} hold on the pre-existing object.  If any of the predicates are
    not true for the existing object, then the operation will have no effect and
    fail with \code{CMPFAIL}.

    All checks are atomic with the write.  HyperDex guarantees that no other
    operation will come between validating the checks, and writing the new
    version of the object..

\item This operation mutates the value of a key-value pair in a map.  This call
    is similar to the equivalent call without the \code{map\_} prefix, but
    operates on the value of a pair in a map, instead of on an attribute's
    value.  If there is no pair with the specified map key, a new pair will be
    created and initialized to its default value.  If this is undesirable, it
    may be avoided by using a conditional operation that requires that the map
    contain the key in question.

\end{itemize}


\paragraph{Definition:}
\begin{rubycode}
cond_map_string_prepend(spacename, key, predicates, mapattributes)
\end{rubycode}

\paragraph{Parameters:}
\begin{itemize}[noitemsep]
\item \code{spacename}\\
The name of the space as a string or symbol.

\item \code{key}\\
The key for the operation where \code{key} is a Javascript value.

\item \code{predicates}\\
A set of predicates to check against.  \code{checks} points to an array of
length \code{checks\_sz}.

\item \code{mapattributes}\\
The set of map attributes to modify and their respective key/values.
\code{mapattrs} points to an array of length \code{mapattrs\_sz}.  Each entry
specify an attribute that is a map and a key within that map.

\end{itemize}

\paragraph{Returns:}
True if the operation succeeded.  False if any provided predicates failed.
Raises an exception on error.


\pagebreak
\subsubsection{\code{async\_cond\_map\_string\_prepend}}
\label{api:ruby:async_cond_map_string_prepend}
\index{async\_cond\_map\_string\_prepend!Ruby API}
Conditionally prepend the specified string to the value of the key-value pair
for each map.

%%% Generated below here
\paragraph{Behavior:}
\begin{itemize}[noitemsep]
\item This operation manipulates an existing object.  If no object exists, the
    operation will fail with \code{NOTFOUND}.

\item This operation will succeed if and only if the predicates specified by
    \code{checks} hold on the pre-existing object.  If any of the predicates are
    not true for the existing object, then the operation will have no effect and
    fail with \code{CMPFAIL}.

    All checks are atomic with the write.  HyperDex guarantees that no other
    operation will come between validating the checks, and writing the new
    version of the object..

\item This operation mutates the value of a key-value pair in a map.  This call
    is similar to the equivalent call without the \code{map\_} prefix, but
    operates on the value of a pair in a map, instead of on an attribute's
    value.  If there is no pair with the specified map key, a new pair will be
    created and initialized to its default value.  If this is undesirable, it
    may be avoided by using a conditional operation that requires that the map
    contain the key in question.

\end{itemize}


\paragraph{Definition:}
\begin{rubycode}
async_cond_map_string_prepend(spacename, key, predicates, mapattributes)
\end{rubycode}

\paragraph{Parameters:}
\begin{itemize}[noitemsep]
\item \code{spacename}\\
The name of the space as a string or symbol.

\item \code{key}\\
The key for the operation where \code{key} is a Javascript value.

\item \code{predicates}\\
A set of predicates to check against.  \code{checks} points to an array of
length \code{checks\_sz}.

\item \code{mapattributes}\\
The set of map attributes to modify and their respective key/values.
\code{mapattrs} points to an array of length \code{mapattrs\_sz}.  Each entry
specify an attribute that is a map and a key within that map.

\end{itemize}

\paragraph{Returns:}
A Deferred object with a \code{wait} method that returns True if the operation
succeeded or False if any provided predicates failed.  Raises an exception on
error.


\paragraph{See also:}  This is the asynchronous form of \code{cond\_map\_string\_prepend}.

%%%%%%%%%%%%%%%%%%%% map_string_append %%%%%%%%%%%%%%%%%%%%
\pagebreak
\subsubsection{\code{map\_string\_append}}
\label{api:ruby:map_string_append}
\index{map\_string\_append!Ruby API}
Append the specified string to the value of the key-value pair for each map.

%%% Generated below here
\paragraph{Behavior:}
\begin{itemize}[noitemsep]
\item This operation manipulates an existing object.  If no object exists, the
    operation will fail with \code{NOTFOUND}.

\item This operation mutates the value of a key-value pair in a map.  This call
    is similar to the equivalent call without the \code{map\_} prefix, but
    operates on the value of a pair in a map, instead of on an attribute's
    value.  If there is no pair with the specified map key, a new pair will be
    created and initialized to its default value.  If this is undesirable, it
    may be avoided by using a conditional operation that requires that the map
    contain the key in question.

\end{itemize}


\paragraph{Definition:}
\begin{rubycode}
map_string_append(spacename, key, mapattributes)
\end{rubycode}

\paragraph{Parameters:}
\begin{itemize}[noitemsep]
\item \code{spacename}\\
The name of the space as a string or symbol.

\item \code{key}\\
The key for the operation where \code{key} is a Javascript value.

\item \code{mapattributes}\\
The set of map attributes to modify and their respective key/values.
\code{mapattrs} points to an array of length \code{mapattrs\_sz}.  Each entry
specify an attribute that is a map and a key within that map.

\end{itemize}

\paragraph{Returns:}
True if the operation succeeded.  False if any provided predicates failed.
Raises an exception on error.


\pagebreak
\subsubsection{\code{async\_map\_string\_append}}
\label{api:ruby:async_map_string_append}
\index{async\_map\_string\_append!Ruby API}
Append the specified string to the value of the key-value pair for each map.

%%% Generated below here
\paragraph{Behavior:}
\begin{itemize}[noitemsep]
\item This operation manipulates an existing object.  If no object exists, the
    operation will fail with \code{NOTFOUND}.

\item This operation mutates the value of a key-value pair in a map.  This call
    is similar to the equivalent call without the \code{map\_} prefix, but
    operates on the value of a pair in a map, instead of on an attribute's
    value.  If there is no pair with the specified map key, a new pair will be
    created and initialized to its default value.  If this is undesirable, it
    may be avoided by using a conditional operation that requires that the map
    contain the key in question.

\end{itemize}


\paragraph{Definition:}
\begin{rubycode}
async_map_string_append(spacename, key, mapattributes)
\end{rubycode}

\paragraph{Parameters:}
\begin{itemize}[noitemsep]
\item \code{spacename}\\
The name of the space as a string or symbol.

\item \code{key}\\
The key for the operation where \code{key} is a Javascript value.

\item \code{mapattributes}\\
The set of map attributes to modify and their respective key/values.
\code{mapattrs} points to an array of length \code{mapattrs\_sz}.  Each entry
specify an attribute that is a map and a key within that map.

\end{itemize}

\paragraph{Returns:}
A Deferred object with a \code{wait} method that returns True if the operation
succeeded or False if any provided predicates failed.  Raises an exception on
error.


\paragraph{See also:}  This is the asynchronous form of \code{map\_string\_append}.

%%%%%%%%%%%%%%%%%%%% cond_map_string_append %%%%%%%%%%%%%%%%%%%%
\pagebreak
\subsubsection{\code{cond\_map\_string\_append}}
\label{api:ruby:cond_map_string_append}
\index{cond\_map\_string\_append!Ruby API}
Conditionally append the specified string to the value of the key-value pair for
each map.

%%% Generated below here
\paragraph{Behavior:}
\begin{itemize}[noitemsep]
\item This operation manipulates an existing object.  If no object exists, the
    operation will fail with \code{NOTFOUND}.

\item This operation will succeed if and only if the predicates specified by
    \code{checks} hold on the pre-existing object.  If any of the predicates are
    not true for the existing object, then the operation will have no effect and
    fail with \code{CMPFAIL}.

    All checks are atomic with the write.  HyperDex guarantees that no other
    operation will come between validating the checks, and writing the new
    version of the object..

\item This operation mutates the value of a key-value pair in a map.  This call
    is similar to the equivalent call without the \code{map\_} prefix, but
    operates on the value of a pair in a map, instead of on an attribute's
    value.  If there is no pair with the specified map key, a new pair will be
    created and initialized to its default value.  If this is undesirable, it
    may be avoided by using a conditional operation that requires that the map
    contain the key in question.

\end{itemize}


\paragraph{Definition:}
\begin{rubycode}
cond_map_string_append(spacename, key, predicates, mapattributes)
\end{rubycode}

\paragraph{Parameters:}
\begin{itemize}[noitemsep]
\item \code{spacename}\\
The name of the space as a string or symbol.

\item \code{key}\\
The key for the operation where \code{key} is a Javascript value.

\item \code{predicates}\\
A set of predicates to check against.  \code{checks} points to an array of
length \code{checks\_sz}.

\item \code{mapattributes}\\
The set of map attributes to modify and their respective key/values.
\code{mapattrs} points to an array of length \code{mapattrs\_sz}.  Each entry
specify an attribute that is a map and a key within that map.

\end{itemize}

\paragraph{Returns:}
True if the operation succeeded.  False if any provided predicates failed.
Raises an exception on error.


\pagebreak
\subsubsection{\code{async\_cond\_map\_string\_append}}
\label{api:ruby:async_cond_map_string_append}
\index{async\_cond\_map\_string\_append!Ruby API}
Conditionally append the specified string to the value of the key-value pair for
each map.

%%% Generated below here
\paragraph{Behavior:}
\begin{itemize}[noitemsep]
\item This operation manipulates an existing object.  If no object exists, the
    operation will fail with \code{NOTFOUND}.

\item This operation will succeed if and only if the predicates specified by
    \code{checks} hold on the pre-existing object.  If any of the predicates are
    not true for the existing object, then the operation will have no effect and
    fail with \code{CMPFAIL}.

    All checks are atomic with the write.  HyperDex guarantees that no other
    operation will come between validating the checks, and writing the new
    version of the object..

\item This operation mutates the value of a key-value pair in a map.  This call
    is similar to the equivalent call without the \code{map\_} prefix, but
    operates on the value of a pair in a map, instead of on an attribute's
    value.  If there is no pair with the specified map key, a new pair will be
    created and initialized to its default value.  If this is undesirable, it
    may be avoided by using a conditional operation that requires that the map
    contain the key in question.

\end{itemize}


\paragraph{Definition:}
\begin{rubycode}
async_cond_map_string_append(spacename, key, predicates, mapattributes)
\end{rubycode}

\paragraph{Parameters:}
\begin{itemize}[noitemsep]
\item \code{spacename}\\
The name of the space as a string or symbol.

\item \code{key}\\
The key for the operation where \code{key} is a Javascript value.

\item \code{predicates}\\
A set of predicates to check against.  \code{checks} points to an array of
length \code{checks\_sz}.

\item \code{mapattributes}\\
The set of map attributes to modify and their respective key/values.
\code{mapattrs} points to an array of length \code{mapattrs\_sz}.  Each entry
specify an attribute that is a map and a key within that map.

\end{itemize}

\paragraph{Returns:}
A Deferred object with a \code{wait} method that returns True if the operation
succeeded or False if any provided predicates failed.  Raises an exception on
error.


\paragraph{See also:}  This is the asynchronous form of \code{cond\_map\_string\_append}.

%%%%%%%%%%%%%%%%%%%% search %%%%%%%%%%%%%%%%%%%%
\pagebreak
\subsubsection{\code{search}}
\label{api:ruby:search}
\index{search!Ruby API}
Return all objects that match the specified \code{checks}.

\paragraph{Behavior:}
\begin{itemize}[noitemsep]
\item This operation behaves as an iterator.  Multiple objects may be returned
    from the single call.

\item This operation return to the user the requested object(s).

\end{itemize}


\paragraph{Definition:}
\begin{rubycode}
search(spacename, predicates)
\end{rubycode}

\paragraph{Parameters:}
\begin{itemize}[noitemsep]
\item \code{spacename}\\
The name of the space as a string or symbol.

\item \code{predicates}\\
A set of predicates to check against.  \code{checks} points to an array of
length \code{checks\_sz}.

\end{itemize}

\paragraph{Returns:}
This asynchronous operation returns an \code{Iterator} object.  The
\code{Iterator} will return the resulting objects as they become available.

Errors will be returned from the Iterator, as it is possible to retreive partial
results in the face of an error.


%%%%%%%%%%%%%%%%%%%% search_describe %%%%%%%%%%%%%%%%%%%%
\pagebreak
\subsubsection{\code{search\_describe}}
\label{api:ruby:search_describe}
\index{search\_describe!Ruby API}
Return a human-readable string suitable for debugging search internals.  This
API is only really relevant for debugging the internals of \code{search}.


\paragraph{Definition:}
\begin{rubycode}
search_describe(spacename, predicates)
\end{rubycode}

\paragraph{Parameters:}
\begin{itemize}[noitemsep]
\item \code{spacename}\\
The name of the space as a string or symbol.

\item \code{predicates}\\
A set of predicates to check against.  \code{checks} points to an array of
length \code{checks\_sz}.

\end{itemize}

\paragraph{Returns:}
This function returns a string describing the internals of the search.

On error, this function will raise a \code{HyperDexClientException} describing
the error.


\pagebreak
\subsubsection{\code{async\_search\_describe}}
\label{api:ruby:async_search_describe}
\index{async\_search\_describe!Ruby API}
Return a human-readable string suitable for debugging search internals.  This
API is only really relevant for debugging the internals of \code{search}.


\paragraph{Definition:}
\begin{rubycode}
async_search_describe(spacename, predicates)
\end{rubycode}

\paragraph{Parameters:}
\begin{itemize}[noitemsep]
\item \code{spacename}\\
The name of the space as a string or symbol.

\item \code{predicates}\\
A set of predicates to check against.  \code{checks} points to an array of
length \code{checks\_sz}.

\end{itemize}

\paragraph{Returns:}
This asynchronous operation returns a \code{Deferred} object with a
\code{waitForIt} method which blocks and returns a string describing the
internals of the search.

On error, this function will raise a \code{HyperDexClientException} describing
the error.


\paragraph{See also:}  This is the asynchronous form of \code{search\_describe}.

%%%%%%%%%%%%%%%%%%%% sorted_search %%%%%%%%%%%%%%%%%%%%
\pagebreak
\subsubsection{\code{sorted\_search}}
\label{api:ruby:sorted_search}
\index{sorted\_search!Ruby API}
Return all objects that match the specified \code{checks}, sorted according to
\code{attr}.

\paragraph{Behavior:}
\begin{itemize}[noitemsep]
\item This operation behaves as an iterator.  Multiple objects may be returned
    from the single call.

\item This operation return to the user the requested object(s).

\end{itemize}


\paragraph{Definition:}
\begin{rubycode}
sorted_search(spacename, predicates, sortby, limit, maxmin)
\end{rubycode}

\paragraph{Parameters:}
\begin{itemize}[noitemsep]
\item \code{spacename}\\
The name of the space as a string or symbol.

\item \code{predicates}\\
A set of predicates to check against.  \code{checks} points to an array of
length \code{checks\_sz}.

\item \code{sortby}\\
The attribute to sort by.

\item \code{limit}\\
The number of results to return.

\item \code{maxmin}\\
Maximize (!= 0) or minimize (== 0).

\end{itemize}

\paragraph{Returns:}
This asynchronous operation returns an \code{Iterator} object.  The
\code{Iterator} will return the resulting objects as they become available.

Errors will be returned from the Iterator, as it is possible to retreive partial
results in the face of an error.


%%%%%%%%%%%%%%%%%%%% group_del %%%%%%%%%%%%%%%%%%%%
\pagebreak
\subsubsection{\code{group\_del}}
\label{api:ruby:group_del}
\index{group\_del!Ruby API}
Asynchronously delete all objects that match the specified \code{checks}.

\paragraph{Behavior:}
\begin{itemize}[noitemsep]
\item This operation is roughly equivalent to a client manually deleting every
    object returned from a search, but saves HyperDex from sending to the client
    objects that are soon to be deleted.
\end{itemize}


\paragraph{Definition:}
\begin{rubycode}
group_del(spacename, predicates)
\end{rubycode}

\paragraph{Parameters:}
\begin{itemize}[noitemsep]
\item \code{spacename}\\
The name of the space as a string or symbol.

\item \code{predicates}\\
A set of predicates to check against.  \code{checks} points to an array of
length \code{checks\_sz}.

\end{itemize}

\paragraph{Returns:}
True if the operation succeeded.  False if any provided predicates failed.
Raises an exception on error.


\pagebreak
\subsubsection{\code{async\_group\_del}}
\label{api:ruby:async_group_del}
\index{async\_group\_del!Ruby API}
Asynchronously delete all objects that match the specified \code{checks}.

\paragraph{Behavior:}
\begin{itemize}[noitemsep]
\item This operation is roughly equivalent to a client manually deleting every
    object returned from a search, but saves HyperDex from sending to the client
    objects that are soon to be deleted.
\end{itemize}


\paragraph{Definition:}
\begin{rubycode}
async_group_del(spacename, predicates)
\end{rubycode}

\paragraph{Parameters:}
\begin{itemize}[noitemsep]
\item \code{spacename}\\
The name of the space as a string or symbol.

\item \code{predicates}\\
A set of predicates to check against.  \code{checks} points to an array of
length \code{checks\_sz}.

\end{itemize}

\paragraph{Returns:}
A Deferred object with a \code{wait} method that returns True if the operation
succeeded or False if any provided predicates failed.  Raises an exception on
error.


\paragraph{See also:}  This is the asynchronous form of \code{group\_del}.

%%%%%%%%%%%%%%%%%%%% count %%%%%%%%%%%%%%%%%%%%
\pagebreak
\subsubsection{\code{count}}
\label{api:ruby:count}
\index{count!Ruby API}
Count the number of objects that match the specified \code{checks}.

\paragraph{Behavior:}
\begin{itemize}[noitemsep]
\item This will return the number of objects counted by the search.  If an error
    occurs during the count, the count will reflect a partial count.  The real
    count will be higher than the returned value.  Some languages will throw an
    exception rather than return the partial count.
\end{itemize}


\paragraph{Definition:}
\begin{rubycode}
count(spacename, predicates)
\end{rubycode}

\paragraph{Parameters:}
\begin{itemize}[noitemsep]
\item \code{spacename}\\
The name of the space as a string or symbol.

\item \code{predicates}\\
A set of predicates to check against.  \code{checks} points to an array of
length \code{checks\_sz}.

\end{itemize}

\paragraph{Returns:}
This function returns a number indicating the number of objects counted.

On error, this function will raise a \code{HyperDexClientException} describing
the error.


\pagebreak
\subsubsection{\code{async\_count}}
\label{api:ruby:async_count}
\index{async\_count!Ruby API}
Count the number of objects that match the specified \code{checks}.

\paragraph{Behavior:}
\begin{itemize}[noitemsep]
\item This will return the number of objects counted by the search.  If an error
    occurs during the count, the count will reflect a partial count.  The real
    count will be higher than the returned value.  Some languages will throw an
    exception rather than return the partial count.
\end{itemize}


\paragraph{Definition:}
\begin{rubycode}
async_count(spacename, predicates)
\end{rubycode}

\paragraph{Parameters:}
\begin{itemize}[noitemsep]
\item \code{spacename}\\
The name of the space as a string or symbol.

\item \code{predicates}\\
A set of predicates to check against.  \code{checks} points to an array of
length \code{checks\_sz}.

\end{itemize}

\paragraph{Returns:}
A Deferred object with a \code{wait} method that returns the number of objects
found.  Raises exception on error.


\paragraph{See also:}  This is the asynchronous form of \code{count}.

\pagebreak

\subsection{Working with Signals}
\label{sec:api:ruby:signals}

The HyperDex client library is signal-safe.  Should a signal interrupt the
client during a blocking operation, it will raise a
\code{HyperDexClientException} with status \code{HYPERDEX\_CLIENT\_INTERRUPTED}.

\subsection{Working with Threads}
\label{sec:api:ruby:threads}

The Ruby module is fully reentrant.  Instances of
\code{HyperDex::Client::Client} and their associated state may be accessed from
multiple threads, provided that the application employs its own synchronization
that provides mutual exclusion.

Put simply, a multi-threaded application should protect each \code{Client}
instance with a mutex or lock to ensure correct operation.
