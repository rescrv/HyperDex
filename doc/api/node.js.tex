\chapter{Node API}
\label{chap:api:node}

\section{Client Library}
\label{sec:api:node:client}

HyperDex provides Node bindings under the module \code{hyperdex-client}.  This
library wraps the HyperDex C Client library and enables use of native Javascript
data types.

This module was re-introduced in HyperDex 1.2.0.

\subsection{Building the HyperDex Node.js Binding}
\label{sec:api:node:building}

The HyperDex Node.js Binding must be built and installed after HyperDex is built
and installed.  After installing HyperDex, you can build the Node.js bindings
from either source or git checkout with:

\begin{consolecode}
% cd bindings/node.js
% node rebuild
\end{consolecode}

\subsection{Using Node.js Within Your Application}
\label{sec:api:node:using}

All client operation are defined in the \code{hyperdex\_client} module.  You can
access this in your program with:

\begin{javascriptcode}
var hyperdex_client = require('hyperdex-client');
\end{javascriptcode}

\subsection{Hello World}
\label{sec:api:node:hello-world}

The following is a minimal application that stores the value "Hello World" and
then immediately retrieves the value:

\inputminted{javascript}{\topdir/api/node.js/hello-world.js}

You can run this example with:

\begin{consolecode}
% node hello-world.js
put: true
get: [object Object]
\end{consolecode}

Right away, there are several points worth noting in this example:

\begin{itemize}
\item Each operation takes a callback.  While the operation is outstanding, your
program is free to execute other code.

\item Javascript types are automatically converted to HyperDex types.  There's
no need to specify information such as the length of each string, as one would
do with the C API.

\item There's no need to manually enter the HyperDex event loop.  HyperDex will
add and remove itself from the event loop as operations start and finish.
\end{itemize}

\subsection{Asynchronous Operations}
\label{sec:api:node:async-ops}

HyperDex provides native integration with the asynchronous world of Node.js.
You can issue several operations concurrently, and Node.js and HyperDex will
work together to complete these operations quickly and efficiently.  It's easy
to work with data concurrently.  A common pattern is to keep a constant number
of operations outstanding concurrently:

\inputminted{javascript}{\topdir/api/node.js/window-pattern.js}

\subsection{Data Structures}
\label{sec:api:node:data-structures}

The Node bindings automatically manage conversion of data types from Javascript
to HyperDex types, enabling applications to be written in idiomatic Javascript.

\subsubsection{Examples}
\label{sec:api:node:examples}

This section shows examples of Java data structures that are recognized by
HyperDex.  The examples here are for illustration purposes and are not
exhaustive.

\paragraph{Strings}

The HyperDex client recognizes Javascript strings and buffers and automatically
converts them to HyperDex strings.  For example, the following two calls have
the same effect:

\begin{javascriptcode}
c.put("kv", "somekey", {v: "somevalue"}, function (success, err) {});
c.put("kv", "somekey", {v: new Buffer("somevalue")}, function (success, err) {});
\end{javascriptcode}

\paragraph{Integers}

The HyperDex client recognizes Javascript numbers and can convert them to
HyperDex integers.  For example:

\begin{javascriptcode}
c.put("kv", "somekey", {v: c.asInt(42)}, function (success, err) {});
\end{javascriptcode}

\paragraph{Floats}

The HyperDex client recognizes Javascript numbers and can convert them to
HyperDex floats.  For example:

\begin{javascriptcode}
c.put("kv", "somekey", {v: c.asFloat(3.1415)}, function (success, err) {});
\end{javascriptcode}

\paragraph{Lists}

The HyperDex client permits users to construct HyperDex lists from Javascript.
For example:

\begin{javascriptcode}
c.put("kv", "somekey", {v: c.asList(['a', 'b', 'c'])}, function (success, err) {});
c.put("kv", "somekey", {v: c.asList([1, 2, 3])}, function (success, err) {});
c.put("kv", "somekey", {v: c.asList([1.0, 0.5, 0.25])}, function (success, err) {});
\end{javascriptcode}

\paragraph{Sets}

The HyperDex client permits users to construct HyperDex sets from Javascript.
For example:

\begin{javascriptcode}
c.put("kv", "somekey", {v: c.asSet(['a', 'b', 'c'])}, function (success, err) {});
c.put("kv", "somekey", {v: c.asSet([1, 2, 3])}, function (success, err) {});
c.put("kv", "somekey", {v: c.asSet([1.0, 0.5, 0.25])}, function (success, err) {});
\end{javascriptcode}

\paragraph{Maps}

The HyperDex client permits users to construct HyperDex maps from Javascript.
For example:

\begin{javascriptcode}
c.put("kv", "somekey", {v1: c.asMap([["k", "v"]])}, function(success, err) {});
c.put("kv", "somekey", {v2: c.asMap([[1, 2]])}, function(success, err) {});
c.put("kv", "somekey", {v3: c.asMap([[3.14, 0.125]])}, function(success, err) {});
c.put("kv", "somekey", {v4: c.asMap([["a", 1]])}, function(success, err) {});
\end{javascriptcode}

\subsection{Attributes}
\label{sec:api:node:attributes}

Attributes in Node are specified in the form of a Javascript object.  As you can
see in the examples above, attributes are specified in the form:

\begin{javascriptcode}
{name: "value"}
\end{javascriptcode}

\subsection{Operations}
\label{sec:api:node:ops}

% Copyright (c) 2014, Cornell University
% All rights reserved.
%
% Redistribution and use in source and binary forms, with or without
% modification, are permitted provided that the following conditions are met:
%
%     * Redistributions of source code must retain the above copyright notice,
%       this list of conditions and the following disclaimer.
%     * Redistributions in binary form must reproduce the above copyright
%       notice, this list of conditions and the following disclaimer in the
%       documentation and/or other materials provided with the distribution.
%     * Neither the name of HyperDex nor the names of its contributors may be
%       used to endorse or promote products derived from this software without
%       specific prior written permission.
%
% THIS SOFTWARE IS PROVIDED BY THE COPYRIGHT HOLDERS AND CONTRIBUTORS "AS IS"
% AND ANY EXPRESS OR IMPLIED WARRANTIES, INCLUDING, BUT NOT LIMITED TO, THE
% IMPLIED WARRANTIES OF MERCHANTABILITY AND FITNESS FOR A PARTICULAR PURPOSE ARE
% DISCLAIMED. IN NO EVENT SHALL THE COPYRIGHT OWNER OR CONTRIBUTORS BE LIABLE
% FOR ANY DIRECT, INDIRECT, INCIDENTAL, SPECIAL, EXEMPLARY, OR CONSEQUENTIAL
% DAMAGES (INCLUDING, BUT NOT LIMITED TO, PROCUREMENT OF SUBSTITUTE GOODS OR
% SERVICES; LOSS OF USE, DATA, OR PROFITS; OR BUSINESS INTERRUPTION) HOWEVER
% CAUSED AND ON ANY THEORY OF LIABILITY, WHETHER IN CONTRACT, STRICT LIABILITY,
% OR TORT (INCLUDING NEGLIGENCE OR OTHERWISE) ARISING IN ANY WAY OUT OF THE USE
% OF THIS SOFTWARE, EVEN IF ADVISED OF THE POSSIBILITY OF SUCH DAMAGE.

% This LaTeX file is generated by bindings/nodejs.py

%%%%%%%%%%%%%%%%%%%% get %%%%%%%%%%%%%%%%%%%%
\pagebreak
\subsubsection{\code{get}}
\label{api:nodejs:get}
\index{get!Node.js API}
Get an object by key.

\paragraph{Behavior:}
\begin{itemize}[noitemsep]
\item This operation return to the user the requested object(s).

\end{itemize}


\paragraph{Definition:}
\begin{javascriptcode}
get(spacename, key, function (obj, done, err) {})
\end{javascriptcode}
\paragraph{Parameters:}
\begin{itemize}[noitemsep]
\item \code{spacename}\\
The name of the space as a string or symbol.

\item \code{key}\\
The key for the operation where \code{key} is a Javascript value.

\end{itemize}

\paragraph{Returns:}
Object if found, nil if not found.  Raises exception on error.


%%%%%%%%%%%%%%%%%%%% put %%%%%%%%%%%%%%%%%%%%
\pagebreak
\subsubsection{\code{put}}
\label{api:nodejs:put}
\index{put!Node.js API}
Store or update an object by key.  Existing values will be overwritten with the
values specified by \code{attrs}.  Values not specified by \code{attrs} will be
preserved.

%%% Generated below here
\paragraph{Behavior:}
\begin{itemize}[noitemsep]
\item An existing object will be updated by the operation.  If no object does
    exists, a new object will be created, with attributes initialized to their
    default values.

\end{itemize}


\paragraph{Definition:}
\begin{javascriptcode}
put(spacename, key, attributes, function (success, err) {})
\end{javascriptcode}
\paragraph{Parameters:}
\begin{itemize}[noitemsep]
\item \code{spacename}\\
The name of the space as a string or symbol.

\item \code{key}\\
The key for the operation where \code{key} is a Javascript value.

\item \code{attributes}\\
The set of attributes to modify and their respective values.  \code{attrs}
points to an array of length \code{attrs\_sz}.

\end{itemize}

\paragraph{Returns:}
True if the operation succeeded.  False if any provided predicates failed.
Raises an exception on error.


%%%%%%%%%%%%%%%%%%%% cond_put %%%%%%%%%%%%%%%%%%%%
\pagebreak
\subsubsection{\code{cond\_put}}
\label{api:nodejs:cond_put}
\index{cond\_put!Node.js API}
Conditionally store or update an object by key.  Existing values will be
overwritten with the values specified by \code{attrs}.  Values not specified by
\code{attrs} will be preserved.

%%% Generated below here
\paragraph{Behavior:}
\begin{itemize}[noitemsep]
\item This operation manipulates an existing object.  If no object exists, the
    operation will fail with \code{NOTFOUND}.

\item This operation will succeed if and only if the predicates specified by
    \code{checks} hold on the pre-existing object.  If any of the predicates are
    not true for the existing object, then the operation will have no effect and
    fail with \code{CMPFAIL}.

    All checks are atomic with the write.  HyperDex guarantees that no other
    operation will come between validating the checks, and writing the new
    version of the object..

\end{itemize}


\paragraph{Definition:}
\begin{javascriptcode}
cond_put(spacename, key, predicates, attributes, function (success, err) {})
\end{javascriptcode}
\paragraph{Parameters:}
\begin{itemize}[noitemsep]
\item \code{spacename}\\
The name of the space as a string or symbol.

\item \code{key}\\
The key for the operation where \code{key} is a Javascript value.

\item \code{predicates}\\
A set of predicates to check against.  \code{checks} points to an array of
length \code{checks\_sz}.

\item \code{attributes}\\
The set of attributes to modify and their respective values.  \code{attrs}
points to an array of length \code{attrs\_sz}.

\end{itemize}

\paragraph{Returns:}
True if the operation succeeded.  False if any provided predicates failed.
Raises an exception on error.


%%%%%%%%%%%%%%%%%%%% put_if_not_exist %%%%%%%%%%%%%%%%%%%%
\pagebreak
\subsubsection{\code{put\_if\_not\_exist}}
\label{api:nodejs:put_if_not_exist}
\index{put\_if\_not\_exist!Node.js API}
Store a new object by key.  Values not specified by \code{attrs} will be
initialized to their defaults.

%%% Generated below here
\paragraph{Behavior:}
\begin{itemize}[noitemsep]
\item This operation creates a new object.  If an object exists, the operation
    will fail with \code{CMPFAIL}.

\end{itemize}


\paragraph{Definition:}
\begin{javascriptcode}
put_if_not_exist(spacename, key, attributes, function (success, err) {})
\end{javascriptcode}
\paragraph{Parameters:}
\begin{itemize}[noitemsep]
\item \code{spacename}\\
The name of the space as a string or symbol.

\item \code{key}\\
The key for the operation where \code{key} is a Javascript value.

\item \code{attributes}\\
The set of attributes to modify and their respective values.  \code{attrs}
points to an array of length \code{attrs\_sz}.

\end{itemize}

\paragraph{Returns:}
True if the operation succeeded.  False if any provided predicates failed.
Raises an exception on error.


%%%%%%%%%%%%%%%%%%%% del %%%%%%%%%%%%%%%%%%%%
\pagebreak
\subsubsection{\code{del}}
\label{api:nodejs:del}
\index{del!Node.js API}
Delete an object by key.

%%% Generated below here
\paragraph{Behavior:}
\begin{itemize}[noitemsep]
\item An existing object stored under the given key will be erased.  If no
    object exists, the operation will fail and report \code{NOTFOUND}.

\end{itemize}


\paragraph{Definition:}
\begin{javascriptcode}
del(spacename, key, function (success, err) {})
\end{javascriptcode}
\paragraph{Parameters:}
\begin{itemize}[noitemsep]
\item \code{spacename}\\
The name of the space as a string or symbol.

\item \code{key}\\
The key for the operation where \code{key} is a Javascript value.

\end{itemize}

\paragraph{Returns:}
True if the operation succeeded.  False if any provided predicates failed.
Raises an exception on error.


%%%%%%%%%%%%%%%%%%%% cond_del %%%%%%%%%%%%%%%%%%%%
\pagebreak
\subsubsection{\code{cond\_del}}
\label{api:nodejs:cond_del}
\index{cond\_del!Node.js API}
Conditionally delete an object by key.

%%% Generated below here
\paragraph{Behavior:}
\begin{itemize}[noitemsep]
\item An existing object stored under the given key will be erased.  If no
    object exists, the operation will fail and report \code{NOTFOUND}.

\item This operation will succeed if and only if the predicates specified by
    \code{checks} hold on the pre-existing object.  If any of the predicates are
    not true for the existing object, then the operation will have no effect and
    fail with \code{CMPFAIL}.

    All checks are atomic with the write.  HyperDex guarantees that no other
    operation will come between validating the checks, and writing the new
    version of the object..

\end{itemize}


\paragraph{Definition:}
\begin{javascriptcode}
cond_del(spacename, key, predicates, function (success, err) {})
\end{javascriptcode}
\paragraph{Parameters:}
\begin{itemize}[noitemsep]
\item \code{spacename}\\
The name of the space as a string or symbol.

\item \code{key}\\
The key for the operation where \code{key} is a Javascript value.

\item \code{predicates}\\
A set of predicates to check against.  \code{checks} points to an array of
length \code{checks\_sz}.

\end{itemize}

\paragraph{Returns:}
True if the operation succeeded.  False if any provided predicates failed.
Raises an exception on error.


%%%%%%%%%%%%%%%%%%%% atomic_add %%%%%%%%%%%%%%%%%%%%
\pagebreak
\subsubsection{\code{atomic\_add}}
\label{api:nodejs:atomic_add}
\index{atomic\_add!Node.js API}
Add the specified number to the existing value for each attribute.

%%% Generated below here
\paragraph{Behavior:}
\begin{itemize}[noitemsep]
\item This operation manipulates an existing object.  If no object exists, the
    operation will fail with \code{NOTFOUND}.

\end{itemize}


\paragraph{Definition:}
\begin{javascriptcode}
atomic_add(spacename, key, attributes, function (success, err) {})
\end{javascriptcode}
\paragraph{Parameters:}
\begin{itemize}[noitemsep]
\item \code{spacename}\\
The name of the space as a string or symbol.

\item \code{key}\\
The key for the operation where \code{key} is a Javascript value.

\item \code{attributes}\\
The set of attributes to modify and their respective values.  \code{attrs}
points to an array of length \code{attrs\_sz}.

\end{itemize}

\paragraph{Returns:}
True if the operation succeeded.  False if any provided predicates failed.
Raises an exception on error.


%%%%%%%%%%%%%%%%%%%% cond_atomic_add %%%%%%%%%%%%%%%%%%%%
\pagebreak
\subsubsection{\code{cond\_atomic\_add}}
\label{api:nodejs:cond_atomic_add}
\index{cond\_atomic\_add!Node.js API}
Conditionally add the specified number to the existing value for each attribute.

%%% Generated below here
\paragraph{Behavior:}
\begin{itemize}[noitemsep]
\item This operation manipulates an existing object.  If no object exists, the
    operation will fail with \code{NOTFOUND}.

\item This operation will succeed if and only if the predicates specified by
    \code{checks} hold on the pre-existing object.  If any of the predicates are
    not true for the existing object, then the operation will have no effect and
    fail with \code{CMPFAIL}.

    All checks are atomic with the write.  HyperDex guarantees that no other
    operation will come between validating the checks, and writing the new
    version of the object..

\end{itemize}


\paragraph{Definition:}
\begin{javascriptcode}
cond_atomic_add(spacename, key, predicates, attributes, function (success, err) {})
\end{javascriptcode}
\paragraph{Parameters:}
\begin{itemize}[noitemsep]
\item \code{spacename}\\
The name of the space as a string or symbol.

\item \code{key}\\
The key for the operation where \code{key} is a Javascript value.

\item \code{predicates}\\
A set of predicates to check against.  \code{checks} points to an array of
length \code{checks\_sz}.

\item \code{attributes}\\
The set of attributes to modify and their respective values.  \code{attrs}
points to an array of length \code{attrs\_sz}.

\end{itemize}

\paragraph{Returns:}
True if the operation succeeded.  False if any provided predicates failed.
Raises an exception on error.


%%%%%%%%%%%%%%%%%%%% atomic_sub %%%%%%%%%%%%%%%%%%%%
\pagebreak
\subsubsection{\code{atomic\_sub}}
\label{api:nodejs:atomic_sub}
\index{atomic\_sub!Node.js API}
Subtract the specified number from the existing value for each attribute.

%%% Generated below here
\paragraph{Behavior:}
\begin{itemize}[noitemsep]
\item This operation manipulates an existing object.  If no object exists, the
    operation will fail with \code{NOTFOUND}.

\end{itemize}


\paragraph{Definition:}
\begin{javascriptcode}
atomic_sub(spacename, key, attributes, function (success, err) {})
\end{javascriptcode}
\paragraph{Parameters:}
\begin{itemize}[noitemsep]
\item \code{spacename}\\
The name of the space as a string or symbol.

\item \code{key}\\
The key for the operation where \code{key} is a Javascript value.

\item \code{attributes}\\
The set of attributes to modify and their respective values.  \code{attrs}
points to an array of length \code{attrs\_sz}.

\end{itemize}

\paragraph{Returns:}
True if the operation succeeded.  False if any provided predicates failed.
Raises an exception on error.


%%%%%%%%%%%%%%%%%%%% cond_atomic_sub %%%%%%%%%%%%%%%%%%%%
\pagebreak
\subsubsection{\code{cond\_atomic\_sub}}
\label{api:nodejs:cond_atomic_sub}
\index{cond\_atomic\_sub!Node.js API}
Conditionally subtract the specified number from the existing value for each attribute.

%%% Generated below here
\paragraph{Behavior:}
\begin{itemize}[noitemsep]
\item This operation manipulates an existing object.  If no object exists, the
    operation will fail with \code{NOTFOUND}.

\item This operation will succeed if and only if the predicates specified by
    \code{checks} hold on the pre-existing object.  If any of the predicates are
    not true for the existing object, then the operation will have no effect and
    fail with \code{CMPFAIL}.

    All checks are atomic with the write.  HyperDex guarantees that no other
    operation will come between validating the checks, and writing the new
    version of the object..

\end{itemize}


\paragraph{Definition:}
\begin{javascriptcode}
cond_atomic_sub(spacename, key, predicates, attributes, function (success, err) {})
\end{javascriptcode}
\paragraph{Parameters:}
\begin{itemize}[noitemsep]
\item \code{spacename}\\
The name of the space as a string or symbol.

\item \code{key}\\
The key for the operation where \code{key} is a Javascript value.

\item \code{predicates}\\
A set of predicates to check against.  \code{checks} points to an array of
length \code{checks\_sz}.

\item \code{attributes}\\
The set of attributes to modify and their respective values.  \code{attrs}
points to an array of length \code{attrs\_sz}.

\end{itemize}

\paragraph{Returns:}
True if the operation succeeded.  False if any provided predicates failed.
Raises an exception on error.


%%%%%%%%%%%%%%%%%%%% atomic_mul %%%%%%%%%%%%%%%%%%%%
\pagebreak
\subsubsection{\code{atomic\_mul}}
\label{api:nodejs:atomic_mul}
\index{atomic\_mul!Node.js API}
Multiply the existing value by the specified number for each attribute.

%%% Generated below here
\paragraph{Behavior:}
\begin{itemize}[noitemsep]
\item This operation manipulates an existing object.  If no object exists, the
    operation will fail with \code{NOTFOUND}.

\end{itemize}


\paragraph{Definition:}
\begin{javascriptcode}
atomic_mul(spacename, key, attributes, function (success, err) {})
\end{javascriptcode}
\paragraph{Parameters:}
\begin{itemize}[noitemsep]
\item \code{spacename}\\
The name of the space as a string or symbol.

\item \code{key}\\
The key for the operation where \code{key} is a Javascript value.

\item \code{attributes}\\
The set of attributes to modify and their respective values.  \code{attrs}
points to an array of length \code{attrs\_sz}.

\end{itemize}

\paragraph{Returns:}
True if the operation succeeded.  False if any provided predicates failed.
Raises an exception on error.


%%%%%%%%%%%%%%%%%%%% cond_atomic_mul %%%%%%%%%%%%%%%%%%%%
\pagebreak
\subsubsection{\code{cond\_atomic\_mul}}
\label{api:nodejs:cond_atomic_mul}
\index{cond\_atomic\_mul!Node.js API}
Conditionally multiply the existing value by the specified number for each
attribute.

%%% Generated below here
\paragraph{Behavior:}
\begin{itemize}[noitemsep]
\item This operation manipulates an existing object.  If no object exists, the
    operation will fail with \code{NOTFOUND}.

\item This operation will succeed if and only if the predicates specified by
    \code{checks} hold on the pre-existing object.  If any of the predicates are
    not true for the existing object, then the operation will have no effect and
    fail with \code{CMPFAIL}.

    All checks are atomic with the write.  HyperDex guarantees that no other
    operation will come between validating the checks, and writing the new
    version of the object..

\end{itemize}


\paragraph{Definition:}
\begin{javascriptcode}
cond_atomic_mul(spacename, key, predicates, attributes, function (success, err) {})
\end{javascriptcode}
\paragraph{Parameters:}
\begin{itemize}[noitemsep]
\item \code{spacename}\\
The name of the space as a string or symbol.

\item \code{key}\\
The key for the operation where \code{key} is a Javascript value.

\item \code{predicates}\\
A set of predicates to check against.  \code{checks} points to an array of
length \code{checks\_sz}.

\item \code{attributes}\\
The set of attributes to modify and their respective values.  \code{attrs}
points to an array of length \code{attrs\_sz}.

\end{itemize}

\paragraph{Returns:}
True if the operation succeeded.  False if any provided predicates failed.
Raises an exception on error.


%%%%%%%%%%%%%%%%%%%% atomic_div %%%%%%%%%%%%%%%%%%%%
\pagebreak
\subsubsection{\code{atomic\_div}}
\label{api:nodejs:atomic_div}
\index{atomic\_div!Node.js API}
Divide the existing value by the specified number for each attribute.

%%% Generated below here
\paragraph{Behavior:}
\begin{itemize}[noitemsep]
\item This operation manipulates an existing object.  If no object exists, the
    operation will fail with \code{NOTFOUND}.

\end{itemize}


\paragraph{Definition:}
\begin{javascriptcode}
atomic_div(spacename, key, attributes, function (success, err) {})
\end{javascriptcode}
\paragraph{Parameters:}
\begin{itemize}[noitemsep]
\item \code{spacename}\\
The name of the space as a string or symbol.

\item \code{key}\\
The key for the operation where \code{key} is a Javascript value.

\item \code{attributes}\\
The set of attributes to modify and their respective values.  \code{attrs}
points to an array of length \code{attrs\_sz}.

\end{itemize}

\paragraph{Returns:}
True if the operation succeeded.  False if any provided predicates failed.
Raises an exception on error.


%%%%%%%%%%%%%%%%%%%% cond_atomic_div %%%%%%%%%%%%%%%%%%%%
\pagebreak
\subsubsection{\code{cond\_atomic\_div}}
\label{api:nodejs:cond_atomic_div}
\index{cond\_atomic\_div!Node.js API}
Conditionally divide the existing value by the specified number for each
attribute.

%%% Generated below here
\paragraph{Behavior:}
\begin{itemize}[noitemsep]
\item This operation manipulates an existing object.  If no object exists, the
    operation will fail with \code{NOTFOUND}.

\item This operation will succeed if and only if the predicates specified by
    \code{checks} hold on the pre-existing object.  If any of the predicates are
    not true for the existing object, then the operation will have no effect and
    fail with \code{CMPFAIL}.

    All checks are atomic with the write.  HyperDex guarantees that no other
    operation will come between validating the checks, and writing the new
    version of the object..

\end{itemize}


\paragraph{Definition:}
\begin{javascriptcode}
cond_atomic_div(spacename, key, predicates, attributes, function (success, err) {})
\end{javascriptcode}
\paragraph{Parameters:}
\begin{itemize}[noitemsep]
\item \code{spacename}\\
The name of the space as a string or symbol.

\item \code{key}\\
The key for the operation where \code{key} is a Javascript value.

\item \code{predicates}\\
A set of predicates to check against.  \code{checks} points to an array of
length \code{checks\_sz}.

\item \code{attributes}\\
The set of attributes to modify and their respective values.  \code{attrs}
points to an array of length \code{attrs\_sz}.

\end{itemize}

\paragraph{Returns:}
True if the operation succeeded.  False if any provided predicates failed.
Raises an exception on error.


%%%%%%%%%%%%%%%%%%%% atomic_mod %%%%%%%%%%%%%%%%%%%%
\pagebreak
\subsubsection{\code{atomic\_mod}}
\label{api:nodejs:atomic_mod}
\index{atomic\_mod!Node.js API}
Store the existing value modulo the specified number for each attribute.

%%% Generated below here
\paragraph{Behavior:}
\begin{itemize}[noitemsep]
\item This operation manipulates an existing object.  If no object exists, the
    operation will fail with \code{NOTFOUND}.

\end{itemize}


\paragraph{Definition:}
\begin{javascriptcode}
atomic_mod(spacename, key, attributes, function (success, err) {})
\end{javascriptcode}
\paragraph{Parameters:}
\begin{itemize}[noitemsep]
\item \code{spacename}\\
The name of the space as a string or symbol.

\item \code{key}\\
The key for the operation where \code{key} is a Javascript value.

\item \code{attributes}\\
The set of attributes to modify and their respective values.  \code{attrs}
points to an array of length \code{attrs\_sz}.

\end{itemize}

\paragraph{Returns:}
True if the operation succeeded.  False if any provided predicates failed.
Raises an exception on error.


%%%%%%%%%%%%%%%%%%%% cond_atomic_mod %%%%%%%%%%%%%%%%%%%%
\pagebreak
\subsubsection{\code{cond\_atomic\_mod}}
\label{api:nodejs:cond_atomic_mod}
\index{cond\_atomic\_mod!Node.js API}
Conditionally store the existing value modulo the specified number for each
attribute.

%%% Generated below here
\paragraph{Behavior:}
\begin{itemize}[noitemsep]
\item This operation manipulates an existing object.  If no object exists, the
    operation will fail with \code{NOTFOUND}.

\item This operation will succeed if and only if the predicates specified by
    \code{checks} hold on the pre-existing object.  If any of the predicates are
    not true for the existing object, then the operation will have no effect and
    fail with \code{CMPFAIL}.

    All checks are atomic with the write.  HyperDex guarantees that no other
    operation will come between validating the checks, and writing the new
    version of the object..

\end{itemize}


\paragraph{Definition:}
\begin{javascriptcode}
cond_atomic_mod(spacename, key, predicates, attributes, function (success, err) {})
\end{javascriptcode}
\paragraph{Parameters:}
\begin{itemize}[noitemsep]
\item \code{spacename}\\
The name of the space as a string or symbol.

\item \code{key}\\
The key for the operation where \code{key} is a Javascript value.

\item \code{predicates}\\
A set of predicates to check against.  \code{checks} points to an array of
length \code{checks\_sz}.

\item \code{attributes}\\
The set of attributes to modify and their respective values.  \code{attrs}
points to an array of length \code{attrs\_sz}.

\end{itemize}

\paragraph{Returns:}
True if the operation succeeded.  False if any provided predicates failed.
Raises an exception on error.


%%%%%%%%%%%%%%%%%%%% atomic_and %%%%%%%%%%%%%%%%%%%%
\pagebreak
\subsubsection{\code{atomic\_and}}
\label{api:nodejs:atomic_and}
\index{atomic\_and!Node.js API}
Store the bitwise AND of the existing value and the specified number for
each attribute.

%%% Generated below here
\paragraph{Behavior:}
\begin{itemize}[noitemsep]
\item This operation manipulates an existing object.  If no object exists, the
    operation will fail with \code{NOTFOUND}.

\end{itemize}


\paragraph{Definition:}
\begin{javascriptcode}
atomic_and(spacename, key, attributes, function (success, err) {})
\end{javascriptcode}
\paragraph{Parameters:}
\begin{itemize}[noitemsep]
\item \code{spacename}\\
The name of the space as a string or symbol.

\item \code{key}\\
The key for the operation where \code{key} is a Javascript value.

\item \code{attributes}\\
The set of attributes to modify and their respective values.  \code{attrs}
points to an array of length \code{attrs\_sz}.

\end{itemize}

\paragraph{Returns:}
True if the operation succeeded.  False if any provided predicates failed.
Raises an exception on error.


%%%%%%%%%%%%%%%%%%%% cond_atomic_and %%%%%%%%%%%%%%%%%%%%
\pagebreak
\subsubsection{\code{cond\_atomic\_and}}
\label{api:nodejs:cond_atomic_and}
\index{cond\_atomic\_and!Node.js API}
Conditionally store the bitwise AND of the existing value and the specified
number for each attribute.

%%% Generated below here
\paragraph{Behavior:}
\begin{itemize}[noitemsep]
\item This operation manipulates an existing object.  If no object exists, the
    operation will fail with \code{NOTFOUND}.

\item This operation will succeed if and only if the predicates specified by
    \code{checks} hold on the pre-existing object.  If any of the predicates are
    not true for the existing object, then the operation will have no effect and
    fail with \code{CMPFAIL}.

    All checks are atomic with the write.  HyperDex guarantees that no other
    operation will come between validating the checks, and writing the new
    version of the object..

\end{itemize}


\paragraph{Definition:}
\begin{javascriptcode}
cond_atomic_and(spacename, key, predicates, attributes, function (success, err) {})
\end{javascriptcode}
\paragraph{Parameters:}
\begin{itemize}[noitemsep]
\item \code{spacename}\\
The name of the space as a string or symbol.

\item \code{key}\\
The key for the operation where \code{key} is a Javascript value.

\item \code{predicates}\\
A set of predicates to check against.  \code{checks} points to an array of
length \code{checks\_sz}.

\item \code{attributes}\\
The set of attributes to modify and their respective values.  \code{attrs}
points to an array of length \code{attrs\_sz}.

\end{itemize}

\paragraph{Returns:}
True if the operation succeeded.  False if any provided predicates failed.
Raises an exception on error.


%%%%%%%%%%%%%%%%%%%% atomic_or %%%%%%%%%%%%%%%%%%%%
\pagebreak
\subsubsection{\code{atomic\_or}}
\label{api:nodejs:atomic_or}
\index{atomic\_or!Node.js API}
Store the bitwise OR of the existing value and the specified number for each
attribute.

%%% Generated below here
\paragraph{Behavior:}
\begin{itemize}[noitemsep]
\item This operation manipulates an existing object.  If no object exists, the
    operation will fail with \code{NOTFOUND}.

\end{itemize}


\paragraph{Definition:}
\begin{javascriptcode}
atomic_or(spacename, key, attributes, function (success, err) {})
\end{javascriptcode}
\paragraph{Parameters:}
\begin{itemize}[noitemsep]
\item \code{spacename}\\
The name of the space as a string or symbol.

\item \code{key}\\
The key for the operation where \code{key} is a Javascript value.

\item \code{attributes}\\
The set of attributes to modify and their respective values.  \code{attrs}
points to an array of length \code{attrs\_sz}.

\end{itemize}

\paragraph{Returns:}
True if the operation succeeded.  False if any provided predicates failed.
Raises an exception on error.


%%%%%%%%%%%%%%%%%%%% cond_atomic_or %%%%%%%%%%%%%%%%%%%%
\pagebreak
\subsubsection{\code{cond\_atomic\_or}}
\label{api:nodejs:cond_atomic_or}
\index{cond\_atomic\_or!Node.js API}
Conditionally store the bitwise OR of the existing value and the specified
number for each attribute.

%%% Generated below here
\paragraph{Behavior:}
\begin{itemize}[noitemsep]
\item This operation manipulates an existing object.  If no object exists, the
    operation will fail with \code{NOTFOUND}.

\item This operation will succeed if and only if the predicates specified by
    \code{checks} hold on the pre-existing object.  If any of the predicates are
    not true for the existing object, then the operation will have no effect and
    fail with \code{CMPFAIL}.

    All checks are atomic with the write.  HyperDex guarantees that no other
    operation will come between validating the checks, and writing the new
    version of the object..

\end{itemize}


\paragraph{Definition:}
\begin{javascriptcode}
cond_atomic_or(spacename, key, predicates, attributes, function (success, err) {})
\end{javascriptcode}
\paragraph{Parameters:}
\begin{itemize}[noitemsep]
\item \code{spacename}\\
The name of the space as a string or symbol.

\item \code{key}\\
The key for the operation where \code{key} is a Javascript value.

\item \code{predicates}\\
A set of predicates to check against.  \code{checks} points to an array of
length \code{checks\_sz}.

\item \code{attributes}\\
The set of attributes to modify and their respective values.  \code{attrs}
points to an array of length \code{attrs\_sz}.

\end{itemize}

\paragraph{Returns:}
True if the operation succeeded.  False if any provided predicates failed.
Raises an exception on error.


%%%%%%%%%%%%%%%%%%%% atomic_xor %%%%%%%%%%%%%%%%%%%%
\pagebreak
\subsubsection{\code{atomic\_xor}}
\label{api:nodejs:atomic_xor}
\index{atomic\_xor!Node.js API}
Store the bitwise XOR of the existing value and the specified number for each
attribute.

%%% Generated below here
\paragraph{Behavior:}
\begin{itemize}[noitemsep]
\item This operation manipulates an existing object.  If no object exists, the
    operation will fail with \code{NOTFOUND}.

\end{itemize}


\paragraph{Definition:}
\begin{javascriptcode}
atomic_xor(spacename, key, attributes, function (success, err) {})
\end{javascriptcode}
\paragraph{Parameters:}
\begin{itemize}[noitemsep]
\item \code{spacename}\\
The name of the space as a string or symbol.

\item \code{key}\\
The key for the operation where \code{key} is a Javascript value.

\item \code{attributes}\\
The set of attributes to modify and their respective values.  \code{attrs}
points to an array of length \code{attrs\_sz}.

\end{itemize}

\paragraph{Returns:}
True if the operation succeeded.  False if any provided predicates failed.
Raises an exception on error.


%%%%%%%%%%%%%%%%%%%% cond_atomic_xor %%%%%%%%%%%%%%%%%%%%
\pagebreak
\subsubsection{\code{cond\_atomic\_xor}}
\label{api:nodejs:cond_atomic_xor}
\index{cond\_atomic\_xor!Node.js API}
Conditionally store the bitwise XOR of the existing value and the specified
number for each attribute.

%%% Generated below here
\paragraph{Behavior:}
\begin{itemize}[noitemsep]
\item This operation manipulates an existing object.  If no object exists, the
    operation will fail with \code{NOTFOUND}.

\item This operation will succeed if and only if the predicates specified by
    \code{checks} hold on the pre-existing object.  If any of the predicates are
    not true for the existing object, then the operation will have no effect and
    fail with \code{CMPFAIL}.

    All checks are atomic with the write.  HyperDex guarantees that no other
    operation will come between validating the checks, and writing the new
    version of the object..

\end{itemize}


\paragraph{Definition:}
\begin{javascriptcode}
cond_atomic_xor(spacename, key, predicates, attributes, function (success, err) {})
\end{javascriptcode}
\paragraph{Parameters:}
\begin{itemize}[noitemsep]
\item \code{spacename}\\
The name of the space as a string or symbol.

\item \code{key}\\
The key for the operation where \code{key} is a Javascript value.

\item \code{predicates}\\
A set of predicates to check against.  \code{checks} points to an array of
length \code{checks\_sz}.

\item \code{attributes}\\
The set of attributes to modify and their respective values.  \code{attrs}
points to an array of length \code{attrs\_sz}.

\end{itemize}

\paragraph{Returns:}
True if the operation succeeded.  False if any provided predicates failed.
Raises an exception on error.


%%%%%%%%%%%%%%%%%%%% string_prepend %%%%%%%%%%%%%%%%%%%%
\pagebreak
\subsubsection{\code{string\_prepend}}
\label{api:nodejs:string_prepend}
\index{string\_prepend!Node.js API}
Prepend the specified string to the existing value for each attribute.

%%% Generated below here
\paragraph{Behavior:}
\begin{itemize}[noitemsep]
\item This operation manipulates an existing object.  If no object exists, the
    operation will fail with \code{NOTFOUND}.

\end{itemize}


\paragraph{Definition:}
\begin{javascriptcode}
string_prepend(spacename, key, attributes, function (success, err) {})
\end{javascriptcode}
\paragraph{Parameters:}
\begin{itemize}[noitemsep]
\item \code{spacename}\\
The name of the space as a string or symbol.

\item \code{key}\\
The key for the operation where \code{key} is a Javascript value.

\item \code{attributes}\\
The set of attributes to modify and their respective values.  \code{attrs}
points to an array of length \code{attrs\_sz}.

\end{itemize}

\paragraph{Returns:}
True if the operation succeeded.  False if any provided predicates failed.
Raises an exception on error.


%%%%%%%%%%%%%%%%%%%% cond_string_prepend %%%%%%%%%%%%%%%%%%%%
\pagebreak
\subsubsection{\code{cond\_string\_prepend}}
\label{api:nodejs:cond_string_prepend}
\index{cond\_string\_prepend!Node.js API}
Conditionally prepend the specified string to the existing value for each
attribute.

%%% Generated below here
\paragraph{Behavior:}
\begin{itemize}[noitemsep]
\item This operation manipulates an existing object.  If no object exists, the
    operation will fail with \code{NOTFOUND}.

\item This operation will succeed if and only if the predicates specified by
    \code{checks} hold on the pre-existing object.  If any of the predicates are
    not true for the existing object, then the operation will have no effect and
    fail with \code{CMPFAIL}.

    All checks are atomic with the write.  HyperDex guarantees that no other
    operation will come between validating the checks, and writing the new
    version of the object..

\end{itemize}


\paragraph{Definition:}
\begin{javascriptcode}
cond_string_prepend(
        spacename, key, predicates, attributes, function (success, err) {})
\end{javascriptcode}
\paragraph{Parameters:}
\begin{itemize}[noitemsep]
\item \code{spacename}\\
The name of the space as a string or symbol.

\item \code{key}\\
The key for the operation where \code{key} is a Javascript value.

\item \code{predicates}\\
A set of predicates to check against.  \code{checks} points to an array of
length \code{checks\_sz}.

\item \code{attributes}\\
The set of attributes to modify and their respective values.  \code{attrs}
points to an array of length \code{attrs\_sz}.

\end{itemize}

\paragraph{Returns:}
True if the operation succeeded.  False if any provided predicates failed.
Raises an exception on error.


%%%%%%%%%%%%%%%%%%%% string_append %%%%%%%%%%%%%%%%%%%%
\pagebreak
\subsubsection{\code{string\_append}}
\label{api:nodejs:string_append}
\index{string\_append!Node.js API}
Append the specified string to the existing value for each attribute.

%%% Generated below here
\paragraph{Behavior:}
\begin{itemize}[noitemsep]
\item This operation manipulates an existing object.  If no object exists, the
    operation will fail with \code{NOTFOUND}.

\end{itemize}


\paragraph{Definition:}
\begin{javascriptcode}
string_append(spacename, key, attributes, function (success, err) {})
\end{javascriptcode}
\paragraph{Parameters:}
\begin{itemize}[noitemsep]
\item \code{spacename}\\
The name of the space as a string or symbol.

\item \code{key}\\
The key for the operation where \code{key} is a Javascript value.

\item \code{attributes}\\
The set of attributes to modify and their respective values.  \code{attrs}
points to an array of length \code{attrs\_sz}.

\end{itemize}

\paragraph{Returns:}
True if the operation succeeded.  False if any provided predicates failed.
Raises an exception on error.


%%%%%%%%%%%%%%%%%%%% cond_string_append %%%%%%%%%%%%%%%%%%%%
\pagebreak
\subsubsection{\code{cond\_string\_append}}
\label{api:nodejs:cond_string_append}
\index{cond\_string\_append!Node.js API}
Conditionally append the specified string to the existing value for each
attribute.

%%% Generated below here
\paragraph{Behavior:}
\begin{itemize}[noitemsep]
\item This operation manipulates an existing object.  If no object exists, the
    operation will fail with \code{NOTFOUND}.

\item This operation will succeed if and only if the predicates specified by
    \code{checks} hold on the pre-existing object.  If any of the predicates are
    not true for the existing object, then the operation will have no effect and
    fail with \code{CMPFAIL}.

    All checks are atomic with the write.  HyperDex guarantees that no other
    operation will come between validating the checks, and writing the new
    version of the object..

\end{itemize}


\paragraph{Definition:}
\begin{javascriptcode}
cond_string_append(
        spacename, key, predicates, attributes, function (success, err) {})
\end{javascriptcode}
\paragraph{Parameters:}
\begin{itemize}[noitemsep]
\item \code{spacename}\\
The name of the space as a string or symbol.

\item \code{key}\\
The key for the operation where \code{key} is a Javascript value.

\item \code{predicates}\\
A set of predicates to check against.  \code{checks} points to an array of
length \code{checks\_sz}.

\item \code{attributes}\\
The set of attributes to modify and their respective values.  \code{attrs}
points to an array of length \code{attrs\_sz}.

\end{itemize}

\paragraph{Returns:}
True if the operation succeeded.  False if any provided predicates failed.
Raises an exception on error.


%%%%%%%%%%%%%%%%%%%% list_lpush %%%%%%%%%%%%%%%%%%%%
\pagebreak
\subsubsection{\code{list\_lpush}}
\label{api:nodejs:list_lpush}
\index{list\_lpush!Node.js API}
Push the specified value onto the front of the list for each attribute.

%%% Generated below here
\paragraph{Behavior:}
\begin{itemize}[noitemsep]
\item This operation manipulates an existing object.  If no object exists, the
    operation will fail with \code{NOTFOUND}.

\end{itemize}


\paragraph{Definition:}
\begin{javascriptcode}
list_lpush(spacename, key, attributes, function (success, err) {})
\end{javascriptcode}
\paragraph{Parameters:}
\begin{itemize}[noitemsep]
\item \code{spacename}\\
The name of the space as a string or symbol.

\item \code{key}\\
The key for the operation where \code{key} is a Javascript value.

\item \code{attributes}\\
The set of attributes to modify and their respective values.  \code{attrs}
points to an array of length \code{attrs\_sz}.

\end{itemize}

\paragraph{Returns:}
True if the operation succeeded.  False if any provided predicates failed.
Raises an exception on error.


%%%%%%%%%%%%%%%%%%%% cond_list_lpush %%%%%%%%%%%%%%%%%%%%
\pagebreak
\subsubsection{\code{cond\_list\_lpush}}
\label{api:nodejs:cond_list_lpush}
\index{cond\_list\_lpush!Node.js API}
Condtitionally push the specified value onto the front of the list for each
attribute.

%%% Generated below here
\paragraph{Behavior:}
\begin{itemize}[noitemsep]
\item This operation manipulates an existing object.  If no object exists, the
    operation will fail with \code{NOTFOUND}.

\item This operation will succeed if and only if the predicates specified by
    \code{checks} hold on the pre-existing object.  If any of the predicates are
    not true for the existing object, then the operation will have no effect and
    fail with \code{CMPFAIL}.

    All checks are atomic with the write.  HyperDex guarantees that no other
    operation will come between validating the checks, and writing the new
    version of the object..

\end{itemize}


\paragraph{Definition:}
\begin{javascriptcode}
cond_list_lpush(spacename, key, predicates, attributes, function (success, err) {})
\end{javascriptcode}
\paragraph{Parameters:}
\begin{itemize}[noitemsep]
\item \code{spacename}\\
The name of the space as a string or symbol.

\item \code{key}\\
The key for the operation where \code{key} is a Javascript value.

\item \code{predicates}\\
A set of predicates to check against.  \code{checks} points to an array of
length \code{checks\_sz}.

\item \code{attributes}\\
The set of attributes to modify and their respective values.  \code{attrs}
points to an array of length \code{attrs\_sz}.

\end{itemize}

\paragraph{Returns:}
True if the operation succeeded.  False if any provided predicates failed.
Raises an exception on error.


%%%%%%%%%%%%%%%%%%%% list_rpush %%%%%%%%%%%%%%%%%%%%
\pagebreak
\subsubsection{\code{list\_rpush}}
\label{api:nodejs:list_rpush}
\index{list\_rpush!Node.js API}
Push the specified value onto the back of the list for each attribute.

%%% Generated below here
\paragraph{Behavior:}
\begin{itemize}[noitemsep]
\item This operation manipulates an existing object.  If no object exists, the
    operation will fail with \code{NOTFOUND}.

\end{itemize}


\paragraph{Definition:}
\begin{javascriptcode}
list_rpush(spacename, key, attributes, function (success, err) {})
\end{javascriptcode}
\paragraph{Parameters:}
\begin{itemize}[noitemsep]
\item \code{spacename}\\
The name of the space as a string or symbol.

\item \code{key}\\
The key for the operation where \code{key} is a Javascript value.

\item \code{attributes}\\
The set of attributes to modify and their respective values.  \code{attrs}
points to an array of length \code{attrs\_sz}.

\end{itemize}

\paragraph{Returns:}
True if the operation succeeded.  False if any provided predicates failed.
Raises an exception on error.


%%%%%%%%%%%%%%%%%%%% cond_list_rpush %%%%%%%%%%%%%%%%%%%%
\pagebreak
\subsubsection{\code{cond\_list\_rpush}}
\label{api:nodejs:cond_list_rpush}
\index{cond\_list\_rpush!Node.js API}
Conditionally push the specified value onto the back of the list for each
attribute.

%%% Generated below here
\paragraph{Behavior:}
\begin{itemize}[noitemsep]
\item This operation manipulates an existing object.  If no object exists, the
    operation will fail with \code{NOTFOUND}.

\item This operation will succeed if and only if the predicates specified by
    \code{checks} hold on the pre-existing object.  If any of the predicates are
    not true for the existing object, then the operation will have no effect and
    fail with \code{CMPFAIL}.

    All checks are atomic with the write.  HyperDex guarantees that no other
    operation will come between validating the checks, and writing the new
    version of the object..

\end{itemize}


\paragraph{Definition:}
\begin{javascriptcode}
cond_list_rpush(spacename, key, predicates, attributes, function (success, err) {})
\end{javascriptcode}
\paragraph{Parameters:}
\begin{itemize}[noitemsep]
\item \code{spacename}\\
The name of the space as a string or symbol.

\item \code{key}\\
The key for the operation where \code{key} is a Javascript value.

\item \code{predicates}\\
A set of predicates to check against.  \code{checks} points to an array of
length \code{checks\_sz}.

\item \code{attributes}\\
The set of attributes to modify and their respective values.  \code{attrs}
points to an array of length \code{attrs\_sz}.

\end{itemize}

\paragraph{Returns:}
True if the operation succeeded.  False if any provided predicates failed.
Raises an exception on error.


%%%%%%%%%%%%%%%%%%%% set_add %%%%%%%%%%%%%%%%%%%%
\pagebreak
\subsubsection{\code{set\_add}}
\label{api:nodejs:set_add}
\index{set\_add!Node.js API}
Add the specified value to the set for each attribute.

%%% Generated below here
\paragraph{Behavior:}
\begin{itemize}[noitemsep]
\item This operation manipulates an existing object.  If no object exists, the
    operation will fail with \code{NOTFOUND}.

\end{itemize}


\paragraph{Definition:}
\begin{javascriptcode}
set_add(spacename, key, attributes, function (success, err) {})
\end{javascriptcode}
\paragraph{Parameters:}
\begin{itemize}[noitemsep]
\item \code{spacename}\\
The name of the space as a string or symbol.

\item \code{key}\\
The key for the operation where \code{key} is a Javascript value.

\item \code{attributes}\\
The set of attributes to modify and their respective values.  \code{attrs}
points to an array of length \code{attrs\_sz}.

\end{itemize}

\paragraph{Returns:}
True if the operation succeeded.  False if any provided predicates failed.
Raises an exception on error.


%%%%%%%%%%%%%%%%%%%% cond_set_add %%%%%%%%%%%%%%%%%%%%
\pagebreak
\subsubsection{\code{cond\_set\_add}}
\label{api:nodejs:cond_set_add}
\index{cond\_set\_add!Node.js API}
Conditionally add the specified value to the set for each attribute.

%%% Generated below here
\paragraph{Behavior:}
\begin{itemize}[noitemsep]
\item This operation manipulates an existing object.  If no object exists, the
    operation will fail with \code{NOTFOUND}.

\item This operation will succeed if and only if the predicates specified by
    \code{checks} hold on the pre-existing object.  If any of the predicates are
    not true for the existing object, then the operation will have no effect and
    fail with \code{CMPFAIL}.

    All checks are atomic with the write.  HyperDex guarantees that no other
    operation will come between validating the checks, and writing the new
    version of the object..

\end{itemize}


\paragraph{Definition:}
\begin{javascriptcode}
cond_set_add(spacename, key, predicates, attributes, function (success, err) {})
\end{javascriptcode}
\paragraph{Parameters:}
\begin{itemize}[noitemsep]
\item \code{spacename}\\
The name of the space as a string or symbol.

\item \code{key}\\
The key for the operation where \code{key} is a Javascript value.

\item \code{predicates}\\
A set of predicates to check against.  \code{checks} points to an array of
length \code{checks\_sz}.

\item \code{attributes}\\
The set of attributes to modify and their respective values.  \code{attrs}
points to an array of length \code{attrs\_sz}.

\end{itemize}

\paragraph{Returns:}
True if the operation succeeded.  False if any provided predicates failed.
Raises an exception on error.


%%%%%%%%%%%%%%%%%%%% set_remove %%%%%%%%%%%%%%%%%%%%
\pagebreak
\subsubsection{\code{set\_remove}}
\label{api:nodejs:set_remove}
\index{set\_remove!Node.js API}
Remove the specified value from the set.  If the value is not contained within
the set, this operation will do nothing.

%%% Generated below here
\paragraph{Behavior:}
\begin{itemize}[noitemsep]
\item This operation manipulates an existing object.  If no object exists, the
    operation will fail with \code{NOTFOUND}.

\end{itemize}


\paragraph{Definition:}
\begin{javascriptcode}
set_remove(spacename, key, attributes, function (success, err) {})
\end{javascriptcode}
\paragraph{Parameters:}
\begin{itemize}[noitemsep]
\item \code{spacename}\\
The name of the space as a string or symbol.

\item \code{key}\\
The key for the operation where \code{key} is a Javascript value.

\item \code{attributes}\\
The set of attributes to modify and their respective values.  \code{attrs}
points to an array of length \code{attrs\_sz}.

\end{itemize}

\paragraph{Returns:}
True if the operation succeeded.  False if any provided predicates failed.
Raises an exception on error.


%%%%%%%%%%%%%%%%%%%% cond_set_remove %%%%%%%%%%%%%%%%%%%%
\pagebreak
\subsubsection{\code{cond\_set\_remove}}
\label{api:nodejs:cond_set_remove}
\index{cond\_set\_remove!Node.js API}
Conditionally remove the specified value from the set.  If the value is not
contained within the set, this operation will do nothing.

%%% Generated below here
\paragraph{Behavior:}
\begin{itemize}[noitemsep]
\item This operation manipulates an existing object.  If no object exists, the
    operation will fail with \code{NOTFOUND}.

\item This operation will succeed if and only if the predicates specified by
    \code{checks} hold on the pre-existing object.  If any of the predicates are
    not true for the existing object, then the operation will have no effect and
    fail with \code{CMPFAIL}.

    All checks are atomic with the write.  HyperDex guarantees that no other
    operation will come between validating the checks, and writing the new
    version of the object..

\end{itemize}


\paragraph{Definition:}
\begin{javascriptcode}
cond_set_remove(spacename, key, predicates, attributes, function (success, err) {})
\end{javascriptcode}
\paragraph{Parameters:}
\begin{itemize}[noitemsep]
\item \code{spacename}\\
The name of the space as a string or symbol.

\item \code{key}\\
The key for the operation where \code{key} is a Javascript value.

\item \code{predicates}\\
A set of predicates to check against.  \code{checks} points to an array of
length \code{checks\_sz}.

\item \code{attributes}\\
The set of attributes to modify and their respective values.  \code{attrs}
points to an array of length \code{attrs\_sz}.

\end{itemize}

\paragraph{Returns:}
True if the operation succeeded.  False if any provided predicates failed.
Raises an exception on error.


%%%%%%%%%%%%%%%%%%%% set_intersect %%%%%%%%%%%%%%%%%%%%
\pagebreak
\subsubsection{\code{set\_intersect}}
\label{api:nodejs:set_intersect}
\index{set\_intersect!Node.js API}
Store the intersection of the specified set and the existing value for each
attribute.

%%% Generated below here
\paragraph{Behavior:}
\begin{itemize}[noitemsep]
\item This operation manipulates an existing object.  If no object exists, the
    operation will fail with \code{NOTFOUND}.

\end{itemize}


\paragraph{Definition:}
\begin{javascriptcode}
set_intersect(spacename, key, attributes, function (success, err) {})
\end{javascriptcode}
\paragraph{Parameters:}
\begin{itemize}[noitemsep]
\item \code{spacename}\\
The name of the space as a string or symbol.

\item \code{key}\\
The key for the operation where \code{key} is a Javascript value.

\item \code{attributes}\\
The set of attributes to modify and their respective values.  \code{attrs}
points to an array of length \code{attrs\_sz}.

\end{itemize}

\paragraph{Returns:}
True if the operation succeeded.  False if any provided predicates failed.
Raises an exception on error.


%%%%%%%%%%%%%%%%%%%% cond_set_intersect %%%%%%%%%%%%%%%%%%%%
\pagebreak
\subsubsection{\code{cond\_set\_intersect}}
\label{api:nodejs:cond_set_intersect}
\index{cond\_set\_intersect!Node.js API}
Conditionally store the intersection of the specified set and the existing value
for each attribute.

%%% Generated below here
\paragraph{Behavior:}
\begin{itemize}[noitemsep]
\item This operation manipulates an existing object.  If no object exists, the
    operation will fail with \code{NOTFOUND}.

\item This operation will succeed if and only if the predicates specified by
    \code{checks} hold on the pre-existing object.  If any of the predicates are
    not true for the existing object, then the operation will have no effect and
    fail with \code{CMPFAIL}.

    All checks are atomic with the write.  HyperDex guarantees that no other
    operation will come between validating the checks, and writing the new
    version of the object..

\end{itemize}


\paragraph{Definition:}
\begin{javascriptcode}
cond_set_intersect(
        spacename, key, predicates, attributes, function (success, err) {})
\end{javascriptcode}
\paragraph{Parameters:}
\begin{itemize}[noitemsep]
\item \code{spacename}\\
The name of the space as a string or symbol.

\item \code{key}\\
The key for the operation where \code{key} is a Javascript value.

\item \code{predicates}\\
A set of predicates to check against.  \code{checks} points to an array of
length \code{checks\_sz}.

\item \code{attributes}\\
The set of attributes to modify and their respective values.  \code{attrs}
points to an array of length \code{attrs\_sz}.

\end{itemize}

\paragraph{Returns:}
True if the operation succeeded.  False if any provided predicates failed.
Raises an exception on error.


%%%%%%%%%%%%%%%%%%%% set_union %%%%%%%%%%%%%%%%%%%%
\pagebreak
\subsubsection{\code{set\_union}}
\label{api:nodejs:set_union}
\index{set\_union!Node.js API}
Store the union of the specified set and the existing value for each attribute.

%%% Generated below here
\paragraph{Behavior:}
\begin{itemize}[noitemsep]
\item This operation manipulates an existing object.  If no object exists, the
    operation will fail with \code{NOTFOUND}.

\end{itemize}


\paragraph{Definition:}
\begin{javascriptcode}
set_union(spacename, key, attributes, function (success, err) {})
\end{javascriptcode}
\paragraph{Parameters:}
\begin{itemize}[noitemsep]
\item \code{spacename}\\
The name of the space as a string or symbol.

\item \code{key}\\
The key for the operation where \code{key} is a Javascript value.

\item \code{attributes}\\
The set of attributes to modify and their respective values.  \code{attrs}
points to an array of length \code{attrs\_sz}.

\end{itemize}

\paragraph{Returns:}
True if the operation succeeded.  False if any provided predicates failed.
Raises an exception on error.


%%%%%%%%%%%%%%%%%%%% cond_set_union %%%%%%%%%%%%%%%%%%%%
\pagebreak
\subsubsection{\code{cond\_set\_union}}
\label{api:nodejs:cond_set_union}
\index{cond\_set\_union!Node.js API}
Conditionally store the union of the specified set and the existing value for
each attribute.

%%% Generated below here
\paragraph{Behavior:}
\begin{itemize}[noitemsep]
\item This operation manipulates an existing object.  If no object exists, the
    operation will fail with \code{NOTFOUND}.

\item This operation will succeed if and only if the predicates specified by
    \code{checks} hold on the pre-existing object.  If any of the predicates are
    not true for the existing object, then the operation will have no effect and
    fail with \code{CMPFAIL}.

    All checks are atomic with the write.  HyperDex guarantees that no other
    operation will come between validating the checks, and writing the new
    version of the object..

\end{itemize}


\paragraph{Definition:}
\begin{javascriptcode}
cond_set_union(spacename, key, predicates, attributes, function (success, err) {})
\end{javascriptcode}
\paragraph{Parameters:}
\begin{itemize}[noitemsep]
\item \code{spacename}\\
The name of the space as a string or symbol.

\item \code{key}\\
The key for the operation where \code{key} is a Javascript value.

\item \code{predicates}\\
A set of predicates to check against.  \code{checks} points to an array of
length \code{checks\_sz}.

\item \code{attributes}\\
The set of attributes to modify and their respective values.  \code{attrs}
points to an array of length \code{attrs\_sz}.

\end{itemize}

\paragraph{Returns:}
True if the operation succeeded.  False if any provided predicates failed.
Raises an exception on error.


%%%%%%%%%%%%%%%%%%%% map_add %%%%%%%%%%%%%%%%%%%%
\pagebreak
\subsubsection{\code{map\_add}}
\label{api:nodejs:map_add}
\index{map\_add!Node.js API}
Insert a key-value pair into the map specified by each map-attribute.

%%% Generated below here
\paragraph{Behavior:}
\begin{itemize}[noitemsep]
\item This operation manipulates an existing object.  If no object exists, the
    operation will fail with \code{NOTFOUND}.

\end{itemize}


\paragraph{Definition:}
\begin{javascriptcode}
map_add(spacename, key, mapattributes, function (success, err) {})
\end{javascriptcode}
\paragraph{Parameters:}
\begin{itemize}[noitemsep]
\item \code{spacename}\\
The name of the space as a string or symbol.

\item \code{key}\\
The key for the operation where \code{key} is a Javascript value.

\item \code{mapattributes}\\
The set of map attributes to modify and their respective key/values.
\code{mapattrs} points to an array of length \code{mapattrs\_sz}.  Each entry
specify an attribute that is a map and a key within that map.

\end{itemize}

\paragraph{Returns:}
True if the operation succeeded.  False if any provided predicates failed.
Raises an exception on error.


%%%%%%%%%%%%%%%%%%%% cond_map_add %%%%%%%%%%%%%%%%%%%%
\pagebreak
\subsubsection{\code{cond\_map\_add}}
\label{api:nodejs:cond_map_add}
\index{cond\_map\_add!Node.js API}
Conditionally insert a key-value pair into the map specified by each
map-attribute.

%%% Generated below here
\paragraph{Behavior:}
\begin{itemize}[noitemsep]
\item This operation manipulates an existing object.  If no object exists, the
    operation will fail with \code{NOTFOUND}.

\item This operation will succeed if and only if the predicates specified by
    \code{checks} hold on the pre-existing object.  If any of the predicates are
    not true for the existing object, then the operation will have no effect and
    fail with \code{CMPFAIL}.

    All checks are atomic with the write.  HyperDex guarantees that no other
    operation will come between validating the checks, and writing the new
    version of the object..

\end{itemize}


\paragraph{Definition:}
\begin{javascriptcode}
cond_map_add(spacename, key, predicates, mapattributes, function (success, err) {})
\end{javascriptcode}
\paragraph{Parameters:}
\begin{itemize}[noitemsep]
\item \code{spacename}\\
The name of the space as a string or symbol.

\item \code{key}\\
The key for the operation where \code{key} is a Javascript value.

\item \code{predicates}\\
A set of predicates to check against.  \code{checks} points to an array of
length \code{checks\_sz}.

\item \code{mapattributes}\\
The set of map attributes to modify and their respective key/values.
\code{mapattrs} points to an array of length \code{mapattrs\_sz}.  Each entry
specify an attribute that is a map and a key within that map.

\end{itemize}

\paragraph{Returns:}
True if the operation succeeded.  False if any provided predicates failed.
Raises an exception on error.


%%%%%%%%%%%%%%%%%%%% map_remove %%%%%%%%%%%%%%%%%%%%
\pagebreak
\subsubsection{\code{map\_remove}}
\label{api:nodejs:map_remove}
\index{map\_remove!Node.js API}
Remove a key-value pair from the map specified by each attribute.  If there is
no pair with the specified key within the map, this operation will do nothing.

%%% Generated below here
\paragraph{Behavior:}
\begin{itemize}[noitemsep]
\item This operation manipulates an existing object.  If no object exists, the
    operation will fail with \code{NOTFOUND}.

\end{itemize}


\paragraph{Definition:}
\begin{javascriptcode}
map_remove(spacename, key, attributes, function (success, err) {})
\end{javascriptcode}
\paragraph{Parameters:}
\begin{itemize}[noitemsep]
\item \code{spacename}\\
The name of the space as a string or symbol.

\item \code{key}\\
The key for the operation where \code{key} is a Javascript value.

\item \code{attributes}\\
The set of attributes to modify and their respective values.  \code{attrs}
points to an array of length \code{attrs\_sz}.

\end{itemize}

\paragraph{Returns:}
True if the operation succeeded.  False if any provided predicates failed.
Raises an exception on error.


%%%%%%%%%%%%%%%%%%%% cond_map_remove %%%%%%%%%%%%%%%%%%%%
\pagebreak
\subsubsection{\code{cond\_map\_remove}}
\label{api:nodejs:cond_map_remove}
\index{cond\_map\_remove!Node.js API}
Conditionally remove a key-value pair from the map specified by each attribute.

%%% Generated below here
\paragraph{Behavior:}
\begin{itemize}[noitemsep]
\item This operation manipulates an existing object.  If no object exists, the
    operation will fail with \code{NOTFOUND}.

\item This operation will succeed if and only if the predicates specified by
    \code{checks} hold on the pre-existing object.  If any of the predicates are
    not true for the existing object, then the operation will have no effect and
    fail with \code{CMPFAIL}.

    All checks are atomic with the write.  HyperDex guarantees that no other
    operation will come between validating the checks, and writing the new
    version of the object..

\end{itemize}


\paragraph{Definition:}
\begin{javascriptcode}
cond_map_remove(spacename, key, predicates, attributes, function (success, err) {})
\end{javascriptcode}
\paragraph{Parameters:}
\begin{itemize}[noitemsep]
\item \code{spacename}\\
The name of the space as a string or symbol.

\item \code{key}\\
The key for the operation where \code{key} is a Javascript value.

\item \code{predicates}\\
A set of predicates to check against.  \code{checks} points to an array of
length \code{checks\_sz}.

\item \code{attributes}\\
The set of attributes to modify and their respective values.  \code{attrs}
points to an array of length \code{attrs\_sz}.

\end{itemize}

\paragraph{Returns:}
True if the operation succeeded.  False if any provided predicates failed.
Raises an exception on error.


%%%%%%%%%%%%%%%%%%%% map_atomic_add %%%%%%%%%%%%%%%%%%%%
\pagebreak
\subsubsection{\code{map\_atomic\_add}}
\label{api:nodejs:map_atomic_add}
\index{map\_atomic\_add!Node.js API}
Add the specified number to the value of a key-value pair within each map.

%%% Generated below here
\paragraph{Behavior:}
\begin{itemize}[noitemsep]
\item This operation manipulates an existing object.  If no object exists, the
    operation will fail with \code{NOTFOUND}.

\item This operation mutates the value of a key-value pair in a map.  This call
    is similar to the equivalent call without the \code{map\_} prefix, but
    operates on the value of a pair in a map, instead of on an attribute's
    value.  If there is no pair with the specified map key, a new pair will be
    created and initialized to its default value.  If this is undesirable, it
    may be avoided by using a conditional operation that requires that the map
    contain the key in question.

\end{itemize}


\paragraph{Definition:}
\begin{javascriptcode}
map_atomic_add(spacename, key, mapattributes, function (success, err) {})
\end{javascriptcode}
\paragraph{Parameters:}
\begin{itemize}[noitemsep]
\item \code{spacename}\\
The name of the space as a string or symbol.

\item \code{key}\\
The key for the operation where \code{key} is a Javascript value.

\item \code{mapattributes}\\
The set of map attributes to modify and their respective key/values.
\code{mapattrs} points to an array of length \code{mapattrs\_sz}.  Each entry
specify an attribute that is a map and a key within that map.

\end{itemize}

\paragraph{Returns:}
True if the operation succeeded.  False if any provided predicates failed.
Raises an exception on error.


%%%%%%%%%%%%%%%%%%%% cond_map_atomic_add %%%%%%%%%%%%%%%%%%%%
\pagebreak
\subsubsection{\code{cond\_map\_atomic\_add}}
\label{api:nodejs:cond_map_atomic_add}
\index{cond\_map\_atomic\_add!Node.js API}
Conditionally add the specified number to the value of a key-value pair within
each map.

%%% Generated below here
\paragraph{Behavior:}
\begin{itemize}[noitemsep]
\item This operation manipulates an existing object.  If no object exists, the
    operation will fail with \code{NOTFOUND}.

\item This operation will succeed if and only if the predicates specified by
    \code{checks} hold on the pre-existing object.  If any of the predicates are
    not true for the existing object, then the operation will have no effect and
    fail with \code{CMPFAIL}.

    All checks are atomic with the write.  HyperDex guarantees that no other
    operation will come between validating the checks, and writing the new
    version of the object..

\item This operation mutates the value of a key-value pair in a map.  This call
    is similar to the equivalent call without the \code{map\_} prefix, but
    operates on the value of a pair in a map, instead of on an attribute's
    value.  If there is no pair with the specified map key, a new pair will be
    created and initialized to its default value.  If this is undesirable, it
    may be avoided by using a conditional operation that requires that the map
    contain the key in question.

\end{itemize}


\paragraph{Definition:}
\begin{javascriptcode}
cond_map_atomic_add(
        spacename, key, predicates, mapattributes, function (success, err) {})
\end{javascriptcode}
\paragraph{Parameters:}
\begin{itemize}[noitemsep]
\item \code{spacename}\\
The name of the space as a string or symbol.

\item \code{key}\\
The key for the operation where \code{key} is a Javascript value.

\item \code{predicates}\\
A set of predicates to check against.  \code{checks} points to an array of
length \code{checks\_sz}.

\item \code{mapattributes}\\
The set of map attributes to modify and their respective key/values.
\code{mapattrs} points to an array of length \code{mapattrs\_sz}.  Each entry
specify an attribute that is a map and a key within that map.

\end{itemize}

\paragraph{Returns:}
True if the operation succeeded.  False if any provided predicates failed.
Raises an exception on error.


%%%%%%%%%%%%%%%%%%%% map_atomic_sub %%%%%%%%%%%%%%%%%%%%
\pagebreak
\subsubsection{\code{map\_atomic\_sub}}
\label{api:nodejs:map_atomic_sub}
\index{map\_atomic\_sub!Node.js API}
Subtract the specified number from the value of a key-value pair within each
map.

%%% Generated below here
\paragraph{Behavior:}
\begin{itemize}[noitemsep]
\item This operation manipulates an existing object.  If no object exists, the
    operation will fail with \code{NOTFOUND}.

\item This operation mutates the value of a key-value pair in a map.  This call
    is similar to the equivalent call without the \code{map\_} prefix, but
    operates on the value of a pair in a map, instead of on an attribute's
    value.  If there is no pair with the specified map key, a new pair will be
    created and initialized to its default value.  If this is undesirable, it
    may be avoided by using a conditional operation that requires that the map
    contain the key in question.

\end{itemize}


\paragraph{Definition:}
\begin{javascriptcode}
map_atomic_sub(spacename, key, mapattributes, function (success, err) {})
\end{javascriptcode}
\paragraph{Parameters:}
\begin{itemize}[noitemsep]
\item \code{spacename}\\
The name of the space as a string or symbol.

\item \code{key}\\
The key for the operation where \code{key} is a Javascript value.

\item \code{mapattributes}\\
The set of map attributes to modify and their respective key/values.
\code{mapattrs} points to an array of length \code{mapattrs\_sz}.  Each entry
specify an attribute that is a map and a key within that map.

\end{itemize}

\paragraph{Returns:}
True if the operation succeeded.  False if any provided predicates failed.
Raises an exception on error.


%%%%%%%%%%%%%%%%%%%% cond_map_atomic_sub %%%%%%%%%%%%%%%%%%%%
\pagebreak
\subsubsection{\code{cond\_map\_atomic\_sub}}
\label{api:nodejs:cond_map_atomic_sub}
\index{cond\_map\_atomic\_sub!Node.js API}
Subtract the specified number from the value of a key-value pair within each
map.

%%% Generated below here
\paragraph{Behavior:}
\begin{itemize}[noitemsep]
\item This operation manipulates an existing object.  If no object exists, the
    operation will fail with \code{NOTFOUND}.

\item This operation will succeed if and only if the predicates specified by
    \code{checks} hold on the pre-existing object.  If any of the predicates are
    not true for the existing object, then the operation will have no effect and
    fail with \code{CMPFAIL}.

    All checks are atomic with the write.  HyperDex guarantees that no other
    operation will come between validating the checks, and writing the new
    version of the object..

\item This operation mutates the value of a key-value pair in a map.  This call
    is similar to the equivalent call without the \code{map\_} prefix, but
    operates on the value of a pair in a map, instead of on an attribute's
    value.  If there is no pair with the specified map key, a new pair will be
    created and initialized to its default value.  If this is undesirable, it
    may be avoided by using a conditional operation that requires that the map
    contain the key in question.

\end{itemize}


\paragraph{Definition:}
\begin{javascriptcode}
cond_map_atomic_sub(
        spacename, key, predicates, mapattributes, function (success, err) {})
\end{javascriptcode}
\paragraph{Parameters:}
\begin{itemize}[noitemsep]
\item \code{spacename}\\
The name of the space as a string or symbol.

\item \code{key}\\
The key for the operation where \code{key} is a Javascript value.

\item \code{predicates}\\
A set of predicates to check against.  \code{checks} points to an array of
length \code{checks\_sz}.

\item \code{mapattributes}\\
The set of map attributes to modify and their respective key/values.
\code{mapattrs} points to an array of length \code{mapattrs\_sz}.  Each entry
specify an attribute that is a map and a key within that map.

\end{itemize}

\paragraph{Returns:}
True if the operation succeeded.  False if any provided predicates failed.
Raises an exception on error.


%%%%%%%%%%%%%%%%%%%% map_atomic_mul %%%%%%%%%%%%%%%%%%%%
\pagebreak
\subsubsection{\code{map\_atomic\_mul}}
\label{api:nodejs:map_atomic_mul}
\index{map\_atomic\_mul!Node.js API}
Multiply the value of each key-value pair by the specified number for each map.

%%% Generated below here
\paragraph{Behavior:}
\begin{itemize}[noitemsep]
\item This operation manipulates an existing object.  If no object exists, the
    operation will fail with \code{NOTFOUND}.

\item This operation mutates the value of a key-value pair in a map.  This call
    is similar to the equivalent call without the \code{map\_} prefix, but
    operates on the value of a pair in a map, instead of on an attribute's
    value.  If there is no pair with the specified map key, a new pair will be
    created and initialized to its default value.  If this is undesirable, it
    may be avoided by using a conditional operation that requires that the map
    contain the key in question.

\end{itemize}


\paragraph{Definition:}
\begin{javascriptcode}
map_atomic_mul(spacename, key, mapattributes, function (success, err) {})
\end{javascriptcode}
\paragraph{Parameters:}
\begin{itemize}[noitemsep]
\item \code{spacename}\\
The name of the space as a string or symbol.

\item \code{key}\\
The key for the operation where \code{key} is a Javascript value.

\item \code{mapattributes}\\
The set of map attributes to modify and their respective key/values.
\code{mapattrs} points to an array of length \code{mapattrs\_sz}.  Each entry
specify an attribute that is a map and a key within that map.

\end{itemize}

\paragraph{Returns:}
True if the operation succeeded.  False if any provided predicates failed.
Raises an exception on error.


%%%%%%%%%%%%%%%%%%%% cond_map_atomic_mul %%%%%%%%%%%%%%%%%%%%
\pagebreak
\subsubsection{\code{cond\_map\_atomic\_mul}}
\label{api:nodejs:cond_map_atomic_mul}
\index{cond\_map\_atomic\_mul!Node.js API}
Conditionally multiply the value of each key-value pair by the specified number
for each map.

%%% Generated below here
\paragraph{Behavior:}
\begin{itemize}[noitemsep]
\item This operation manipulates an existing object.  If no object exists, the
    operation will fail with \code{NOTFOUND}.

\item This operation will succeed if and only if the predicates specified by
    \code{checks} hold on the pre-existing object.  If any of the predicates are
    not true for the existing object, then the operation will have no effect and
    fail with \code{CMPFAIL}.

    All checks are atomic with the write.  HyperDex guarantees that no other
    operation will come between validating the checks, and writing the new
    version of the object..

\item This operation mutates the value of a key-value pair in a map.  This call
    is similar to the equivalent call without the \code{map\_} prefix, but
    operates on the value of a pair in a map, instead of on an attribute's
    value.  If there is no pair with the specified map key, a new pair will be
    created and initialized to its default value.  If this is undesirable, it
    may be avoided by using a conditional operation that requires that the map
    contain the key in question.

\end{itemize}


\paragraph{Definition:}
\begin{javascriptcode}
cond_map_atomic_mul(
        spacename, key, predicates, mapattributes, function (success, err) {})
\end{javascriptcode}
\paragraph{Parameters:}
\begin{itemize}[noitemsep]
\item \code{spacename}\\
The name of the space as a string or symbol.

\item \code{key}\\
The key for the operation where \code{key} is a Javascript value.

\item \code{predicates}\\
A set of predicates to check against.  \code{checks} points to an array of
length \code{checks\_sz}.

\item \code{mapattributes}\\
The set of map attributes to modify and their respective key/values.
\code{mapattrs} points to an array of length \code{mapattrs\_sz}.  Each entry
specify an attribute that is a map and a key within that map.

\end{itemize}

\paragraph{Returns:}
True if the operation succeeded.  False if any provided predicates failed.
Raises an exception on error.


%%%%%%%%%%%%%%%%%%%% map_atomic_div %%%%%%%%%%%%%%%%%%%%
\pagebreak
\subsubsection{\code{map\_atomic\_div}}
\label{api:nodejs:map_atomic_div}
\index{map\_atomic\_div!Node.js API}
Divide the value of each key-value pair by the specified number for each map.

%%% Generated below here
\paragraph{Behavior:}
\begin{itemize}[noitemsep]
\item This operation manipulates an existing object.  If no object exists, the
    operation will fail with \code{NOTFOUND}.

\item This operation mutates the value of a key-value pair in a map.  This call
    is similar to the equivalent call without the \code{map\_} prefix, but
    operates on the value of a pair in a map, instead of on an attribute's
    value.  If there is no pair with the specified map key, a new pair will be
    created and initialized to its default value.  If this is undesirable, it
    may be avoided by using a conditional operation that requires that the map
    contain the key in question.

\end{itemize}


\paragraph{Definition:}
\begin{javascriptcode}
map_atomic_div(spacename, key, mapattributes, function (success, err) {})
\end{javascriptcode}
\paragraph{Parameters:}
\begin{itemize}[noitemsep]
\item \code{spacename}\\
The name of the space as a string or symbol.

\item \code{key}\\
The key for the operation where \code{key} is a Javascript value.

\item \code{mapattributes}\\
The set of map attributes to modify and their respective key/values.
\code{mapattrs} points to an array of length \code{mapattrs\_sz}.  Each entry
specify an attribute that is a map and a key within that map.

\end{itemize}

\paragraph{Returns:}
True if the operation succeeded.  False if any provided predicates failed.
Raises an exception on error.


%%%%%%%%%%%%%%%%%%%% cond_map_atomic_div %%%%%%%%%%%%%%%%%%%%
\pagebreak
\subsubsection{\code{cond\_map\_atomic\_div}}
\label{api:nodejs:cond_map_atomic_div}
\index{cond\_map\_atomic\_div!Node.js API}
Conditionally divide the value of each key-value pair by the specified number for each map.

%%% Generated below here
\paragraph{Behavior:}
\begin{itemize}[noitemsep]
\item This operation manipulates an existing object.  If no object exists, the
    operation will fail with \code{NOTFOUND}.

\item This operation will succeed if and only if the predicates specified by
    \code{checks} hold on the pre-existing object.  If any of the predicates are
    not true for the existing object, then the operation will have no effect and
    fail with \code{CMPFAIL}.

    All checks are atomic with the write.  HyperDex guarantees that no other
    operation will come between validating the checks, and writing the new
    version of the object..

\item This operation mutates the value of a key-value pair in a map.  This call
    is similar to the equivalent call without the \code{map\_} prefix, but
    operates on the value of a pair in a map, instead of on an attribute's
    value.  If there is no pair with the specified map key, a new pair will be
    created and initialized to its default value.  If this is undesirable, it
    may be avoided by using a conditional operation that requires that the map
    contain the key in question.

\end{itemize}


\paragraph{Definition:}
\begin{javascriptcode}
cond_map_atomic_div(
        spacename, key, predicates, mapattributes, function (success, err) {})
\end{javascriptcode}
\paragraph{Parameters:}
\begin{itemize}[noitemsep]
\item \code{spacename}\\
The name of the space as a string or symbol.

\item \code{key}\\
The key for the operation where \code{key} is a Javascript value.

\item \code{predicates}\\
A set of predicates to check against.  \code{checks} points to an array of
length \code{checks\_sz}.

\item \code{mapattributes}\\
The set of map attributes to modify and their respective key/values.
\code{mapattrs} points to an array of length \code{mapattrs\_sz}.  Each entry
specify an attribute that is a map and a key within that map.

\end{itemize}

\paragraph{Returns:}
True if the operation succeeded.  False if any provided predicates failed.
Raises an exception on error.


%%%%%%%%%%%%%%%%%%%% map_atomic_mod %%%%%%%%%%%%%%%%%%%%
\pagebreak
\subsubsection{\code{map\_atomic\_mod}}
\label{api:nodejs:map_atomic_mod}
\index{map\_atomic\_mod!Node.js API}
Store the value of the key-value pair modulo the specified number for each map.

%%% Generated below here
\paragraph{Behavior:}
\begin{itemize}[noitemsep]
\item This operation manipulates an existing object.  If no object exists, the
    operation will fail with \code{NOTFOUND}.

\item This operation mutates the value of a key-value pair in a map.  This call
    is similar to the equivalent call without the \code{map\_} prefix, but
    operates on the value of a pair in a map, instead of on an attribute's
    value.  If there is no pair with the specified map key, a new pair will be
    created and initialized to its default value.  If this is undesirable, it
    may be avoided by using a conditional operation that requires that the map
    contain the key in question.

\end{itemize}


\paragraph{Definition:}
\begin{javascriptcode}
map_atomic_mod(spacename, key, mapattributes, function (success, err) {})
\end{javascriptcode}
\paragraph{Parameters:}
\begin{itemize}[noitemsep]
\item \code{spacename}\\
The name of the space as a string or symbol.

\item \code{key}\\
The key for the operation where \code{key} is a Javascript value.

\item \code{mapattributes}\\
The set of map attributes to modify and their respective key/values.
\code{mapattrs} points to an array of length \code{mapattrs\_sz}.  Each entry
specify an attribute that is a map and a key within that map.

\end{itemize}

\paragraph{Returns:}
True if the operation succeeded.  False if any provided predicates failed.
Raises an exception on error.


%%%%%%%%%%%%%%%%%%%% cond_map_atomic_mod %%%%%%%%%%%%%%%%%%%%
\pagebreak
\subsubsection{\code{cond\_map\_atomic\_mod}}
\label{api:nodejs:cond_map_atomic_mod}
\index{cond\_map\_atomic\_mod!Node.js API}
Conditionally store the value of the key-value pair modulo the specified number
for each map.

%%% Generated below here
\paragraph{Behavior:}
\begin{itemize}[noitemsep]
\item This operation manipulates an existing object.  If no object exists, the
    operation will fail with \code{NOTFOUND}.

\item This operation will succeed if and only if the predicates specified by
    \code{checks} hold on the pre-existing object.  If any of the predicates are
    not true for the existing object, then the operation will have no effect and
    fail with \code{CMPFAIL}.

    All checks are atomic with the write.  HyperDex guarantees that no other
    operation will come between validating the checks, and writing the new
    version of the object..

\item This operation mutates the value of a key-value pair in a map.  This call
    is similar to the equivalent call without the \code{map\_} prefix, but
    operates on the value of a pair in a map, instead of on an attribute's
    value.  If there is no pair with the specified map key, a new pair will be
    created and initialized to its default value.  If this is undesirable, it
    may be avoided by using a conditional operation that requires that the map
    contain the key in question.

\end{itemize}


\paragraph{Definition:}
\begin{javascriptcode}
cond_map_atomic_mod(
        spacename, key, predicates, mapattributes, function (success, err) {})
\end{javascriptcode}
\paragraph{Parameters:}
\begin{itemize}[noitemsep]
\item \code{spacename}\\
The name of the space as a string or symbol.

\item \code{key}\\
The key for the operation where \code{key} is a Javascript value.

\item \code{predicates}\\
A set of predicates to check against.  \code{checks} points to an array of
length \code{checks\_sz}.

\item \code{mapattributes}\\
The set of map attributes to modify and their respective key/values.
\code{mapattrs} points to an array of length \code{mapattrs\_sz}.  Each entry
specify an attribute that is a map and a key within that map.

\end{itemize}

\paragraph{Returns:}
True if the operation succeeded.  False if any provided predicates failed.
Raises an exception on error.


%%%%%%%%%%%%%%%%%%%% map_atomic_and %%%%%%%%%%%%%%%%%%%%
\pagebreak
\subsubsection{\code{map\_atomic\_and}}
\label{api:nodejs:map_atomic_and}
\index{map\_atomic\_and!Node.js API}
Store the bitwise AND of the value of the key-value pair and the specified
number for each map.

%%% Generated below here
\paragraph{Behavior:}
\begin{itemize}[noitemsep]
\item This operation manipulates an existing object.  If no object exists, the
    operation will fail with \code{NOTFOUND}.

\item This operation mutates the value of a key-value pair in a map.  This call
    is similar to the equivalent call without the \code{map\_} prefix, but
    operates on the value of a pair in a map, instead of on an attribute's
    value.  If there is no pair with the specified map key, a new pair will be
    created and initialized to its default value.  If this is undesirable, it
    may be avoided by using a conditional operation that requires that the map
    contain the key in question.

\end{itemize}


\paragraph{Definition:}
\begin{javascriptcode}
map_atomic_and(spacename, key, mapattributes, function (success, err) {})
\end{javascriptcode}
\paragraph{Parameters:}
\begin{itemize}[noitemsep]
\item \code{spacename}\\
The name of the space as a string or symbol.

\item \code{key}\\
The key for the operation where \code{key} is a Javascript value.

\item \code{mapattributes}\\
The set of map attributes to modify and their respective key/values.
\code{mapattrs} points to an array of length \code{mapattrs\_sz}.  Each entry
specify an attribute that is a map and a key within that map.

\end{itemize}

\paragraph{Returns:}
True if the operation succeeded.  False if any provided predicates failed.
Raises an exception on error.


%%%%%%%%%%%%%%%%%%%% cond_map_atomic_and %%%%%%%%%%%%%%%%%%%%
\pagebreak
\subsubsection{\code{cond\_map\_atomic\_and}}
\label{api:nodejs:cond_map_atomic_and}
\index{cond\_map\_atomic\_and!Node.js API}
Conditionally store the bitwise AND of the value of the key-value pair and the
specified number for each map.

%%% Generated below here
\paragraph{Behavior:}
\begin{itemize}[noitemsep]
\item This operation manipulates an existing object.  If no object exists, the
    operation will fail with \code{NOTFOUND}.

\item This operation will succeed if and only if the predicates specified by
    \code{checks} hold on the pre-existing object.  If any of the predicates are
    not true for the existing object, then the operation will have no effect and
    fail with \code{CMPFAIL}.

    All checks are atomic with the write.  HyperDex guarantees that no other
    operation will come between validating the checks, and writing the new
    version of the object..

\item This operation mutates the value of a key-value pair in a map.  This call
    is similar to the equivalent call without the \code{map\_} prefix, but
    operates on the value of a pair in a map, instead of on an attribute's
    value.  If there is no pair with the specified map key, a new pair will be
    created and initialized to its default value.  If this is undesirable, it
    may be avoided by using a conditional operation that requires that the map
    contain the key in question.

\end{itemize}


\paragraph{Definition:}
\begin{javascriptcode}
cond_map_atomic_and(
        spacename, key, predicates, mapattributes, function (success, err) {})
\end{javascriptcode}
\paragraph{Parameters:}
\begin{itemize}[noitemsep]
\item \code{spacename}\\
The name of the space as a string or symbol.

\item \code{key}\\
The key for the operation where \code{key} is a Javascript value.

\item \code{predicates}\\
A set of predicates to check against.  \code{checks} points to an array of
length \code{checks\_sz}.

\item \code{mapattributes}\\
The set of map attributes to modify and their respective key/values.
\code{mapattrs} points to an array of length \code{mapattrs\_sz}.  Each entry
specify an attribute that is a map and a key within that map.

\end{itemize}

\paragraph{Returns:}
True if the operation succeeded.  False if any provided predicates failed.
Raises an exception on error.


%%%%%%%%%%%%%%%%%%%% map_atomic_or %%%%%%%%%%%%%%%%%%%%
\pagebreak
\subsubsection{\code{map\_atomic\_or}}
\label{api:nodejs:map_atomic_or}
\index{map\_atomic\_or!Node.js API}
Store the bitwise OR of the value of the key-value pair and the specified number
for each map.

%%% Generated below here
\paragraph{Behavior:}
\begin{itemize}[noitemsep]
\item This operation manipulates an existing object.  If no object exists, the
    operation will fail with \code{NOTFOUND}.

\item This operation mutates the value of a key-value pair in a map.  This call
    is similar to the equivalent call without the \code{map\_} prefix, but
    operates on the value of a pair in a map, instead of on an attribute's
    value.  If there is no pair with the specified map key, a new pair will be
    created and initialized to its default value.  If this is undesirable, it
    may be avoided by using a conditional operation that requires that the map
    contain the key in question.

\end{itemize}


\paragraph{Definition:}
\begin{javascriptcode}
map_atomic_or(spacename, key, mapattributes, function (success, err) {})
\end{javascriptcode}
\paragraph{Parameters:}
\begin{itemize}[noitemsep]
\item \code{spacename}\\
The name of the space as a string or symbol.

\item \code{key}\\
The key for the operation where \code{key} is a Javascript value.

\item \code{mapattributes}\\
The set of map attributes to modify and their respective key/values.
\code{mapattrs} points to an array of length \code{mapattrs\_sz}.  Each entry
specify an attribute that is a map and a key within that map.

\end{itemize}

\paragraph{Returns:}
True if the operation succeeded.  False if any provided predicates failed.
Raises an exception on error.


%%%%%%%%%%%%%%%%%%%% cond_map_atomic_or %%%%%%%%%%%%%%%%%%%%
\pagebreak
\subsubsection{\code{cond\_map\_atomic\_or}}
\label{api:nodejs:cond_map_atomic_or}
\index{cond\_map\_atomic\_or!Node.js API}
Conditionally store the bitwise OR of the value of the key-value pair and the
specified number for each map.

%%% Generated below here
\paragraph{Behavior:}
\begin{itemize}[noitemsep]
\item This operation manipulates an existing object.  If no object exists, the
    operation will fail with \code{NOTFOUND}.

\item This operation will succeed if and only if the predicates specified by
    \code{checks} hold on the pre-existing object.  If any of the predicates are
    not true for the existing object, then the operation will have no effect and
    fail with \code{CMPFAIL}.

    All checks are atomic with the write.  HyperDex guarantees that no other
    operation will come between validating the checks, and writing the new
    version of the object..

\item This operation mutates the value of a key-value pair in a map.  This call
    is similar to the equivalent call without the \code{map\_} prefix, but
    operates on the value of a pair in a map, instead of on an attribute's
    value.  If there is no pair with the specified map key, a new pair will be
    created and initialized to its default value.  If this is undesirable, it
    may be avoided by using a conditional operation that requires that the map
    contain the key in question.

\end{itemize}


\paragraph{Definition:}
\begin{javascriptcode}
cond_map_atomic_or(
        spacename, key, predicates, mapattributes, function (success, err) {})
\end{javascriptcode}
\paragraph{Parameters:}
\begin{itemize}[noitemsep]
\item \code{spacename}\\
The name of the space as a string or symbol.

\item \code{key}\\
The key for the operation where \code{key} is a Javascript value.

\item \code{predicates}\\
A set of predicates to check against.  \code{checks} points to an array of
length \code{checks\_sz}.

\item \code{mapattributes}\\
The set of map attributes to modify and their respective key/values.
\code{mapattrs} points to an array of length \code{mapattrs\_sz}.  Each entry
specify an attribute that is a map and a key within that map.

\end{itemize}

\paragraph{Returns:}
True if the operation succeeded.  False if any provided predicates failed.
Raises an exception on error.


%%%%%%%%%%%%%%%%%%%% map_atomic_xor %%%%%%%%%%%%%%%%%%%%
\pagebreak
\subsubsection{\code{map\_atomic\_xor}}
\label{api:nodejs:map_atomic_xor}
\index{map\_atomic\_xor!Node.js API}
Store the bitwise XOR of the value of the key-value pair and the specified
number for each map.

%%% Generated below here
\paragraph{Behavior:}
\begin{itemize}[noitemsep]
\item This operation manipulates an existing object.  If no object exists, the
    operation will fail with \code{NOTFOUND}.

\item This operation mutates the value of a key-value pair in a map.  This call
    is similar to the equivalent call without the \code{map\_} prefix, but
    operates on the value of a pair in a map, instead of on an attribute's
    value.  If there is no pair with the specified map key, a new pair will be
    created and initialized to its default value.  If this is undesirable, it
    may be avoided by using a conditional operation that requires that the map
    contain the key in question.

\end{itemize}


\paragraph{Definition:}
\begin{javascriptcode}
map_atomic_xor(spacename, key, mapattributes, function (success, err) {})
\end{javascriptcode}
\paragraph{Parameters:}
\begin{itemize}[noitemsep]
\item \code{spacename}\\
The name of the space as a string or symbol.

\item \code{key}\\
The key for the operation where \code{key} is a Javascript value.

\item \code{mapattributes}\\
The set of map attributes to modify and their respective key/values.
\code{mapattrs} points to an array of length \code{mapattrs\_sz}.  Each entry
specify an attribute that is a map and a key within that map.

\end{itemize}

\paragraph{Returns:}
True if the operation succeeded.  False if any provided predicates failed.
Raises an exception on error.


%%%%%%%%%%%%%%%%%%%% cond_map_atomic_xor %%%%%%%%%%%%%%%%%%%%
\pagebreak
\subsubsection{\code{cond\_map\_atomic\_xor}}
\label{api:nodejs:cond_map_atomic_xor}
\index{cond\_map\_atomic\_xor!Node.js API}
Conditionally store the bitwise XOR of the value of the key-value pair and the
specified number for each map.

%%% Generated below here
\paragraph{Behavior:}
\begin{itemize}[noitemsep]
\item This operation manipulates an existing object.  If no object exists, the
    operation will fail with \code{NOTFOUND}.

\item This operation will succeed if and only if the predicates specified by
    \code{checks} hold on the pre-existing object.  If any of the predicates are
    not true for the existing object, then the operation will have no effect and
    fail with \code{CMPFAIL}.

    All checks are atomic with the write.  HyperDex guarantees that no other
    operation will come between validating the checks, and writing the new
    version of the object..

\item This operation mutates the value of a key-value pair in a map.  This call
    is similar to the equivalent call without the \code{map\_} prefix, but
    operates on the value of a pair in a map, instead of on an attribute's
    value.  If there is no pair with the specified map key, a new pair will be
    created and initialized to its default value.  If this is undesirable, it
    may be avoided by using a conditional operation that requires that the map
    contain the key in question.

\end{itemize}


\paragraph{Definition:}
\begin{javascriptcode}
cond_map_atomic_xor(
        spacename, key, predicates, mapattributes, function (success, err) {})
\end{javascriptcode}
\paragraph{Parameters:}
\begin{itemize}[noitemsep]
\item \code{spacename}\\
The name of the space as a string or symbol.

\item \code{key}\\
The key for the operation where \code{key} is a Javascript value.

\item \code{predicates}\\
A set of predicates to check against.  \code{checks} points to an array of
length \code{checks\_sz}.

\item \code{mapattributes}\\
The set of map attributes to modify and their respective key/values.
\code{mapattrs} points to an array of length \code{mapattrs\_sz}.  Each entry
specify an attribute that is a map and a key within that map.

\end{itemize}

\paragraph{Returns:}
True if the operation succeeded.  False if any provided predicates failed.
Raises an exception on error.


%%%%%%%%%%%%%%%%%%%% map_string_prepend %%%%%%%%%%%%%%%%%%%%
\pagebreak
\subsubsection{\code{map\_string\_prepend}}
\label{api:nodejs:map_string_prepend}
\index{map\_string\_prepend!Node.js API}
Prepend the specified string to the value of the key-value pair for each map.

%%% Generated below here
\paragraph{Behavior:}
\begin{itemize}[noitemsep]
\item This operation manipulates an existing object.  If no object exists, the
    operation will fail with \code{NOTFOUND}.

\item This operation mutates the value of a key-value pair in a map.  This call
    is similar to the equivalent call without the \code{map\_} prefix, but
    operates on the value of a pair in a map, instead of on an attribute's
    value.  If there is no pair with the specified map key, a new pair will be
    created and initialized to its default value.  If this is undesirable, it
    may be avoided by using a conditional operation that requires that the map
    contain the key in question.

\end{itemize}


\paragraph{Definition:}
\begin{javascriptcode}
map_string_prepend(spacename, key, mapattributes, function (success, err) {})
\end{javascriptcode}
\paragraph{Parameters:}
\begin{itemize}[noitemsep]
\item \code{spacename}\\
The name of the space as a string or symbol.

\item \code{key}\\
The key for the operation where \code{key} is a Javascript value.

\item \code{mapattributes}\\
The set of map attributes to modify and their respective key/values.
\code{mapattrs} points to an array of length \code{mapattrs\_sz}.  Each entry
specify an attribute that is a map and a key within that map.

\end{itemize}

\paragraph{Returns:}
True if the operation succeeded.  False if any provided predicates failed.
Raises an exception on error.


%%%%%%%%%%%%%%%%%%%% cond_map_string_prepend %%%%%%%%%%%%%%%%%%%%
\pagebreak
\subsubsection{\code{cond\_map\_string\_prepend}}
\label{api:nodejs:cond_map_string_prepend}
\index{cond\_map\_string\_prepend!Node.js API}
Conditionally prepend the specified string to the value of the key-value pair
for each map.

%%% Generated below here
\paragraph{Behavior:}
\begin{itemize}[noitemsep]
\item This operation manipulates an existing object.  If no object exists, the
    operation will fail with \code{NOTFOUND}.

\item This operation will succeed if and only if the predicates specified by
    \code{checks} hold on the pre-existing object.  If any of the predicates are
    not true for the existing object, then the operation will have no effect and
    fail with \code{CMPFAIL}.

    All checks are atomic with the write.  HyperDex guarantees that no other
    operation will come between validating the checks, and writing the new
    version of the object..

\item This operation mutates the value of a key-value pair in a map.  This call
    is similar to the equivalent call without the \code{map\_} prefix, but
    operates on the value of a pair in a map, instead of on an attribute's
    value.  If there is no pair with the specified map key, a new pair will be
    created and initialized to its default value.  If this is undesirable, it
    may be avoided by using a conditional operation that requires that the map
    contain the key in question.

\end{itemize}


\paragraph{Definition:}
\begin{javascriptcode}
cond_map_string_prepend(
        spacename, key, predicates, mapattributes, function (success, err) {})
\end{javascriptcode}
\paragraph{Parameters:}
\begin{itemize}[noitemsep]
\item \code{spacename}\\
The name of the space as a string or symbol.

\item \code{key}\\
The key for the operation where \code{key} is a Javascript value.

\item \code{predicates}\\
A set of predicates to check against.  \code{checks} points to an array of
length \code{checks\_sz}.

\item \code{mapattributes}\\
The set of map attributes to modify and their respective key/values.
\code{mapattrs} points to an array of length \code{mapattrs\_sz}.  Each entry
specify an attribute that is a map and a key within that map.

\end{itemize}

\paragraph{Returns:}
True if the operation succeeded.  False if any provided predicates failed.
Raises an exception on error.


%%%%%%%%%%%%%%%%%%%% map_string_append %%%%%%%%%%%%%%%%%%%%
\pagebreak
\subsubsection{\code{map\_string\_append}}
\label{api:nodejs:map_string_append}
\index{map\_string\_append!Node.js API}
Append the specified string to the value of the key-value pair for each map.

%%% Generated below here
\paragraph{Behavior:}
\begin{itemize}[noitemsep]
\item This operation manipulates an existing object.  If no object exists, the
    operation will fail with \code{NOTFOUND}.

\item This operation mutates the value of a key-value pair in a map.  This call
    is similar to the equivalent call without the \code{map\_} prefix, but
    operates on the value of a pair in a map, instead of on an attribute's
    value.  If there is no pair with the specified map key, a new pair will be
    created and initialized to its default value.  If this is undesirable, it
    may be avoided by using a conditional operation that requires that the map
    contain the key in question.

\end{itemize}


\paragraph{Definition:}
\begin{javascriptcode}
map_string_append(spacename, key, mapattributes, function (success, err) {})
\end{javascriptcode}
\paragraph{Parameters:}
\begin{itemize}[noitemsep]
\item \code{spacename}\\
The name of the space as a string or symbol.

\item \code{key}\\
The key for the operation where \code{key} is a Javascript value.

\item \code{mapattributes}\\
The set of map attributes to modify and their respective key/values.
\code{mapattrs} points to an array of length \code{mapattrs\_sz}.  Each entry
specify an attribute that is a map and a key within that map.

\end{itemize}

\paragraph{Returns:}
True if the operation succeeded.  False if any provided predicates failed.
Raises an exception on error.


%%%%%%%%%%%%%%%%%%%% cond_map_string_append %%%%%%%%%%%%%%%%%%%%
\pagebreak
\subsubsection{\code{cond\_map\_string\_append}}
\label{api:nodejs:cond_map_string_append}
\index{cond\_map\_string\_append!Node.js API}
Conditionally append the specified string to the value of the key-value pair for
each map.

%%% Generated below here
\paragraph{Behavior:}
\begin{itemize}[noitemsep]
\item This operation manipulates an existing object.  If no object exists, the
    operation will fail with \code{NOTFOUND}.

\item This operation will succeed if and only if the predicates specified by
    \code{checks} hold on the pre-existing object.  If any of the predicates are
    not true for the existing object, then the operation will have no effect and
    fail with \code{CMPFAIL}.

    All checks are atomic with the write.  HyperDex guarantees that no other
    operation will come between validating the checks, and writing the new
    version of the object..

\item This operation mutates the value of a key-value pair in a map.  This call
    is similar to the equivalent call without the \code{map\_} prefix, but
    operates on the value of a pair in a map, instead of on an attribute's
    value.  If there is no pair with the specified map key, a new pair will be
    created and initialized to its default value.  If this is undesirable, it
    may be avoided by using a conditional operation that requires that the map
    contain the key in question.

\end{itemize}


\paragraph{Definition:}
\begin{javascriptcode}
cond_map_string_append(
        spacename, key, predicates, mapattributes, function (success, err) {})
\end{javascriptcode}
\paragraph{Parameters:}
\begin{itemize}[noitemsep]
\item \code{spacename}\\
The name of the space as a string or symbol.

\item \code{key}\\
The key for the operation where \code{key} is a Javascript value.

\item \code{predicates}\\
A set of predicates to check against.  \code{checks} points to an array of
length \code{checks\_sz}.

\item \code{mapattributes}\\
The set of map attributes to modify and their respective key/values.
\code{mapattrs} points to an array of length \code{mapattrs\_sz}.  Each entry
specify an attribute that is a map and a key within that map.

\end{itemize}

\paragraph{Returns:}
True if the operation succeeded.  False if any provided predicates failed.
Raises an exception on error.


%%%%%%%%%%%%%%%%%%%% search %%%%%%%%%%%%%%%%%%%%
\pagebreak
\subsubsection{\code{search}}
\label{api:nodejs:search}
\index{search!Node.js API}
Return all objects that match the specified \code{checks}.

\paragraph{Behavior:}
\begin{itemize}[noitemsep]
\item This operation behaves as an iterator.  Multiple objects may be returned
    from the single call.

\item This operation return to the user the requested object(s).

\end{itemize}


\paragraph{Definition:}
\begin{javascriptcode}
search(spacename, predicates, function (obj, err) {})
\end{javascriptcode}
\paragraph{Parameters:}
\begin{itemize}[noitemsep]
\item \code{spacename}\\
The name of the space as a string or symbol.

\item \code{predicates}\\
A set of predicates to check against.  \code{checks} points to an array of
length \code{checks\_sz}.

\end{itemize}

\paragraph{Returns:}
This asynchronous operation returns an \code{Iterator} object.  The
\code{Iterator} will return the resulting objects as they become available.

Errors will be returned from the Iterator, as it is possible to retreive partial
results in the face of an error.


%%%%%%%%%%%%%%%%%%%% search_describe %%%%%%%%%%%%%%%%%%%%
\pagebreak
\subsubsection{\code{search\_describe}}
\label{api:nodejs:search_describe}
\index{search\_describe!Node.js API}
Return a human-readable string suitable for debugging search internals.  This
API is only really relevant for debugging the internals of \code{search}.


\paragraph{Definition:}
\begin{javascriptcode}
search_describe(spacename, predicates, function (desc, err) {})
\end{javascriptcode}
\paragraph{Parameters:}
\begin{itemize}[noitemsep]
\item \code{spacename}\\
The name of the space as a string or symbol.

\item \code{predicates}\\
A set of predicates to check against.  \code{checks} points to an array of
length \code{checks\_sz}.

\end{itemize}

\paragraph{Returns:}
This function returns a string describing the internals of the search.

On error, this function will raise a \code{HyperDexClientException} describing
the error.


%%%%%%%%%%%%%%%%%%%% sorted_search %%%%%%%%%%%%%%%%%%%%
\pagebreak
\subsubsection{\code{sorted\_search}}
\label{api:nodejs:sorted_search}
\index{sorted\_search!Node.js API}
Return all objects that match the specified \code{checks}, sorted according to
\code{attr}.

\paragraph{Behavior:}
\begin{itemize}[noitemsep]
\item This operation behaves as an iterator.  Multiple objects may be returned
    from the single call.

\item This operation return to the user the requested object(s).

\end{itemize}


\paragraph{Definition:}
\begin{javascriptcode}
sorted_search(spacename, predicates, sortby, limit, maxmin, function (obj, err) {})
\end{javascriptcode}
\paragraph{Parameters:}
\begin{itemize}[noitemsep]
\item \code{spacename}\\
The name of the space as a string or symbol.

\item \code{predicates}\\
A set of predicates to check against.  \code{checks} points to an array of
length \code{checks\_sz}.

\item \code{sortby}\\
The attribute to sort by.

\item \code{limit}\\
The number of results to return.

\item \code{maxmin}\\
Maximize (!= 0) or minimize (== 0).

\end{itemize}

\paragraph{Returns:}
This asynchronous operation returns an \code{Iterator} object.  The
\code{Iterator} will return the resulting objects as they become available.

Errors will be returned from the Iterator, as it is possible to retreive partial
results in the face of an error.


%%%%%%%%%%%%%%%%%%%% group_del %%%%%%%%%%%%%%%%%%%%
\pagebreak
\subsubsection{\code{group\_del}}
\label{api:nodejs:group_del}
\index{group\_del!Node.js API}
Asynchronously delete all objects that match the specified \code{checks}.

\paragraph{Behavior:}
\begin{itemize}[noitemsep]
\item This operation is roughly equivalent to a client manually deleting every
    object returned from a search, but saves HyperDex from sending to the client
    objects that are soon to be deleted.
\end{itemize}


\paragraph{Definition:}
\begin{javascriptcode}
group_del(spacename, predicates, function (success, err) {})
\end{javascriptcode}
\paragraph{Parameters:}
\begin{itemize}[noitemsep]
\item \code{spacename}\\
The name of the space as a string or symbol.

\item \code{predicates}\\
A set of predicates to check against.  \code{checks} points to an array of
length \code{checks\_sz}.

\end{itemize}

\paragraph{Returns:}
True if the operation succeeded.  False if any provided predicates failed.
Raises an exception on error.


%%%%%%%%%%%%%%%%%%%% count %%%%%%%%%%%%%%%%%%%%
\pagebreak
\subsubsection{\code{count}}
\label{api:nodejs:count}
\index{count!Node.js API}
Count the number of objects that match the specified \code{checks}.

\paragraph{Behavior:}
\begin{itemize}[noitemsep]
\item This will return the number of objects counted by the search.  If an error
    occurs during the count, the count will reflect a partial count.  The real
    count will be higher than the returned value.  Some languages will throw an
    exception rather than return the partial count.
\end{itemize}


\paragraph{Definition:}
\begin{javascriptcode}
count(spacename, predicates, function (count, err) {})
\end{javascriptcode}
\paragraph{Parameters:}
\begin{itemize}[noitemsep]
\item \code{spacename}\\
The name of the space as a string or symbol.

\item \code{predicates}\\
A set of predicates to check against.  \code{checks} points to an array of
length \code{checks\_sz}.

\end{itemize}

\paragraph{Returns:}
This function returns a number indicating the number of objects counted.

On error, this function will raise a \code{HyperDexClientException} describing
the error.


\pagebreak

\subsection{Working with Signals}
\label{sec:api:node:signals}

The HyperDex client module is signal-safe.  Should a signal interrupt the
client, it will raise an exception with status
\code{HYPERDEX\_CLIENT\_INTERRUPTED}.

\subsection{Working with Events}
\label{sec:api:node:threads}

The Node module naturally integrates with the Node.js event loop.  Each instance
of \code{Client} registers itself with the Node event loop and makes callbacks
as soon as events complete on the HyperDex side.

Put simply, a Node.js application can use \code{Client} instances in a
straight-forward fashion without worrying about threading or manual integration.
