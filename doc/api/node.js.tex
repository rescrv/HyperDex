\chapter{Node API}
\label{chap:api:node}

\section{Client Library}
\label{sec:api:node:client}

HyperDex provides Node bindings under the module \code{HyperDex::Client}.  This
library wraps the HyperDex C Client library and enables use of native Javascript
data types.

This module was re-introduced in HyperDex 1.2.0.

\subsection{Building the HyperDex Node.js Binding}
\label{sec:api:node:building}

The HyperDex Node.js Binding must be built and installed after HyperDex is
built.  After installing HyperDex, you can build the Node.js bindings from
either source or git checkout with:

\begin{consolecode}
% cd bindings/node.js
% node rebuild
\end{consolecode}

\subsection{Using Node.js Within Your Application}
\label{sec:api:node:using}

All client operation are defined in the \code{hyperdex\_client} module.  You can
access this in your program with:

\begin{javascriptcode}
var hyperdex_client = require('hyperdex-client');
\end{javascriptcode}

\subsection{Hello World}
\label{sec:api:node:hello-world}

The following is a minimal application that stores the value "Hello World" and
then immediately retrieves the value:

\inputminted{javascript}{\topdir/api/node.js/hello-world.js}

You can run this example with:

\begin{consolecode}
% node hello-world.js
put: true
get: [object Object]
\end{consolecode}

Right away, there are several points worth noting in this example:

\begin{itemize}
\item Each operation takes a callback.  While the operation is outstanding, your
program is free to execute other code.

\item Javascript types are automatically converted to HyperDex types.  There's
no need to specify information such as the length of each string, as one would
do with the C API.

\item There's no need to manually enter the HyperDex event loop.  HyperDex will
add and remove itself from the event loop as operations start and finish.
\end{itemize}

\subsection{Asynchronous Operations}
\label{sec:api:node:async-ops}

HyperDex provides native integration with the asynchronous world of Node.js.
You can issue several operations concurrently, and Node.js and HyperDex will
work together to complete these operations quickly and efficiently.  It's easy
to work with data concurrently.  A common pattern is to keep a constant number
of operations outstanding concurrently:

\inputminted{javascript}{\topdir/api/node.js/window-pattern.js}

\subsection{Data Structures}
\label{sec:api:node:data-structures}

The Node bindings automatically manage conversion of data types from Javascript
to HyperDex types, enabling applications to be written in idiomatic Javascript.

\subsubsection{Examples}
\label{sec:api:node:examples}

This section shows examples of Java data structures that are recognized by
HyperDex.  The examples here are for illustration purposes and are not
exhaustive.

\paragraph{Strings}

The HyperDex client recognizes Nodes's strings and buffers and automatically
converts them to HyperDex strings.  For example, the following two calls
have the same effect:

\begin{javascriptcode}
c.put("kv", "somekey", {v: "somevalue"}, function (success, err) {});
c.put("kv", "somekey", {v: new Buffer("somevalue")}, function (success, err) {});
\end{javascriptcode}

\paragraph{Other Types}

At the current point in time, other data types are not supported within Node.js.
This is primarily because Javascript doesn't support the majority of HyperDex
types without third party libraries.  We are looking for feedback from the
Node.js communtiy about the best way to support all HyperDex types.

\subsection{Attributes}
\label{sec:api:node:attributes}

Attributes in Node are specified in the form of a Javascript object.  As you can
see in the examples above, attributes are specified in the form:

\begin{javascriptcode}
{name: "value"}
\end{javascriptcode}

\subsection{Operations}
\label{sec:api:node:ops}

% Copyright (c) 2014, Cornell University
% All rights reserved.
%
% Redistribution and use in source and binary forms, with or without
% modification, are permitted provided that the following conditions are met:
%
%     * Redistributions of source code must retain the above copyright notice,
%       this list of conditions and the following disclaimer.
%     * Redistributions in binary form must reproduce the above copyright
%       notice, this list of conditions and the following disclaimer in the
%       documentation and/or other materials provided with the distribution.
%     * Neither the name of HyperDex nor the names of its contributors may be
%       used to endorse or promote products derived from this software without
%       specific prior written permission.
%
% THIS SOFTWARE IS PROVIDED BY THE COPYRIGHT HOLDERS AND CONTRIBUTORS "AS IS"
% AND ANY EXPRESS OR IMPLIED WARRANTIES, INCLUDING, BUT NOT LIMITED TO, THE
% IMPLIED WARRANTIES OF MERCHANTABILITY AND FITNESS FOR A PARTICULAR PURPOSE ARE
% DISCLAIMED. IN NO EVENT SHALL THE COPYRIGHT OWNER OR CONTRIBUTORS BE LIABLE
% FOR ANY DIRECT, INDIRECT, INCIDENTAL, SPECIAL, EXEMPLARY, OR CONSEQUENTIAL
% DAMAGES (INCLUDING, BUT NOT LIMITED TO, PROCUREMENT OF SUBSTITUTE GOODS OR
% SERVICES; LOSS OF USE, DATA, OR PROFITS; OR BUSINESS INTERRUPTION) HOWEVER
% CAUSED AND ON ANY THEORY OF LIABILITY, WHETHER IN CONTRACT, STRICT LIABILITY,
% OR TORT (INCLUDING NEGLIGENCE OR OTHERWISE) ARISING IN ANY WAY OUT OF THE USE
% OF THIS SOFTWARE, EVEN IF ADVISED OF THE POSSIBILITY OF SUCH DAMAGE.

% This LaTeX file is generated by bindings/nodejs.py

\subsubsection{\code{get}}
\label{api:nodejs:get}
\index{get!Node.js API}
\begin{javascriptcode}
Client :: get(spacename, key)
\end{javascriptcode}
Retreive the object with key "key" from space "space".

\noindent\textbf{Cost:}  Approximately one network round trip.


\input{api/shards/linearizable}


\noindent\textbf{Parameters:}
\begin{description}[labelindent=\widthof{{\code{spacename}}},leftmargin=*,noitemsep,nolistsep,align=right]
\item[\code{spacename}] The name of the space as a string or buffer.
\item[\code{key}] The key for the operation as a Node type
\end{description}

\noindent\textbf{Returns:}
Object if found, nil if not found.  Raises exception on error.

\subsubsection{\code{put}}
\label{api:nodejs:put}
\index{put!Node.js API}
\begin{javascriptcode}
Client :: put(spacename, key, attributes)
\end{javascriptcode}
Store or update an object by key.  The object's attributes will be set to the
values specified by \code{attrs}.
If the object exists, it will be updated and all existing values not altered by
\code{attrs} will be preserved.  If the object does not exist, a new object will
be created, with its attributes initialized to their default values.



\noindent\textbf{Parameters:}
\begin{description}[labelindent=\widthof{{\code{attributes}}},leftmargin=*,noitemsep,nolistsep,align=right]
\item[\code{spacename}] The name of the space as a string or buffer.
\item[\code{key}] The key for the operation as a Node type
\item[\code{attributes}] An object specifying attributes to modify and their respective values.
\end{description}

\noindent\textbf{Returns:}
True.  Raises exception on error.

\subsubsection{\code{cond\_put}}
\label{api:nodejs:cond_put}
\index{cond\_put!Node.js API}
\begin{javascriptcode}
Client :: cond_put(spacename, key, predicates, attributes)
\end{javascriptcode}
Conditionally update an the object stored under \code{key} in \code{space}.
Existing values will be overwritten with the values specified by \code{attrs}.
Values not specified by \code{attrs} will remain unchanged.
This operation requires a pre-existing object in order to complete successfully.
If no object exists, the operation will fail with \code{NOTFOUND}.


This operation will succeed if and only if the predicates specified by
\code{checks} hold on the pre-existing object.  If any of the predicates are not
true for the existing object, then the operation will have no effect and fail
with \code{CMPFAIL}.

All checks are atomic with the write.  HyperDex guarantees that no other
operation will come between validating the checks, and writing the new version
of the object.



\noindent\textbf{Parameters:}
\begin{description}[labelindent=\widthof{{\code{predicates}}},leftmargin=*,noitemsep,nolistsep,align=right]
\item[\code{spacename}] The name of the space as a string or buffer.
\item[\code{key}] The key for the operation as a Node type
\item[\code{predicates}] An object of predicates to check against.
\item[\code{attributes}] An object specifying attributes to modify and their respective values.
\end{description}

\noindent\textbf{Returns:}
True if predicate, False if not predicate.  Raises exception on error.

\subsubsection{\code{put\_if\_not\_exist}}
\label{api:nodejs:put_if_not_exist}
\index{put\_if\_not\_exist!Node.js API}
\begin{javascriptcode}
Client :: put_if_not_exist(spacename, key, attributes)
\end{javascriptcode}
Store an object in space "space" under key "key" if and only if it does not
already exist.

The object will be create, and the attributes specified by \texttt{attrs} will
be set to their respective values.  Any attributes not specified by
\texttt{attrs} will be initialized to their default values.  The check is atomic
with the write, and is guaranteed to never overwrite an existing object.

\noindent\textbf{Cost:}  Approximately one traversal of the value-dependent
chain.


\input{api/shards/linearizable}


\noindent\textbf{Parameters:}
\begin{description}[labelindent=\widthof{{\code{attributes}}},leftmargin=*,noitemsep,nolistsep,align=right]
\item[\code{spacename}] The name of the space as a string or buffer.
\item[\code{key}] The key for the operation as a Node type
\item[\code{attributes}] An object specifying attributes to modify and their respective values.
\end{description}

\noindent\textbf{Returns:}
True.  Raises exception on error.

\subsubsection{\code{del}}
\label{api:nodejs:del}
\index{del!Node.js API}
\begin{javascriptcode}
Client :: del(spacename, key)
\end{javascriptcode}
Delete an object by key.

%%% Generated below here
\paragraph{Behavior:}
\begin{itemize}[noitemsep]
If no object exists, the operation will fail with \code{NOTFOUND}.

\end{itemize}


\noindent\textbf{Parameters:}
\begin{description}[labelindent=\widthof{{\code{spacename}}},leftmargin=*,noitemsep,nolistsep,align=right]
\item[\code{spacename}] The name of the space as a string or buffer.
\item[\code{key}] The key for the operation as a Node type
\end{description}

\noindent\textbf{Returns:}
True.  Raises exception on error.

\subsubsection{\code{cond\_del}}
\label{api:nodejs:cond_del}
\index{cond\_del!Node.js API}
\begin{javascriptcode}
Client :: cond_del(spacename, key, predicates)
\end{javascriptcode}
Conditionally delete an object by key.

%%% Generated below here
\paragraph{Behavior:}
\begin{itemize}[noitemsep]
If no object exists, the operation will fail with \code{NOTFOUND}.

This operation will succeed if and only if the predicates specified by
\code{checks} hold on the pre-existing object.  If any of the predicates are not
true for the existing object, then the operation will have no effect and fail
with \code{CMPFAIL}.

All checks are atomic with the write.  HyperDex guarantees that no other
operation will come between validating the checks, and writing the new version
of the object.

\end{itemize}


\noindent\textbf{Parameters:}
\begin{description}[labelindent=\widthof{{\code{predicates}}},leftmargin=*,noitemsep,nolistsep,align=right]
\item[\code{spacename}] The name of the space as a string or buffer.
\item[\code{key}] The key for the operation as a Node type
\item[\code{predicates}] An object of predicates to check against.
\end{description}

\noindent\textbf{Returns:}
True if predicate, False if not predicate.  Raises exception on error.

\subsubsection{\code{atomic\_add}}
\label{api:nodejs:atomic_add}
\index{atomic\_add!Node.js API}
\begin{javascriptcode}
Client :: atomic_add(spacename, key, attributes)
\end{javascriptcode}
Add the specified number to the existing value for each attribute.
This operation requires a pre-existing object in order to complete successfully.
If no object exists, the operation will fail with \code{NOTFOUND}.



\noindent\textbf{Parameters:}
\begin{description}[labelindent=\widthof{{\code{attributes}}},leftmargin=*,noitemsep,nolistsep,align=right]
\item[\code{spacename}] The name of the space as a string or buffer.
\item[\code{key}] The key for the operation as a Node type
\item[\code{attributes}] An object specifying attributes to modify and their respective values.
\end{description}

\noindent\textbf{Returns:}
True.  Raises exception on error.

\subsubsection{\code{cond\_atomic\_add}}
\label{api:nodejs:cond_atomic_add}
\index{cond\_atomic\_add!Node.js API}
\begin{javascriptcode}
Client :: cond_atomic_add(spacename, key, predicates, attributes)
\end{javascriptcode}
Conditionally add the specified number to the existing value for each attribute.

%%% Generated below here
\paragraph{Behavior:}
\begin{itemize}[noitemsep]
This operation requires a pre-existing object in order to complete successfully.
If no object exists, the operation will fail with \code{NOTFOUND}.

This operation will succeed if and only if the predicates specified by
\code{checks} hold on the pre-existing object.  If any of the predicates are not
true for the existing object, then the operation will have no effect and fail
with \code{CMPFAIL}.

All checks are atomic with the write.  HyperDex guarantees that no other
operation will come between validating the checks, and writing the new version
of the object.

\end{itemize}


\noindent\textbf{Parameters:}
\begin{description}[labelindent=\widthof{{\code{predicates}}},leftmargin=*,noitemsep,nolistsep,align=right]
\item[\code{spacename}] The name of the space as a string or buffer.
\item[\code{key}] The key for the operation as a Node type
\item[\code{predicates}] An object of predicates to check against.
\item[\code{attributes}] An object specifying attributes to modify and their respective values.
\end{description}

\noindent\textbf{Returns:}
True if predicate, False if not predicate.  Raises exception on error.

\subsubsection{\code{atomic\_sub}}
\label{api:nodejs:atomic_sub}
\index{atomic\_sub!Node.js API}
\begin{javascriptcode}
Client :: atomic_sub(spacename, key, attributes)
\end{javascriptcode}
Subtract the specified number from the existing value for each attribute.

%%% Generated below here
\paragraph{Behavior:}
\begin{itemize}[noitemsep]
This operation requires a pre-existing object in order to complete successfully.
If no object exists, the operation will fail with \code{NOTFOUND}.

\end{itemize}


\noindent\textbf{Parameters:}
\begin{description}[labelindent=\widthof{{\code{attributes}}},leftmargin=*,noitemsep,nolistsep,align=right]
\item[\code{spacename}] The name of the space as a string or buffer.
\item[\code{key}] The key for the operation as a Node type
\item[\code{attributes}] An object specifying attributes to modify and their respective values.
\end{description}

\noindent\textbf{Returns:}
True.  Raises exception on error.

\subsubsection{\code{cond\_atomic\_sub}}
\label{api:nodejs:cond_atomic_sub}
\index{cond\_atomic\_sub!Node.js API}
\begin{javascriptcode}
Client :: cond_atomic_sub(spacename, key, predicates, attributes)
\end{javascriptcode}
Conditionally subtract the specified number from the existing value for each attribute.

%%% Generated below here
\paragraph{Behavior:}
\begin{itemize}[noitemsep]
This operation requires a pre-existing object in order to complete successfully.
If no object exists, the operation will fail with \code{NOTFOUND}.

This operation will succeed if and only if the predicates specified by
\code{checks} hold on the pre-existing object.  If any of the predicates are not
true for the existing object, then the operation will have no effect and fail
with \code{CMPFAIL}.

All checks are atomic with the write.  HyperDex guarantees that no other
operation will come between validating the checks, and writing the new version
of the object.

\end{itemize}


\noindent\textbf{Parameters:}
\begin{description}[labelindent=\widthof{{\code{predicates}}},leftmargin=*,noitemsep,nolistsep,align=right]
\item[\code{spacename}] The name of the space as a string or buffer.
\item[\code{key}] The key for the operation as a Node type
\item[\code{predicates}] An object of predicates to check against.
\item[\code{attributes}] An object specifying attributes to modify and their respective values.
\end{description}

\noindent\textbf{Returns:}
True if predicate, False if not predicate.  Raises exception on error.

\subsubsection{\code{atomic\_mul}}
\label{api:nodejs:atomic_mul}
\index{atomic\_mul!Node.js API}
\begin{javascriptcode}
Client :: atomic_mul(spacename, key, attributes)
\end{javascriptcode}
\input{\topdir/api/desc/atomic_mul}

\noindent\textbf{Parameters:}
\begin{description}[labelindent=\widthof{{\code{attributes}}},leftmargin=*,noitemsep,nolistsep,align=right]
\item[\code{spacename}] The name of the space as a string or buffer.
\item[\code{key}] The key for the operation as a Node type
\item[\code{attributes}] An object specifying attributes to modify and their respective values.
\end{description}

\noindent\textbf{Returns:}
True.  Raises exception on error.

\subsubsection{\code{cond\_atomic\_mul}}
\label{api:nodejs:cond_atomic_mul}
\index{cond\_atomic\_mul!Node.js API}
\begin{javascriptcode}
Client :: cond_atomic_mul(spacename, key, predicates, attributes)
\end{javascriptcode}
Multiply the existing value by the specified number for each attribute if and
only if the \code{checks} hold on the object.
This operation requires a pre-existing object in order to complete successfully.
If no object exists, the operation will fail with \code{NOTFOUND}.


This operation will succeed if and only if the predicates specified by
\code{checks} hold on the pre-existing object.  If any of the predicates are not
true for the existing object, then the operation will have no effect and fail
with \code{CMPFAIL}.

All checks are atomic with the write.  HyperDex guarantees that no other
operation will come between validating the checks, and writing the new version
of the object.



\noindent\textbf{Parameters:}
\begin{description}[labelindent=\widthof{{\code{predicates}}},leftmargin=*,noitemsep,nolistsep,align=right]
\item[\code{spacename}] The name of the space as a string or buffer.
\item[\code{key}] The key for the operation as a Node type
\item[\code{predicates}] An object of predicates to check against.
\item[\code{attributes}] An object specifying attributes to modify and their respective values.
\end{description}

\noindent\textbf{Returns:}
True if predicate, False if not predicate.  Raises exception on error.

\subsubsection{\code{atomic\_div}}
\label{api:nodejs:atomic_div}
\index{atomic\_div!Node.js API}
\begin{javascriptcode}
Client :: atomic_div(spacename, key, attributes)
\end{javascriptcode}
Divide the existing value by the number specified for each attribute.

The division is atomic with the write.  If the object does not exist, the
operation will fail.

\noindent\textbf{Cost:}  Approximately one traversal of the value-dependent
chain.


\input{\topdir/api/shards/linearizable}


\noindent\textbf{Parameters:}
\begin{description}[labelindent=\widthof{{\code{attributes}}},leftmargin=*,noitemsep,nolistsep,align=right]
\item[\code{spacename}] The name of the space as a string or buffer.
\item[\code{key}] The key for the operation as a Node type
\item[\code{attributes}] An object specifying attributes to modify and their respective values.
\end{description}

\noindent\textbf{Returns:}
True.  Raises exception on error.

\subsubsection{\code{cond\_atomic\_div}}
\label{api:nodejs:cond_atomic_div}
\index{cond\_atomic\_div!Node.js API}
\begin{javascriptcode}
Client :: cond_atomic_div(spacename, key, predicates, attributes)
\end{javascriptcode}
\input{\topdir/api/desc/cond_atomic_div}

\noindent\textbf{Parameters:}
\begin{description}[labelindent=\widthof{{\code{predicates}}},leftmargin=*,noitemsep,nolistsep,align=right]
\item[\code{spacename}] The name of the space as a string or buffer.
\item[\code{key}] The key for the operation as a Node type
\item[\code{predicates}] An object of predicates to check against.
\item[\code{attributes}] An object specifying attributes to modify and their respective values.
\end{description}

\noindent\textbf{Returns:}
True if predicate, False if not predicate.  Raises exception on error.

\subsubsection{\code{atomic\_mod}}
\label{api:nodejs:atomic_mod}
\index{atomic\_mod!Node.js API}
\begin{javascriptcode}
Client :: atomic_mod(spacename, key, attributes)
\end{javascriptcode}
Store the existing value modulo the specified number for each attribute.
This operation requires a pre-existing object in order to complete successfully.
If no object exists, the operation will fail with \code{NOTFOUND}.



\noindent\textbf{Parameters:}
\begin{description}[labelindent=\widthof{{\code{attributes}}},leftmargin=*,noitemsep,nolistsep,align=right]
\item[\code{spacename}] The name of the space as a string or buffer.
\item[\code{key}] The key for the operation as a Node type
\item[\code{attributes}] An object specifying attributes to modify and their respective values.
\end{description}

\noindent\textbf{Returns:}
True.  Raises exception on error.

\subsubsection{\code{cond\_atomic\_mod}}
\label{api:nodejs:cond_atomic_mod}
\index{cond\_atomic\_mod!Node.js API}
\begin{javascriptcode}
Client :: cond_atomic_mod(spacename, key, predicates, attributes)
\end{javascriptcode}
Conditionally store the existing value modulo the specified number for each
attribute.

%%% Generated below here
\paragraph{Behavior:}
\begin{itemize}[noitemsep]
This operation requires a pre-existing object in order to complete successfully.
If no object exists, the operation will fail with \code{NOTFOUND}.

This operation will succeed if and only if the predicates specified by
\code{checks} hold on the pre-existing object.  If any of the predicates are not
true for the existing object, then the operation will have no effect and fail
with \code{CMPFAIL}.

All checks are atomic with the write.  HyperDex guarantees that no other
operation will come between validating the checks, and writing the new version
of the object.

\end{itemize}


\noindent\textbf{Parameters:}
\begin{description}[labelindent=\widthof{{\code{predicates}}},leftmargin=*,noitemsep,nolistsep,align=right]
\item[\code{spacename}] The name of the space as a string or buffer.
\item[\code{key}] The key for the operation as a Node type
\item[\code{predicates}] An object of predicates to check against.
\item[\code{attributes}] An object specifying attributes to modify and their respective values.
\end{description}

\noindent\textbf{Returns:}
True if predicate, False if not predicate.  Raises exception on error.

\subsubsection{\code{atomic\_and}}
\label{api:nodejs:atomic_and}
\index{atomic\_and!Node.js API}
\begin{javascriptcode}
Client :: atomic_and(spacename, key, attributes)
\end{javascriptcode}
Store the bitwise AND of the existing value and the specified number for
each attribute.
This operation requires a pre-existing object in order to complete successfully.
If no object exists, the operation will fail with \code{NOTFOUND}.



\noindent\textbf{Parameters:}
\begin{description}[labelindent=\widthof{{\code{attributes}}},leftmargin=*,noitemsep,nolistsep,align=right]
\item[\code{spacename}] The name of the space as a string or buffer.
\item[\code{key}] The key for the operation as a Node type
\item[\code{attributes}] An object specifying attributes to modify and their respective values.
\end{description}

\noindent\textbf{Returns:}
True.  Raises exception on error.

\subsubsection{\code{cond\_atomic\_and}}
\label{api:nodejs:cond_atomic_and}
\index{cond\_atomic\_and!Node.js API}
\begin{javascriptcode}
Client :: cond_atomic_and(spacename, key, predicates, attributes)
\end{javascriptcode}
Conditionally store the bitwise AND of the existing value and the specified
number for each attribute.

%%% Generated below here
\paragraph{Behavior:}
\begin{itemize}[noitemsep]
This operation requires a pre-existing object in order to complete successfully.
If no object exists, the operation will fail with \code{NOTFOUND}.

This operation will succeed if and only if the predicates specified by
\code{checks} hold on the pre-existing object.  If any of the predicates are not
true for the existing object, then the operation will have no effect and fail
with \code{CMPFAIL}.

All checks are atomic with the write.  HyperDex guarantees that no other
operation will come between validating the checks, and writing the new version
of the object.

\end{itemize}


\noindent\textbf{Parameters:}
\begin{description}[labelindent=\widthof{{\code{predicates}}},leftmargin=*,noitemsep,nolistsep,align=right]
\item[\code{spacename}] The name of the space as a string or buffer.
\item[\code{key}] The key for the operation as a Node type
\item[\code{predicates}] An object of predicates to check against.
\item[\code{attributes}] An object specifying attributes to modify and their respective values.
\end{description}

\noindent\textbf{Returns:}
True if predicate, False if not predicate.  Raises exception on error.

\subsubsection{\code{atomic\_or}}
\label{api:nodejs:atomic_or}
\index{atomic\_or!Node.js API}
\begin{javascriptcode}
Client :: atomic_or(spacename, key, attributes)
\end{javascriptcode}
Store the bitwise OR of the existing value and the specified number for each
attribute.

%%% Generated below here
\paragraph{Behavior:}
\begin{itemize}[noitemsep]
This operation requires a pre-existing object in order to complete successfully.
If no object exists, the operation will fail with \code{NOTFOUND}.

\end{itemize}


\noindent\textbf{Parameters:}
\begin{description}[labelindent=\widthof{{\code{attributes}}},leftmargin=*,noitemsep,nolistsep,align=right]
\item[\code{spacename}] The name of the space as a string or buffer.
\item[\code{key}] The key for the operation as a Node type
\item[\code{attributes}] An object specifying attributes to modify and their respective values.
\end{description}

\noindent\textbf{Returns:}
True.  Raises exception on error.

\subsubsection{\code{cond\_atomic\_or}}
\label{api:nodejs:cond_atomic_or}
\index{cond\_atomic\_or!Node.js API}
\begin{javascriptcode}
Client :: cond_atomic_or(spacename, key, predicates, attributes)
\end{javascriptcode}
\input{\topdir/api/desc/cond_atomic_or}

\noindent\textbf{Parameters:}
\begin{description}[labelindent=\widthof{{\code{predicates}}},leftmargin=*,noitemsep,nolistsep,align=right]
\item[\code{spacename}] The name of the space as a string or buffer.
\item[\code{key}] The key for the operation as a Node type
\item[\code{predicates}] An object of predicates to check against.
\item[\code{attributes}] An object specifying attributes to modify and their respective values.
\end{description}

\noindent\textbf{Returns:}
True if predicate, False if not predicate.  Raises exception on error.

\subsubsection{\code{atomic\_xor}}
\label{api:nodejs:atomic_xor}
\index{atomic\_xor!Node.js API}
\begin{javascriptcode}
Client :: atomic_xor(spacename, key, attributes)
\end{javascriptcode}
Store the bitwise XOR of the existing value and the specified number for each
attribute.
This operation requires a pre-existing object in order to complete successfully.
If no object exists, the operation will fail with \code{NOTFOUND}.



\noindent\textbf{Parameters:}
\begin{description}[labelindent=\widthof{{\code{attributes}}},leftmargin=*,noitemsep,nolistsep,align=right]
\item[\code{spacename}] The name of the space as a string or buffer.
\item[\code{key}] The key for the operation as a Node type
\item[\code{attributes}] An object specifying attributes to modify and their respective values.
\end{description}

\noindent\textbf{Returns:}
True.  Raises exception on error.

\subsubsection{\code{cond\_atomic\_xor}}
\label{api:nodejs:cond_atomic_xor}
\index{cond\_atomic\_xor!Node.js API}
\begin{javascriptcode}
Client :: cond_atomic_xor(spacename, key, predicates, attributes)
\end{javascriptcode}
Conditionally store the bitwise XOR of the existing value and the specified
number for each attribute.

%%% Generated below here
\paragraph{Behavior:}
\begin{itemize}[noitemsep]
This operation requires a pre-existing object in order to complete successfully.
If no object exists, the operation will fail with \code{NOTFOUND}.

This operation will succeed if and only if the predicates specified by
\code{checks} hold on the pre-existing object.  If any of the predicates are not
true for the existing object, then the operation will have no effect and fail
with \code{CMPFAIL}.

All checks are atomic with the write.  HyperDex guarantees that no other
operation will come between validating the checks, and writing the new version
of the object.

\end{itemize}


\noindent\textbf{Parameters:}
\begin{description}[labelindent=\widthof{{\code{predicates}}},leftmargin=*,noitemsep,nolistsep,align=right]
\item[\code{spacename}] The name of the space as a string or buffer.
\item[\code{key}] The key for the operation as a Node type
\item[\code{predicates}] An object of predicates to check against.
\item[\code{attributes}] An object specifying attributes to modify and their respective values.
\end{description}

\noindent\textbf{Returns:}
True if predicate, False if not predicate.  Raises exception on error.

\subsubsection{\code{string\_prepend}}
\label{api:nodejs:string_prepend}
\index{string\_prepend!Node.js API}
\begin{javascriptcode}
Client :: string_prepend(spacename, key, attributes)
\end{javascriptcode}
Prepend the specified string to the existing value for each attribute.

%%% Generated below here
\paragraph{Behavior:}
\begin{itemize}[noitemsep]
This operation requires a pre-existing object in order to complete successfully.
If no object exists, the operation will fail with \code{NOTFOUND}.

\end{itemize}


\noindent\textbf{Parameters:}
\begin{description}[labelindent=\widthof{{\code{attributes}}},leftmargin=*,noitemsep,nolistsep,align=right]
\item[\code{spacename}] The name of the space as a string or buffer.
\item[\code{key}] The key for the operation as a Node type
\item[\code{attributes}] An object specifying attributes to modify and their respective values.
\end{description}

\noindent\textbf{Returns:}
True.  Raises exception on error.

\subsubsection{\code{cond\_string\_prepend}}
\label{api:nodejs:cond_string_prepend}
\index{cond\_string\_prepend!Node.js API}
\begin{javascriptcode}
Client :: cond_string_prepend(spacename, key, predicates, attributes)
\end{javascriptcode}
Conditionally prepend the specified string to the existing value for each
attribute.

%%% Generated below here
\paragraph{Behavior:}
\begin{itemize}[noitemsep]
This operation requires a pre-existing object in order to complete successfully.
If no object exists, the operation will fail with \code{NOTFOUND}.

This operation will succeed if and only if the predicates specified by
\code{checks} hold on the pre-existing object.  If any of the predicates are not
true for the existing object, then the operation will have no effect and fail
with \code{CMPFAIL}.

All checks are atomic with the write.  HyperDex guarantees that no other
operation will come between validating the checks, and writing the new version
of the object.

\end{itemize}


\noindent\textbf{Parameters:}
\begin{description}[labelindent=\widthof{{\code{predicates}}},leftmargin=*,noitemsep,nolistsep,align=right]
\item[\code{spacename}] The name of the space as a string or buffer.
\item[\code{key}] The key for the operation as a Node type
\item[\code{predicates}] An object of predicates to check against.
\item[\code{attributes}] An object specifying attributes to modify and their respective values.
\end{description}

\noindent\textbf{Returns:}
True if predicate, False if not predicate.  Raises exception on error.

\subsubsection{\code{string\_append}}
\label{api:nodejs:string_append}
\index{string\_append!Node.js API}
\begin{javascriptcode}
Client :: string_append(spacename, key, attributes)
\end{javascriptcode}
Append the specified string to the existing value for each attribute.
This operation requires a pre-existing object in order to complete successfully.
If no object exists, the operation will fail with \code{NOTFOUND}.



\noindent\textbf{Parameters:}
\begin{description}[labelindent=\widthof{{\code{attributes}}},leftmargin=*,noitemsep,nolistsep,align=right]
\item[\code{spacename}] The name of the space as a string or buffer.
\item[\code{key}] The key for the operation as a Node type
\item[\code{attributes}] An object specifying attributes to modify and their respective values.
\end{description}

\noindent\textbf{Returns:}
True.  Raises exception on error.

\subsubsection{\code{cond\_string\_append}}
\label{api:nodejs:cond_string_append}
\index{cond\_string\_append!Node.js API}
\begin{javascriptcode}
Client :: cond_string_append(spacename, key, predicates, attributes)
\end{javascriptcode}
Append the specified string to the existing value for each attribute if and only
if \code{checks} hold on the object.
This operation requires a pre-existing object in order to complete successfully.
If no object exists, the operation will fail with \code{NOTFOUND}.


This operation will succeed if and only if the predicates specified by
\code{checks} hold on the pre-existing object.  If any of the predicates are not
true for the existing object, then the operation will have no effect and fail
with \code{CMPFAIL}.

All checks are atomic with the write.  HyperDex guarantees that no other
operation will come between validating the checks, and writing the new version
of the object.



\noindent\textbf{Parameters:}
\begin{description}[labelindent=\widthof{{\code{predicates}}},leftmargin=*,noitemsep,nolistsep,align=right]
\item[\code{spacename}] The name of the space as a string or buffer.
\item[\code{key}] The key for the operation as a Node type
\item[\code{predicates}] An object of predicates to check against.
\item[\code{attributes}] An object specifying attributes to modify and their respective values.
\end{description}

\noindent\textbf{Returns:}
True if predicate, False if not predicate.  Raises exception on error.

\subsubsection{\code{list\_lpush}}
\label{api:nodejs:list_lpush}
\index{list\_lpush!Node.js API}
\begin{javascriptcode}
Client :: list_lpush(spacename, key, attributes)
\end{javascriptcode}
Push the specified value onto the front of the list for each attribute.

%%% Generated below here
\paragraph{Behavior:}
\begin{itemize}[noitemsep]
This operation requires a pre-existing object in order to complete successfully.
If no object exists, the operation will fail with \code{NOTFOUND}.

\end{itemize}


\noindent\textbf{Parameters:}
\begin{description}[labelindent=\widthof{{\code{attributes}}},leftmargin=*,noitemsep,nolistsep,align=right]
\item[\code{spacename}] The name of the space as a string or buffer.
\item[\code{key}] The key for the operation as a Node type
\item[\code{attributes}] An object specifying attributes to modify and their respective values.
\end{description}

\noindent\textbf{Returns:}
True.  Raises exception on error.

\subsubsection{\code{cond\_list\_lpush}}
\label{api:nodejs:cond_list_lpush}
\index{cond\_list\_lpush!Node.js API}
\begin{javascriptcode}
Client :: cond_list_lpush(spacename, key, predicates, attributes)
\end{javascriptcode}
Condtitionally push the specified value onto the front of the list for each
attribute.

%%% Generated below here
\paragraph{Behavior:}
\begin{itemize}[noitemsep]
This operation requires a pre-existing object in order to complete successfully.
If no object exists, the operation will fail with \code{NOTFOUND}.

This operation will succeed if and only if the predicates specified by
\code{checks} hold on the pre-existing object.  If any of the predicates are not
true for the existing object, then the operation will have no effect and fail
with \code{CMPFAIL}.

All checks are atomic with the write.  HyperDex guarantees that no other
operation will come between validating the checks, and writing the new version
of the object.

\end{itemize}


\noindent\textbf{Parameters:}
\begin{description}[labelindent=\widthof{{\code{predicates}}},leftmargin=*,noitemsep,nolistsep,align=right]
\item[\code{spacename}] The name of the space as a string or buffer.
\item[\code{key}] The key for the operation as a Node type
\item[\code{predicates}] An object of predicates to check against.
\item[\code{attributes}] An object specifying attributes to modify and their respective values.
\end{description}

\noindent\textbf{Returns:}
True if predicate, False if not predicate.  Raises exception on error.

\subsubsection{\code{list\_rpush}}
\label{api:nodejs:list_rpush}
\index{list\_rpush!Node.js API}
\begin{javascriptcode}
Client :: list_rpush(spacename, key, attributes)
\end{javascriptcode}
Push the specified value onto the back of the list for each attribute.

%%% Generated below here
\paragraph{Behavior:}
\begin{itemize}[noitemsep]
This operation requires a pre-existing object in order to complete successfully.
If no object exists, the operation will fail with \code{NOTFOUND}.

\end{itemize}


\noindent\textbf{Parameters:}
\begin{description}[labelindent=\widthof{{\code{attributes}}},leftmargin=*,noitemsep,nolistsep,align=right]
\item[\code{spacename}] The name of the space as a string or buffer.
\item[\code{key}] The key for the operation as a Node type
\item[\code{attributes}] An object specifying attributes to modify and their respective values.
\end{description}

\noindent\textbf{Returns:}
True.  Raises exception on error.

\subsubsection{\code{cond\_list\_rpush}}
\label{api:nodejs:cond_list_rpush}
\index{cond\_list\_rpush!Node.js API}
\begin{javascriptcode}
Client :: cond_list_rpush(spacename, key, predicates, attributes)
\end{javascriptcode}
Push the specified value onto the back of the list for each attribute if and
only if the \code{checks} hold on the object.
This operation requires a pre-existing object in order to complete successfully.
If no object exists, the operation will fail with \code{NOTFOUND}.


This operation will succeed if and only if the predicates specified by
\code{checks} hold on the pre-existing object.  If any of the predicates are not
true for the existing object, then the operation will have no effect and fail
with \code{CMPFAIL}.

All checks are atomic with the write.  HyperDex guarantees that no other
operation will come between validating the checks, and writing the new version
of the object.



\noindent\textbf{Parameters:}
\begin{description}[labelindent=\widthof{{\code{predicates}}},leftmargin=*,noitemsep,nolistsep,align=right]
\item[\code{spacename}] The name of the space as a string or buffer.
\item[\code{key}] The key for the operation as a Node type
\item[\code{predicates}] An object of predicates to check against.
\item[\code{attributes}] An object specifying attributes to modify and their respective values.
\end{description}

\noindent\textbf{Returns:}
True if predicate, False if not predicate.  Raises exception on error.

\subsubsection{\code{set\_add}}
\label{api:nodejs:set_add}
\index{set\_add!Node.js API}
\begin{javascriptcode}
Client :: set_add(spacename, key, attributes)
\end{javascriptcode}
Add the specified value to the set for each attribute.

%%% Generated below here
\paragraph{Behavior:}
\begin{itemize}[noitemsep]
This operation requires a pre-existing object in order to complete successfully.
If no object exists, the operation will fail with \code{NOTFOUND}.

\end{itemize}


\noindent\textbf{Parameters:}
\begin{description}[labelindent=\widthof{{\code{attributes}}},leftmargin=*,noitemsep,nolistsep,align=right]
\item[\code{spacename}] The name of the space as a string or buffer.
\item[\code{key}] The key for the operation as a Node type
\item[\code{attributes}] An object specifying attributes to modify and their respective values.
\end{description}

\noindent\textbf{Returns:}
True.  Raises exception on error.

\subsubsection{\code{cond\_set\_add}}
\label{api:nodejs:cond_set_add}
\index{cond\_set\_add!Node.js API}
\begin{javascriptcode}
Client :: cond_set_add(spacename, key, predicates, attributes)
\end{javascriptcode}
Conditionally add the specified value to the set for each attribute.

%%% Generated below here
\paragraph{Behavior:}
\begin{itemize}[noitemsep]
This operation requires a pre-existing object in order to complete successfully.
If no object exists, the operation will fail with \code{NOTFOUND}.

This operation will succeed if and only if the predicates specified by
\code{checks} hold on the pre-existing object.  If any of the predicates are not
true for the existing object, then the operation will have no effect and fail
with \code{CMPFAIL}.

All checks are atomic with the write.  HyperDex guarantees that no other
operation will come between validating the checks, and writing the new version
of the object.

\end{itemize}


\noindent\textbf{Parameters:}
\begin{description}[labelindent=\widthof{{\code{predicates}}},leftmargin=*,noitemsep,nolistsep,align=right]
\item[\code{spacename}] The name of the space as a string or buffer.
\item[\code{key}] The key for the operation as a Node type
\item[\code{predicates}] An object of predicates to check against.
\item[\code{attributes}] An object specifying attributes to modify and their respective values.
\end{description}

\noindent\textbf{Returns:}
True if predicate, False if not predicate.  Raises exception on error.

\subsubsection{\code{set\_remove}}
\label{api:nodejs:set_remove}
\index{set\_remove!Node.js API}
\begin{javascriptcode}
Client :: set_remove(spacename, key, attributes)
\end{javascriptcode}
Remove the specified value from the set.  If the value is not contained within
the set, this operation will do nothing.

%%% Generated below here
\paragraph{Behavior:}
\begin{itemize}[noitemsep]
This operation requires a pre-existing object in order to complete successfully.
If no object exists, the operation will fail with \code{NOTFOUND}.

\end{itemize}


\noindent\textbf{Parameters:}
\begin{description}[labelindent=\widthof{{\code{attributes}}},leftmargin=*,noitemsep,nolistsep,align=right]
\item[\code{spacename}] The name of the space as a string or buffer.
\item[\code{key}] The key for the operation as a Node type
\item[\code{attributes}] An object specifying attributes to modify and their respective values.
\end{description}

\noindent\textbf{Returns:}
True.  Raises exception on error.

\subsubsection{\code{cond\_set\_remove}}
\label{api:nodejs:cond_set_remove}
\index{cond\_set\_remove!Node.js API}
\begin{javascriptcode}
Client :: cond_set_remove(spacename, key, predicates, attributes)
\end{javascriptcode}
Conditionally remove the specified value from the set.  If the value is not
contained within the set, this operation will do nothing.

%%% Generated below here
\paragraph{Behavior:}
\begin{itemize}[noitemsep]
This operation requires a pre-existing object in order to complete successfully.
If no object exists, the operation will fail with \code{NOTFOUND}.

This operation will succeed if and only if the predicates specified by
\code{checks} hold on the pre-existing object.  If any of the predicates are not
true for the existing object, then the operation will have no effect and fail
with \code{CMPFAIL}.

All checks are atomic with the write.  HyperDex guarantees that no other
operation will come between validating the checks, and writing the new version
of the object.

\end{itemize}


\noindent\textbf{Parameters:}
\begin{description}[labelindent=\widthof{{\code{predicates}}},leftmargin=*,noitemsep,nolistsep,align=right]
\item[\code{spacename}] The name of the space as a string or buffer.
\item[\code{key}] The key for the operation as a Node type
\item[\code{predicates}] An object of predicates to check against.
\item[\code{attributes}] An object specifying attributes to modify and their respective values.
\end{description}

\noindent\textbf{Returns:}
True if predicate, False if not predicate.  Raises exception on error.

\subsubsection{\code{set\_intersect}}
\label{api:nodejs:set_intersect}
\index{set\_intersect!Node.js API}
\begin{javascriptcode}
Client :: set_intersect(spacename, key, attributes)
\end{javascriptcode}
Store the intersection of the specified set and the existing value for each
attribute.

%%% Generated below here
\paragraph{Behavior:}
\begin{itemize}[noitemsep]
This operation requires a pre-existing object in order to complete successfully.
If no object exists, the operation will fail with \code{NOTFOUND}.

\end{itemize}


\noindent\textbf{Parameters:}
\begin{description}[labelindent=\widthof{{\code{attributes}}},leftmargin=*,noitemsep,nolistsep,align=right]
\item[\code{spacename}] The name of the space as a string or buffer.
\item[\code{key}] The key for the operation as a Node type
\item[\code{attributes}] An object specifying attributes to modify and their respective values.
\end{description}

\noindent\textbf{Returns:}
True.  Raises exception on error.

\subsubsection{\code{cond\_set\_intersect}}
\label{api:nodejs:cond_set_intersect}
\index{cond\_set\_intersect!Node.js API}
\begin{javascriptcode}
Client :: cond_set_intersect(spacename, key, predicates, attributes)
\end{javascriptcode}
\input{\topdir/api/desc/cond_set_intersect}

\noindent\textbf{Parameters:}
\begin{description}[labelindent=\widthof{{\code{predicates}}},leftmargin=*,noitemsep,nolistsep,align=right]
\item[\code{spacename}] The name of the space as a string or buffer.
\item[\code{key}] The key for the operation as a Node type
\item[\code{predicates}] An object of predicates to check against.
\item[\code{attributes}] An object specifying attributes to modify and their respective values.
\end{description}

\noindent\textbf{Returns:}
True if predicate, False if not predicate.  Raises exception on error.

\subsubsection{\code{set\_union}}
\label{api:nodejs:set_union}
\index{set\_union!Node.js API}
\begin{javascriptcode}
Client :: set_union(spacename, key, attributes)
\end{javascriptcode}
Store the union of the specified set and the existing value for each attribute.

%%% Generated below here
\paragraph{Behavior:}
\begin{itemize}[noitemsep]
This operation requires a pre-existing object in order to complete successfully.
If no object exists, the operation will fail with \code{NOTFOUND}.

\end{itemize}


\noindent\textbf{Parameters:}
\begin{description}[labelindent=\widthof{{\code{attributes}}},leftmargin=*,noitemsep,nolistsep,align=right]
\item[\code{spacename}] The name of the space as a string or buffer.
\item[\code{key}] The key for the operation as a Node type
\item[\code{attributes}] An object specifying attributes to modify and their respective values.
\end{description}

\noindent\textbf{Returns:}
True.  Raises exception on error.

\subsubsection{\code{cond\_set\_union}}
\label{api:nodejs:cond_set_union}
\index{cond\_set\_union!Node.js API}
\begin{javascriptcode}
Client :: cond_set_union(spacename, key, predicates, attributes)
\end{javascriptcode}
Conditionally store the union of the specified set and the existing value for
each attribute.

%%% Generated below here
\paragraph{Behavior:}
\begin{itemize}[noitemsep]
This operation requires a pre-existing object in order to complete successfully.
If no object exists, the operation will fail with \code{NOTFOUND}.

This operation will succeed if and only if the predicates specified by
\code{checks} hold on the pre-existing object.  If any of the predicates are not
true for the existing object, then the operation will have no effect and fail
with \code{CMPFAIL}.

All checks are atomic with the write.  HyperDex guarantees that no other
operation will come between validating the checks, and writing the new version
of the object.

\end{itemize}


\noindent\textbf{Parameters:}
\begin{description}[labelindent=\widthof{{\code{predicates}}},leftmargin=*,noitemsep,nolistsep,align=right]
\item[\code{spacename}] The name of the space as a string or buffer.
\item[\code{key}] The key for the operation as a Node type
\item[\code{predicates}] An object of predicates to check against.
\item[\code{attributes}] An object specifying attributes to modify and their respective values.
\end{description}

\noindent\textbf{Returns:}
True if predicate, False if not predicate.  Raises exception on error.

\subsubsection{\code{map\_add}}
\label{api:nodejs:map_add}
\index{map\_add!Node.js API}
\begin{javascriptcode}
Client :: map_add(spacename, key, mapattributes)
\end{javascriptcode}
Insert a key-value pair into the map specified by each map-attribute.

%%% Generated below here
\paragraph{Behavior:}
\begin{itemize}[noitemsep]
This operation requires a pre-existing object in order to complete successfully.
If no object exists, the operation will fail with \code{NOTFOUND}.

\end{itemize}


\noindent\textbf{Parameters:}
\begin{description}[labelindent=\widthof{{\code{mapattributes}}},leftmargin=*,noitemsep,nolistsep,align=right]
\item[\code{spacename}] The name of the space as a string or buffer.
\item[\code{key}] The key for the operation as a Node type
\item[\code{mapattributes}] An object specifying map attributes to modify and their respective key/values.
\end{description}

\noindent\textbf{Returns:}
True.  Raises exception on error.

\subsubsection{\code{cond\_map\_add}}
\label{api:nodejs:cond_map_add}
\index{cond\_map\_add!Node.js API}
\begin{javascriptcode}
Client :: cond_map_add(spacename, key, predicates, mapattributes)
\end{javascriptcode}
Conditionally insert a key-value pair into the map specified by each
map-attribute.

%%% Generated below here
\paragraph{Behavior:}
\begin{itemize}[noitemsep]
This operation requires a pre-existing object in order to complete successfully.
If no object exists, the operation will fail with \code{NOTFOUND}.

This operation will succeed if and only if the predicates specified by
\code{checks} hold on the pre-existing object.  If any of the predicates are not
true for the existing object, then the operation will have no effect and fail
with \code{CMPFAIL}.

All checks are atomic with the write.  HyperDex guarantees that no other
operation will come between validating the checks, and writing the new version
of the object.

\end{itemize}


\noindent\textbf{Parameters:}
\begin{description}[labelindent=\widthof{{\code{mapattributes}}},leftmargin=*,noitemsep,nolistsep,align=right]
\item[\code{spacename}] The name of the space as a string or buffer.
\item[\code{key}] The key for the operation as a Node type
\item[\code{predicates}] An object of predicates to check against.
\item[\code{mapattributes}] An object specifying map attributes to modify and their respective key/values.
\end{description}

\noindent\textbf{Returns:}
True if predicate, False if not predicate.  Raises exception on error.

\subsubsection{\code{map\_remove}}
\label{api:nodejs:map_remove}
\index{map\_remove!Node.js API}
\begin{javascriptcode}
Client :: map_remove(spacename, key, attributes)
\end{javascriptcode}
Remove a key-value pair from the map specified by each attribute.  If there is
no pair with the specified key within the map, this operation will do nothing.
This operation requires a pre-existing object in order to complete successfully.
If no object exists, the operation will fail with \code{NOTFOUND}.



\noindent\textbf{Parameters:}
\begin{description}[labelindent=\widthof{{\code{attributes}}},leftmargin=*,noitemsep,nolistsep,align=right]
\item[\code{spacename}] The name of the space as a string or buffer.
\item[\code{key}] The key for the operation as a Node type
\item[\code{attributes}] An object specifying attributes to modify and their respective values.
\end{description}

\noindent\textbf{Returns:}
True.  Raises exception on error.

\subsubsection{\code{cond\_map\_remove}}
\label{api:nodejs:cond_map_remove}
\index{cond\_map\_remove!Node.js API}
\begin{javascriptcode}
Client :: cond_map_remove(spacename, key, predicates, attributes)
\end{javascriptcode}
Remove a key-value pair from the map specified by each attribute if and only if
\code{checks} hold on the object.  If there is no pair with the specified key
within the map, this operation will do nothing.
This operation requires a pre-existing object in order to complete successfully.
If no object exists, the operation will fail with \code{NOTFOUND}.


This operation will succeed if and only if the predicates specified by
\code{checks} hold on the pre-existing object.  If any of the predicates are not
true for the existing object, then the operation will have no effect and fail
with \code{CMPFAIL}.

All checks are atomic with the write.  HyperDex guarantees that no other
operation will come between validating the checks, and writing the new version
of the object.



\noindent\textbf{Parameters:}
\begin{description}[labelindent=\widthof{{\code{predicates}}},leftmargin=*,noitemsep,nolistsep,align=right]
\item[\code{spacename}] The name of the space as a string or buffer.
\item[\code{key}] The key for the operation as a Node type
\item[\code{predicates}] An object of predicates to check against.
\item[\code{attributes}] An object specifying attributes to modify and their respective values.
\end{description}

\noindent\textbf{Returns:}
True if predicate, False if not predicate.  Raises exception on error.

\subsubsection{\code{map\_atomic\_add}}
\label{api:nodejs:map_atomic_add}
\index{map\_atomic\_add!Node.js API}
\begin{javascriptcode}
Client :: map_atomic_add(spacename, key, mapattributes)
\end{javascriptcode}
Add the specified number to the value of a key-value pair within each map.

%%% Generated below here
\paragraph{Behavior:}
\begin{itemize}[noitemsep]
This operation requires a pre-existing object in order to complete successfully.
If no object exists, the operation will fail with \code{NOTFOUND}.

\item This operation mutates the value of a key-value pair in a map.  This call
    is similar to the equivalent call without the \code{map\_} prefix, but
    operates on the value of a pair in a map, instead of on an attribute's
    value.  If there is no pair with the specified map key, a new pair will be
    created and initialized to its default value.  If this is undesirable, it
    may be avoided by using a conditional operation that requires that the map
    contain the key in question.

\end{itemize}


\noindent\textbf{Parameters:}
\begin{description}[labelindent=\widthof{{\code{mapattributes}}},leftmargin=*,noitemsep,nolistsep,align=right]
\item[\code{spacename}] The name of the space as a string or buffer.
\item[\code{key}] The key for the operation as a Node type
\item[\code{mapattributes}] An object specifying map attributes to modify and their respective key/values.
\end{description}

\noindent\textbf{Returns:}
True.  Raises exception on error.

\subsubsection{\code{cond\_map\_atomic\_add}}
\label{api:nodejs:cond_map_atomic_add}
\index{cond\_map\_atomic\_add!Node.js API}
\begin{javascriptcode}
Client :: cond_map_atomic_add(spacename, key, predicates, mapattributes)
\end{javascriptcode}
Conditionally add the specified number to the value of a key-value pair within
each map.

%%% Generated below here
\paragraph{Behavior:}
\begin{itemize}[noitemsep]
This operation requires a pre-existing object in order to complete successfully.
If no object exists, the operation will fail with \code{NOTFOUND}.

This operation will succeed if and only if the predicates specified by
\code{checks} hold on the pre-existing object.  If any of the predicates are not
true for the existing object, then the operation will have no effect and fail
with \code{CMPFAIL}.

All checks are atomic with the write.  HyperDex guarantees that no other
operation will come between validating the checks, and writing the new version
of the object.

\item This operation mutates the value of a key-value pair in a map.  This call
    is similar to the equivalent call without the \code{map\_} prefix, but
    operates on the value of a pair in a map, instead of on an attribute's
    value.  If there is no pair with the specified map key, a new pair will be
    created and initialized to its default value.  If this is undesirable, it
    may be avoided by using a conditional operation that requires that the map
    contain the key in question.

\end{itemize}


\noindent\textbf{Parameters:}
\begin{description}[labelindent=\widthof{{\code{mapattributes}}},leftmargin=*,noitemsep,nolistsep,align=right]
\item[\code{spacename}] The name of the space as a string or buffer.
\item[\code{key}] The key for the operation as a Node type
\item[\code{predicates}] An object of predicates to check against.
\item[\code{mapattributes}] An object specifying map attributes to modify and their respective key/values.
\end{description}

\noindent\textbf{Returns:}
True if predicate, False if not predicate.  Raises exception on error.

\subsubsection{\code{map\_atomic\_sub}}
\label{api:nodejs:map_atomic_sub}
\index{map\_atomic\_sub!Node.js API}
\begin{javascriptcode}
Client :: map_atomic_sub(spacename, key, mapattributes)
\end{javascriptcode}
Subtract the specified number from the value of a key-value pair within each
map.
This operation requires a pre-existing object in order to complete successfully.
If no object exists, the operation will fail with \code{NOTFOUND}.



\noindent\textbf{Parameters:}
\begin{description}[labelindent=\widthof{{\code{mapattributes}}},leftmargin=*,noitemsep,nolistsep,align=right]
\item[\code{spacename}] The name of the space as a string or buffer.
\item[\code{key}] The key for the operation as a Node type
\item[\code{mapattributes}] An object specifying map attributes to modify and their respective key/values.
\end{description}

\noindent\textbf{Returns:}
True.  Raises exception on error.

\subsubsection{\code{cond\_map\_atomic\_sub}}
\label{api:nodejs:cond_map_atomic_sub}
\index{cond\_map\_atomic\_sub!Node.js API}
\begin{javascriptcode}
Client :: cond_map_atomic_sub(spacename, key, predicates, mapattributes)
\end{javascriptcode}
Subtract the specified number from the value of a key-value pair within each
map if and only if the \code{checks} hold on the object.
This operation requires a pre-existing object in order to complete successfully.
If no object exists, the operation will fail with \code{NOTFOUND}.


This operation will succeed if and only if the predicates specified by
\code{checks} hold on the pre-existing object.  If any of the predicates are not
true for the existing object, then the operation will have no effect and fail
with \code{CMPFAIL}.

All checks are atomic with the write.  HyperDex guarantees that no other
operation will come between validating the checks, and writing the new version
of the object.



\noindent\textbf{Parameters:}
\begin{description}[labelindent=\widthof{{\code{mapattributes}}},leftmargin=*,noitemsep,nolistsep,align=right]
\item[\code{spacename}] The name of the space as a string or buffer.
\item[\code{key}] The key for the operation as a Node type
\item[\code{predicates}] An object of predicates to check against.
\item[\code{mapattributes}] An object specifying map attributes to modify and their respective key/values.
\end{description}

\noindent\textbf{Returns:}
True if predicate, False if not predicate.  Raises exception on error.

\subsubsection{\code{map\_atomic\_mul}}
\label{api:nodejs:map_atomic_mul}
\index{map\_atomic\_mul!Node.js API}
\begin{javascriptcode}
Client :: map_atomic_mul(spacename, key, mapattributes)
\end{javascriptcode}
\input{\topdir/api/desc/map_atomic_mul}

\noindent\textbf{Parameters:}
\begin{description}[labelindent=\widthof{{\code{mapattributes}}},leftmargin=*,noitemsep,nolistsep,align=right]
\item[\code{spacename}] The name of the space as a string or buffer.
\item[\code{key}] The key for the operation as a Node type
\item[\code{mapattributes}] An object specifying map attributes to modify and their respective key/values.
\end{description}

\noindent\textbf{Returns:}
True.  Raises exception on error.

\subsubsection{\code{cond\_map\_atomic\_mul}}
\label{api:nodejs:cond_map_atomic_mul}
\index{cond\_map\_atomic\_mul!Node.js API}
\begin{javascriptcode}
Client :: cond_map_atomic_mul(spacename, key, predicates, mapattributes)
\end{javascriptcode}
\input{\topdir/api/desc/cond_map_atomic_mul}

\noindent\textbf{Parameters:}
\begin{description}[labelindent=\widthof{{\code{mapattributes}}},leftmargin=*,noitemsep,nolistsep,align=right]
\item[\code{spacename}] The name of the space as a string or buffer.
\item[\code{key}] The key for the operation as a Node type
\item[\code{predicates}] An object of predicates to check against.
\item[\code{mapattributes}] An object specifying map attributes to modify and their respective key/values.
\end{description}

\noindent\textbf{Returns:}
True if predicate, False if not predicate.  Raises exception on error.

\subsubsection{\code{map\_atomic\_div}}
\label{api:nodejs:map_atomic_div}
\index{map\_atomic\_div!Node.js API}
\begin{javascriptcode}
Client :: map_atomic_div(spacename, key, mapattributes)
\end{javascriptcode}
Divide the value of each key-value pair by the specified number for each map.

%%% Generated below here
\paragraph{Behavior:}
\begin{itemize}[noitemsep]
This operation requires a pre-existing object in order to complete successfully.
If no object exists, the operation will fail with \code{NOTFOUND}.

\item This operation mutates the value of a key-value pair in a map.  This call
    is similar to the equivalent call without the \code{map\_} prefix, but
    operates on the value of a pair in a map, instead of on an attribute's
    value.  If there is no pair with the specified map key, a new pair will be
    created and initialized to its default value.  If this is undesirable, it
    may be avoided by using a conditional operation that requires that the map
    contain the key in question.

\end{itemize}


\noindent\textbf{Parameters:}
\begin{description}[labelindent=\widthof{{\code{mapattributes}}},leftmargin=*,noitemsep,nolistsep,align=right]
\item[\code{spacename}] The name of the space as a string or buffer.
\item[\code{key}] The key for the operation as a Node type
\item[\code{mapattributes}] An object specifying map attributes to modify and their respective key/values.
\end{description}

\noindent\textbf{Returns:}
True.  Raises exception on error.

\subsubsection{\code{cond\_map\_atomic\_div}}
\label{api:nodejs:cond_map_atomic_div}
\index{cond\_map\_atomic\_div!Node.js API}
\begin{javascriptcode}
Client :: cond_map_atomic_div(spacename, key, predicates, mapattributes)
\end{javascriptcode}
Divide the value of each key-value pair by the specified number for each map if
and only if the \code{checks} hold on the object.
This operation requires a pre-existing object in order to complete successfully.
If no object exists, the operation will fail with \code{NOTFOUND}.


This operation will succeed if and only if the predicates specified by
\code{checks} hold on the pre-existing object.  If any of the predicates are not
true for the existing object, then the operation will have no effect and fail
with \code{CMPFAIL}.

All checks are atomic with the write.  HyperDex guarantees that no other
operation will come between validating the checks, and writing the new version
of the object.



\noindent\textbf{Parameters:}
\begin{description}[labelindent=\widthof{{\code{mapattributes}}},leftmargin=*,noitemsep,nolistsep,align=right]
\item[\code{spacename}] The name of the space as a string or buffer.
\item[\code{key}] The key for the operation as a Node type
\item[\code{predicates}] An object of predicates to check against.
\item[\code{mapattributes}] An object specifying map attributes to modify and their respective key/values.
\end{description}

\noindent\textbf{Returns:}
True if predicate, False if not predicate.  Raises exception on error.

\subsubsection{\code{map\_atomic\_mod}}
\label{api:nodejs:map_atomic_mod}
\index{map\_atomic\_mod!Node.js API}
\begin{javascriptcode}
Client :: map_atomic_mod(spacename, key, mapattributes)
\end{javascriptcode}
Store the value of the key-value pair modulo the specified number for each map.
This operation requires a pre-existing object in order to complete successfully.
If no object exists, the operation will fail with \code{NOTFOUND}.



\noindent\textbf{Parameters:}
\begin{description}[labelindent=\widthof{{\code{mapattributes}}},leftmargin=*,noitemsep,nolistsep,align=right]
\item[\code{spacename}] The name of the space as a string or buffer.
\item[\code{key}] The key for the operation as a Node type
\item[\code{mapattributes}] An object specifying map attributes to modify and their respective key/values.
\end{description}

\noindent\textbf{Returns:}
True.  Raises exception on error.

\subsubsection{\code{cond\_map\_atomic\_mod}}
\label{api:nodejs:cond_map_atomic_mod}
\index{cond\_map\_atomic\_mod!Node.js API}
\begin{javascriptcode}
Client :: cond_map_atomic_mod(spacename, key, predicates, mapattributes)
\end{javascriptcode}
Conditionally store the value of the key-value pair modulo the specified number
for each map.

%%% Generated below here
\paragraph{Behavior:}
\begin{itemize}[noitemsep]
This operation requires a pre-existing object in order to complete successfully.
If no object exists, the operation will fail with \code{NOTFOUND}.

This operation will succeed if and only if the predicates specified by
\code{checks} hold on the pre-existing object.  If any of the predicates are not
true for the existing object, then the operation will have no effect and fail
with \code{CMPFAIL}.

All checks are atomic with the write.  HyperDex guarantees that no other
operation will come between validating the checks, and writing the new version
of the object.

\item This operation mutates the value of a key-value pair in a map.  This call
    is similar to the equivalent call without the \code{map\_} prefix, but
    operates on the value of a pair in a map, instead of on an attribute's
    value.  If there is no pair with the specified map key, a new pair will be
    created and initialized to its default value.  If this is undesirable, it
    may be avoided by using a conditional operation that requires that the map
    contain the key in question.

\end{itemize}


\noindent\textbf{Parameters:}
\begin{description}[labelindent=\widthof{{\code{mapattributes}}},leftmargin=*,noitemsep,nolistsep,align=right]
\item[\code{spacename}] The name of the space as a string or buffer.
\item[\code{key}] The key for the operation as a Node type
\item[\code{predicates}] An object of predicates to check against.
\item[\code{mapattributes}] An object specifying map attributes to modify and their respective key/values.
\end{description}

\noindent\textbf{Returns:}
True if predicate, False if not predicate.  Raises exception on error.

\subsubsection{\code{map\_atomic\_and}}
\label{api:nodejs:map_atomic_and}
\index{map\_atomic\_and!Node.js API}
\begin{javascriptcode}
Client :: map_atomic_and(spacename, key, mapattributes)
\end{javascriptcode}
Store the bitwise AND of the value of the key-value pair and the specified
number for each map.
This operation requires a pre-existing object in order to complete successfully.
If no object exists, the operation will fail with \code{NOTFOUND}.



\noindent\textbf{Parameters:}
\begin{description}[labelindent=\widthof{{\code{mapattributes}}},leftmargin=*,noitemsep,nolistsep,align=right]
\item[\code{spacename}] The name of the space as a string or buffer.
\item[\code{key}] The key for the operation as a Node type
\item[\code{mapattributes}] An object specifying map attributes to modify and their respective key/values.
\end{description}

\noindent\textbf{Returns:}
True.  Raises exception on error.

\subsubsection{\code{cond\_map\_atomic\_and}}
\label{api:nodejs:cond_map_atomic_and}
\index{cond\_map\_atomic\_and!Node.js API}
\begin{javascriptcode}
Client :: cond_map_atomic_and(spacename, key, predicates, mapattributes)
\end{javascriptcode}
\input{\topdir/api/desc/cond_map_atomic_and}

\noindent\textbf{Parameters:}
\begin{description}[labelindent=\widthof{{\code{mapattributes}}},leftmargin=*,noitemsep,nolistsep,align=right]
\item[\code{spacename}] The name of the space as a string or buffer.
\item[\code{key}] The key for the operation as a Node type
\item[\code{predicates}] An object of predicates to check against.
\item[\code{mapattributes}] An object specifying map attributes to modify and their respective key/values.
\end{description}

\noindent\textbf{Returns:}
True if predicate, False if not predicate.  Raises exception on error.

\subsubsection{\code{map\_atomic\_or}}
\label{api:nodejs:map_atomic_or}
\index{map\_atomic\_or!Node.js API}
\begin{javascriptcode}
Client :: map_atomic_or(spacename, key, mapattributes)
\end{javascriptcode}
Store the bitwise OR of the value of the key-value pair and the specified number
for each map.

%%% Generated below here
\paragraph{Behavior:}
\begin{itemize}[noitemsep]
This operation requires a pre-existing object in order to complete successfully.
If no object exists, the operation will fail with \code{NOTFOUND}.

\item This operation mutates the value of a key-value pair in a map.  This call
    is similar to the equivalent call without the \code{map\_} prefix, but
    operates on the value of a pair in a map, instead of on an attribute's
    value.  If there is no pair with the specified map key, a new pair will be
    created and initialized to its default value.  If this is undesirable, it
    may be avoided by using a conditional operation that requires that the map
    contain the key in question.

\end{itemize}


\noindent\textbf{Parameters:}
\begin{description}[labelindent=\widthof{{\code{mapattributes}}},leftmargin=*,noitemsep,nolistsep,align=right]
\item[\code{spacename}] The name of the space as a string or buffer.
\item[\code{key}] The key for the operation as a Node type
\item[\code{mapattributes}] An object specifying map attributes to modify and their respective key/values.
\end{description}

\noindent\textbf{Returns:}
True.  Raises exception on error.

\subsubsection{\code{cond\_map\_atomic\_or}}
\label{api:nodejs:cond_map_atomic_or}
\index{cond\_map\_atomic\_or!Node.js API}
\begin{javascriptcode}
Client :: cond_map_atomic_or(spacename, key, predicates, mapattributes)
\end{javascriptcode}
\input{\topdir/api/desc/cond_map_atomic_or}

\noindent\textbf{Parameters:}
\begin{description}[labelindent=\widthof{{\code{mapattributes}}},leftmargin=*,noitemsep,nolistsep,align=right]
\item[\code{spacename}] The name of the space as a string or buffer.
\item[\code{key}] The key for the operation as a Node type
\item[\code{predicates}] An object of predicates to check against.
\item[\code{mapattributes}] An object specifying map attributes to modify and their respective key/values.
\end{description}

\noindent\textbf{Returns:}
True if predicate, False if not predicate.  Raises exception on error.

\subsubsection{\code{map\_atomic\_xor}}
\label{api:nodejs:map_atomic_xor}
\index{map\_atomic\_xor!Node.js API}
\begin{javascriptcode}
Client :: map_atomic_xor(spacename, key, mapattributes)
\end{javascriptcode}
Store the bitwise XOR of the value of the key-value pair and the specified
number for each map attribute.
This operation requires a pre-existing object in order to complete successfully.
If no object exists, the operation will fail with \code{NOTFOUND}.



\noindent\textbf{Parameters:}
\begin{description}[labelindent=\widthof{{\code{mapattributes}}},leftmargin=*,noitemsep,nolistsep,align=right]
\item[\code{spacename}] The name of the space as a string or buffer.
\item[\code{key}] The key for the operation as a Node type
\item[\code{mapattributes}] An object specifying map attributes to modify and their respective key/values.
\end{description}

\noindent\textbf{Returns:}
True.  Raises exception on error.

\subsubsection{\code{cond\_map\_atomic\_xor}}
\label{api:nodejs:cond_map_atomic_xor}
\index{cond\_map\_atomic\_xor!Node.js API}
\begin{javascriptcode}
Client :: cond_map_atomic_xor(spacename, key, predicates, mapattributes)
\end{javascriptcode}
Conditionally store the bitwise XOR of the value of the key-value pair and the
specified number for each map.

%%% Generated below here
\paragraph{Behavior:}
\begin{itemize}[noitemsep]
This operation requires a pre-existing object in order to complete successfully.
If no object exists, the operation will fail with \code{NOTFOUND}.

This operation will succeed if and only if the predicates specified by
\code{checks} hold on the pre-existing object.  If any of the predicates are not
true for the existing object, then the operation will have no effect and fail
with \code{CMPFAIL}.

All checks are atomic with the write.  HyperDex guarantees that no other
operation will come between validating the checks, and writing the new version
of the object.

\item This operation mutates the value of a key-value pair in a map.  This call
    is similar to the equivalent call without the \code{map\_} prefix, but
    operates on the value of a pair in a map, instead of on an attribute's
    value.  If there is no pair with the specified map key, a new pair will be
    created and initialized to its default value.  If this is undesirable, it
    may be avoided by using a conditional operation that requires that the map
    contain the key in question.

\end{itemize}


\noindent\textbf{Parameters:}
\begin{description}[labelindent=\widthof{{\code{mapattributes}}},leftmargin=*,noitemsep,nolistsep,align=right]
\item[\code{spacename}] The name of the space as a string or buffer.
\item[\code{key}] The key for the operation as a Node type
\item[\code{predicates}] An object of predicates to check against.
\item[\code{mapattributes}] An object specifying map attributes to modify and their respective key/values.
\end{description}

\noindent\textbf{Returns:}
True if predicate, False if not predicate.  Raises exception on error.

\subsubsection{\code{map\_string\_prepend}}
\label{api:nodejs:map_string_prepend}
\index{map\_string\_prepend!Node.js API}
\begin{javascriptcode}
Client :: map_string_prepend(spacename, key, mapattributes)
\end{javascriptcode}
Prepend the specified string to the value of the key-value pair for each map.

%%% Generated below here
\paragraph{Behavior:}
\begin{itemize}[noitemsep]
This operation requires a pre-existing object in order to complete successfully.
If no object exists, the operation will fail with \code{NOTFOUND}.

\item This operation mutates the value of a key-value pair in a map.  This call
    is similar to the equivalent call without the \code{map\_} prefix, but
    operates on the value of a pair in a map, instead of on an attribute's
    value.  If there is no pair with the specified map key, a new pair will be
    created and initialized to its default value.  If this is undesirable, it
    may be avoided by using a conditional operation that requires that the map
    contain the key in question.

\end{itemize}


\noindent\textbf{Parameters:}
\begin{description}[labelindent=\widthof{{\code{mapattributes}}},leftmargin=*,noitemsep,nolistsep,align=right]
\item[\code{spacename}] The name of the space as a string or buffer.
\item[\code{key}] The key for the operation as a Node type
\item[\code{mapattributes}] An object specifying map attributes to modify and their respective key/values.
\end{description}

\noindent\textbf{Returns:}
True.  Raises exception on error.

\subsubsection{\code{cond\_map\_string\_prepend}}
\label{api:nodejs:cond_map_string_prepend}
\index{cond\_map\_string\_prepend!Node.js API}
\begin{javascriptcode}
Client :: cond_map_string_prepend(spacename, key, predicates, mapattributes)
\end{javascriptcode}
\input{\topdir/api/desc/cond_map_string_prepend}

\noindent\textbf{Parameters:}
\begin{description}[labelindent=\widthof{{\code{mapattributes}}},leftmargin=*,noitemsep,nolistsep,align=right]
\item[\code{spacename}] The name of the space as a string or buffer.
\item[\code{key}] The key for the operation as a Node type
\item[\code{predicates}] An object of predicates to check against.
\item[\code{mapattributes}] An object specifying map attributes to modify and their respective key/values.
\end{description}

\noindent\textbf{Returns:}
True if predicate, False if not predicate.  Raises exception on error.

\subsubsection{\code{map\_string\_append}}
\label{api:nodejs:map_string_append}
\index{map\_string\_append!Node.js API}
\begin{javascriptcode}
Client :: map_string_append(spacename, key, mapattributes)
\end{javascriptcode}
Append the specified string to the value of the key-value pair for each map
attribute.
This operation requires a pre-existing object in order to complete successfully.
If no object exists, the operation will fail with \code{NOTFOUND}.



\noindent\textbf{Parameters:}
\begin{description}[labelindent=\widthof{{\code{mapattributes}}},leftmargin=*,noitemsep,nolistsep,align=right]
\item[\code{spacename}] The name of the space as a string or buffer.
\item[\code{key}] The key for the operation as a Node type
\item[\code{mapattributes}] An object specifying map attributes to modify and their respective key/values.
\end{description}

\noindent\textbf{Returns:}
True.  Raises exception on error.

\subsubsection{\code{cond\_map\_string\_append}}
\label{api:nodejs:cond_map_string_append}
\index{cond\_map\_string\_append!Node.js API}
\begin{javascriptcode}
Client :: cond_map_string_append(spacename, key, predicates, mapattributes)
\end{javascriptcode}
Append the specified string to the value of the key-value pair for each map
attribute if and only if the \code{checks} hold on the object.
This operation requires a pre-existing object in order to complete successfully.
If no object exists, the operation will fail with \code{NOTFOUND}.


This operation will succeed if and only if the predicates specified by
\code{checks} hold on the pre-existing object.  If any of the predicates are not
true for the existing object, then the operation will have no effect and fail
with \code{CMPFAIL}.

All checks are atomic with the write.  HyperDex guarantees that no other
operation will come between validating the checks, and writing the new version
of the object.



\noindent\textbf{Parameters:}
\begin{description}[labelindent=\widthof{{\code{mapattributes}}},leftmargin=*,noitemsep,nolistsep,align=right]
\item[\code{spacename}] The name of the space as a string or buffer.
\item[\code{key}] The key for the operation as a Node type
\item[\code{predicates}] An object of predicates to check against.
\item[\code{mapattributes}] An object specifying map attributes to modify and their respective key/values.
\end{description}

\noindent\textbf{Returns:}
True if predicate, False if not predicate.  Raises exception on error.

\subsubsection{\code{search}}
\label{api:nodejs:search}
\index{search!Node.js API}
\begin{javascriptcode}
Client :: search(spacename, predicates)
\end{javascriptcode}
Return all objects that match the specified \code{checks}.

\paragraph{Behavior:}
\begin{itemize}[noitemsep]
This operation behaves as an iterator and may return multiple objects from the
single call.

\item This operation return to the user the requested object(s).

\end{itemize}


\noindent\textbf{Parameters:}
\begin{description}[labelindent=\widthof{{\code{predicates}}},leftmargin=*,noitemsep,nolistsep,align=right]
\item[\code{spacename}] The name of the space as string or buffer.
\item[\code{predicates}] An object of predicates to check against.
\end{description}

\noindent\textbf{Returns:}
Object if found, nil if not found.  Raises exception on error.

\subsubsection{\code{search\_describe}}
\label{api:nodejs:search_describe}
\index{search\_describe!Node.js API}
\begin{javascriptcode}
Client :: search_describe(spacename, predicates)
\end{javascriptcode}
Return a human-readable string suitable for debugging search internals.  This
API is only really relevant for debugging the internals of \code{search}.


\noindent\textbf{Parameters:}
\begin{description}[labelindent=\widthof{{\code{predicates}}},leftmargin=*,noitemsep,nolistsep,align=right]
\item[\code{spacename}] The name of the space as a string or buffer.
\item[\code{predicates}] An object of predicates to check against.
\end{description}

\noindent\textbf{Returns:}
Description of search.  Raises exception on error.

\subsubsection{\code{sorted\_search}}
\label{api:nodejs:sorted_search}
\index{sorted\_search!Node.js API}
\begin{javascriptcode}
Client :: sorted_search(spacename, predicates, sortby, limit, maxmin)
\end{javascriptcode}
Return all objects that match the specified \code{checks}, sorted according to
\code{attr}.
This operation behaves as an iterator and may return multiple objects from the
single call.



\noindent\textbf{Parameters:}
\begin{description}[labelindent=\widthof{{\code{predicates}}},leftmargin=*,noitemsep,nolistsep,align=right]
\item[\code{spacename}] The name of the space as string or buffer.
\item[\code{predicates}] An object of predicates to check against.
\item[\code{sortby}] The attribute to sort by.
\item[\code{limit}] The number of results to return.
\item[\code{maxmin}] Maximize or minimize (e.g., "max" or "min").
\end{description}

\noindent\textbf{Returns:}
Object if found, nil if not found.  Raises exception on error.

\subsubsection{\code{group\_del}}
\label{api:nodejs:group_del}
\index{group\_del!Node.js API}
\begin{javascriptcode}
Client :: group_del(spacename, predicates)
\end{javascriptcode}
Asynchronously delete all objects that match the specified \code{checks}.

\paragraph{Behavior:}
\begin{itemize}[noitemsep]
\item This operation is roughly equivalent to a client manually deleting every
    object returned from a search, but saves HyperDex from sending to the client
    objects that are soon to be deleted.
\end{itemize}


\noindent\textbf{Parameters:}
\begin{description}[labelindent=\widthof{{\code{predicates}}},leftmargin=*,noitemsep,nolistsep,align=right]
\item[\code{spacename}] The name of the space as a string or buffer.
\item[\code{predicates}] An object of predicates to check against.
\end{description}

\noindent\textbf{Returns:}
True if predicate, False if not predicate.  Raises exception on error.

\subsubsection{\code{count}}
\label{api:nodejs:count}
\index{count!Node.js API}
\begin{javascriptcode}
Client :: count(spacename, predicates)
\end{javascriptcode}
Count the number of objects that match the specified \code{checks}.

\paragraph{Behavior:}
\begin{itemize}[noitemsep]
\item This will return the number of objects counted by the search.  If an error
    occurs during the count, the count will reflect a partial count.  The real
    count will be higher than the returned value.  Some languages will throw an
    exception rather than return the partial count.
\end{itemize}


\noindent\textbf{Parameters:}
\begin{description}[labelindent=\widthof{{\code{predicates}}},leftmargin=*,noitemsep,nolistsep,align=right]
\item[\code{spacename}] The name of the space as a string or buffer.
\item[\code{predicates}] An object of predicates to check against.
\end{description}

\noindent\textbf{Returns:}
Number of objects found.  Raises exception on error.


\subsection{Working with Signals}
\label{sec:api:node:signals}

Your application must mask all signals prior to making any calls into the Node
bindings.  The Node bindings will unmask the signals during blocking operations
and raise a \code{HyperDexClientException} with status
\code{'HYPERDEX\_CLIENT\_INTERRUPTED'} should any signals be received.

\subsection{Working with Events}
\label{sec:api:node:threads}

The Node module naturally integrates with the Node.js event loop.  Each instance
of \code{Client} registers itself with the Node event loop and makes callbacks
as soon as events complete on the HyperDex side.

Put simply, a Node.js application can use \code{Client} instances in a
straight-forward fashion without worrying about threading or manual integration.
