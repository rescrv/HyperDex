% Copyright (c) 2013-2014, Cornell University
% All rights reserved.
%
% Redistribution and use in source and binary forms, with or without
% modification, are permitted provided that the following conditions are met:
%
%     * Redistributions of source code must retain the above copyright notice,
%       this list of conditions and the following disclaimer.
%     * Redistributions in binary form must reproduce the above copyright
%       notice, this list of conditions and the following disclaimer in the
%       documentation and/or other materials provided with the distribution.
%     * Neither the name of HyperDex nor the names of its contributors may be
%       used to endorse or promote products derived from this software without
%       specific prior written permission.
%
% THIS SOFTWARE IS PROVIDED BY THE COPYRIGHT HOLDERS AND CONTRIBUTORS "AS IS"
% AND ANY EXPRESS OR IMPLIED WARRANTIES, INCLUDING, BUT NOT LIMITED TO, THE
% IMPLIED WARRANTIES OF MERCHANTABILITY AND FITNESS FOR A PARTICULAR PURPOSE ARE
% DISCLAIMED. IN NO EVENT SHALL THE COPYRIGHT OWNER OR CONTRIBUTORS BE LIABLE
% FOR ANY DIRECT, INDIRECT, INCIDENTAL, SPECIAL, EXEMPLARY, OR CONSEQUENTIAL
% DAMAGES (INCLUDING, BUT NOT LIMITED TO, PROCUREMENT OF SUBSTITUTE GOODS OR
% SERVICES; LOSS OF USE, DATA, OR PROFITS; OR BUSINESS INTERRUPTION) HOWEVER
% CAUSED AND ON ANY THEORY OF LIABILITY, WHETHER IN CONTRACT, STRICT LIABILITY,
% OR TORT (INCLUDING NEGLIGENCE OR OTHERWISE) ARISING IN ANY WAY OUT OF THE USE
% OF THIS SOFTWARE, EVEN IF ADVISED OF THE POSSIBILITY OF SUCH DAMAGE.

% This LaTeX file is generated by bindings/c.py

%%%%%%%%%%%%%%%%%%%% get %%%%%%%%%%%%%%%%%%%%
\pagebreak
\subsubsection{\code{get}}
\label{api:c:get}
\index{get!C API}
Retreive the object with key "key" from space "space".

\noindent\textbf{Cost:}  Approximately one network round trip.


\input{api/shards/linearizable}


\paragraph{Definition:}
\begin{ccode}
int64_t hyperdex_client_get(struct hyperdex_client* client,
        const char* space,
        const char* key, size_t key_sz,
        enum hyperdex_client_returncode* status,
        const struct hyperdex_client_attribute** attrs, size_t* attrs_sz);
\end{ccode}

\paragraph{Parameters:}
\begin{itemize}[noitemsep]
\item \code{space}\\
The name of the space as a c-string.
\item \code{key}, \code{key\_sz}\\
The key for the operation where \code{key} is a bytestring and \code{key\_sz} specifies the number of bytes in \code{key}.
\end{itemize}

\paragraph{Returns:}
\begin{itemize}[noitemsep]
\item \code{status}\\
The status of the operation.  The client library will fill in this variable before returning this operation's request id from \code{hyperdex\_client\_loop}.  The pointer must remain valid until then, and the pointer should not be aliased to the status for any other outstanding operation.
\item \code{attrs}, \code{attrs\_sz}\\
An array of attributes that comprise a returned object.  The application must free the returned values with \code{hyperdex\_client\_destroy\_attrs}.  The pointers must remain valid until the operation completes.
\end{itemize}

%%%%%%%%%%%%%%%%%%%% put %%%%%%%%%%%%%%%%%%%%
\pagebreak
\subsubsection{\code{put}}
\label{api:c:put}
\index{put!C API}
Store or update an object by key.  The object's attributes will be set to the
values specified by \code{attrs}.
If the object exists, it will be updated and all existing values not altered by
\code{attrs} will be preserved.  If the object does not exist, a new object will
be created, with its attributes initialized to their default values.



\paragraph{Definition:}
\begin{ccode}
int64_t hyperdex_client_put(struct hyperdex_client* client,
        const char* space,
        const char* key, size_t key_sz,
        const struct hyperdex_client_attribute* attrs, size_t attrs_sz,
        enum hyperdex_client_returncode* status);
\end{ccode}

\paragraph{Parameters:}
\begin{itemize}[noitemsep]
\item \code{space}\\
The name of the space as a c-string.
\item \code{key}, \code{key\_sz}\\
The key for the operation where \code{key} is a bytestring and \code{key\_sz} specifies the number of bytes in \code{key}.
\item \code{attrs}, \code{attrs\_sz}\\
The set of attributes to modify and their respective values.  \code{attrs} points to an array of length \code{attrs\_sz}.
\end{itemize}

\paragraph{Returns:}
\begin{itemize}[noitemsep]
\item \code{status}\\
The status of the operation.  The client library will fill in this variable before returning this operation's request id from \code{hyperdex\_client\_loop}.  The pointer must remain valid until then, and the pointer should not be aliased to the status for any other outstanding operation.
\end{itemize}

%%%%%%%%%%%%%%%%%%%% cond_put %%%%%%%%%%%%%%%%%%%%
\pagebreak
\subsubsection{\code{cond\_put}}
\label{api:c:cond_put}
\index{cond\_put!C API}
Conditionally update an the object stored under \code{key} in \code{space}.
Existing values will be overwritten with the values specified by \code{attrs}.
Values not specified by \code{attrs} will remain unchanged.
This operation requires a pre-existing object in order to complete successfully.
If no object exists, the operation will fail with \code{NOTFOUND}.


This operation will succeed if and only if the predicates specified by
\code{checks} hold on the pre-existing object.  If any of the predicates are not
true for the existing object, then the operation will have no effect and fail
with \code{CMPFAIL}.

All checks are atomic with the write.  HyperDex guarantees that no other
operation will come between validating the checks, and writing the new version
of the object.



\paragraph{Definition:}
\begin{ccode}
int64_t hyperdex_client_cond_put(struct hyperdex_client* client,
        const char* space,
        const char* key, size_t key_sz,
        const struct hyperdex_client_attribute_check* checks, size_t checks_sz,
        const struct hyperdex_client_attribute* attrs, size_t attrs_sz,
        enum hyperdex_client_returncode* status);
\end{ccode}

\paragraph{Parameters:}
\begin{itemize}[noitemsep]
\item \code{space}\\
The name of the space as a c-string.
\item \code{key}, \code{key\_sz}\\
The key for the operation where \code{key} is a bytestring and \code{key\_sz} specifies the number of bytes in \code{key}.
\item \code{checks}, \code{checks\_sz}\\
A set of predicates to check against.  \code{checks} points to an array of length \code{checks\_sz}.
\item \code{attrs}, \code{attrs\_sz}\\
The set of attributes to modify and their respective values.  \code{attrs} points to an array of length \code{attrs\_sz}.
\end{itemize}

\paragraph{Returns:}
\begin{itemize}[noitemsep]
\item \code{status}\\
The status of the operation.  The client library will fill in this variable before returning this operation's request id from \code{hyperdex\_client\_loop}.  The pointer must remain valid until then, and the pointer should not be aliased to the status for any other outstanding operation.
\end{itemize}

%%%%%%%%%%%%%%%%%%%% put_if_not_exist %%%%%%%%%%%%%%%%%%%%
\pagebreak
\subsubsection{\code{put\_if\_not\_exist}}
\label{api:c:put_if_not_exist}
\index{put\_if\_not\_exist!C API}
Store an object in space "space" under key "key" if and only if it does not
already exist.

The object will be create, and the attributes specified by \texttt{attrs} will
be set to their respective values.  Any attributes not specified by
\texttt{attrs} will be initialized to their default values.  The check is atomic
with the write, and is guaranteed to never overwrite an existing object.

\noindent\textbf{Cost:}  Approximately one traversal of the value-dependent
chain.


\input{api/shards/linearizable}


\paragraph{Definition:}
\begin{ccode}
int64_t hyperdex_client_put_if_not_exist(struct hyperdex_client* client,
        const char* space,
        const char* key, size_t key_sz,
        const struct hyperdex_client_attribute* attrs, size_t attrs_sz,
        enum hyperdex_client_returncode* status);
\end{ccode}

\paragraph{Parameters:}
\begin{itemize}[noitemsep]
\item \code{space}\\
The name of the space as a c-string.
\item \code{key}, \code{key\_sz}\\
The key for the operation where \code{key} is a bytestring and \code{key\_sz} specifies the number of bytes in \code{key}.
\item \code{attrs}, \code{attrs\_sz}\\
The set of attributes to modify and their respective values.  \code{attrs} points to an array of length \code{attrs\_sz}.
\end{itemize}

\paragraph{Returns:}
\begin{itemize}[noitemsep]
\item \code{status}\\
The status of the operation.  The client library will fill in this variable before returning this operation's request id from \code{hyperdex\_client\_loop}.  The pointer must remain valid until then, and the pointer should not be aliased to the status for any other outstanding operation.
\end{itemize}

%%%%%%%%%%%%%%%%%%%% del %%%%%%%%%%%%%%%%%%%%
\pagebreak
\subsubsection{\code{del}}
\label{api:c:del}
\index{del!C API}
Delete an object by key.

%%% Generated below here
\paragraph{Behavior:}
\begin{itemize}[noitemsep]
If no object exists, the operation will fail with \code{NOTFOUND}.

\end{itemize}


\paragraph{Definition:}
\begin{ccode}
int64_t hyperdex_client_del(struct hyperdex_client* client,
        const char* space,
        const char* key, size_t key_sz,
        enum hyperdex_client_returncode* status);
\end{ccode}

\paragraph{Parameters:}
\begin{itemize}[noitemsep]
\item \code{space}\\
The name of the space as a c-string.
\item \code{key}, \code{key\_sz}\\
The key for the operation where \code{key} is a bytestring and \code{key\_sz} specifies the number of bytes in \code{key}.
\end{itemize}

\paragraph{Returns:}
\begin{itemize}[noitemsep]
\item \code{status}\\
The status of the operation.  The client library will fill in this variable before returning this operation's request id from \code{hyperdex\_client\_loop}.  The pointer must remain valid until then, and the pointer should not be aliased to the status for any other outstanding operation.
\end{itemize}

%%%%%%%%%%%%%%%%%%%% cond_del %%%%%%%%%%%%%%%%%%%%
\pagebreak
\subsubsection{\code{cond\_del}}
\label{api:c:cond_del}
\index{cond\_del!C API}
Conditionally delete an object by key.

%%% Generated below here
\paragraph{Behavior:}
\begin{itemize}[noitemsep]
If no object exists, the operation will fail with \code{NOTFOUND}.

This operation will succeed if and only if the predicates specified by
\code{checks} hold on the pre-existing object.  If any of the predicates are not
true for the existing object, then the operation will have no effect and fail
with \code{CMPFAIL}.

All checks are atomic with the write.  HyperDex guarantees that no other
operation will come between validating the checks, and writing the new version
of the object.

\end{itemize}


\paragraph{Definition:}
\begin{ccode}
int64_t hyperdex_client_cond_del(struct hyperdex_client* client,
        const char* space,
        const char* key, size_t key_sz,
        const struct hyperdex_client_attribute_check* checks, size_t checks_sz,
        enum hyperdex_client_returncode* status);
\end{ccode}

\paragraph{Parameters:}
\begin{itemize}[noitemsep]
\item \code{space}\\
The name of the space as a c-string.
\item \code{key}, \code{key\_sz}\\
The key for the operation where \code{key} is a bytestring and \code{key\_sz} specifies the number of bytes in \code{key}.
\item \code{checks}, \code{checks\_sz}\\
A set of predicates to check against.  \code{checks} points to an array of length \code{checks\_sz}.
\end{itemize}

\paragraph{Returns:}
\begin{itemize}[noitemsep]
\item \code{status}\\
The status of the operation.  The client library will fill in this variable before returning this operation's request id from \code{hyperdex\_client\_loop}.  The pointer must remain valid until then, and the pointer should not be aliased to the status for any other outstanding operation.
\end{itemize}

%%%%%%%%%%%%%%%%%%%% atomic_add %%%%%%%%%%%%%%%%%%%%
\pagebreak
\subsubsection{\code{atomic\_add}}
\label{api:c:atomic_add}
\index{atomic\_add!C API}
Add the specified number to the existing value for each attribute.
This operation requires a pre-existing object in order to complete successfully.
If no object exists, the operation will fail with \code{NOTFOUND}.



\paragraph{Definition:}
\begin{ccode}
int64_t hyperdex_client_atomic_add(struct hyperdex_client* client,
        const char* space,
        const char* key, size_t key_sz,
        const struct hyperdex_client_attribute* attrs, size_t attrs_sz,
        enum hyperdex_client_returncode* status);
\end{ccode}

\paragraph{Parameters:}
\begin{itemize}[noitemsep]
\item \code{space}\\
The name of the space as a c-string.
\item \code{key}, \code{key\_sz}\\
The key for the operation where \code{key} is a bytestring and \code{key\_sz} specifies the number of bytes in \code{key}.
\item \code{attrs}, \code{attrs\_sz}\\
The set of attributes to modify and their respective values.  \code{attrs} points to an array of length \code{attrs\_sz}.
\end{itemize}

\paragraph{Returns:}
\begin{itemize}[noitemsep]
\item \code{status}\\
The status of the operation.  The client library will fill in this variable before returning this operation's request id from \code{hyperdex\_client\_loop}.  The pointer must remain valid until then, and the pointer should not be aliased to the status for any other outstanding operation.
\end{itemize}

%%%%%%%%%%%%%%%%%%%% cond_atomic_add %%%%%%%%%%%%%%%%%%%%
\pagebreak
\subsubsection{\code{cond\_atomic\_add}}
\label{api:c:cond_atomic_add}
\index{cond\_atomic\_add!C API}
Conditionally add the specified number to the existing value for each attribute.

%%% Generated below here
\paragraph{Behavior:}
\begin{itemize}[noitemsep]
This operation requires a pre-existing object in order to complete successfully.
If no object exists, the operation will fail with \code{NOTFOUND}.

This operation will succeed if and only if the predicates specified by
\code{checks} hold on the pre-existing object.  If any of the predicates are not
true for the existing object, then the operation will have no effect and fail
with \code{CMPFAIL}.

All checks are atomic with the write.  HyperDex guarantees that no other
operation will come between validating the checks, and writing the new version
of the object.

\end{itemize}


\paragraph{Definition:}
\begin{ccode}
int64_t hyperdex_client_cond_atomic_add(struct hyperdex_client* client,
        const char* space,
        const char* key, size_t key_sz,
        const struct hyperdex_client_attribute_check* checks, size_t checks_sz,
        const struct hyperdex_client_attribute* attrs, size_t attrs_sz,
        enum hyperdex_client_returncode* status);
\end{ccode}

\paragraph{Parameters:}
\begin{itemize}[noitemsep]
\item \code{space}\\
The name of the space as a c-string.
\item \code{key}, \code{key\_sz}\\
The key for the operation where \code{key} is a bytestring and \code{key\_sz} specifies the number of bytes in \code{key}.
\item \code{checks}, \code{checks\_sz}\\
A set of predicates to check against.  \code{checks} points to an array of length \code{checks\_sz}.
\item \code{attrs}, \code{attrs\_sz}\\
The set of attributes to modify and their respective values.  \code{attrs} points to an array of length \code{attrs\_sz}.
\end{itemize}

\paragraph{Returns:}
\begin{itemize}[noitemsep]
\item \code{status}\\
The status of the operation.  The client library will fill in this variable before returning this operation's request id from \code{hyperdex\_client\_loop}.  The pointer must remain valid until then, and the pointer should not be aliased to the status for any other outstanding operation.
\end{itemize}

%%%%%%%%%%%%%%%%%%%% atomic_sub %%%%%%%%%%%%%%%%%%%%
\pagebreak
\subsubsection{\code{atomic\_sub}}
\label{api:c:atomic_sub}
\index{atomic\_sub!C API}
Subtract the specified number from the existing value for each attribute.

%%% Generated below here
\paragraph{Behavior:}
\begin{itemize}[noitemsep]
This operation requires a pre-existing object in order to complete successfully.
If no object exists, the operation will fail with \code{NOTFOUND}.

\end{itemize}


\paragraph{Definition:}
\begin{ccode}
int64_t hyperdex_client_atomic_sub(struct hyperdex_client* client,
        const char* space,
        const char* key, size_t key_sz,
        const struct hyperdex_client_attribute* attrs, size_t attrs_sz,
        enum hyperdex_client_returncode* status);
\end{ccode}

\paragraph{Parameters:}
\begin{itemize}[noitemsep]
\item \code{space}\\
The name of the space as a c-string.
\item \code{key}, \code{key\_sz}\\
The key for the operation where \code{key} is a bytestring and \code{key\_sz} specifies the number of bytes in \code{key}.
\item \code{attrs}, \code{attrs\_sz}\\
The set of attributes to modify and their respective values.  \code{attrs} points to an array of length \code{attrs\_sz}.
\end{itemize}

\paragraph{Returns:}
\begin{itemize}[noitemsep]
\item \code{status}\\
The status of the operation.  The client library will fill in this variable before returning this operation's request id from \code{hyperdex\_client\_loop}.  The pointer must remain valid until then, and the pointer should not be aliased to the status for any other outstanding operation.
\end{itemize}

%%%%%%%%%%%%%%%%%%%% cond_atomic_sub %%%%%%%%%%%%%%%%%%%%
\pagebreak
\subsubsection{\code{cond\_atomic\_sub}}
\label{api:c:cond_atomic_sub}
\index{cond\_atomic\_sub!C API}
Conditionally subtract the specified number from the existing value for each attribute.

%%% Generated below here
\paragraph{Behavior:}
\begin{itemize}[noitemsep]
This operation requires a pre-existing object in order to complete successfully.
If no object exists, the operation will fail with \code{NOTFOUND}.

This operation will succeed if and only if the predicates specified by
\code{checks} hold on the pre-existing object.  If any of the predicates are not
true for the existing object, then the operation will have no effect and fail
with \code{CMPFAIL}.

All checks are atomic with the write.  HyperDex guarantees that no other
operation will come between validating the checks, and writing the new version
of the object.

\end{itemize}


\paragraph{Definition:}
\begin{ccode}
int64_t hyperdex_client_cond_atomic_sub(struct hyperdex_client* client,
        const char* space,
        const char* key, size_t key_sz,
        const struct hyperdex_client_attribute_check* checks, size_t checks_sz,
        const struct hyperdex_client_attribute* attrs, size_t attrs_sz,
        enum hyperdex_client_returncode* status);
\end{ccode}

\paragraph{Parameters:}
\begin{itemize}[noitemsep]
\item \code{space}\\
The name of the space as a c-string.
\item \code{key}, \code{key\_sz}\\
The key for the operation where \code{key} is a bytestring and \code{key\_sz} specifies the number of bytes in \code{key}.
\item \code{checks}, \code{checks\_sz}\\
A set of predicates to check against.  \code{checks} points to an array of length \code{checks\_sz}.
\item \code{attrs}, \code{attrs\_sz}\\
The set of attributes to modify and their respective values.  \code{attrs} points to an array of length \code{attrs\_sz}.
\end{itemize}

\paragraph{Returns:}
\begin{itemize}[noitemsep]
\item \code{status}\\
The status of the operation.  The client library will fill in this variable before returning this operation's request id from \code{hyperdex\_client\_loop}.  The pointer must remain valid until then, and the pointer should not be aliased to the status for any other outstanding operation.
\end{itemize}

%%%%%%%%%%%%%%%%%%%% atomic_mul %%%%%%%%%%%%%%%%%%%%
\pagebreak
\subsubsection{\code{atomic\_mul}}
\label{api:c:atomic_mul}
\index{atomic\_mul!C API}
\input{\topdir/api/desc/atomic_mul}

\paragraph{Definition:}
\begin{ccode}
int64_t hyperdex_client_atomic_mul(struct hyperdex_client* client,
        const char* space,
        const char* key, size_t key_sz,
        const struct hyperdex_client_attribute* attrs, size_t attrs_sz,
        enum hyperdex_client_returncode* status);
\end{ccode}

\paragraph{Parameters:}
\begin{itemize}[noitemsep]
\item \code{space}\\
The name of the space as a c-string.
\item \code{key}, \code{key\_sz}\\
The key for the operation where \code{key} is a bytestring and \code{key\_sz} specifies the number of bytes in \code{key}.
\item \code{attrs}, \code{attrs\_sz}\\
The set of attributes to modify and their respective values.  \code{attrs} points to an array of length \code{attrs\_sz}.
\end{itemize}

\paragraph{Returns:}
\begin{itemize}[noitemsep]
\item \code{status}\\
The status of the operation.  The client library will fill in this variable before returning this operation's request id from \code{hyperdex\_client\_loop}.  The pointer must remain valid until then, and the pointer should not be aliased to the status for any other outstanding operation.
\end{itemize}

%%%%%%%%%%%%%%%%%%%% cond_atomic_mul %%%%%%%%%%%%%%%%%%%%
\pagebreak
\subsubsection{\code{cond\_atomic\_mul}}
\label{api:c:cond_atomic_mul}
\index{cond\_atomic\_mul!C API}
Multiply the existing value by the specified number for each attribute if and
only if the \code{checks} hold on the object.
This operation requires a pre-existing object in order to complete successfully.
If no object exists, the operation will fail with \code{NOTFOUND}.


This operation will succeed if and only if the predicates specified by
\code{checks} hold on the pre-existing object.  If any of the predicates are not
true for the existing object, then the operation will have no effect and fail
with \code{CMPFAIL}.

All checks are atomic with the write.  HyperDex guarantees that no other
operation will come between validating the checks, and writing the new version
of the object.



\paragraph{Definition:}
\begin{ccode}
int64_t hyperdex_client_cond_atomic_mul(struct hyperdex_client* client,
        const char* space,
        const char* key, size_t key_sz,
        const struct hyperdex_client_attribute_check* checks, size_t checks_sz,
        const struct hyperdex_client_attribute* attrs, size_t attrs_sz,
        enum hyperdex_client_returncode* status);
\end{ccode}

\paragraph{Parameters:}
\begin{itemize}[noitemsep]
\item \code{space}\\
The name of the space as a c-string.
\item \code{key}, \code{key\_sz}\\
The key for the operation where \code{key} is a bytestring and \code{key\_sz} specifies the number of bytes in \code{key}.
\item \code{checks}, \code{checks\_sz}\\
A set of predicates to check against.  \code{checks} points to an array of length \code{checks\_sz}.
\item \code{attrs}, \code{attrs\_sz}\\
The set of attributes to modify and their respective values.  \code{attrs} points to an array of length \code{attrs\_sz}.
\end{itemize}

\paragraph{Returns:}
\begin{itemize}[noitemsep]
\item \code{status}\\
The status of the operation.  The client library will fill in this variable before returning this operation's request id from \code{hyperdex\_client\_loop}.  The pointer must remain valid until then, and the pointer should not be aliased to the status for any other outstanding operation.
\end{itemize}

%%%%%%%%%%%%%%%%%%%% atomic_div %%%%%%%%%%%%%%%%%%%%
\pagebreak
\subsubsection{\code{atomic\_div}}
\label{api:c:atomic_div}
\index{atomic\_div!C API}
Divide the existing value by the number specified for each attribute.

The division is atomic with the write.  If the object does not exist, the
operation will fail.

\noindent\textbf{Cost:}  Approximately one traversal of the value-dependent
chain.


\input{\topdir/api/shards/linearizable}


\paragraph{Definition:}
\begin{ccode}
int64_t hyperdex_client_atomic_div(struct hyperdex_client* client,
        const char* space,
        const char* key, size_t key_sz,
        const struct hyperdex_client_attribute* attrs, size_t attrs_sz,
        enum hyperdex_client_returncode* status);
\end{ccode}

\paragraph{Parameters:}
\begin{itemize}[noitemsep]
\item \code{space}\\
The name of the space as a c-string.
\item \code{key}, \code{key\_sz}\\
The key for the operation where \code{key} is a bytestring and \code{key\_sz} specifies the number of bytes in \code{key}.
\item \code{attrs}, \code{attrs\_sz}\\
The set of attributes to modify and their respective values.  \code{attrs} points to an array of length \code{attrs\_sz}.
\end{itemize}

\paragraph{Returns:}
\begin{itemize}[noitemsep]
\item \code{status}\\
The status of the operation.  The client library will fill in this variable before returning this operation's request id from \code{hyperdex\_client\_loop}.  The pointer must remain valid until then, and the pointer should not be aliased to the status for any other outstanding operation.
\end{itemize}

%%%%%%%%%%%%%%%%%%%% cond_atomic_div %%%%%%%%%%%%%%%%%%%%
\pagebreak
\subsubsection{\code{cond\_atomic\_div}}
\label{api:c:cond_atomic_div}
\index{cond\_atomic\_div!C API}
\input{\topdir/api/desc/cond_atomic_div}

\paragraph{Definition:}
\begin{ccode}
int64_t hyperdex_client_cond_atomic_div(struct hyperdex_client* client,
        const char* space,
        const char* key, size_t key_sz,
        const struct hyperdex_client_attribute_check* checks, size_t checks_sz,
        const struct hyperdex_client_attribute* attrs, size_t attrs_sz,
        enum hyperdex_client_returncode* status);
\end{ccode}

\paragraph{Parameters:}
\begin{itemize}[noitemsep]
\item \code{space}\\
The name of the space as a c-string.
\item \code{key}, \code{key\_sz}\\
The key for the operation where \code{key} is a bytestring and \code{key\_sz} specifies the number of bytes in \code{key}.
\item \code{checks}, \code{checks\_sz}\\
A set of predicates to check against.  \code{checks} points to an array of length \code{checks\_sz}.
\item \code{attrs}, \code{attrs\_sz}\\
The set of attributes to modify and their respective values.  \code{attrs} points to an array of length \code{attrs\_sz}.
\end{itemize}

\paragraph{Returns:}
\begin{itemize}[noitemsep]
\item \code{status}\\
The status of the operation.  The client library will fill in this variable before returning this operation's request id from \code{hyperdex\_client\_loop}.  The pointer must remain valid until then, and the pointer should not be aliased to the status for any other outstanding operation.
\end{itemize}

%%%%%%%%%%%%%%%%%%%% atomic_mod %%%%%%%%%%%%%%%%%%%%
\pagebreak
\subsubsection{\code{atomic\_mod}}
\label{api:c:atomic_mod}
\index{atomic\_mod!C API}
Store the existing value modulo the specified number for each attribute.
This operation requires a pre-existing object in order to complete successfully.
If no object exists, the operation will fail with \code{NOTFOUND}.



\paragraph{Definition:}
\begin{ccode}
int64_t hyperdex_client_atomic_mod(struct hyperdex_client* client,
        const char* space,
        const char* key, size_t key_sz,
        const struct hyperdex_client_attribute* attrs, size_t attrs_sz,
        enum hyperdex_client_returncode* status);
\end{ccode}

\paragraph{Parameters:}
\begin{itemize}[noitemsep]
\item \code{space}\\
The name of the space as a c-string.
\item \code{key}, \code{key\_sz}\\
The key for the operation where \code{key} is a bytestring and \code{key\_sz} specifies the number of bytes in \code{key}.
\item \code{attrs}, \code{attrs\_sz}\\
The set of attributes to modify and their respective values.  \code{attrs} points to an array of length \code{attrs\_sz}.
\end{itemize}

\paragraph{Returns:}
\begin{itemize}[noitemsep]
\item \code{status}\\
The status of the operation.  The client library will fill in this variable before returning this operation's request id from \code{hyperdex\_client\_loop}.  The pointer must remain valid until then, and the pointer should not be aliased to the status for any other outstanding operation.
\end{itemize}

%%%%%%%%%%%%%%%%%%%% cond_atomic_mod %%%%%%%%%%%%%%%%%%%%
\pagebreak
\subsubsection{\code{cond\_atomic\_mod}}
\label{api:c:cond_atomic_mod}
\index{cond\_atomic\_mod!C API}
Conditionally store the existing value modulo the specified number for each
attribute.

%%% Generated below here
\paragraph{Behavior:}
\begin{itemize}[noitemsep]
This operation requires a pre-existing object in order to complete successfully.
If no object exists, the operation will fail with \code{NOTFOUND}.

This operation will succeed if and only if the predicates specified by
\code{checks} hold on the pre-existing object.  If any of the predicates are not
true for the existing object, then the operation will have no effect and fail
with \code{CMPFAIL}.

All checks are atomic with the write.  HyperDex guarantees that no other
operation will come between validating the checks, and writing the new version
of the object.

\end{itemize}


\paragraph{Definition:}
\begin{ccode}
int64_t hyperdex_client_cond_atomic_mod(struct hyperdex_client* client,
        const char* space,
        const char* key, size_t key_sz,
        const struct hyperdex_client_attribute_check* checks, size_t checks_sz,
        const struct hyperdex_client_attribute* attrs, size_t attrs_sz,
        enum hyperdex_client_returncode* status);
\end{ccode}

\paragraph{Parameters:}
\begin{itemize}[noitemsep]
\item \code{space}\\
The name of the space as a c-string.
\item \code{key}, \code{key\_sz}\\
The key for the operation where \code{key} is a bytestring and \code{key\_sz} specifies the number of bytes in \code{key}.
\item \code{checks}, \code{checks\_sz}\\
A set of predicates to check against.  \code{checks} points to an array of length \code{checks\_sz}.
\item \code{attrs}, \code{attrs\_sz}\\
The set of attributes to modify and their respective values.  \code{attrs} points to an array of length \code{attrs\_sz}.
\end{itemize}

\paragraph{Returns:}
\begin{itemize}[noitemsep]
\item \code{status}\\
The status of the operation.  The client library will fill in this variable before returning this operation's request id from \code{hyperdex\_client\_loop}.  The pointer must remain valid until then, and the pointer should not be aliased to the status for any other outstanding operation.
\end{itemize}

%%%%%%%%%%%%%%%%%%%% atomic_and %%%%%%%%%%%%%%%%%%%%
\pagebreak
\subsubsection{\code{atomic\_and}}
\label{api:c:atomic_and}
\index{atomic\_and!C API}
Store the bitwise AND of the existing value and the specified number for
each attribute.
This operation requires a pre-existing object in order to complete successfully.
If no object exists, the operation will fail with \code{NOTFOUND}.



\paragraph{Definition:}
\begin{ccode}
int64_t hyperdex_client_atomic_and(struct hyperdex_client* client,
        const char* space,
        const char* key, size_t key_sz,
        const struct hyperdex_client_attribute* attrs, size_t attrs_sz,
        enum hyperdex_client_returncode* status);
\end{ccode}

\paragraph{Parameters:}
\begin{itemize}[noitemsep]
\item \code{space}\\
The name of the space as a c-string.
\item \code{key}, \code{key\_sz}\\
The key for the operation where \code{key} is a bytestring and \code{key\_sz} specifies the number of bytes in \code{key}.
\item \code{attrs}, \code{attrs\_sz}\\
The set of attributes to modify and their respective values.  \code{attrs} points to an array of length \code{attrs\_sz}.
\end{itemize}

\paragraph{Returns:}
\begin{itemize}[noitemsep]
\item \code{status}\\
The status of the operation.  The client library will fill in this variable before returning this operation's request id from \code{hyperdex\_client\_loop}.  The pointer must remain valid until then, and the pointer should not be aliased to the status for any other outstanding operation.
\end{itemize}

%%%%%%%%%%%%%%%%%%%% cond_atomic_and %%%%%%%%%%%%%%%%%%%%
\pagebreak
\subsubsection{\code{cond\_atomic\_and}}
\label{api:c:cond_atomic_and}
\index{cond\_atomic\_and!C API}
Conditionally store the bitwise AND of the existing value and the specified
number for each attribute.

%%% Generated below here
\paragraph{Behavior:}
\begin{itemize}[noitemsep]
This operation requires a pre-existing object in order to complete successfully.
If no object exists, the operation will fail with \code{NOTFOUND}.

This operation will succeed if and only if the predicates specified by
\code{checks} hold on the pre-existing object.  If any of the predicates are not
true for the existing object, then the operation will have no effect and fail
with \code{CMPFAIL}.

All checks are atomic with the write.  HyperDex guarantees that no other
operation will come between validating the checks, and writing the new version
of the object.

\end{itemize}


\paragraph{Definition:}
\begin{ccode}
int64_t hyperdex_client_cond_atomic_and(struct hyperdex_client* client,
        const char* space,
        const char* key, size_t key_sz,
        const struct hyperdex_client_attribute_check* checks, size_t checks_sz,
        const struct hyperdex_client_attribute* attrs, size_t attrs_sz,
        enum hyperdex_client_returncode* status);
\end{ccode}

\paragraph{Parameters:}
\begin{itemize}[noitemsep]
\item \code{space}\\
The name of the space as a c-string.
\item \code{key}, \code{key\_sz}\\
The key for the operation where \code{key} is a bytestring and \code{key\_sz} specifies the number of bytes in \code{key}.
\item \code{checks}, \code{checks\_sz}\\
A set of predicates to check against.  \code{checks} points to an array of length \code{checks\_sz}.
\item \code{attrs}, \code{attrs\_sz}\\
The set of attributes to modify and their respective values.  \code{attrs} points to an array of length \code{attrs\_sz}.
\end{itemize}

\paragraph{Returns:}
\begin{itemize}[noitemsep]
\item \code{status}\\
The status of the operation.  The client library will fill in this variable before returning this operation's request id from \code{hyperdex\_client\_loop}.  The pointer must remain valid until then, and the pointer should not be aliased to the status for any other outstanding operation.
\end{itemize}

%%%%%%%%%%%%%%%%%%%% atomic_or %%%%%%%%%%%%%%%%%%%%
\pagebreak
\subsubsection{\code{atomic\_or}}
\label{api:c:atomic_or}
\index{atomic\_or!C API}
Store the bitwise OR of the existing value and the specified number for each
attribute.

%%% Generated below here
\paragraph{Behavior:}
\begin{itemize}[noitemsep]
This operation requires a pre-existing object in order to complete successfully.
If no object exists, the operation will fail with \code{NOTFOUND}.

\end{itemize}


\paragraph{Definition:}
\begin{ccode}
int64_t hyperdex_client_atomic_or(struct hyperdex_client* client,
        const char* space,
        const char* key, size_t key_sz,
        const struct hyperdex_client_attribute* attrs, size_t attrs_sz,
        enum hyperdex_client_returncode* status);
\end{ccode}

\paragraph{Parameters:}
\begin{itemize}[noitemsep]
\item \code{space}\\
The name of the space as a c-string.
\item \code{key}, \code{key\_sz}\\
The key for the operation where \code{key} is a bytestring and \code{key\_sz} specifies the number of bytes in \code{key}.
\item \code{attrs}, \code{attrs\_sz}\\
The set of attributes to modify and their respective values.  \code{attrs} points to an array of length \code{attrs\_sz}.
\end{itemize}

\paragraph{Returns:}
\begin{itemize}[noitemsep]
\item \code{status}\\
The status of the operation.  The client library will fill in this variable before returning this operation's request id from \code{hyperdex\_client\_loop}.  The pointer must remain valid until then, and the pointer should not be aliased to the status for any other outstanding operation.
\end{itemize}

%%%%%%%%%%%%%%%%%%%% cond_atomic_or %%%%%%%%%%%%%%%%%%%%
\pagebreak
\subsubsection{\code{cond\_atomic\_or}}
\label{api:c:cond_atomic_or}
\index{cond\_atomic\_or!C API}
\input{\topdir/api/desc/cond_atomic_or}

\paragraph{Definition:}
\begin{ccode}
int64_t hyperdex_client_cond_atomic_or(struct hyperdex_client* client,
        const char* space,
        const char* key, size_t key_sz,
        const struct hyperdex_client_attribute_check* checks, size_t checks_sz,
        const struct hyperdex_client_attribute* attrs, size_t attrs_sz,
        enum hyperdex_client_returncode* status);
\end{ccode}

\paragraph{Parameters:}
\begin{itemize}[noitemsep]
\item \code{space}\\
The name of the space as a c-string.
\item \code{key}, \code{key\_sz}\\
The key for the operation where \code{key} is a bytestring and \code{key\_sz} specifies the number of bytes in \code{key}.
\item \code{checks}, \code{checks\_sz}\\
A set of predicates to check against.  \code{checks} points to an array of length \code{checks\_sz}.
\item \code{attrs}, \code{attrs\_sz}\\
The set of attributes to modify and their respective values.  \code{attrs} points to an array of length \code{attrs\_sz}.
\end{itemize}

\paragraph{Returns:}
\begin{itemize}[noitemsep]
\item \code{status}\\
The status of the operation.  The client library will fill in this variable before returning this operation's request id from \code{hyperdex\_client\_loop}.  The pointer must remain valid until then, and the pointer should not be aliased to the status for any other outstanding operation.
\end{itemize}

%%%%%%%%%%%%%%%%%%%% atomic_xor %%%%%%%%%%%%%%%%%%%%
\pagebreak
\subsubsection{\code{atomic\_xor}}
\label{api:c:atomic_xor}
\index{atomic\_xor!C API}
Store the bitwise XOR of the existing value and the specified number for each
attribute.
This operation requires a pre-existing object in order to complete successfully.
If no object exists, the operation will fail with \code{NOTFOUND}.



\paragraph{Definition:}
\begin{ccode}
int64_t hyperdex_client_atomic_xor(struct hyperdex_client* client,
        const char* space,
        const char* key, size_t key_sz,
        const struct hyperdex_client_attribute* attrs, size_t attrs_sz,
        enum hyperdex_client_returncode* status);
\end{ccode}

\paragraph{Parameters:}
\begin{itemize}[noitemsep]
\item \code{space}\\
The name of the space as a c-string.
\item \code{key}, \code{key\_sz}\\
The key for the operation where \code{key} is a bytestring and \code{key\_sz} specifies the number of bytes in \code{key}.
\item \code{attrs}, \code{attrs\_sz}\\
The set of attributes to modify and their respective values.  \code{attrs} points to an array of length \code{attrs\_sz}.
\end{itemize}

\paragraph{Returns:}
\begin{itemize}[noitemsep]
\item \code{status}\\
The status of the operation.  The client library will fill in this variable before returning this operation's request id from \code{hyperdex\_client\_loop}.  The pointer must remain valid until then, and the pointer should not be aliased to the status for any other outstanding operation.
\end{itemize}

%%%%%%%%%%%%%%%%%%%% cond_atomic_xor %%%%%%%%%%%%%%%%%%%%
\pagebreak
\subsubsection{\code{cond\_atomic\_xor}}
\label{api:c:cond_atomic_xor}
\index{cond\_atomic\_xor!C API}
Conditionally store the bitwise XOR of the existing value and the specified
number for each attribute.

%%% Generated below here
\paragraph{Behavior:}
\begin{itemize}[noitemsep]
This operation requires a pre-existing object in order to complete successfully.
If no object exists, the operation will fail with \code{NOTFOUND}.

This operation will succeed if and only if the predicates specified by
\code{checks} hold on the pre-existing object.  If any of the predicates are not
true for the existing object, then the operation will have no effect and fail
with \code{CMPFAIL}.

All checks are atomic with the write.  HyperDex guarantees that no other
operation will come between validating the checks, and writing the new version
of the object.

\end{itemize}


\paragraph{Definition:}
\begin{ccode}
int64_t hyperdex_client_cond_atomic_xor(struct hyperdex_client* client,
        const char* space,
        const char* key, size_t key_sz,
        const struct hyperdex_client_attribute_check* checks, size_t checks_sz,
        const struct hyperdex_client_attribute* attrs, size_t attrs_sz,
        enum hyperdex_client_returncode* status);
\end{ccode}

\paragraph{Parameters:}
\begin{itemize}[noitemsep]
\item \code{space}\\
The name of the space as a c-string.
\item \code{key}, \code{key\_sz}\\
The key for the operation where \code{key} is a bytestring and \code{key\_sz} specifies the number of bytes in \code{key}.
\item \code{checks}, \code{checks\_sz}\\
A set of predicates to check against.  \code{checks} points to an array of length \code{checks\_sz}.
\item \code{attrs}, \code{attrs\_sz}\\
The set of attributes to modify and their respective values.  \code{attrs} points to an array of length \code{attrs\_sz}.
\end{itemize}

\paragraph{Returns:}
\begin{itemize}[noitemsep]
\item \code{status}\\
The status of the operation.  The client library will fill in this variable before returning this operation's request id from \code{hyperdex\_client\_loop}.  The pointer must remain valid until then, and the pointer should not be aliased to the status for any other outstanding operation.
\end{itemize}

%%%%%%%%%%%%%%%%%%%% string_prepend %%%%%%%%%%%%%%%%%%%%
\pagebreak
\subsubsection{\code{string\_prepend}}
\label{api:c:string_prepend}
\index{string\_prepend!C API}
Prepend the specified string to the existing value for each attribute.

%%% Generated below here
\paragraph{Behavior:}
\begin{itemize}[noitemsep]
This operation requires a pre-existing object in order to complete successfully.
If no object exists, the operation will fail with \code{NOTFOUND}.

\end{itemize}


\paragraph{Definition:}
\begin{ccode}
int64_t hyperdex_client_string_prepend(struct hyperdex_client* client,
        const char* space,
        const char* key, size_t key_sz,
        const struct hyperdex_client_attribute* attrs, size_t attrs_sz,
        enum hyperdex_client_returncode* status);
\end{ccode}

\paragraph{Parameters:}
\begin{itemize}[noitemsep]
\item \code{space}\\
The name of the space as a c-string.
\item \code{key}, \code{key\_sz}\\
The key for the operation where \code{key} is a bytestring and \code{key\_sz} specifies the number of bytes in \code{key}.
\item \code{attrs}, \code{attrs\_sz}\\
The set of attributes to modify and their respective values.  \code{attrs} points to an array of length \code{attrs\_sz}.
\end{itemize}

\paragraph{Returns:}
\begin{itemize}[noitemsep]
\item \code{status}\\
The status of the operation.  The client library will fill in this variable before returning this operation's request id from \code{hyperdex\_client\_loop}.  The pointer must remain valid until then, and the pointer should not be aliased to the status for any other outstanding operation.
\end{itemize}

%%%%%%%%%%%%%%%%%%%% cond_string_prepend %%%%%%%%%%%%%%%%%%%%
\pagebreak
\subsubsection{\code{cond\_string\_prepend}}
\label{api:c:cond_string_prepend}
\index{cond\_string\_prepend!C API}
Conditionally prepend the specified string to the existing value for each
attribute.

%%% Generated below here
\paragraph{Behavior:}
\begin{itemize}[noitemsep]
This operation requires a pre-existing object in order to complete successfully.
If no object exists, the operation will fail with \code{NOTFOUND}.

This operation will succeed if and only if the predicates specified by
\code{checks} hold on the pre-existing object.  If any of the predicates are not
true for the existing object, then the operation will have no effect and fail
with \code{CMPFAIL}.

All checks are atomic with the write.  HyperDex guarantees that no other
operation will come between validating the checks, and writing the new version
of the object.

\end{itemize}


\paragraph{Definition:}
\begin{ccode}
int64_t hyperdex_client_cond_string_prepend(struct hyperdex_client* client,
        const char* space,
        const char* key, size_t key_sz,
        const struct hyperdex_client_attribute_check* checks, size_t checks_sz,
        const struct hyperdex_client_attribute* attrs, size_t attrs_sz,
        enum hyperdex_client_returncode* status);
\end{ccode}

\paragraph{Parameters:}
\begin{itemize}[noitemsep]
\item \code{space}\\
The name of the space as a c-string.
\item \code{key}, \code{key\_sz}\\
The key for the operation where \code{key} is a bytestring and \code{key\_sz} specifies the number of bytes in \code{key}.
\item \code{checks}, \code{checks\_sz}\\
A set of predicates to check against.  \code{checks} points to an array of length \code{checks\_sz}.
\item \code{attrs}, \code{attrs\_sz}\\
The set of attributes to modify and their respective values.  \code{attrs} points to an array of length \code{attrs\_sz}.
\end{itemize}

\paragraph{Returns:}
\begin{itemize}[noitemsep]
\item \code{status}\\
The status of the operation.  The client library will fill in this variable before returning this operation's request id from \code{hyperdex\_client\_loop}.  The pointer must remain valid until then, and the pointer should not be aliased to the status for any other outstanding operation.
\end{itemize}

%%%%%%%%%%%%%%%%%%%% string_append %%%%%%%%%%%%%%%%%%%%
\pagebreak
\subsubsection{\code{string\_append}}
\label{api:c:string_append}
\index{string\_append!C API}
Append the specified string to the existing value for each attribute.
This operation requires a pre-existing object in order to complete successfully.
If no object exists, the operation will fail with \code{NOTFOUND}.



\paragraph{Definition:}
\begin{ccode}
int64_t hyperdex_client_string_append(struct hyperdex_client* client,
        const char* space,
        const char* key, size_t key_sz,
        const struct hyperdex_client_attribute* attrs, size_t attrs_sz,
        enum hyperdex_client_returncode* status);
\end{ccode}

\paragraph{Parameters:}
\begin{itemize}[noitemsep]
\item \code{space}\\
The name of the space as a c-string.
\item \code{key}, \code{key\_sz}\\
The key for the operation where \code{key} is a bytestring and \code{key\_sz} specifies the number of bytes in \code{key}.
\item \code{attrs}, \code{attrs\_sz}\\
The set of attributes to modify and their respective values.  \code{attrs} points to an array of length \code{attrs\_sz}.
\end{itemize}

\paragraph{Returns:}
\begin{itemize}[noitemsep]
\item \code{status}\\
The status of the operation.  The client library will fill in this variable before returning this operation's request id from \code{hyperdex\_client\_loop}.  The pointer must remain valid until then, and the pointer should not be aliased to the status for any other outstanding operation.
\end{itemize}

%%%%%%%%%%%%%%%%%%%% cond_string_append %%%%%%%%%%%%%%%%%%%%
\pagebreak
\subsubsection{\code{cond\_string\_append}}
\label{api:c:cond_string_append}
\index{cond\_string\_append!C API}
Append the specified string to the existing value for each attribute if and only
if \code{checks} hold on the object.
This operation requires a pre-existing object in order to complete successfully.
If no object exists, the operation will fail with \code{NOTFOUND}.


This operation will succeed if and only if the predicates specified by
\code{checks} hold on the pre-existing object.  If any of the predicates are not
true for the existing object, then the operation will have no effect and fail
with \code{CMPFAIL}.

All checks are atomic with the write.  HyperDex guarantees that no other
operation will come between validating the checks, and writing the new version
of the object.



\paragraph{Definition:}
\begin{ccode}
int64_t hyperdex_client_cond_string_append(struct hyperdex_client* client,
        const char* space,
        const char* key, size_t key_sz,
        const struct hyperdex_client_attribute_check* checks, size_t checks_sz,
        const struct hyperdex_client_attribute* attrs, size_t attrs_sz,
        enum hyperdex_client_returncode* status);
\end{ccode}

\paragraph{Parameters:}
\begin{itemize}[noitemsep]
\item \code{space}\\
The name of the space as a c-string.
\item \code{key}, \code{key\_sz}\\
The key for the operation where \code{key} is a bytestring and \code{key\_sz} specifies the number of bytes in \code{key}.
\item \code{checks}, \code{checks\_sz}\\
A set of predicates to check against.  \code{checks} points to an array of length \code{checks\_sz}.
\item \code{attrs}, \code{attrs\_sz}\\
The set of attributes to modify and their respective values.  \code{attrs} points to an array of length \code{attrs\_sz}.
\end{itemize}

\paragraph{Returns:}
\begin{itemize}[noitemsep]
\item \code{status}\\
The status of the operation.  The client library will fill in this variable before returning this operation's request id from \code{hyperdex\_client\_loop}.  The pointer must remain valid until then, and the pointer should not be aliased to the status for any other outstanding operation.
\end{itemize}

%%%%%%%%%%%%%%%%%%%% list_lpush %%%%%%%%%%%%%%%%%%%%
\pagebreak
\subsubsection{\code{list\_lpush}}
\label{api:c:list_lpush}
\index{list\_lpush!C API}
Push the specified value onto the front of the list for each attribute.

%%% Generated below here
\paragraph{Behavior:}
\begin{itemize}[noitemsep]
This operation requires a pre-existing object in order to complete successfully.
If no object exists, the operation will fail with \code{NOTFOUND}.

\end{itemize}


\paragraph{Definition:}
\begin{ccode}
int64_t hyperdex_client_list_lpush(struct hyperdex_client* client,
        const char* space,
        const char* key, size_t key_sz,
        const struct hyperdex_client_attribute* attrs, size_t attrs_sz,
        enum hyperdex_client_returncode* status);
\end{ccode}

\paragraph{Parameters:}
\begin{itemize}[noitemsep]
\item \code{space}\\
The name of the space as a c-string.
\item \code{key}, \code{key\_sz}\\
The key for the operation where \code{key} is a bytestring and \code{key\_sz} specifies the number of bytes in \code{key}.
\item \code{attrs}, \code{attrs\_sz}\\
The set of attributes to modify and their respective values.  \code{attrs} points to an array of length \code{attrs\_sz}.
\end{itemize}

\paragraph{Returns:}
\begin{itemize}[noitemsep]
\item \code{status}\\
The status of the operation.  The client library will fill in this variable before returning this operation's request id from \code{hyperdex\_client\_loop}.  The pointer must remain valid until then, and the pointer should not be aliased to the status for any other outstanding operation.
\end{itemize}

%%%%%%%%%%%%%%%%%%%% cond_list_lpush %%%%%%%%%%%%%%%%%%%%
\pagebreak
\subsubsection{\code{cond\_list\_lpush}}
\label{api:c:cond_list_lpush}
\index{cond\_list\_lpush!C API}
Condtitionally push the specified value onto the front of the list for each
attribute.

%%% Generated below here
\paragraph{Behavior:}
\begin{itemize}[noitemsep]
This operation requires a pre-existing object in order to complete successfully.
If no object exists, the operation will fail with \code{NOTFOUND}.

This operation will succeed if and only if the predicates specified by
\code{checks} hold on the pre-existing object.  If any of the predicates are not
true for the existing object, then the operation will have no effect and fail
with \code{CMPFAIL}.

All checks are atomic with the write.  HyperDex guarantees that no other
operation will come between validating the checks, and writing the new version
of the object.

\end{itemize}


\paragraph{Definition:}
\begin{ccode}
int64_t hyperdex_client_cond_list_lpush(struct hyperdex_client* client,
        const char* space,
        const char* key, size_t key_sz,
        const struct hyperdex_client_attribute_check* checks, size_t checks_sz,
        const struct hyperdex_client_attribute* attrs, size_t attrs_sz,
        enum hyperdex_client_returncode* status);
\end{ccode}

\paragraph{Parameters:}
\begin{itemize}[noitemsep]
\item \code{space}\\
The name of the space as a c-string.
\item \code{key}, \code{key\_sz}\\
The key for the operation where \code{key} is a bytestring and \code{key\_sz} specifies the number of bytes in \code{key}.
\item \code{checks}, \code{checks\_sz}\\
A set of predicates to check against.  \code{checks} points to an array of length \code{checks\_sz}.
\item \code{attrs}, \code{attrs\_sz}\\
The set of attributes to modify and their respective values.  \code{attrs} points to an array of length \code{attrs\_sz}.
\end{itemize}

\paragraph{Returns:}
\begin{itemize}[noitemsep]
\item \code{status}\\
The status of the operation.  The client library will fill in this variable before returning this operation's request id from \code{hyperdex\_client\_loop}.  The pointer must remain valid until then, and the pointer should not be aliased to the status for any other outstanding operation.
\end{itemize}

%%%%%%%%%%%%%%%%%%%% list_rpush %%%%%%%%%%%%%%%%%%%%
\pagebreak
\subsubsection{\code{list\_rpush}}
\label{api:c:list_rpush}
\index{list\_rpush!C API}
Push the specified value onto the back of the list for each attribute.

%%% Generated below here
\paragraph{Behavior:}
\begin{itemize}[noitemsep]
This operation requires a pre-existing object in order to complete successfully.
If no object exists, the operation will fail with \code{NOTFOUND}.

\end{itemize}


\paragraph{Definition:}
\begin{ccode}
int64_t hyperdex_client_list_rpush(struct hyperdex_client* client,
        const char* space,
        const char* key, size_t key_sz,
        const struct hyperdex_client_attribute* attrs, size_t attrs_sz,
        enum hyperdex_client_returncode* status);
\end{ccode}

\paragraph{Parameters:}
\begin{itemize}[noitemsep]
\item \code{space}\\
The name of the space as a c-string.
\item \code{key}, \code{key\_sz}\\
The key for the operation where \code{key} is a bytestring and \code{key\_sz} specifies the number of bytes in \code{key}.
\item \code{attrs}, \code{attrs\_sz}\\
The set of attributes to modify and their respective values.  \code{attrs} points to an array of length \code{attrs\_sz}.
\end{itemize}

\paragraph{Returns:}
\begin{itemize}[noitemsep]
\item \code{status}\\
The status of the operation.  The client library will fill in this variable before returning this operation's request id from \code{hyperdex\_client\_loop}.  The pointer must remain valid until then, and the pointer should not be aliased to the status for any other outstanding operation.
\end{itemize}

%%%%%%%%%%%%%%%%%%%% cond_list_rpush %%%%%%%%%%%%%%%%%%%%
\pagebreak
\subsubsection{\code{cond\_list\_rpush}}
\label{api:c:cond_list_rpush}
\index{cond\_list\_rpush!C API}
Push the specified value onto the back of the list for each attribute if and
only if the \code{checks} hold on the object.
This operation requires a pre-existing object in order to complete successfully.
If no object exists, the operation will fail with \code{NOTFOUND}.


This operation will succeed if and only if the predicates specified by
\code{checks} hold on the pre-existing object.  If any of the predicates are not
true for the existing object, then the operation will have no effect and fail
with \code{CMPFAIL}.

All checks are atomic with the write.  HyperDex guarantees that no other
operation will come between validating the checks, and writing the new version
of the object.



\paragraph{Definition:}
\begin{ccode}
int64_t hyperdex_client_cond_list_rpush(struct hyperdex_client* client,
        const char* space,
        const char* key, size_t key_sz,
        const struct hyperdex_client_attribute_check* checks, size_t checks_sz,
        const struct hyperdex_client_attribute* attrs, size_t attrs_sz,
        enum hyperdex_client_returncode* status);
\end{ccode}

\paragraph{Parameters:}
\begin{itemize}[noitemsep]
\item \code{space}\\
The name of the space as a c-string.
\item \code{key}, \code{key\_sz}\\
The key for the operation where \code{key} is a bytestring and \code{key\_sz} specifies the number of bytes in \code{key}.
\item \code{checks}, \code{checks\_sz}\\
A set of predicates to check against.  \code{checks} points to an array of length \code{checks\_sz}.
\item \code{attrs}, \code{attrs\_sz}\\
The set of attributes to modify and their respective values.  \code{attrs} points to an array of length \code{attrs\_sz}.
\end{itemize}

\paragraph{Returns:}
\begin{itemize}[noitemsep]
\item \code{status}\\
The status of the operation.  The client library will fill in this variable before returning this operation's request id from \code{hyperdex\_client\_loop}.  The pointer must remain valid until then, and the pointer should not be aliased to the status for any other outstanding operation.
\end{itemize}

%%%%%%%%%%%%%%%%%%%% set_add %%%%%%%%%%%%%%%%%%%%
\pagebreak
\subsubsection{\code{set\_add}}
\label{api:c:set_add}
\index{set\_add!C API}
Add the specified value to the set for each attribute.

%%% Generated below here
\paragraph{Behavior:}
\begin{itemize}[noitemsep]
This operation requires a pre-existing object in order to complete successfully.
If no object exists, the operation will fail with \code{NOTFOUND}.

\end{itemize}


\paragraph{Definition:}
\begin{ccode}
int64_t hyperdex_client_set_add(struct hyperdex_client* client,
        const char* space,
        const char* key, size_t key_sz,
        const struct hyperdex_client_attribute* attrs, size_t attrs_sz,
        enum hyperdex_client_returncode* status);
\end{ccode}

\paragraph{Parameters:}
\begin{itemize}[noitemsep]
\item \code{space}\\
The name of the space as a c-string.
\item \code{key}, \code{key\_sz}\\
The key for the operation where \code{key} is a bytestring and \code{key\_sz} specifies the number of bytes in \code{key}.
\item \code{attrs}, \code{attrs\_sz}\\
The set of attributes to modify and their respective values.  \code{attrs} points to an array of length \code{attrs\_sz}.
\end{itemize}

\paragraph{Returns:}
\begin{itemize}[noitemsep]
\item \code{status}\\
The status of the operation.  The client library will fill in this variable before returning this operation's request id from \code{hyperdex\_client\_loop}.  The pointer must remain valid until then, and the pointer should not be aliased to the status for any other outstanding operation.
\end{itemize}

%%%%%%%%%%%%%%%%%%%% cond_set_add %%%%%%%%%%%%%%%%%%%%
\pagebreak
\subsubsection{\code{cond\_set\_add}}
\label{api:c:cond_set_add}
\index{cond\_set\_add!C API}
Conditionally add the specified value to the set for each attribute.

%%% Generated below here
\paragraph{Behavior:}
\begin{itemize}[noitemsep]
This operation requires a pre-existing object in order to complete successfully.
If no object exists, the operation will fail with \code{NOTFOUND}.

This operation will succeed if and only if the predicates specified by
\code{checks} hold on the pre-existing object.  If any of the predicates are not
true for the existing object, then the operation will have no effect and fail
with \code{CMPFAIL}.

All checks are atomic with the write.  HyperDex guarantees that no other
operation will come between validating the checks, and writing the new version
of the object.

\end{itemize}


\paragraph{Definition:}
\begin{ccode}
int64_t hyperdex_client_cond_set_add(struct hyperdex_client* client,
        const char* space,
        const char* key, size_t key_sz,
        const struct hyperdex_client_attribute_check* checks, size_t checks_sz,
        const struct hyperdex_client_attribute* attrs, size_t attrs_sz,
        enum hyperdex_client_returncode* status);
\end{ccode}

\paragraph{Parameters:}
\begin{itemize}[noitemsep]
\item \code{space}\\
The name of the space as a c-string.
\item \code{key}, \code{key\_sz}\\
The key for the operation where \code{key} is a bytestring and \code{key\_sz} specifies the number of bytes in \code{key}.
\item \code{checks}, \code{checks\_sz}\\
A set of predicates to check against.  \code{checks} points to an array of length \code{checks\_sz}.
\item \code{attrs}, \code{attrs\_sz}\\
The set of attributes to modify and their respective values.  \code{attrs} points to an array of length \code{attrs\_sz}.
\end{itemize}

\paragraph{Returns:}
\begin{itemize}[noitemsep]
\item \code{status}\\
The status of the operation.  The client library will fill in this variable before returning this operation's request id from \code{hyperdex\_client\_loop}.  The pointer must remain valid until then, and the pointer should not be aliased to the status for any other outstanding operation.
\end{itemize}

%%%%%%%%%%%%%%%%%%%% set_remove %%%%%%%%%%%%%%%%%%%%
\pagebreak
\subsubsection{\code{set\_remove}}
\label{api:c:set_remove}
\index{set\_remove!C API}
Remove the specified value from the set.  If the value is not contained within
the set, this operation will do nothing.

%%% Generated below here
\paragraph{Behavior:}
\begin{itemize}[noitemsep]
This operation requires a pre-existing object in order to complete successfully.
If no object exists, the operation will fail with \code{NOTFOUND}.

\end{itemize}


\paragraph{Definition:}
\begin{ccode}
int64_t hyperdex_client_set_remove(struct hyperdex_client* client,
        const char* space,
        const char* key, size_t key_sz,
        const struct hyperdex_client_attribute* attrs, size_t attrs_sz,
        enum hyperdex_client_returncode* status);
\end{ccode}

\paragraph{Parameters:}
\begin{itemize}[noitemsep]
\item \code{space}\\
The name of the space as a c-string.
\item \code{key}, \code{key\_sz}\\
The key for the operation where \code{key} is a bytestring and \code{key\_sz} specifies the number of bytes in \code{key}.
\item \code{attrs}, \code{attrs\_sz}\\
The set of attributes to modify and their respective values.  \code{attrs} points to an array of length \code{attrs\_sz}.
\end{itemize}

\paragraph{Returns:}
\begin{itemize}[noitemsep]
\item \code{status}\\
The status of the operation.  The client library will fill in this variable before returning this operation's request id from \code{hyperdex\_client\_loop}.  The pointer must remain valid until then, and the pointer should not be aliased to the status for any other outstanding operation.
\end{itemize}

%%%%%%%%%%%%%%%%%%%% cond_set_remove %%%%%%%%%%%%%%%%%%%%
\pagebreak
\subsubsection{\code{cond\_set\_remove}}
\label{api:c:cond_set_remove}
\index{cond\_set\_remove!C API}
Conditionally remove the specified value from the set.  If the value is not
contained within the set, this operation will do nothing.

%%% Generated below here
\paragraph{Behavior:}
\begin{itemize}[noitemsep]
This operation requires a pre-existing object in order to complete successfully.
If no object exists, the operation will fail with \code{NOTFOUND}.

This operation will succeed if and only if the predicates specified by
\code{checks} hold on the pre-existing object.  If any of the predicates are not
true for the existing object, then the operation will have no effect and fail
with \code{CMPFAIL}.

All checks are atomic with the write.  HyperDex guarantees that no other
operation will come between validating the checks, and writing the new version
of the object.

\end{itemize}


\paragraph{Definition:}
\begin{ccode}
int64_t hyperdex_client_cond_set_remove(struct hyperdex_client* client,
        const char* space,
        const char* key, size_t key_sz,
        const struct hyperdex_client_attribute_check* checks, size_t checks_sz,
        const struct hyperdex_client_attribute* attrs, size_t attrs_sz,
        enum hyperdex_client_returncode* status);
\end{ccode}

\paragraph{Parameters:}
\begin{itemize}[noitemsep]
\item \code{space}\\
The name of the space as a c-string.
\item \code{key}, \code{key\_sz}\\
The key for the operation where \code{key} is a bytestring and \code{key\_sz} specifies the number of bytes in \code{key}.
\item \code{checks}, \code{checks\_sz}\\
A set of predicates to check against.  \code{checks} points to an array of length \code{checks\_sz}.
\item \code{attrs}, \code{attrs\_sz}\\
The set of attributes to modify and their respective values.  \code{attrs} points to an array of length \code{attrs\_sz}.
\end{itemize}

\paragraph{Returns:}
\begin{itemize}[noitemsep]
\item \code{status}\\
The status of the operation.  The client library will fill in this variable before returning this operation's request id from \code{hyperdex\_client\_loop}.  The pointer must remain valid until then, and the pointer should not be aliased to the status for any other outstanding operation.
\end{itemize}

%%%%%%%%%%%%%%%%%%%% set_intersect %%%%%%%%%%%%%%%%%%%%
\pagebreak
\subsubsection{\code{set\_intersect}}
\label{api:c:set_intersect}
\index{set\_intersect!C API}
Store the intersection of the specified set and the existing value for each
attribute.

%%% Generated below here
\paragraph{Behavior:}
\begin{itemize}[noitemsep]
This operation requires a pre-existing object in order to complete successfully.
If no object exists, the operation will fail with \code{NOTFOUND}.

\end{itemize}


\paragraph{Definition:}
\begin{ccode}
int64_t hyperdex_client_set_intersect(struct hyperdex_client* client,
        const char* space,
        const char* key, size_t key_sz,
        const struct hyperdex_client_attribute* attrs, size_t attrs_sz,
        enum hyperdex_client_returncode* status);
\end{ccode}

\paragraph{Parameters:}
\begin{itemize}[noitemsep]
\item \code{space}\\
The name of the space as a c-string.
\item \code{key}, \code{key\_sz}\\
The key for the operation where \code{key} is a bytestring and \code{key\_sz} specifies the number of bytes in \code{key}.
\item \code{attrs}, \code{attrs\_sz}\\
The set of attributes to modify and their respective values.  \code{attrs} points to an array of length \code{attrs\_sz}.
\end{itemize}

\paragraph{Returns:}
\begin{itemize}[noitemsep]
\item \code{status}\\
The status of the operation.  The client library will fill in this variable before returning this operation's request id from \code{hyperdex\_client\_loop}.  The pointer must remain valid until then, and the pointer should not be aliased to the status for any other outstanding operation.
\end{itemize}

%%%%%%%%%%%%%%%%%%%% cond_set_intersect %%%%%%%%%%%%%%%%%%%%
\pagebreak
\subsubsection{\code{cond\_set\_intersect}}
\label{api:c:cond_set_intersect}
\index{cond\_set\_intersect!C API}
\input{\topdir/api/desc/cond_set_intersect}

\paragraph{Definition:}
\begin{ccode}
int64_t hyperdex_client_cond_set_intersect(struct hyperdex_client* client,
        const char* space,
        const char* key, size_t key_sz,
        const struct hyperdex_client_attribute_check* checks, size_t checks_sz,
        const struct hyperdex_client_attribute* attrs, size_t attrs_sz,
        enum hyperdex_client_returncode* status);
\end{ccode}

\paragraph{Parameters:}
\begin{itemize}[noitemsep]
\item \code{space}\\
The name of the space as a c-string.
\item \code{key}, \code{key\_sz}\\
The key for the operation where \code{key} is a bytestring and \code{key\_sz} specifies the number of bytes in \code{key}.
\item \code{checks}, \code{checks\_sz}\\
A set of predicates to check against.  \code{checks} points to an array of length \code{checks\_sz}.
\item \code{attrs}, \code{attrs\_sz}\\
The set of attributes to modify and their respective values.  \code{attrs} points to an array of length \code{attrs\_sz}.
\end{itemize}

\paragraph{Returns:}
\begin{itemize}[noitemsep]
\item \code{status}\\
The status of the operation.  The client library will fill in this variable before returning this operation's request id from \code{hyperdex\_client\_loop}.  The pointer must remain valid until then, and the pointer should not be aliased to the status for any other outstanding operation.
\end{itemize}

%%%%%%%%%%%%%%%%%%%% set_union %%%%%%%%%%%%%%%%%%%%
\pagebreak
\subsubsection{\code{set\_union}}
\label{api:c:set_union}
\index{set\_union!C API}
Store the union of the specified set and the existing value for each attribute.

%%% Generated below here
\paragraph{Behavior:}
\begin{itemize}[noitemsep]
This operation requires a pre-existing object in order to complete successfully.
If no object exists, the operation will fail with \code{NOTFOUND}.

\end{itemize}


\paragraph{Definition:}
\begin{ccode}
int64_t hyperdex_client_set_union(struct hyperdex_client* client,
        const char* space,
        const char* key, size_t key_sz,
        const struct hyperdex_client_attribute* attrs, size_t attrs_sz,
        enum hyperdex_client_returncode* status);
\end{ccode}

\paragraph{Parameters:}
\begin{itemize}[noitemsep]
\item \code{space}\\
The name of the space as a c-string.
\item \code{key}, \code{key\_sz}\\
The key for the operation where \code{key} is a bytestring and \code{key\_sz} specifies the number of bytes in \code{key}.
\item \code{attrs}, \code{attrs\_sz}\\
The set of attributes to modify and their respective values.  \code{attrs} points to an array of length \code{attrs\_sz}.
\end{itemize}

\paragraph{Returns:}
\begin{itemize}[noitemsep]
\item \code{status}\\
The status of the operation.  The client library will fill in this variable before returning this operation's request id from \code{hyperdex\_client\_loop}.  The pointer must remain valid until then, and the pointer should not be aliased to the status for any other outstanding operation.
\end{itemize}

%%%%%%%%%%%%%%%%%%%% cond_set_union %%%%%%%%%%%%%%%%%%%%
\pagebreak
\subsubsection{\code{cond\_set\_union}}
\label{api:c:cond_set_union}
\index{cond\_set\_union!C API}
Conditionally store the union of the specified set and the existing value for
each attribute.

%%% Generated below here
\paragraph{Behavior:}
\begin{itemize}[noitemsep]
This operation requires a pre-existing object in order to complete successfully.
If no object exists, the operation will fail with \code{NOTFOUND}.

This operation will succeed if and only if the predicates specified by
\code{checks} hold on the pre-existing object.  If any of the predicates are not
true for the existing object, then the operation will have no effect and fail
with \code{CMPFAIL}.

All checks are atomic with the write.  HyperDex guarantees that no other
operation will come between validating the checks, and writing the new version
of the object.

\end{itemize}


\paragraph{Definition:}
\begin{ccode}
int64_t hyperdex_client_cond_set_union(struct hyperdex_client* client,
        const char* space,
        const char* key, size_t key_sz,
        const struct hyperdex_client_attribute_check* checks, size_t checks_sz,
        const struct hyperdex_client_attribute* attrs, size_t attrs_sz,
        enum hyperdex_client_returncode* status);
\end{ccode}

\paragraph{Parameters:}
\begin{itemize}[noitemsep]
\item \code{space}\\
The name of the space as a c-string.
\item \code{key}, \code{key\_sz}\\
The key for the operation where \code{key} is a bytestring and \code{key\_sz} specifies the number of bytes in \code{key}.
\item \code{checks}, \code{checks\_sz}\\
A set of predicates to check against.  \code{checks} points to an array of length \code{checks\_sz}.
\item \code{attrs}, \code{attrs\_sz}\\
The set of attributes to modify and their respective values.  \code{attrs} points to an array of length \code{attrs\_sz}.
\end{itemize}

\paragraph{Returns:}
\begin{itemize}[noitemsep]
\item \code{status}\\
The status of the operation.  The client library will fill in this variable before returning this operation's request id from \code{hyperdex\_client\_loop}.  The pointer must remain valid until then, and the pointer should not be aliased to the status for any other outstanding operation.
\end{itemize}

%%%%%%%%%%%%%%%%%%%% map_add %%%%%%%%%%%%%%%%%%%%
\pagebreak
\subsubsection{\code{map\_add}}
\label{api:c:map_add}
\index{map\_add!C API}
Insert a key-value pair into the map specified by each map-attribute.

%%% Generated below here
\paragraph{Behavior:}
\begin{itemize}[noitemsep]
This operation requires a pre-existing object in order to complete successfully.
If no object exists, the operation will fail with \code{NOTFOUND}.

\end{itemize}


\paragraph{Definition:}
\begin{ccode}
int64_t hyperdex_client_map_add(struct hyperdex_client* client,
        const char* space,
        const char* key, size_t key_sz,
        const struct hyperdex_client_map_attribute* mapattrs, size_t mapattrs_sz,
        enum hyperdex_client_returncode* status);
\end{ccode}

\paragraph{Parameters:}
\begin{itemize}[noitemsep]
\item \code{space}\\
The name of the space as a c-string.
\item \code{key}, \code{key\_sz}\\
The key for the operation where \code{key} is a bytestring and \code{key\_sz} specifies the number of bytes in \code{key}.
\item \code{mapattrs}, \code{mapattrs\_sz}\\
The set of map attributes to modify and their respective key/values.  \code{mapattrs} points to an array of length \code{mapattrs\_sz}.  Each entry specify an attribute that is a map and a key within that map.
\end{itemize}

\paragraph{Returns:}
\begin{itemize}[noitemsep]
\item \code{status}\\
The status of the operation.  The client library will fill in this variable before returning this operation's request id from \code{hyperdex\_client\_loop}.  The pointer must remain valid until then, and the pointer should not be aliased to the status for any other outstanding operation.
\end{itemize}

%%%%%%%%%%%%%%%%%%%% cond_map_add %%%%%%%%%%%%%%%%%%%%
\pagebreak
\subsubsection{\code{cond\_map\_add}}
\label{api:c:cond_map_add}
\index{cond\_map\_add!C API}
Conditionally insert a key-value pair into the map specified by each
map-attribute.

%%% Generated below here
\paragraph{Behavior:}
\begin{itemize}[noitemsep]
This operation requires a pre-existing object in order to complete successfully.
If no object exists, the operation will fail with \code{NOTFOUND}.

This operation will succeed if and only if the predicates specified by
\code{checks} hold on the pre-existing object.  If any of the predicates are not
true for the existing object, then the operation will have no effect and fail
with \code{CMPFAIL}.

All checks are atomic with the write.  HyperDex guarantees that no other
operation will come between validating the checks, and writing the new version
of the object.

\end{itemize}


\paragraph{Definition:}
\begin{ccode}
int64_t hyperdex_client_cond_map_add(struct hyperdex_client* client,
        const char* space,
        const char* key, size_t key_sz,
        const struct hyperdex_client_attribute_check* checks, size_t checks_sz,
        const struct hyperdex_client_map_attribute* mapattrs, size_t mapattrs_sz,
        enum hyperdex_client_returncode* status);
\end{ccode}

\paragraph{Parameters:}
\begin{itemize}[noitemsep]
\item \code{space}\\
The name of the space as a c-string.
\item \code{key}, \code{key\_sz}\\
The key for the operation where \code{key} is a bytestring and \code{key\_sz} specifies the number of bytes in \code{key}.
\item \code{checks}, \code{checks\_sz}\\
A set of predicates to check against.  \code{checks} points to an array of length \code{checks\_sz}.
\item \code{mapattrs}, \code{mapattrs\_sz}\\
The set of map attributes to modify and their respective key/values.  \code{mapattrs} points to an array of length \code{mapattrs\_sz}.  Each entry specify an attribute that is a map and a key within that map.
\end{itemize}

\paragraph{Returns:}
\begin{itemize}[noitemsep]
\item \code{status}\\
The status of the operation.  The client library will fill in this variable before returning this operation's request id from \code{hyperdex\_client\_loop}.  The pointer must remain valid until then, and the pointer should not be aliased to the status for any other outstanding operation.
\end{itemize}

%%%%%%%%%%%%%%%%%%%% map_remove %%%%%%%%%%%%%%%%%%%%
\pagebreak
\subsubsection{\code{map\_remove}}
\label{api:c:map_remove}
\index{map\_remove!C API}
Remove a key-value pair from the map specified by each attribute.  If there is
no pair with the specified key within the map, this operation will do nothing.
This operation requires a pre-existing object in order to complete successfully.
If no object exists, the operation will fail with \code{NOTFOUND}.



\paragraph{Definition:}
\begin{ccode}
int64_t hyperdex_client_map_remove(struct hyperdex_client* client,
        const char* space,
        const char* key, size_t key_sz,
        const struct hyperdex_client_attribute* attrs, size_t attrs_sz,
        enum hyperdex_client_returncode* status);
\end{ccode}

\paragraph{Parameters:}
\begin{itemize}[noitemsep]
\item \code{space}\\
The name of the space as a c-string.
\item \code{key}, \code{key\_sz}\\
The key for the operation where \code{key} is a bytestring and \code{key\_sz} specifies the number of bytes in \code{key}.
\item \code{attrs}, \code{attrs\_sz}\\
The set of attributes to modify and their respective values.  \code{attrs} points to an array of length \code{attrs\_sz}.
\end{itemize}

\paragraph{Returns:}
\begin{itemize}[noitemsep]
\item \code{status}\\
The status of the operation.  The client library will fill in this variable before returning this operation's request id from \code{hyperdex\_client\_loop}.  The pointer must remain valid until then, and the pointer should not be aliased to the status for any other outstanding operation.
\end{itemize}

%%%%%%%%%%%%%%%%%%%% cond_map_remove %%%%%%%%%%%%%%%%%%%%
\pagebreak
\subsubsection{\code{cond\_map\_remove}}
\label{api:c:cond_map_remove}
\index{cond\_map\_remove!C API}
Remove a key-value pair from the map specified by each attribute if and only if
\code{checks} hold on the object.  If there is no pair with the specified key
within the map, this operation will do nothing.
This operation requires a pre-existing object in order to complete successfully.
If no object exists, the operation will fail with \code{NOTFOUND}.


This operation will succeed if and only if the predicates specified by
\code{checks} hold on the pre-existing object.  If any of the predicates are not
true for the existing object, then the operation will have no effect and fail
with \code{CMPFAIL}.

All checks are atomic with the write.  HyperDex guarantees that no other
operation will come between validating the checks, and writing the new version
of the object.



\paragraph{Definition:}
\begin{ccode}
int64_t hyperdex_client_cond_map_remove(struct hyperdex_client* client,
        const char* space,
        const char* key, size_t key_sz,
        const struct hyperdex_client_attribute_check* checks, size_t checks_sz,
        const struct hyperdex_client_attribute* attrs, size_t attrs_sz,
        enum hyperdex_client_returncode* status);
\end{ccode}

\paragraph{Parameters:}
\begin{itemize}[noitemsep]
\item \code{space}\\
The name of the space as a c-string.
\item \code{key}, \code{key\_sz}\\
The key for the operation where \code{key} is a bytestring and \code{key\_sz} specifies the number of bytes in \code{key}.
\item \code{checks}, \code{checks\_sz}\\
A set of predicates to check against.  \code{checks} points to an array of length \code{checks\_sz}.
\item \code{attrs}, \code{attrs\_sz}\\
The set of attributes to modify and their respective values.  \code{attrs} points to an array of length \code{attrs\_sz}.
\end{itemize}

\paragraph{Returns:}
\begin{itemize}[noitemsep]
\item \code{status}\\
The status of the operation.  The client library will fill in this variable before returning this operation's request id from \code{hyperdex\_client\_loop}.  The pointer must remain valid until then, and the pointer should not be aliased to the status for any other outstanding operation.
\end{itemize}

%%%%%%%%%%%%%%%%%%%% map_atomic_add %%%%%%%%%%%%%%%%%%%%
\pagebreak
\subsubsection{\code{map\_atomic\_add}}
\label{api:c:map_atomic_add}
\index{map\_atomic\_add!C API}
Add the specified number to the value of a key-value pair within each map.

%%% Generated below here
\paragraph{Behavior:}
\begin{itemize}[noitemsep]
This operation requires a pre-existing object in order to complete successfully.
If no object exists, the operation will fail with \code{NOTFOUND}.

\item This operation mutates the value of a key-value pair in a map.  This call
    is similar to the equivalent call without the \code{map\_} prefix, but
    operates on the value of a pair in a map, instead of on an attribute's
    value.  If there is no pair with the specified map key, a new pair will be
    created and initialized to its default value.  If this is undesirable, it
    may be avoided by using a conditional operation that requires that the map
    contain the key in question.

\end{itemize}


\paragraph{Definition:}
\begin{ccode}
int64_t hyperdex_client_map_atomic_add(struct hyperdex_client* client,
        const char* space,
        const char* key, size_t key_sz,
        const struct hyperdex_client_map_attribute* mapattrs, size_t mapattrs_sz,
        enum hyperdex_client_returncode* status);
\end{ccode}

\paragraph{Parameters:}
\begin{itemize}[noitemsep]
\item \code{space}\\
The name of the space as a c-string.
\item \code{key}, \code{key\_sz}\\
The key for the operation where \code{key} is a bytestring and \code{key\_sz} specifies the number of bytes in \code{key}.
\item \code{mapattrs}, \code{mapattrs\_sz}\\
The set of map attributes to modify and their respective key/values.  \code{mapattrs} points to an array of length \code{mapattrs\_sz}.  Each entry specify an attribute that is a map and a key within that map.
\end{itemize}

\paragraph{Returns:}
\begin{itemize}[noitemsep]
\item \code{status}\\
The status of the operation.  The client library will fill in this variable before returning this operation's request id from \code{hyperdex\_client\_loop}.  The pointer must remain valid until then, and the pointer should not be aliased to the status for any other outstanding operation.
\end{itemize}

%%%%%%%%%%%%%%%%%%%% cond_map_atomic_add %%%%%%%%%%%%%%%%%%%%
\pagebreak
\subsubsection{\code{cond\_map\_atomic\_add}}
\label{api:c:cond_map_atomic_add}
\index{cond\_map\_atomic\_add!C API}
Conditionally add the specified number to the value of a key-value pair within
each map.

%%% Generated below here
\paragraph{Behavior:}
\begin{itemize}[noitemsep]
This operation requires a pre-existing object in order to complete successfully.
If no object exists, the operation will fail with \code{NOTFOUND}.

This operation will succeed if and only if the predicates specified by
\code{checks} hold on the pre-existing object.  If any of the predicates are not
true for the existing object, then the operation will have no effect and fail
with \code{CMPFAIL}.

All checks are atomic with the write.  HyperDex guarantees that no other
operation will come between validating the checks, and writing the new version
of the object.

\item This operation mutates the value of a key-value pair in a map.  This call
    is similar to the equivalent call without the \code{map\_} prefix, but
    operates on the value of a pair in a map, instead of on an attribute's
    value.  If there is no pair with the specified map key, a new pair will be
    created and initialized to its default value.  If this is undesirable, it
    may be avoided by using a conditional operation that requires that the map
    contain the key in question.

\end{itemize}


\paragraph{Definition:}
\begin{ccode}
int64_t hyperdex_client_cond_map_atomic_add(struct hyperdex_client* client,
        const char* space,
        const char* key, size_t key_sz,
        const struct hyperdex_client_attribute_check* checks, size_t checks_sz,
        const struct hyperdex_client_map_attribute* mapattrs, size_t mapattrs_sz,
        enum hyperdex_client_returncode* status);
\end{ccode}

\paragraph{Parameters:}
\begin{itemize}[noitemsep]
\item \code{space}\\
The name of the space as a c-string.
\item \code{key}, \code{key\_sz}\\
The key for the operation where \code{key} is a bytestring and \code{key\_sz} specifies the number of bytes in \code{key}.
\item \code{checks}, \code{checks\_sz}\\
A set of predicates to check against.  \code{checks} points to an array of length \code{checks\_sz}.
\item \code{mapattrs}, \code{mapattrs\_sz}\\
The set of map attributes to modify and their respective key/values.  \code{mapattrs} points to an array of length \code{mapattrs\_sz}.  Each entry specify an attribute that is a map and a key within that map.
\end{itemize}

\paragraph{Returns:}
\begin{itemize}[noitemsep]
\item \code{status}\\
The status of the operation.  The client library will fill in this variable before returning this operation's request id from \code{hyperdex\_client\_loop}.  The pointer must remain valid until then, and the pointer should not be aliased to the status for any other outstanding operation.
\end{itemize}

%%%%%%%%%%%%%%%%%%%% map_atomic_sub %%%%%%%%%%%%%%%%%%%%
\pagebreak
\subsubsection{\code{map\_atomic\_sub}}
\label{api:c:map_atomic_sub}
\index{map\_atomic\_sub!C API}
Subtract the specified number from the value of a key-value pair within each
map.
This operation requires a pre-existing object in order to complete successfully.
If no object exists, the operation will fail with \code{NOTFOUND}.



\paragraph{Definition:}
\begin{ccode}
int64_t hyperdex_client_map_atomic_sub(struct hyperdex_client* client,
        const char* space,
        const char* key, size_t key_sz,
        const struct hyperdex_client_map_attribute* mapattrs, size_t mapattrs_sz,
        enum hyperdex_client_returncode* status);
\end{ccode}

\paragraph{Parameters:}
\begin{itemize}[noitemsep]
\item \code{space}\\
The name of the space as a c-string.
\item \code{key}, \code{key\_sz}\\
The key for the operation where \code{key} is a bytestring and \code{key\_sz} specifies the number of bytes in \code{key}.
\item \code{mapattrs}, \code{mapattrs\_sz}\\
The set of map attributes to modify and their respective key/values.  \code{mapattrs} points to an array of length \code{mapattrs\_sz}.  Each entry specify an attribute that is a map and a key within that map.
\end{itemize}

\paragraph{Returns:}
\begin{itemize}[noitemsep]
\item \code{status}\\
The status of the operation.  The client library will fill in this variable before returning this operation's request id from \code{hyperdex\_client\_loop}.  The pointer must remain valid until then, and the pointer should not be aliased to the status for any other outstanding operation.
\end{itemize}

%%%%%%%%%%%%%%%%%%%% cond_map_atomic_sub %%%%%%%%%%%%%%%%%%%%
\pagebreak
\subsubsection{\code{cond\_map\_atomic\_sub}}
\label{api:c:cond_map_atomic_sub}
\index{cond\_map\_atomic\_sub!C API}
Subtract the specified number from the value of a key-value pair within each
map if and only if the \code{checks} hold on the object.
This operation requires a pre-existing object in order to complete successfully.
If no object exists, the operation will fail with \code{NOTFOUND}.


This operation will succeed if and only if the predicates specified by
\code{checks} hold on the pre-existing object.  If any of the predicates are not
true for the existing object, then the operation will have no effect and fail
with \code{CMPFAIL}.

All checks are atomic with the write.  HyperDex guarantees that no other
operation will come between validating the checks, and writing the new version
of the object.



\paragraph{Definition:}
\begin{ccode}
int64_t hyperdex_client_cond_map_atomic_sub(struct hyperdex_client* client,
        const char* space,
        const char* key, size_t key_sz,
        const struct hyperdex_client_attribute_check* checks, size_t checks_sz,
        const struct hyperdex_client_map_attribute* mapattrs, size_t mapattrs_sz,
        enum hyperdex_client_returncode* status);
\end{ccode}

\paragraph{Parameters:}
\begin{itemize}[noitemsep]
\item \code{space}\\
The name of the space as a c-string.
\item \code{key}, \code{key\_sz}\\
The key for the operation where \code{key} is a bytestring and \code{key\_sz} specifies the number of bytes in \code{key}.
\item \code{checks}, \code{checks\_sz}\\
A set of predicates to check against.  \code{checks} points to an array of length \code{checks\_sz}.
\item \code{mapattrs}, \code{mapattrs\_sz}\\
The set of map attributes to modify and their respective key/values.  \code{mapattrs} points to an array of length \code{mapattrs\_sz}.  Each entry specify an attribute that is a map and a key within that map.
\end{itemize}

\paragraph{Returns:}
\begin{itemize}[noitemsep]
\item \code{status}\\
The status of the operation.  The client library will fill in this variable before returning this operation's request id from \code{hyperdex\_client\_loop}.  The pointer must remain valid until then, and the pointer should not be aliased to the status for any other outstanding operation.
\end{itemize}

%%%%%%%%%%%%%%%%%%%% map_atomic_mul %%%%%%%%%%%%%%%%%%%%
\pagebreak
\subsubsection{\code{map\_atomic\_mul}}
\label{api:c:map_atomic_mul}
\index{map\_atomic\_mul!C API}
\input{\topdir/api/desc/map_atomic_mul}

\paragraph{Definition:}
\begin{ccode}
int64_t hyperdex_client_map_atomic_mul(struct hyperdex_client* client,
        const char* space,
        const char* key, size_t key_sz,
        const struct hyperdex_client_map_attribute* mapattrs, size_t mapattrs_sz,
        enum hyperdex_client_returncode* status);
\end{ccode}

\paragraph{Parameters:}
\begin{itemize}[noitemsep]
\item \code{space}\\
The name of the space as a c-string.
\item \code{key}, \code{key\_sz}\\
The key for the operation where \code{key} is a bytestring and \code{key\_sz} specifies the number of bytes in \code{key}.
\item \code{mapattrs}, \code{mapattrs\_sz}\\
The set of map attributes to modify and their respective key/values.  \code{mapattrs} points to an array of length \code{mapattrs\_sz}.  Each entry specify an attribute that is a map and a key within that map.
\end{itemize}

\paragraph{Returns:}
\begin{itemize}[noitemsep]
\item \code{status}\\
The status of the operation.  The client library will fill in this variable before returning this operation's request id from \code{hyperdex\_client\_loop}.  The pointer must remain valid until then, and the pointer should not be aliased to the status for any other outstanding operation.
\end{itemize}

%%%%%%%%%%%%%%%%%%%% cond_map_atomic_mul %%%%%%%%%%%%%%%%%%%%
\pagebreak
\subsubsection{\code{cond\_map\_atomic\_mul}}
\label{api:c:cond_map_atomic_mul}
\index{cond\_map\_atomic\_mul!C API}
\input{\topdir/api/desc/cond_map_atomic_mul}

\paragraph{Definition:}
\begin{ccode}
int64_t hyperdex_client_cond_map_atomic_mul(struct hyperdex_client* client,
        const char* space,
        const char* key, size_t key_sz,
        const struct hyperdex_client_attribute_check* checks, size_t checks_sz,
        const struct hyperdex_client_map_attribute* mapattrs, size_t mapattrs_sz,
        enum hyperdex_client_returncode* status);
\end{ccode}

\paragraph{Parameters:}
\begin{itemize}[noitemsep]
\item \code{space}\\
The name of the space as a c-string.
\item \code{key}, \code{key\_sz}\\
The key for the operation where \code{key} is a bytestring and \code{key\_sz} specifies the number of bytes in \code{key}.
\item \code{checks}, \code{checks\_sz}\\
A set of predicates to check against.  \code{checks} points to an array of length \code{checks\_sz}.
\item \code{mapattrs}, \code{mapattrs\_sz}\\
The set of map attributes to modify and their respective key/values.  \code{mapattrs} points to an array of length \code{mapattrs\_sz}.  Each entry specify an attribute that is a map and a key within that map.
\end{itemize}

\paragraph{Returns:}
\begin{itemize}[noitemsep]
\item \code{status}\\
The status of the operation.  The client library will fill in this variable before returning this operation's request id from \code{hyperdex\_client\_loop}.  The pointer must remain valid until then, and the pointer should not be aliased to the status for any other outstanding operation.
\end{itemize}

%%%%%%%%%%%%%%%%%%%% map_atomic_div %%%%%%%%%%%%%%%%%%%%
\pagebreak
\subsubsection{\code{map\_atomic\_div}}
\label{api:c:map_atomic_div}
\index{map\_atomic\_div!C API}
Divide the value of each key-value pair by the specified number for each map.

%%% Generated below here
\paragraph{Behavior:}
\begin{itemize}[noitemsep]
This operation requires a pre-existing object in order to complete successfully.
If no object exists, the operation will fail with \code{NOTFOUND}.

\item This operation mutates the value of a key-value pair in a map.  This call
    is similar to the equivalent call without the \code{map\_} prefix, but
    operates on the value of a pair in a map, instead of on an attribute's
    value.  If there is no pair with the specified map key, a new pair will be
    created and initialized to its default value.  If this is undesirable, it
    may be avoided by using a conditional operation that requires that the map
    contain the key in question.

\end{itemize}


\paragraph{Definition:}
\begin{ccode}
int64_t hyperdex_client_map_atomic_div(struct hyperdex_client* client,
        const char* space,
        const char* key, size_t key_sz,
        const struct hyperdex_client_map_attribute* mapattrs, size_t mapattrs_sz,
        enum hyperdex_client_returncode* status);
\end{ccode}

\paragraph{Parameters:}
\begin{itemize}[noitemsep]
\item \code{space}\\
The name of the space as a c-string.
\item \code{key}, \code{key\_sz}\\
The key for the operation where \code{key} is a bytestring and \code{key\_sz} specifies the number of bytes in \code{key}.
\item \code{mapattrs}, \code{mapattrs\_sz}\\
The set of map attributes to modify and their respective key/values.  \code{mapattrs} points to an array of length \code{mapattrs\_sz}.  Each entry specify an attribute that is a map and a key within that map.
\end{itemize}

\paragraph{Returns:}
\begin{itemize}[noitemsep]
\item \code{status}\\
The status of the operation.  The client library will fill in this variable before returning this operation's request id from \code{hyperdex\_client\_loop}.  The pointer must remain valid until then, and the pointer should not be aliased to the status for any other outstanding operation.
\end{itemize}

%%%%%%%%%%%%%%%%%%%% cond_map_atomic_div %%%%%%%%%%%%%%%%%%%%
\pagebreak
\subsubsection{\code{cond\_map\_atomic\_div}}
\label{api:c:cond_map_atomic_div}
\index{cond\_map\_atomic\_div!C API}
Divide the value of each key-value pair by the specified number for each map if
and only if the \code{checks} hold on the object.
This operation requires a pre-existing object in order to complete successfully.
If no object exists, the operation will fail with \code{NOTFOUND}.


This operation will succeed if and only if the predicates specified by
\code{checks} hold on the pre-existing object.  If any of the predicates are not
true for the existing object, then the operation will have no effect and fail
with \code{CMPFAIL}.

All checks are atomic with the write.  HyperDex guarantees that no other
operation will come between validating the checks, and writing the new version
of the object.



\paragraph{Definition:}
\begin{ccode}
int64_t hyperdex_client_cond_map_atomic_div(struct hyperdex_client* client,
        const char* space,
        const char* key, size_t key_sz,
        const struct hyperdex_client_attribute_check* checks, size_t checks_sz,
        const struct hyperdex_client_map_attribute* mapattrs, size_t mapattrs_sz,
        enum hyperdex_client_returncode* status);
\end{ccode}

\paragraph{Parameters:}
\begin{itemize}[noitemsep]
\item \code{space}\\
The name of the space as a c-string.
\item \code{key}, \code{key\_sz}\\
The key for the operation where \code{key} is a bytestring and \code{key\_sz} specifies the number of bytes in \code{key}.
\item \code{checks}, \code{checks\_sz}\\
A set of predicates to check against.  \code{checks} points to an array of length \code{checks\_sz}.
\item \code{mapattrs}, \code{mapattrs\_sz}\\
The set of map attributes to modify and their respective key/values.  \code{mapattrs} points to an array of length \code{mapattrs\_sz}.  Each entry specify an attribute that is a map and a key within that map.
\end{itemize}

\paragraph{Returns:}
\begin{itemize}[noitemsep]
\item \code{status}\\
The status of the operation.  The client library will fill in this variable before returning this operation's request id from \code{hyperdex\_client\_loop}.  The pointer must remain valid until then, and the pointer should not be aliased to the status for any other outstanding operation.
\end{itemize}

%%%%%%%%%%%%%%%%%%%% map_atomic_mod %%%%%%%%%%%%%%%%%%%%
\pagebreak
\subsubsection{\code{map\_atomic\_mod}}
\label{api:c:map_atomic_mod}
\index{map\_atomic\_mod!C API}
Store the value of the key-value pair modulo the specified number for each map.
This operation requires a pre-existing object in order to complete successfully.
If no object exists, the operation will fail with \code{NOTFOUND}.



\paragraph{Definition:}
\begin{ccode}
int64_t hyperdex_client_map_atomic_mod(struct hyperdex_client* client,
        const char* space,
        const char* key, size_t key_sz,
        const struct hyperdex_client_map_attribute* mapattrs, size_t mapattrs_sz,
        enum hyperdex_client_returncode* status);
\end{ccode}

\paragraph{Parameters:}
\begin{itemize}[noitemsep]
\item \code{space}\\
The name of the space as a c-string.
\item \code{key}, \code{key\_sz}\\
The key for the operation where \code{key} is a bytestring and \code{key\_sz} specifies the number of bytes in \code{key}.
\item \code{mapattrs}, \code{mapattrs\_sz}\\
The set of map attributes to modify and their respective key/values.  \code{mapattrs} points to an array of length \code{mapattrs\_sz}.  Each entry specify an attribute that is a map and a key within that map.
\end{itemize}

\paragraph{Returns:}
\begin{itemize}[noitemsep]
\item \code{status}\\
The status of the operation.  The client library will fill in this variable before returning this operation's request id from \code{hyperdex\_client\_loop}.  The pointer must remain valid until then, and the pointer should not be aliased to the status for any other outstanding operation.
\end{itemize}

%%%%%%%%%%%%%%%%%%%% cond_map_atomic_mod %%%%%%%%%%%%%%%%%%%%
\pagebreak
\subsubsection{\code{cond\_map\_atomic\_mod}}
\label{api:c:cond_map_atomic_mod}
\index{cond\_map\_atomic\_mod!C API}
Conditionally store the value of the key-value pair modulo the specified number
for each map.

%%% Generated below here
\paragraph{Behavior:}
\begin{itemize}[noitemsep]
This operation requires a pre-existing object in order to complete successfully.
If no object exists, the operation will fail with \code{NOTFOUND}.

This operation will succeed if and only if the predicates specified by
\code{checks} hold on the pre-existing object.  If any of the predicates are not
true for the existing object, then the operation will have no effect and fail
with \code{CMPFAIL}.

All checks are atomic with the write.  HyperDex guarantees that no other
operation will come between validating the checks, and writing the new version
of the object.

\item This operation mutates the value of a key-value pair in a map.  This call
    is similar to the equivalent call without the \code{map\_} prefix, but
    operates on the value of a pair in a map, instead of on an attribute's
    value.  If there is no pair with the specified map key, a new pair will be
    created and initialized to its default value.  If this is undesirable, it
    may be avoided by using a conditional operation that requires that the map
    contain the key in question.

\end{itemize}


\paragraph{Definition:}
\begin{ccode}
int64_t hyperdex_client_cond_map_atomic_mod(struct hyperdex_client* client,
        const char* space,
        const char* key, size_t key_sz,
        const struct hyperdex_client_attribute_check* checks, size_t checks_sz,
        const struct hyperdex_client_map_attribute* mapattrs, size_t mapattrs_sz,
        enum hyperdex_client_returncode* status);
\end{ccode}

\paragraph{Parameters:}
\begin{itemize}[noitemsep]
\item \code{space}\\
The name of the space as a c-string.
\item \code{key}, \code{key\_sz}\\
The key for the operation where \code{key} is a bytestring and \code{key\_sz} specifies the number of bytes in \code{key}.
\item \code{checks}, \code{checks\_sz}\\
A set of predicates to check against.  \code{checks} points to an array of length \code{checks\_sz}.
\item \code{mapattrs}, \code{mapattrs\_sz}\\
The set of map attributes to modify and their respective key/values.  \code{mapattrs} points to an array of length \code{mapattrs\_sz}.  Each entry specify an attribute that is a map and a key within that map.
\end{itemize}

\paragraph{Returns:}
\begin{itemize}[noitemsep]
\item \code{status}\\
The status of the operation.  The client library will fill in this variable before returning this operation's request id from \code{hyperdex\_client\_loop}.  The pointer must remain valid until then, and the pointer should not be aliased to the status for any other outstanding operation.
\end{itemize}

%%%%%%%%%%%%%%%%%%%% map_atomic_and %%%%%%%%%%%%%%%%%%%%
\pagebreak
\subsubsection{\code{map\_atomic\_and}}
\label{api:c:map_atomic_and}
\index{map\_atomic\_and!C API}
Store the bitwise AND of the value of the key-value pair and the specified
number for each map.
This operation requires a pre-existing object in order to complete successfully.
If no object exists, the operation will fail with \code{NOTFOUND}.



\paragraph{Definition:}
\begin{ccode}
int64_t hyperdex_client_map_atomic_and(struct hyperdex_client* client,
        const char* space,
        const char* key, size_t key_sz,
        const struct hyperdex_client_map_attribute* mapattrs, size_t mapattrs_sz,
        enum hyperdex_client_returncode* status);
\end{ccode}

\paragraph{Parameters:}
\begin{itemize}[noitemsep]
\item \code{space}\\
The name of the space as a c-string.
\item \code{key}, \code{key\_sz}\\
The key for the operation where \code{key} is a bytestring and \code{key\_sz} specifies the number of bytes in \code{key}.
\item \code{mapattrs}, \code{mapattrs\_sz}\\
The set of map attributes to modify and their respective key/values.  \code{mapattrs} points to an array of length \code{mapattrs\_sz}.  Each entry specify an attribute that is a map and a key within that map.
\end{itemize}

\paragraph{Returns:}
\begin{itemize}[noitemsep]
\item \code{status}\\
The status of the operation.  The client library will fill in this variable before returning this operation's request id from \code{hyperdex\_client\_loop}.  The pointer must remain valid until then, and the pointer should not be aliased to the status for any other outstanding operation.
\end{itemize}

%%%%%%%%%%%%%%%%%%%% cond_map_atomic_and %%%%%%%%%%%%%%%%%%%%
\pagebreak
\subsubsection{\code{cond\_map\_atomic\_and}}
\label{api:c:cond_map_atomic_and}
\index{cond\_map\_atomic\_and!C API}
\input{\topdir/api/desc/cond_map_atomic_and}

\paragraph{Definition:}
\begin{ccode}
int64_t hyperdex_client_cond_map_atomic_and(struct hyperdex_client* client,
        const char* space,
        const char* key, size_t key_sz,
        const struct hyperdex_client_attribute_check* checks, size_t checks_sz,
        const struct hyperdex_client_map_attribute* mapattrs, size_t mapattrs_sz,
        enum hyperdex_client_returncode* status);
\end{ccode}

\paragraph{Parameters:}
\begin{itemize}[noitemsep]
\item \code{space}\\
The name of the space as a c-string.
\item \code{key}, \code{key\_sz}\\
The key for the operation where \code{key} is a bytestring and \code{key\_sz} specifies the number of bytes in \code{key}.
\item \code{checks}, \code{checks\_sz}\\
A set of predicates to check against.  \code{checks} points to an array of length \code{checks\_sz}.
\item \code{mapattrs}, \code{mapattrs\_sz}\\
The set of map attributes to modify and their respective key/values.  \code{mapattrs} points to an array of length \code{mapattrs\_sz}.  Each entry specify an attribute that is a map and a key within that map.
\end{itemize}

\paragraph{Returns:}
\begin{itemize}[noitemsep]
\item \code{status}\\
The status of the operation.  The client library will fill in this variable before returning this operation's request id from \code{hyperdex\_client\_loop}.  The pointer must remain valid until then, and the pointer should not be aliased to the status for any other outstanding operation.
\end{itemize}

%%%%%%%%%%%%%%%%%%%% map_atomic_or %%%%%%%%%%%%%%%%%%%%
\pagebreak
\subsubsection{\code{map\_atomic\_or}}
\label{api:c:map_atomic_or}
\index{map\_atomic\_or!C API}
Store the bitwise OR of the value of the key-value pair and the specified number
for each map.

%%% Generated below here
\paragraph{Behavior:}
\begin{itemize}[noitemsep]
This operation requires a pre-existing object in order to complete successfully.
If no object exists, the operation will fail with \code{NOTFOUND}.

\item This operation mutates the value of a key-value pair in a map.  This call
    is similar to the equivalent call without the \code{map\_} prefix, but
    operates on the value of a pair in a map, instead of on an attribute's
    value.  If there is no pair with the specified map key, a new pair will be
    created and initialized to its default value.  If this is undesirable, it
    may be avoided by using a conditional operation that requires that the map
    contain the key in question.

\end{itemize}


\paragraph{Definition:}
\begin{ccode}
int64_t hyperdex_client_map_atomic_or(struct hyperdex_client* client,
        const char* space,
        const char* key, size_t key_sz,
        const struct hyperdex_client_map_attribute* mapattrs, size_t mapattrs_sz,
        enum hyperdex_client_returncode* status);
\end{ccode}

\paragraph{Parameters:}
\begin{itemize}[noitemsep]
\item \code{space}\\
The name of the space as a c-string.
\item \code{key}, \code{key\_sz}\\
The key for the operation where \code{key} is a bytestring and \code{key\_sz} specifies the number of bytes in \code{key}.
\item \code{mapattrs}, \code{mapattrs\_sz}\\
The set of map attributes to modify and their respective key/values.  \code{mapattrs} points to an array of length \code{mapattrs\_sz}.  Each entry specify an attribute that is a map and a key within that map.
\end{itemize}

\paragraph{Returns:}
\begin{itemize}[noitemsep]
\item \code{status}\\
The status of the operation.  The client library will fill in this variable before returning this operation's request id from \code{hyperdex\_client\_loop}.  The pointer must remain valid until then, and the pointer should not be aliased to the status for any other outstanding operation.
\end{itemize}

%%%%%%%%%%%%%%%%%%%% cond_map_atomic_or %%%%%%%%%%%%%%%%%%%%
\pagebreak
\subsubsection{\code{cond\_map\_atomic\_or}}
\label{api:c:cond_map_atomic_or}
\index{cond\_map\_atomic\_or!C API}
\input{\topdir/api/desc/cond_map_atomic_or}

\paragraph{Definition:}
\begin{ccode}
int64_t hyperdex_client_cond_map_atomic_or(struct hyperdex_client* client,
        const char* space,
        const char* key, size_t key_sz,
        const struct hyperdex_client_attribute_check* checks, size_t checks_sz,
        const struct hyperdex_client_map_attribute* mapattrs, size_t mapattrs_sz,
        enum hyperdex_client_returncode* status);
\end{ccode}

\paragraph{Parameters:}
\begin{itemize}[noitemsep]
\item \code{space}\\
The name of the space as a c-string.
\item \code{key}, \code{key\_sz}\\
The key for the operation where \code{key} is a bytestring and \code{key\_sz} specifies the number of bytes in \code{key}.
\item \code{checks}, \code{checks\_sz}\\
A set of predicates to check against.  \code{checks} points to an array of length \code{checks\_sz}.
\item \code{mapattrs}, \code{mapattrs\_sz}\\
The set of map attributes to modify and their respective key/values.  \code{mapattrs} points to an array of length \code{mapattrs\_sz}.  Each entry specify an attribute that is a map and a key within that map.
\end{itemize}

\paragraph{Returns:}
\begin{itemize}[noitemsep]
\item \code{status}\\
The status of the operation.  The client library will fill in this variable before returning this operation's request id from \code{hyperdex\_client\_loop}.  The pointer must remain valid until then, and the pointer should not be aliased to the status for any other outstanding operation.
\end{itemize}

%%%%%%%%%%%%%%%%%%%% map_atomic_xor %%%%%%%%%%%%%%%%%%%%
\pagebreak
\subsubsection{\code{map\_atomic\_xor}}
\label{api:c:map_atomic_xor}
\index{map\_atomic\_xor!C API}
Store the bitwise XOR of the value of the key-value pair and the specified
number for each map attribute.
This operation requires a pre-existing object in order to complete successfully.
If no object exists, the operation will fail with \code{NOTFOUND}.



\paragraph{Definition:}
\begin{ccode}
int64_t hyperdex_client_map_atomic_xor(struct hyperdex_client* client,
        const char* space,
        const char* key, size_t key_sz,
        const struct hyperdex_client_map_attribute* mapattrs, size_t mapattrs_sz,
        enum hyperdex_client_returncode* status);
\end{ccode}

\paragraph{Parameters:}
\begin{itemize}[noitemsep]
\item \code{space}\\
The name of the space as a c-string.
\item \code{key}, \code{key\_sz}\\
The key for the operation where \code{key} is a bytestring and \code{key\_sz} specifies the number of bytes in \code{key}.
\item \code{mapattrs}, \code{mapattrs\_sz}\\
The set of map attributes to modify and their respective key/values.  \code{mapattrs} points to an array of length \code{mapattrs\_sz}.  Each entry specify an attribute that is a map and a key within that map.
\end{itemize}

\paragraph{Returns:}
\begin{itemize}[noitemsep]
\item \code{status}\\
The status of the operation.  The client library will fill in this variable before returning this operation's request id from \code{hyperdex\_client\_loop}.  The pointer must remain valid until then, and the pointer should not be aliased to the status for any other outstanding operation.
\end{itemize}

%%%%%%%%%%%%%%%%%%%% cond_map_atomic_xor %%%%%%%%%%%%%%%%%%%%
\pagebreak
\subsubsection{\code{cond\_map\_atomic\_xor}}
\label{api:c:cond_map_atomic_xor}
\index{cond\_map\_atomic\_xor!C API}
Conditionally store the bitwise XOR of the value of the key-value pair and the
specified number for each map.

%%% Generated below here
\paragraph{Behavior:}
\begin{itemize}[noitemsep]
This operation requires a pre-existing object in order to complete successfully.
If no object exists, the operation will fail with \code{NOTFOUND}.

This operation will succeed if and only if the predicates specified by
\code{checks} hold on the pre-existing object.  If any of the predicates are not
true for the existing object, then the operation will have no effect and fail
with \code{CMPFAIL}.

All checks are atomic with the write.  HyperDex guarantees that no other
operation will come between validating the checks, and writing the new version
of the object.

\item This operation mutates the value of a key-value pair in a map.  This call
    is similar to the equivalent call without the \code{map\_} prefix, but
    operates on the value of a pair in a map, instead of on an attribute's
    value.  If there is no pair with the specified map key, a new pair will be
    created and initialized to its default value.  If this is undesirable, it
    may be avoided by using a conditional operation that requires that the map
    contain the key in question.

\end{itemize}


\paragraph{Definition:}
\begin{ccode}
int64_t hyperdex_client_cond_map_atomic_xor(struct hyperdex_client* client,
        const char* space,
        const char* key, size_t key_sz,
        const struct hyperdex_client_attribute_check* checks, size_t checks_sz,
        const struct hyperdex_client_map_attribute* mapattrs, size_t mapattrs_sz,
        enum hyperdex_client_returncode* status);
\end{ccode}

\paragraph{Parameters:}
\begin{itemize}[noitemsep]
\item \code{space}\\
The name of the space as a c-string.
\item \code{key}, \code{key\_sz}\\
The key for the operation where \code{key} is a bytestring and \code{key\_sz} specifies the number of bytes in \code{key}.
\item \code{checks}, \code{checks\_sz}\\
A set of predicates to check against.  \code{checks} points to an array of length \code{checks\_sz}.
\item \code{mapattrs}, \code{mapattrs\_sz}\\
The set of map attributes to modify and their respective key/values.  \code{mapattrs} points to an array of length \code{mapattrs\_sz}.  Each entry specify an attribute that is a map and a key within that map.
\end{itemize}

\paragraph{Returns:}
\begin{itemize}[noitemsep]
\item \code{status}\\
The status of the operation.  The client library will fill in this variable before returning this operation's request id from \code{hyperdex\_client\_loop}.  The pointer must remain valid until then, and the pointer should not be aliased to the status for any other outstanding operation.
\end{itemize}

%%%%%%%%%%%%%%%%%%%% map_string_prepend %%%%%%%%%%%%%%%%%%%%
\pagebreak
\subsubsection{\code{map\_string\_prepend}}
\label{api:c:map_string_prepend}
\index{map\_string\_prepend!C API}
Prepend the specified string to the value of the key-value pair for each map.

%%% Generated below here
\paragraph{Behavior:}
\begin{itemize}[noitemsep]
This operation requires a pre-existing object in order to complete successfully.
If no object exists, the operation will fail with \code{NOTFOUND}.

\item This operation mutates the value of a key-value pair in a map.  This call
    is similar to the equivalent call without the \code{map\_} prefix, but
    operates on the value of a pair in a map, instead of on an attribute's
    value.  If there is no pair with the specified map key, a new pair will be
    created and initialized to its default value.  If this is undesirable, it
    may be avoided by using a conditional operation that requires that the map
    contain the key in question.

\end{itemize}


\paragraph{Definition:}
\begin{ccode}
int64_t hyperdex_client_map_string_prepend(struct hyperdex_client* client,
        const char* space,
        const char* key, size_t key_sz,
        const struct hyperdex_client_map_attribute* mapattrs, size_t mapattrs_sz,
        enum hyperdex_client_returncode* status);
\end{ccode}

\paragraph{Parameters:}
\begin{itemize}[noitemsep]
\item \code{space}\\
The name of the space as a c-string.
\item \code{key}, \code{key\_sz}\\
The key for the operation where \code{key} is a bytestring and \code{key\_sz} specifies the number of bytes in \code{key}.
\item \code{mapattrs}, \code{mapattrs\_sz}\\
The set of map attributes to modify and their respective key/values.  \code{mapattrs} points to an array of length \code{mapattrs\_sz}.  Each entry specify an attribute that is a map and a key within that map.
\end{itemize}

\paragraph{Returns:}
\begin{itemize}[noitemsep]
\item \code{status}\\
The status of the operation.  The client library will fill in this variable before returning this operation's request id from \code{hyperdex\_client\_loop}.  The pointer must remain valid until then, and the pointer should not be aliased to the status for any other outstanding operation.
\end{itemize}

%%%%%%%%%%%%%%%%%%%% cond_map_string_prepend %%%%%%%%%%%%%%%%%%%%
\pagebreak
\subsubsection{\code{cond\_map\_string\_prepend}}
\label{api:c:cond_map_string_prepend}
\index{cond\_map\_string\_prepend!C API}
\input{\topdir/api/desc/cond_map_string_prepend}

\paragraph{Definition:}
\begin{ccode}
int64_t hyperdex_client_cond_map_string_prepend(struct hyperdex_client* client,
        const char* space,
        const char* key, size_t key_sz,
        const struct hyperdex_client_attribute_check* checks, size_t checks_sz,
        const struct hyperdex_client_map_attribute* mapattrs, size_t mapattrs_sz,
        enum hyperdex_client_returncode* status);
\end{ccode}

\paragraph{Parameters:}
\begin{itemize}[noitemsep]
\item \code{space}\\
The name of the space as a c-string.
\item \code{key}, \code{key\_sz}\\
The key for the operation where \code{key} is a bytestring and \code{key\_sz} specifies the number of bytes in \code{key}.
\item \code{checks}, \code{checks\_sz}\\
A set of predicates to check against.  \code{checks} points to an array of length \code{checks\_sz}.
\item \code{mapattrs}, \code{mapattrs\_sz}\\
The set of map attributes to modify and their respective key/values.  \code{mapattrs} points to an array of length \code{mapattrs\_sz}.  Each entry specify an attribute that is a map and a key within that map.
\end{itemize}

\paragraph{Returns:}
\begin{itemize}[noitemsep]
\item \code{status}\\
The status of the operation.  The client library will fill in this variable before returning this operation's request id from \code{hyperdex\_client\_loop}.  The pointer must remain valid until then, and the pointer should not be aliased to the status for any other outstanding operation.
\end{itemize}

%%%%%%%%%%%%%%%%%%%% map_string_append %%%%%%%%%%%%%%%%%%%%
\pagebreak
\subsubsection{\code{map\_string\_append}}
\label{api:c:map_string_append}
\index{map\_string\_append!C API}
Append the specified string to the value of the key-value pair for each map
attribute.
This operation requires a pre-existing object in order to complete successfully.
If no object exists, the operation will fail with \code{NOTFOUND}.



\paragraph{Definition:}
\begin{ccode}
int64_t hyperdex_client_map_string_append(struct hyperdex_client* client,
        const char* space,
        const char* key, size_t key_sz,
        const struct hyperdex_client_map_attribute* mapattrs, size_t mapattrs_sz,
        enum hyperdex_client_returncode* status);
\end{ccode}

\paragraph{Parameters:}
\begin{itemize}[noitemsep]
\item \code{space}\\
The name of the space as a c-string.
\item \code{key}, \code{key\_sz}\\
The key for the operation where \code{key} is a bytestring and \code{key\_sz} specifies the number of bytes in \code{key}.
\item \code{mapattrs}, \code{mapattrs\_sz}\\
The set of map attributes to modify and their respective key/values.  \code{mapattrs} points to an array of length \code{mapattrs\_sz}.  Each entry specify an attribute that is a map and a key within that map.
\end{itemize}

\paragraph{Returns:}
\begin{itemize}[noitemsep]
\item \code{status}\\
The status of the operation.  The client library will fill in this variable before returning this operation's request id from \code{hyperdex\_client\_loop}.  The pointer must remain valid until then, and the pointer should not be aliased to the status for any other outstanding operation.
\end{itemize}

%%%%%%%%%%%%%%%%%%%% cond_map_string_append %%%%%%%%%%%%%%%%%%%%
\pagebreak
\subsubsection{\code{cond\_map\_string\_append}}
\label{api:c:cond_map_string_append}
\index{cond\_map\_string\_append!C API}
Append the specified string to the value of the key-value pair for each map
attribute if and only if the \code{checks} hold on the object.
This operation requires a pre-existing object in order to complete successfully.
If no object exists, the operation will fail with \code{NOTFOUND}.


This operation will succeed if and only if the predicates specified by
\code{checks} hold on the pre-existing object.  If any of the predicates are not
true for the existing object, then the operation will have no effect and fail
with \code{CMPFAIL}.

All checks are atomic with the write.  HyperDex guarantees that no other
operation will come between validating the checks, and writing the new version
of the object.



\paragraph{Definition:}
\begin{ccode}
int64_t hyperdex_client_cond_map_string_append(struct hyperdex_client* client,
        const char* space,
        const char* key, size_t key_sz,
        const struct hyperdex_client_attribute_check* checks, size_t checks_sz,
        const struct hyperdex_client_map_attribute* mapattrs, size_t mapattrs_sz,
        enum hyperdex_client_returncode* status);
\end{ccode}

\paragraph{Parameters:}
\begin{itemize}[noitemsep]
\item \code{space}\\
The name of the space as a c-string.
\item \code{key}, \code{key\_sz}\\
The key for the operation where \code{key} is a bytestring and \code{key\_sz} specifies the number of bytes in \code{key}.
\item \code{checks}, \code{checks\_sz}\\
A set of predicates to check against.  \code{checks} points to an array of length \code{checks\_sz}.
\item \code{mapattrs}, \code{mapattrs\_sz}\\
The set of map attributes to modify and their respective key/values.  \code{mapattrs} points to an array of length \code{mapattrs\_sz}.  Each entry specify an attribute that is a map and a key within that map.
\end{itemize}

\paragraph{Returns:}
\begin{itemize}[noitemsep]
\item \code{status}\\
The status of the operation.  The client library will fill in this variable before returning this operation's request id from \code{hyperdex\_client\_loop}.  The pointer must remain valid until then, and the pointer should not be aliased to the status for any other outstanding operation.
\end{itemize}

%%%%%%%%%%%%%%%%%%%% search %%%%%%%%%%%%%%%%%%%%
\pagebreak
\subsubsection{\code{search}}
\label{api:c:search}
\index{search!C API}
Return all objects that match the specified \code{checks}.

\paragraph{Behavior:}
\begin{itemize}[noitemsep]
This operation behaves as an iterator and may return multiple objects from the
single call.

\item This operation return to the user the requested object(s).

\end{itemize}


\paragraph{Definition:}
\begin{ccode}
int64_t hyperdex_client_search(struct hyperdex_client* client,
        const char* space,
        const struct hyperdex_client_attribute_check* checks, size_t checks_sz,
        enum hyperdex_client_returncode* status,
        const struct hyperdex_client_attribute** attrs, size_t* attrs_sz);
\end{ccode}

\paragraph{Parameters:}
\begin{itemize}[noitemsep]
\item \code{space}\\
The name of the space as a c-string.
\item \code{checks}, \code{checks\_sz}\\
A set of predicates to check against.  \code{checks} points to an array of length \code{checks\_sz}.
\end{itemize}

\paragraph{Returns:}
\begin{itemize}[noitemsep]
\item \code{status}\\
The status of the operation.  The client library will fill in this variable before returning this operation's request id from \code{hyperdex\_client\_loop}.  The pointer must remain valid until the operation completes, and the pointer should not be aliased to the status for any other outstanding operation.
\item \code{attrs}, \code{attrs\_sz}\\
An array of attributes that comprise a returned object.  The application must free the returned values with \code{hyperdex\_client\_destroy\_attrs}.  The pointers must remain valid until the operation completes.
\end{itemize}

%%%%%%%%%%%%%%%%%%%% search_describe %%%%%%%%%%%%%%%%%%%%
\pagebreak
\subsubsection{\code{search\_describe}}
\label{api:c:search_describe}
\index{search\_describe!C API}
Return a human-readable string suitable for debugging search internals.  This
API is only really relevant for debugging the internals of \code{search}.


\paragraph{Definition:}
\begin{ccode}
int64_t hyperdex_client_search_describe(struct hyperdex_client* client,
        const char* space,
        const struct hyperdex_client_attribute_check* checks, size_t checks_sz,
        enum hyperdex_client_returncode* status,
        const char** description);
\end{ccode}

\paragraph{Parameters:}
\begin{itemize}[noitemsep]
\item \code{space}\\
The name of the space as a c-string.
\item \code{checks}, \code{checks\_sz}\\
A set of predicates to check against.  \code{checks} points to an array of length \code{checks\_sz}.
\end{itemize}

\paragraph{Returns:}
\begin{itemize}[noitemsep]
\item \code{status}\\
The status of the operation.  The client library will fill in this variable before returning this operation's request id from \code{hyperdex\_client\_loop}.  The pointer must remain valid until then, and the pointer should not be aliased to the status for any other outstanding operation.
\item \code{description}\\
The description of the search.  This is a c-string that the client must free.
\end{itemize}

%%%%%%%%%%%%%%%%%%%% sorted_search %%%%%%%%%%%%%%%%%%%%
\pagebreak
\subsubsection{\code{sorted\_search}}
\label{api:c:sorted_search}
\index{sorted\_search!C API}
Return all objects that match the specified \code{checks}, sorted according to
\code{attr}.
This operation behaves as an iterator and may return multiple objects from the
single call.



\paragraph{Definition:}
\begin{ccode}
int64_t hyperdex_client_sorted_search(struct hyperdex_client* client,
        const char* space,
        const struct hyperdex_client_attribute_check* checks, size_t checks_sz,
        const char* sort_by,
        uint64_t limit,
        int maxmin,
        enum hyperdex_client_returncode* status,
        const struct hyperdex_client_attribute** attrs, size_t* attrs_sz);
\end{ccode}

\paragraph{Parameters:}
\begin{itemize}[noitemsep]
\item \code{space}\\
The name of the space as a c-string.
\item \code{checks}, \code{checks\_sz}\\
A set of predicates to check against.  \code{checks} points to an array of length \code{checks\_sz}.
\item \code{sort\_by}\\
The attribute to sort by.
\item \code{limit}\\
The number of results to return.
\item \code{maxmin}\\
Maximize (!= 0) or minimize (== 0).
\end{itemize}

\paragraph{Returns:}
\begin{itemize}[noitemsep]
\item \code{status}\\
The status of the operation.  The client library will fill in this variable before returning this operation's request id from \code{hyperdex\_client\_loop}.  The pointer must remain valid until the operation completes, and the pointer should not be aliased to the status for any other outstanding operation.
\item \code{attrs}, \code{attrs\_sz}\\
An array of attributes that comprise a returned object.  The application must free the returned values with \code{hyperdex\_client\_destroy\_attrs}.  The pointers must remain valid until the operation completes.
\end{itemize}

%%%%%%%%%%%%%%%%%%%% group_del %%%%%%%%%%%%%%%%%%%%
\pagebreak
\subsubsection{\code{group\_del}}
\label{api:c:group_del}
\index{group\_del!C API}
Asynchronously delete all objects that match the specified \code{checks}.

\paragraph{Behavior:}
\begin{itemize}[noitemsep]
\item This operation is roughly equivalent to a client manually deleting every
    object returned from a search, but saves HyperDex from sending to the client
    objects that are soon to be deleted.
\end{itemize}


\paragraph{Definition:}
\begin{ccode}
int64_t hyperdex_client_group_del(struct hyperdex_client* client,
        const char* space,
        const struct hyperdex_client_attribute_check* checks, size_t checks_sz,
        enum hyperdex_client_returncode* status);
\end{ccode}

\paragraph{Parameters:}
\begin{itemize}[noitemsep]
\item \code{space}\\
The name of the space as a c-string.
\item \code{checks}, \code{checks\_sz}\\
A set of predicates to check against.  \code{checks} points to an array of length \code{checks\_sz}.
\end{itemize}

\paragraph{Returns:}
\begin{itemize}[noitemsep]
\item \code{status}\\
The status of the operation.  The client library will fill in this variable before returning this operation's request id from \code{hyperdex\_client\_loop}.  The pointer must remain valid until then, and the pointer should not be aliased to the status for any other outstanding operation.
\end{itemize}

%%%%%%%%%%%%%%%%%%%% count %%%%%%%%%%%%%%%%%%%%
\pagebreak
\subsubsection{\code{count}}
\label{api:c:count}
\index{count!C API}
Count the number of objects that match the specified \code{checks}.

\paragraph{Behavior:}
\begin{itemize}[noitemsep]
\item This will return the number of objects counted by the search.  If an error
    occurs during the count, the count will reflect a partial count.  The real
    count will be higher than the returned value.  Some languages will throw an
    exception rather than return the partial count.
\end{itemize}


\paragraph{Definition:}
\begin{ccode}
int64_t hyperdex_client_count(struct hyperdex_client* client,
        const char* space,
        const struct hyperdex_client_attribute_check* checks, size_t checks_sz,
        enum hyperdex_client_returncode* status,
        uint64_t* count);
\end{ccode}

\paragraph{Parameters:}
\begin{itemize}[noitemsep]
\item \code{space}\\
The name of the space as a c-string.
\item \code{checks}, \code{checks\_sz}\\
A set of predicates to check against.  \code{checks} points to an array of length \code{checks\_sz}.
\end{itemize}

\paragraph{Returns:}
\begin{itemize}[noitemsep]
\item \code{status}\\
The status of the operation.  The client library will fill in this variable before returning this operation's request id from \code{hyperdex\_client\_loop}.  The pointer must remain valid until then, and the pointer should not be aliased to the status for any other outstanding operation.
\item \code{count}\\
The number of objects which match the predicates.
\end{itemize}
